\documentclass[12pt,a4paper]{article}
\usepackage{fontspec}
\usepackage[margin=1in]{geometry}
\usepackage{setspace}
\onehalfspacing

% Set the main font to Adishila Vedic with Devanagari script support
\setmainfont{Adishila Vedic}[
    Path = C:/Users/sekha/OneDrive/Documents/GitHub/jaimineeyasamavedam/Rik_Processing/AdishilaVedic/,
    Extension = .ttf,
    UprightFont = AdishilaVedic,
    BoldFont = AdishilaVedicBold,
    Script=Devanagari,
    Renderer=HarfBuzz,
]

% Command to format accent marks with larger size and bold
\newcommand{\accentmark}[2]{%
    {\fontsize{#1pt}{#1pt}\selectfont\bfseries\addfontfeature{FakeBold=3}#2}%
}

% --- ACCENT OVERLAP ADJUSTMENT ---
% Defines a small negative space (kerning) to pull the next character closer
% Used when two "danger" accents (Anudatta/Kampa) appear consecutively.
\newcommand{\accentadj}{\kern0.15ex}



% --- ADD THESE LINES FOR AGGRESSIVE LINE BREAKING ---
\setlength{\emergencystretch}{1em}
\tolerance=10000 % Allow high tolerance for stretching/shrinking lines
\pretolerance=10000 % Allow high pretolerance
\emergencystretch=1in % Allow emergency stretching to fit text
\setlength{\parindent}{0pt} % Removes the indentation at start of paragraphs
\raggedright                % Forces text to align left, ignoring right margin evenness

\begin{document}
\fontsize{18pt}{27pt}\selectfont

\begin{center}
\Huge\textbf{जैमिनीय सामवेद संहिता}\\
\Large\textbf{श्रीमतेजैमिन्याचार्याय नमः ।\ \ }\\
\Large\textbf{श्रीतलवकार गुरवे नमः ।\ \ }\\
\Large\textbf{अस्मत्गुरुभ्योनमः।\ \ }\\
\end{center}
\vspace{2\baselineskip}

अग्ना\accentmark{22}{\char"0951}\kern0.15emआ\accentmark{27}{\char"1CD2}याहीवी\accentmark{27}{\char"1CD2}\kern0.15emता\accentmark{27}{\char"1CD2}\kern0.15emये।\ गृ\accentmark{27}{\char"1CD2}\kern0.15emणा\accentmark{27}{\char"1CD2}नोहव्यादा\accentmark{22}{\char"0951}तये।\ नीहो\accentmark{27}{\char"1CD2}ता\accentmark{22}{\char"0951}सत्सीबर्हीषि\accentmark{22}{\char"0951}\mbox{॥ १\hspace{0pt}॥} \\
त्वा\accentmark{22}{\char"0951}माग्नेयज्ञा\accentmark{27}{\char"1CD2}नां।\ हो\accentmark{22}{\char"0951}ताविश्वे\accentmark{22}{\char"0951}षांहितः।\ देवेभि\accentmark{22}{\char"0951}र्मानूषे\accentmark{22}{\char"0951}जने\mbox{॥ २\hspace{0pt}॥} \\
अग्निंदूतं\accentmark{27}{\char"1CD2}वृ\accentmark{22}{\char"0951}णीमहे।\ हो\accentmark{27}{\char"1CD2}ता\accentmark{22}{\char"0951}रंविश्वा\accentmark{27}{\char"1CD2}वे\accentmark{22}{\char"0951}दसं।\ अस्या\accentmark{20}{\char"1CF9}यज्ञास्या\accentmark{22}{\char"0951}सुक्रा\accentmark{22}{\char"0951}तुम्\mbox{॥ ३\hspace{0pt}॥} \\
अग्निर्वृ॑\accentmark{22}{\char"0951}णि\accentmark{22}{\char"0951}जङ्घनात्।\ द्रविणास्यु\accentmark{20}{\char"1CF9}र्वीपन्या\accentmark{27}{\char"1CD2}या।\ सामी\accentmark{27}{\char"1CD2}द्धश्शुक्रा\accentmark{27}{\char"1CD2}आहूतः\mbox{॥ ४\hspace{0pt}॥} \\प्रेष्ठं\accentmark{20}{\char"1CF8}वोआतीथिं।\ स्तू\accentmark{27}{\char"1CD2}\kern0.15emषे\accentmark{27}{\char"1CD2}मित्रामीवाप्रीयं।\ अ\accentmark{27}{\char"1CD2}ग्नेराथन्ना\accentmark{27}{\char"1CD2}वेद्यम्\mbox{॥ ५\hspace{0pt}॥} \\त्वन्नोआग्नेमा\accentmark{27}{\char"1CD2}हो\accentmark{22}{\char"0951}भिः।\ पाहीविश्वा\accentmark{22}{\char"0951}\kern0.15emस्याआ\accentmark{27}{\char"1CD2}रा\accentmark{22}{\char"0951}तेः।\ ऊत\accentmark{27}{\char"1CD2}द्वीषोमर्क्त्या\accentmark{22}{\char"0951}स्या\mbox{॥ ६\hspace{0pt}॥} \\
एह्यू\accentmark{27}{\char"1CD2}षुब्रा\accentmark{27}{\char"1CD2}वा\accentmark{22}{\char"0951}णिते।\ अग्नाइत्थे\accentmark{20}{\char"1CF9}ता\accentmark{22}{\char"0951}रा\accentmark{22}{\char"0951}गीराः।\ ए\accentmark{27}{\char"1CD2}भिर्वार्धा\accentmark{22}{\char"0951}\kern0.15emसाइ\accentmark{27}{\char"1CD2}न्दू\accentmark{22}{\char"0951}भिः\mbox{॥ ७\hspace{0pt}॥} \\आ\accentmark{20}{\char"1CF8}तेवात्सो\accentmark{27}{\char"1CD2}मानोयमात्।\ परामा\accentmark{20}{\char"1CF9}ञ्चित्साधास्थात्।\ अ\accentmark{27}{\char"1CD2}\kern0.15emग्ने\accentmark{27}{\char"1CD2}त्वांकामयोगी\accentmark{27}{\char"1CD2}रा\accentmark{22}{\char"0951}\mbox{॥ ८\hspace{0pt}॥} \\
त्वामाग्नेपु\accentmark{20}{\char"1CF9}ष्काराद\accentmark{27}{\char"1CD2}धी\accentmark{22}{\char"0951}।\ आ\accentmark{20}{\char"1CF8}थ\accentmark{22}{\char"0951}र्वाणीरा\accentmark{22}{\char"0951}मन्धत।\ मूर्ध्नो\accentmark{27}{\char"1CD2}विश्वास्यावाघा\accentmark{27}{\char"1CD2}ता\accentmark{22}{\char"0951}:\mbox{॥ ९\hspace{0pt}॥} \\अग्नेवी\accentmark{20}{\char"1CF8}वा\accentmark{27}{\char"1CD2}स्वादा\accentmark{27}{\char"1CD2}भारा।\ अस्मभ्यामू\accentmark{27}{\char"1CD2}\kern0.15emता\accentmark{20}{\char"1CF9}ये\accentmark{22}{\char"0951}\kern0.15emमाहे\accentmark{27}{\char"1CD2}।\ देवोह्यासी\accentmark{22}{\char"0951}नोदृशे\mbox{॥ १०\hspace{0pt}॥} \\
\mbox{॥ इति प्रथमः खण्डः\hspace{0pt}॥} \\ 
नामास्ते\accentmark{27}{\char"1CD2}\kern0.15emअग्ना\accentmark{27}{\char"1CD2}ओज\accentmark{22}{\char"0951}से।\ गृणन्तीदे\accentmark{27}{\char"1CD2}वा\accentmark{22}{\char"0951}कृष्टायाः।\ आमै\accentmark{27}{\char"1CD2}रामित्रा\accentmark{27}{\char"1CD2}मर्दया\mbox{॥ १\hspace{0pt}॥} \\दूतंवो\accentmark{22}{\char"0951}विश्वावे\accentmark{22}{\char"0951}दसम्।\ हव्यावाहामामा\accentmark{27}{\char"1CD2}र्क्त्यम्\accentmark{22}{\char"0951}।\ याजीष्ठमृन्जसे\accentmark{27}{\char"1CD2}गीरा\mbox{॥ २\hspace{0pt}॥} \\
ऊपात्वाजामायो\accentmark{27}{\char"1CD2}गीराः\accentmark{22}{\char"0951}।\ देदीशतिर्हा\accentmark{22}{\char"0951}विष्कृताः।\ वायोरा\accentmark{22}{\char"0951}नीकेअस्थिरन्\mbox{॥ ३\hspace{0pt}॥} \\

ऊ\accentmark{22}{\char"0951}पात्वाग्नेदीवे\accentmark{27}{\char"1CD2}दीवे।\ दो\accentmark{27}{\char"1CD2}षा\accentmark{22}{\char"0951}वस्तार्धीयावायम्\accentmark{27}{\char"1CD2}\kern0.15em।\ ना\accentmark{27}{\char"1CD2}\kern0.15emमोभा\accentmark{20}{\char"1CF9}र\accentmark{22}{\char"0951}न्ताएमा\accentmark{22}{\char"0951}सि \mbox{॥ ४\hspace{0pt}॥} \\
जरा\accentmark{22}{\char"0951}बोधाताव्दीविड्ढी।\ वीशेवी\accentmark{27}{\char"1CD2}शेय\accentmark{22}{\char"0951}ज्ञीयाया।\ स्तो\accentmark{20}{\char"1CF8}मंरुद्रायादृशीक\accentmark{27}{\char"1CD2}म्\mbox{॥ ५\hspace{0pt}॥} \\प्रातित्यञ्चा\accentmark{27}{\char"1CD2}रुमाध्वारं\accentmark{27}{\char"1CD2}\kern0.15em।\ गो\accentmark{20}{\char"1CF8}पी\accentmark{22}{\char"0951}\kern0.15emथा\accentmark{27}{\char"1CD2}\kern0.15emयप्रा\accentmark{27}{\char"1CD2}हू\accentmark{22}{\char"0951}यसे।\ मा\accentmark{27}{\char"1CD2}रुद्भीरग्नाआग\accentmark{22}{\char"0951}हि\mbox{॥ ६\hspace{0pt}॥} \\
अ\accentmark{22}{\char"0951}श्वन्न\accentmark{27}{\char"1CD2}त्वावारा\accentmark{22}{\char"0951}\kern0.15emव\accentmark{27}{\char"1CD2}न्तम्।\ व\accentmark{20}{\char"1CF8}न्द\accentmark{20}{\char"1CF9}ध्याआग्नि\accentmark{27}{\char"1CD2}न्नामोभिः।\ स\accentmark{27}{\char"1CD2}म्म्रा\accentmark{27}{\char"1CD2}जन्तमाध्वाराणा\accentmark{22}{\char"0951}म्\mbox{॥ ७\hspace{0pt}॥} \\और्वा\accentmark{27}{\char"1CD2}भृगूवच्छूचिम्।\ आप्नावा\accentmark{27}{\char"1CD2}नावादाहु\accentmark{22}{\char"0951}वे।\ अग्निं\accentmark{20}{\char"1CF9}सामुद्रावाससम्\mbox{॥ ८\hspace{0pt}॥} \\अ\accentmark{27}{\char"1CD2}\kern0.15emग्नी\accentmark{27}{\char"1CD2}मिन्धानो\accentmark{22}{\char"0951}माना\accentmark{22}{\char"0951}सा।\ धी\accentmark{22}{\char"0951}यंसचेतामा\accentmark{27}{\char"1CD2}र्त्याः\accentmark{22}{\char"0951}\kern0.15em।\ अ\accentmark{27}{\char"1CD2}\kern0.15emग्नी\accentmark{27}{\char"1CD2}मिन्धेवीवा\accentmark{27}{\char"1CD2}स्वा\accentmark{22}{\char"0951}भिः\mbox{॥ ९\hspace{0pt}॥} \\आदी\accentmark{27}{\char"1CD2}त्प्रत्ना\accentmark{27}{\char"1CD2}स्यारे\accentmark{27}{\char"1CD2}ता\accentmark{22}{\char"0951}सः।\ ज्योतिःपश्यन्तीवासा\accentmark{27}{\char"1CD2}रम्।\ पा\accentmark{27}{\char"1CD2}रोयातिध्या\accentmark{20}{\char"1CF9}तेदीवि\accentmark{27}{\char"1CD2}\mbox{॥ १०\hspace{0pt}॥} \\
\mbox{॥ इति द्वितीयः खण्डः\hspace{0pt}॥} \\
अग्निंवोवृधन्तं।\ अध्वाराणांपुरूतामं।\ अच्छानाप्नेसाहा\accentmark{27}{\char"1CD2}स्वते\mbox{॥ १\hspace{0pt}॥} \\अग्निस्तिग्मे\accentmark{20}{\char"1CF9}ना\accentmark{22}{\char"0951}शोचीषा\accentmark{22}{\char"0951}।\ यंसद्वि\accentmark{27}{\char"1CD2}श्वंन्याटट्यत्रीणं।\ अ\accentmark{27}{\char"1CD2}\kern0.15emग्नि\accentmark{27}{\char"1CD2}र्नो\accentmark{22}{\char"0951}वंसतेरायिम्\accentmark{20}{\char"1CF8} \mbox{॥ २\hspace{0pt}॥} \\
अग्नेमृ\accentmark{20}{\char"1CF9}ळामाहंआ\accentmark{22}{\char"0951}सि\accentmark{22}{\char"0951}\kern0.15em।\ आ\accentmark{27}{\char"1CD2}\kern0.15emयाआ\accentmark{20}{\char"1CF9}दे\accentmark{22}{\char"0951}वायुन्ज\accentmark{22}{\char"0951}नम्।\ ईये\accentmark{20}{\char"1CF8}थाबर्ही\accentmark{22}{\char"0951}रासादम्\mbox{॥ ३\hspace{0pt}॥} \\
अ\accentmark{27}{\char"1CD2}\kern0.15emग्नेर\accentmark{20}{\char"1CF9}क्षाणोअं\accentmark{27}{\char"1CD2}हा\accentmark{22}{\char"0951}सः।\ 
प्रा\accentmark{27}{\char"1CD2}तीस्मदेवरिषाताः।\ ता\accentmark{27}{\char"1CD2}पीष्ठैराज\accentmark{27}{\char"1CD2}रोदहा \mbox{॥ ४\hspace{0pt}॥} \\
अ\accentmark{20}{\char"1CF8}ग्नेयुंक्ष्वाहीयेता\accentmark{27}{\char"1CD2}वा\accentmark{22}{\char"0951}।\ अश्वा\accentmark{22}{\char"0951}सोदेवासाधावाः।\ आ\accentmark{27}{\char"1CD2}रं\accentmark{22}{\char"0951}वा\accentmark{20}{\char"1CF9}ह\accentmark{22}{\char"0951}न्त्याशा\accentmark{27}{\char"1CD2}वाः\accentmark{22}{\char"0951} \mbox{॥ ५\hspace{0pt}॥} \\
नित्वा\accentmark{22}{\char"0951}नक्ष्यविस्पते।\ धू\accentmark{27}{\char"1CD2}म\accentmark{22}{\char"0951}\kern0.15emन्त\accentmark{27}{\char"1CD2}न्धीमहेवायम्।\ सु\accentmark{27}{\char"1CD2}\kern0.15emवी\accentmark{27}{\char"1CD2}रामग्नआ\accentmark{20}{\char"1CF8}हुतः\mbox{॥ ६\hspace{0pt}॥} \\
अ\accentmark{27}{\char"1CD2}\kern0.15emग्नि\accentmark{27}{\char"1CD2}र्मूर्द्धादीवाःकाकू\accentmark{27}{\char"1CD2}\kern0.15emत्।\ पा\accentmark{27}{\char"1CD2}तीःपृथिव्या\accentmark{27}{\char"1CD2}\kern0.15emआ\accentmark{27}{\char"1CD2}यम्।\ आपांरेतांसिजिन्वति\mbox{॥ ७\hspace{0pt}॥} \\
ई\accentmark{27}{\char"1CD2}मामू\accentmark{22}{\char"0951}षुत्वाम\accentmark{27}{\char"1CD2}स्मा\accentmark{22}{\char"0951}कम्।\ सा\accentmark{27}{\char"1CD2}\kern0.15emनि\accentmark{20}{\char"1CF9}ङ्गा\accentmark{22}{\char"0951}\kern0.15emयात्र\accentmark{27}{\char"1CD2}न्नव्या\accentmark{22}{\char"0951}सम्।\ अग्ने\accentmark{22}{\char"0951}\kern0.15emदे\accentmark{27}{\char"1CD2}\kern0.15emवे\accentmark{27}{\char"1CD2}षुप्रा\accentmark{27}{\char"1CD2}वोचः\mbox{॥ ८\hspace{0pt}॥} \\
तन्त्वा\accentmark{22}{\char"0951}\kern0.15emगोपा\accentmark{27}{\char"1CD2}वा\accentmark{22}{\char"0951}नोगिरा।\ जा\accentmark{27}{\char"1CD2}नि\accentmark{22}{\char"0951}ष्ठदग्नेअङ्गिरः।\ सा\accentmark{27}{\char"1CD2}पा\accentmark{22}{\char"0951}वकाश्रू\accentmark{22}{\char"0951}\kern0.15emधीहा\accentmark{27}{\char"1CD2}व\accentmark{22}{\char"0951}म्\mbox{॥ ९\hspace{0pt}॥} \\
पा\accentmark{27}{\char"1CD2}\kern0.15emरिवा\accentmark{27}{\char"1CD2}जपतीःकाविः\accentmark{27}{\char"1CD2}\kern0.15em।\ अ\accentmark{27}{\char"1CD2}\kern0.15emग्नि\accentmark{27}{\char"1CD2}\kern0.15emऋ\accentmark{27}{\char"1CD2}\kern0.15emहव्या\accentmark{27}{\char"1CD2}न्या\accentmark{22}{\char"0951}क्रमीत्।\ दध\accentmark{27}{\char"1CD2}द्रत्ना\accentmark{27}{\char"1CD2}नीदाशू\accentmark{27}{\char"1CD2}\kern0.15emषे\accentmark{27}{\char"1CD2}\mbox{॥ १०\hspace{0pt}॥} \\
ऊदूत्यं\accentmark{27}{\char"1CD2}\kern0.15emजाता\accentmark{27}{\char"1CD2}वे\accentmark{22}{\char"0951}दसम्।\ दे\accentmark{27}{\char"1CD2}\kern0.15emवं\accentmark{27}{\char"1CD2}वाहन्तींकेता\accentmark{27}{\char"1CD2}वाः\accentmark{22}{\char"0951}।\ दृशे\accentmark{20}{\char"1CF9}वि\accentmark{20}{\char"1CF9}श्वा\accentmark{22}{\char"0951}यासूर्यम्\mbox{॥ ११\hspace{0pt}॥} \\
का\accentmark{27}{\char"1CD2}\kern0.15emवी\accentmark{27}{\char"1CD2}माग्नी\accentmark{27}{\char"1CD2}\kern0.15emमू\accentmark{27}{\char"1CD2}पा\accentmark{22}{\char"0951}स्तुहि।\ स\accentmark{27}{\char"1CD2}\kern0.15emत्या\accentmark{27}{\char"1CD2}धर्मा\accentmark{22}{\char"0951}णमाध्वारे।\ दे\accentmark{27}{\char"1CD2}\kern0.15emवा\accentmark{27}{\char"1CD2}मा\accentmark{22}{\char"0951}मीवाचा\accentmark{27}{\char"1CD2}ता\accentmark{22}{\char"0951}नम्\mbox{॥ १२\hspace{0pt}॥} \\
श\accentmark{20}{\char"1CF8}न्नो\accentmark{22}{\char"0951}\kern0.15emदेवी\accentmark{27}{\char"1CD2}राभीष्टा\accentmark{22}{\char"0951}\kern0.15emये।\ श\accentmark{27}{\char"1CD2}न्नो\accentmark{22}{\char"0951}भूवन्तूपीता\accentmark{27}{\char"1CD2}ये\accentmark{22}{\char"0951}\kern0.15em।\ श\accentmark{27}{\char"1CD2}य्योराभि\accentmark{27}{\char"1CD2}स्रावन्तूनः\mbox{॥ १३\hspace{0pt}॥} \\
कस्यानू\accentmark{27}{\char"1CD2}नम्पारी\accentmark{22}{\char"0951}णसि।\ धीयो\accentmark{22}{\char"0951}जिन्वसिसत्पते।\ गो\accentmark{27}{\char"1CD2}\kern0.15emषा\accentmark{27}{\char"1CD2}तायस्यातागी\accentmark{27}{\char"1CD2}\kern0.15emराः\accentmark{27}{\char"1CD2}\mbox{॥ १४\hspace{0pt}॥} \\
\mbox{॥ इति तृतीयः खण्डः\hspace{0pt}॥} \\ 
यज्ञा\accentmark{27}{\char"1CD2}याज्ञा\accentmark{27}{\char"1CD2}वोअग्ना\accentmark{27}{\char"1CD2}ये\accentmark{22}{\char"0951}\kern0.15em।\ गी\accentmark{27}{\char"1CD2}\kern0.15emरा\accentmark{27}{\char"1CD2}गी\accentmark{22}{\char"0951}\kern0.15emराचाद\accentmark{27}{\char"1CD2}क्षासे।\ प्र\accentmark{20}{\char"1CF8}प्रा\accentmark{27}{\char"1CD2}\kern0.15emवाया\accentmark{27}{\char"1CD2}\kern0.15emमामृ\accentmark{20}{\char"1CF9}त\accentmark{27}{\char"1CD2}ञ्जा\accentmark{22}{\char"0951}तावेदसम्।\ प्रीयम्मित्र\accentmark{27}{\char"1CD2}\kern0.15emन्ना\accentmark{27}{\char"1CD2}शंसिषम्\mbox{॥ १\hspace{0pt}॥} \\
पाहिनोअग्नाएका\accentmark{22}{\char"0951}\kern0.15emया।\ पा\accentmark{27}{\char"1CD2}ह्यूताद्वि\accentmark{27}{\char"1CD2}\kern0.15emतीया\accentmark{27}{\char"1CD2}\kern0.15emया।\ पा\accentmark{27}{\char"1CD2}हीगीर्भीस्तीसृ\accentmark{27}{\char"1CD2}भीरूर्जां\accentmark{22}{\char"0951}पते।\ पाही\accentmark{20}{\char"1CF9}चातासृभी\accentmark{22}{\char"0951}र्वसो\mbox{॥ २\hspace{0pt}॥} \\
बृ\accentmark{27}{\char"1CD2}\kern0.15emहा\accentmark{27}{\char"1CD2}त्भी\accentmark{22}{\char"0951}रग्नेरर्च्ची\accentmark{22}{\char"0951}भिः।\ शुक्रेणा\accentmark{22}{\char"0951}देवाशोची\accentmark{27}{\char"1CD2}षा\accentmark{22}{\char"0951}\kern0.15em।\ भारा\accentmark{27}{\char"1CD2}द्वा\accentmark{22}{\char"0951}जेसमीधा।\ नो\accentmark{27}{\char"1CD2}या\accentmark{22}{\char"0951}विष्यारेवा\accentmark{27}{\char"1CD2}त्पा\accentmark{22}{\char"0951}वकदीदिहि\mbox{॥ ३\hspace{0pt}॥} \\त्वय्या\accentmark{22}{\char"0951}ग्नेस्वा\accentmark{27}{\char"1CD2}हुतः।\ प्रीया\accentmark{27}{\char"1CD2}\kern0.15emसा\accentmark{27}{\char"1CD2}सन्तूसूरायाः\accentmark{22}{\char"0951}।\ यन्ता\accentmark{27}{\char"1CD2}रोयेमाघा\accentmark{20}{\char"1CF9}वा\accentmark{22}{\char"0951}\kern0.15emनोज\accentmark{27}{\char"1CD2}ना\accentmark{22}{\char"0951}नाम्।\ ऊर्वन्दय\accentmark{20}{\char"1CF8}न्तागो\accentmark{27}{\char"1CD2}नाम्\mbox{॥ ४\hspace{0pt}॥} \\अग्नेजरी\accentmark{27}{\char"1CD2}तार्वीस्पातिः।\ तपानो\accentmark{27}{\char"1CD2}देवारक्षा\accentmark{27}{\char"1CD2}\kern0.15emसाः।\ अ\accentmark{27}{\char"1CD2}प्रो\accentmark{22}{\char"0951}षिवान्ग्रहापा\accentmark{22}{\char"0951}तिर्माहंआ\accentmark{22}{\char"0951}सि।\ दी\accentmark{22}{\char"0951}\kern0.15emवा\accentmark{27}{\char"1CD2}स्पायू\accentmark{27}{\char"1CD2}र्दुरोणायुः\mbox{॥ ५\hspace{0pt}॥} \\अग्नेवी\accentmark{27}{\char"1CD2}वा\accentmark{22}{\char"0951}स्वादूषा\accentmark{27}{\char"1CD2}साः।\ चि\accentmark{27}{\char"1CD2}\kern0.15emत्रं\accentmark{27}{\char"1CD2}\kern0.15emरा\accentmark{27}{\char"1CD2}\kern0.15emधोआ\accentmark{27}{\char"1CD2}मा\accentmark{22}{\char"0951}र्क्त्याम्\accentmark{22}{\char"0951}।\ 
आदाशू\accentmark{27}{\char"1CD2}षेजातवेदोवा\accentmark{27}{\char"1CD2}हात्वम्\accentmark{27}{\char"1CD2}।\ अद्या\accentmark{27}{\char"1CD2}\kern0.15emदेवं\accentmark{20}{\char"1CF9}ऊ\accentmark{22}{\char"0951}ष\accentmark{20}{\char"1CF8}र्बुधाः\mbox{॥ ६\hspace{0pt}॥} \\
त्व\accentmark{20}{\char"1CF8}न्नाश्चित्रावूत्य।\ वा\accentmark{27}{\char"1CD2}\kern0.15emसो\accentmark{27}{\char"1CD2}राधांसिचो\accentmark{27}{\char"1CD2}\kern0.15emदया।\ अ\accentmark{27}{\char"1CD2}\kern0.15emस्या\accentmark{27}{\char"1CD2}रायेत्वा\accentmark{27}{\char"1CD2}मा\accentmark{22}{\char"0951}ग्नेरा\accentmark{22}{\char"0951}थीरासी\accentmark{27}{\char"1CD2}।\ वीदागा\accentmark{27}{\char"1CD2}धन्तूचेतूनाः\mbox{॥ ७\hspace{0pt}॥} \\
त्वा\accentmark{27}{\char"1CD2}मीत्सप्रा\accentmark{27}{\char"1CD2}था\accentmark{22}{\char"0951}\accentmark{27}{\char"1CD2}असि।\ अग्नेत्राता\accentmark{27}{\char"1CD2}ऋताःकाविः।\ त्वांविप्रा\accentmark{22}{\char"0951}सस्समीधानदीदिवः।\ आ\accentmark{27}{\char"1CD2}वीवासन्तीवेधासाः\mbox{॥ ८\hspace{0pt}॥} \\
आनो\accentmark{22}{\char"0951}आग्नेवायो\accentmark{27}{\char"1CD2}\kern0.15emवृ\accentmark{27}{\char"1CD2}धं\accentmark{22}{\char"0951}।\ रायिंपा\accentmark{22}{\char"0951}वाका\accentmark{22}{\char"0951}शंस्यम्\accentmark{22}{\char"0951}।\ रास्वा\accentmark{22}{\char"0951}\kern0.15emचन\accentmark{27}{\char"1CD2}उपमातेपूरूस्पृहं\accentmark{22}{\char"0951}।\ सू\accentmark{22}{\char"0951}\kern0.15emनीं\accentmark{27}{\char"1CD2}\kern0.15emतीसू\accentmark{27}{\char"1CD2}याशस्तरम्\mbox{॥ ९\hspace{0pt}॥} \\
यो\accentmark{27}{\char"1CD2}विश्वाद\accentmark{20}{\char"1CF9}यातेवासू\accentmark{22}{\char"0951}।\ हो\accentmark{20}{\char"1CF8}ता\accentmark{22}{\char"0951}मन्द्रोजना\accentmark{22}{\char"0951}नां।\ मा\accentmark{27}{\char"1CD2}धोर्नपात्राप्राथा\accentmark{27}{\char"1CD2}मान्यस्मै।\ प्रस्तो\accentmark{27}{\char"1CD2}मायन्त्वग्ना\accentmark{27}{\char"1CD2}ये\mbox{॥ १०\hspace{0pt}॥} \\
\mbox{॥ इति चतुर्थः खण्डः\hspace{0pt}॥} \\ 
ए\accentmark{27}{\char"1CD2}\kern0.15emनावो\accentmark{27}{\char"1CD2}\kern0.15emअग्नि\accentmark{27}{\char"1CD2}\kern0.15emन्ना\accentmark{27}{\char"1CD2}मा\accentmark{22}{\char"0951}सा।\ ऊर्जो\accentmark{20}{\char"1CF9}नापातामाहू\accentmark{22}{\char"0951}वे।\ प्री\accentmark{27}{\char"1CD2}यञ्चेतिष्ठामारातिं\accentmark{20}{\char"1CF9}स्वाध्वारं\accentmark{27}{\char"1CD2}।\ विश्वा\accentmark{22}{\char"0951}स्यादूता\accentmark{20}{\char"1CF8}मामृत\accentmark{22}{\char"0951}म्\mbox{॥ १\hspace{0pt}॥} \\
शे\accentmark{27}{\char"1CD2}\kern0.15emषेवा\accentmark{27}{\char"1CD2}ने\accentmark{22}{\char"0951}षूमातृ\accentmark{22}{\char"0951}\kern0.15emषु।\ स\accentmark{27}{\char"1CD2}न्त्वामर्क्ता\accentmark{22}{\char"0951}सइन्धते।\ आत\accentmark{22}{\char"0951}न्द्रोहव्यं\accentmark{27}{\char"1CD2}वा\accentmark{22}{\char"0951}हसिहाविष्कृताः\accentmark{22}{\char"0951}\kern0.15em।\ आ\accentmark{27}{\char"1CD2}\kern0.15emदि\accentmark{27}{\char"1CD2}द्देवेषु\accentmark{22}{\char"0951}राजसि\mbox{॥ २\hspace{0pt}॥} \\
अ\accentmark{27}{\char"1CD2}द\accentmark{22}{\char"0951}र्शीगातूवि\accentmark{27}{\char"1CD2}क्ता\accentmark{22}{\char"0951}मः।\ यस्मि\accentmark{22}{\char"0951}न्व्राता\accentmark{20}{\char"1CF9}न्याद\accentmark{27}{\char"1CD2}\kern0.15emधुः\accentmark{27}{\char"1CD2}।\ ऊपोषूजाता\accentmark{20}{\char"1CF9}मा\accentmark{20}{\char"1CF9}र्यास्यावार्थानं।\ अ\accentmark{27}{\char"1CD2}\kern0.15emग्नि\accentmark{27}{\char"1CD2}न्ना\accentmark{22}{\char"0951}क्षन्तूनोगीराः\accentmark{22}{\char"0951}\mbox{॥ ३\hspace{0pt}॥} \\
अग्नी\accentmark{27}{\char"1CD2}रूत्थेपूरो\accentmark{27}{\char"1CD2}\kern0.15emहितः\accentmark{27}{\char"1CD2}।\ ग्रावा\accentmark{22}{\char"0951}णोबर्हीरा\accentmark{22}{\char"0951}ध्वारे\accentmark{27}{\char"1CD2}\kern0.15em।\ ऋचा\accentmark{27}{\char"1CD2}या\accentmark{22}{\char"0951}मिमरुतोब्रह्मणस्पते।\ देवंआ\accentmark{27}{\char"1CD2}\kern0.15emवोवा\accentmark{27}{\char"1CD2}रेण्यम्\mbox{॥ ४\hspace{0pt}॥} \\
अग्नी\accentmark{20}{\char"1CF9}मीळिष्वा\accentmark{27}{\char"1CD2}वा\accentmark{22}{\char"0951}\kern0.15emसे।\ गा\accentmark{27}{\char"1CD2}था\accentmark{22}{\char"0951}भीश्शीरा\accentmark{27}{\char"1CD2}शो\accentmark{22}{\char"0951}चिषं।\ अ\accentmark{27}{\char"1CD2}\kern0.15emग्निं\accentmark{27}{\char"1CD2}\kern0.15emराये\accentmark{27}{\char"1CD2}पू\accentmark{22}{\char"0951}रूमिढाश्रूतन्ना\accentmark{27}{\char"1CD2}राः\accentmark{22}{\char"0951}।\ अग्निस्सू\accentmark{22}{\char"0951}दीताये\accentmark{20}{\char"1CF9}च्छार्दिः\mbox{॥ ५\hspace{0pt}॥} \\
श्रूधि\accentmark{27}{\char"1CD2}श्रुत्कर्णावन्ही\accentmark{22}{\char"0951}भिः।\ देवैरा\accentmark{22}{\char"0951}\kern0.15emग्ने\accentmark{27}{\char"1CD2}सायावा\accentmark{22}{\char"0951}भिः।\ आसी\accentmark{22}{\char"0951}दतुबर्ही\accentmark{27}{\char"1CD2}षी\accentmark{22}{\char"0951}मित्रो\accentmark{20}{\char"1CF9}अर्यामा।\ प्रात\accentmark{27}{\char"1CD2}\kern0.15emर्या\accentmark{27}{\char"1CD2}वा\accentmark{22}{\char"0951}भीराध्वारे\accentmark{27}{\char"1CD2}\mbox{॥ ६\hspace{0pt}॥} \\
प्रदै\accentmark{27}{\char"1CD2}वो\accentmark{22}{\char"0951}दासोअग्निः।\ दे\accentmark{27}{\char"1CD2}वइन्द्रोनाम\accentmark{27}{\char"1CD2}\kern0.15emज्मा\accentmark{27}{\char"1CD2}\kern0.15emना।\ आ\accentmark{20}{\char"1CF8}नू\accentmark{27}{\char"1CD2}मातारं\accentmark{22}{\char"0951}पृथिवींवीवा\accentmark{22}{\char"0951}वृते।\ तस्थौ\accentmark{20}{\char"1CF9}ना\accentmark{20}{\char"1CF9}का\accentmark{22}{\char"0951}स्याशर्माणि\mbox{॥ ७\hspace{0pt}॥} \\
आधा\accentmark{22}{\char"0951}\kern0.15emज्मो\accentmark{27}{\char"1CD2}आधा\accentmark{22}{\char"0951}वादी\accentmark{22}{\char"0951}\kern0.15emवः।\ बृ\accentmark{27}{\char"1CD2}\kern0.15emहातो\accentmark{20}{\char"1CF9}रो\accentmark{22}{\char"0951}\kern0.15emचानाद\accentmark{27}{\char"1CD2}धी\accentmark{22}{\char"0951}\kern0.15em।\ आया\accentmark{27}{\char"1CD2}\kern0.15emवा\accentmark{20}{\char"1CF8}र्धस्वा\accentmark{27}{\char"1CD2}\kern0.15emतन्वा\accentmark{27}{\char"1CD2}\kern0.15emगीरा\accentmark{27}{\char"1CD2}मामा।\ आजाता\accentmark{27}{\char"1CD2}सूकृतॊ\accentmark{22}{\char"0951}पृणा\mbox{॥ ८\hspace{0pt}॥} \\
काया\accentmark{22}{\char"0951}मानोवा\accentmark{27}{\char"1CD2}नात्वं\accentmark{27}{\char"1CD2}\kern0.15em।\ य\accentmark{20}{\char"1CF8}न्मा\accentmark{22}{\char"0951}\kern0.15emत्ररा\accentmark{27}{\char"1CD2}ज\accentmark{22}{\char"0951}गन्नापः\accentmark{27}{\char"1CD2}\kern0.15em।\ ना\accentmark{27}{\char"1CD2}ताक्ते\accentmark{22}{\char"0951}अग्नेप्रामृ\accentmark{20}{\char"1CF9}षेनीवार्त्ता\accentmark{22}{\char"0951}\kern0.15emनं।\ य\accentmark{27}{\char"1CD2}द्दूरेस\accentmark{27}{\char"1CD2}\kern0.15emन्नी\accentmark{27}{\char"1CD2}हाभूवः\mbox{॥ ९\hspace{0pt}॥} \\
नित्वा\accentmark{22}{\char"0951}माग्ने\accentmark{27}{\char"1CD2}मानू\accentmark{22}{\char"0951}र्दधे।\ ज्यो\accentmark{27}{\char"1CD2}तीर्ज\accentmark{22}{\char"0951}ना\accentmark{22}{\char"0951}\kern0.15emयाशा\accentmark{27}{\char"1CD2}श्वा\accentmark{22}{\char"0951}ते।\ दीदेथाक\accentmark{20}{\char"1CF9}ण्वऋताजातउक्षीतः।\ यन्ना\accentmark{22}{\char"0951}मास्य\accentmark{20}{\char"1CF9}न्तीकृष्टायाः\mbox{॥ १०\hspace{0pt}॥} \\
\mbox{॥ इति पञ्चमः खण्डः\hspace{0pt}॥} \\ 
दे\accentmark{27}{\char"1CD2}\kern0.15emवो\accentmark{27}{\char"1CD2}वो\accentmark{22}{\char"0951}द्रविणोदाः।\ पूर्णां\accentmark{22}{\char"0951}वी\accentmark{22}{\char"0951}वाष्ट्‌वासी\accentmark{27}{\char"1CD2}च\accentmark{22}{\char"0951}म्।\ उद्वा\accentmark{22}{\char"0951}सिन्चा\accentmark{27}{\char"1CD2}ध्वामू\accentmark{27}{\char"1CD2}पा\accentmark{22}{\char"0951}वापृणध्वं।\ आदीद्वो\accentmark{22}{\char"0951}\kern0.15emदेवा\accentmark{27}{\char"1CD2}ओहते\mbox{॥ १\hspace{0pt}॥} \\
प्रैतू\accentmark{20}{\char"1CF9}ब्रह्माण\accentmark{27}{\char"1CD2}\kern0.15emस्पा\accentmark{27}{\char"1CD2}तीः।\ प्रा\accentmark{27}{\char"1CD2}देव्येतूसूनृ\accentmark{27}{\char"1CD2}ता।\ अच्छावीरं\accentmark{20}{\char"1CF9}नर्यं\accentmark{22}{\char"0951}पङ्तीरा\accentmark{22}{\char"0951}धसं।\ दे\accentmark{27}{\char"1CD2}\kern0.15emवा\accentmark{27}{\char"1CD2}\kern0.15emयज्ञ\accentmark{27}{\char"1CD2}\kern0.15emन्ना\accentmark{27}{\char"1CD2}यन्तुनः\mbox{॥ २\hspace{0pt}॥} \\
ऊर्ध्वाऊषू\accentmark{20}{\char"1CF9}णा\accentmark{22}{\char"0951}\kern0.15emऊता\accentmark{27}{\char"1CD2}ये\accentmark{22}{\char"0951}।\ तिष्ठा\accentmark{22}{\char"0951}देवो\accentmark{20}{\char"1CF9}ना\accentmark{20}{\char"1CF9}सा\accentmark{22}{\char"0951}वीता।\ ऊर्ध्वोवा\accentmark{27}{\char"1CD2}ज\accentmark{22}{\char"0951}स्यासानीता\accentmark{20}{\char"1CF9}यादन्जी\accentmark{27}{\char"1CD2}भिः\accentmark{22}{\char"0951}\kern0.15em।\ वा\accentmark{27}{\char"1CD2}\kern0.15emघा\accentmark{20}{\char"1CF9}त्भी\accentmark{22}{\char"0951}र्वीभ्वाया\accentmark{22}{\char"0951}महे\mbox{॥ ३\hspace{0pt}॥} \\
प्र\accentmark{20}{\char"1CF8}योरा\accentmark{22}{\char"0951}येनीनीषते।\ म\accentmark{27}{\char"1CD2}र्तोयस्तेवासोदा\accentmark{22}{\char"0951}शा\accentmark{22}{\char"0951}त्।\ सावीरन्धाक्तेअ\accentmark{27}{\char"1CD2}ग्निउत्थाशंसी\accentmark{22}{\char"0951}नं।\ इत्माना\accentmark{22}{\char"0951}सहस्रापोषी\accentmark{27}{\char"1CD2}णम्\mbox{॥ ४\hspace{0pt}॥} \\
प्रावो\accentmark{22}{\char"0951}यंभंपू\accentmark{22}{\char"0951}\kern0.15emरूणां\accentmark{27}{\char"1CD2}\kern0.15em।\ वीशा\accentmark{27}{\char"1CD2}न्दे\accentmark{22}{\char"0951}वायातीनां।\ अग्निंसूक्तेभिर्वाचो\accentmark{22}{\char"0951}भिर्वृणीमहे।\ यंसमीदन्याइ\accentmark{27}{\char"1CD2}\kern0.15emन्धा\accentmark{27}{\char"1CD2}ते\mbox{॥ ५\hspace{0pt}॥} \\
आया\accentmark{27}{\char"1CD2}\kern0.15emमग्नी\accentmark{27}{\char"1CD2}\kern0.15emसूवी\accentmark{27}{\char"1CD2}र्या\accentmark{22}{\char"0951}स्या।\ ई\accentmark{27}{\char"1CD2}शेहिसौ\accentmark{27}{\char"1CD2}\kern0.15emभा\accentmark{27}{\char"1CD2}गस्य।\ रायाई\accentmark{27}{\char"1CD2}शेस्वापत्यास्यागो\accentmark{27}{\char"1CD2}मा\accentmark{22}{\char"0951}तः।\ ईशे\accentmark{22}{\char"0951}\kern0.15emहिवृ\accentmark{27}{\char"1CD2}त्राहा\accentmark{27}{\char"1CD2}थानाम्\mbox{॥ ६\hspace{0pt}॥} \\
त्वा\accentmark{20}{\char"1CF8}मा\accentmark{22}{\char"0951}ग्नेगृहा\accentmark{27}{\char"1CD2}पा\accentmark{22}{\char"0951}तिः।\ त्वं\accentmark{27}{\char"1CD2}होता\accentmark{22}{\char"0951}नोआ\accentmark{20}{\char"1CF8}ध्वारे।\ त्वां\accentmark{27}{\char"1CD2}\kern0.15emपो\accentmark{27}{\char"1CD2}ता\accentmark{22}{\char"0951}विश्वावारप्रा\accentmark{27}{\char"1CD2}चे\accentmark{22}{\char"0951}\kern0.15emतः।\ य\accentmark{27}{\char"1CD2}क्षीया\accentmark{20}{\char"1CF9}सी\accentmark{22}{\char"0951}चावार्यम्\mbox{॥ ७\hspace{0pt}॥} \\
साखायस्त्वाववृमहे।\ दे\accentmark{27}{\char"1CD2}वंमर्ता\accentmark{27}{\char"1CD2}\kern0.15emसाऊता\accentmark{27}{\char"1CD2}\kern0.15emये।\ आ\accentmark{27}{\char"1CD2}पान्ना\accentmark{22}{\char"0951}पातंसूभा\accentmark{20}{\char"1CF9}गं\accentmark{22}{\char"0951}सूदंसा\accentmark{22}{\char"0951}\kern0.15emसं।\ सु\accentmark{27}{\char"1CD2}\kern0.15emप्रा\accentmark{27}{\char"1CD2}तू\accentmark{22}{\char"0951}र्तीमानेहा\accentmark{27}{\char"1CD2}सम्\mbox{॥ ८\hspace{0pt}॥} \\
\mbox{॥ इति षष्ठः खण्डः॥} \\
आजू\accentmark{22}{\char"0951}होताहविषा\accentmark{22}{\char"0951}मर्जयध्वं।\ नीहो\accentmark{27}{\char"1CD2}ता\accentmark{22}{\char"0951}रंगृहा\accentmark{27}{\char"1CD2}पा\accentmark{22}{\char"0951}तिंदधिध्वम्।\ ईळा\accentmark{22}{\char"0951}स्पातेनामा\accentmark{22}{\char"0951}\kern0.15emसारा\accentmark{27}{\char"1CD2}ताहा\accentmark{22}{\char"0951}व्यं।\ सपर्यातायाजतम्पस्त्याठ\accentmark{22}{\char"0951}नाम्\mbox{॥ १\hspace{0pt}॥} \\
चित्र\accentmark{27}{\char"1CD2}इच्छीशो\accentmark{27}{\char"1CD2}\kern0.15emस्ता\accentmark{27}{\char"1CD2}रुणास्यावक्षा\accentmark{27}{\char"1CD2}थाः\accentmark{22}{\char"0951}\kern0.15em।\ न\accentmark{27}{\char"1CD2}\kern0.15emयो\accentmark{27}{\char"1CD2}\kern0.15emमाता\accentmark{20}{\char"1CF9}रा\accentmark{22}{\char"0951}\kern0.15emवन्वे\accentmark{27}{\char"1CD2}\kern0.15emतीधा\accentmark{27}{\char"1CD2}ता\accentmark{22}{\char"0951}\kern0.15emवे।\ अ\accentmark{27}{\char"1CD2}\kern0.15emनुधा\accentmark{27}{\char"1CD2}\kern0.15emयाद\accentmark{27}{\char"1CD2}जीज\accentmark{22}{\char"0951}\kern0.15emनाद\accentmark{27}{\char"1CD2}धा\accentmark{22}{\char"0951}\kern0.15emचीदा\accentmark{27}{\char"1CD2}।\ 
वावक्षस्सद्योमा\accentmark{22}{\char"0951}हीदूत्याटट्यञ्चा\accentmark{27}{\char"1CD2}र॑\accentmark{22}{\char"0951}न्\mbox{॥ २\hspace{0pt}॥} \\
ईद\accentmark{27}{\char"1CD2}न्ताएकं\accentmark{22}{\char"0951}पारा\accentmark{20}{\char"1CF9}ऊ\accentmark{22}{\char"0951}ताएकं\accentmark{22}{\char"0951}।\ तृतीये\accentmark{22}{\char"0951}नज्योती\accentmark{22}{\char"0951}\kern0.15emषासं\accentmark{27}{\char"1CD2}वीशस्वा।\ सं\accentmark{27}{\char"1CD2}\kern0.15emवे\accentmark{27}{\char"1CD2}\kern0.15emशान\accentmark{20}{\char"1CF9}स्तन्वेटट्यचारू\accentmark{27}{\char"1CD2}रेधि।\ 
प्रियो\accentmark{22}{\char"0951}\kern0.15emदेवा\accentmark{27}{\char"1CD2}नां\accentmark{22}{\char"0951}पा\accentmark{22}{\char"0951}\kern0.15emरामेज\accentmark{27}{\char"1CD2}नित्रे\mbox{॥ ३\hspace{0pt}॥} \\
इमं\accentmark{27}{\char"1CD2}स्तोमामर्हातेजाता\accentmark{27}{\char"1CD2}वेदसे।\ राथा\accentmark{22}{\char"0951}मी\accentmark{20}{\char"1CF9}वा\accentmark{27}{\char"1CD2}संमा\accentmark{22}{\char"0951}हेमामा\accentmark{22}{\char"0951}\kern0.15emनीषा\accentmark{27}{\char"1CD2}\kern0.15emया।\ भ\accentmark{27}{\char"1CD2}\kern0.15emद्रा\accentmark{27}{\char"1CD2}हीनःप्रामा\accentmark{22}{\char"0951}तीरस्यासंसादि।\ 
अग्ने\accentmark{22}{\char"0951}सख्येमारी\accentmark{22}{\char"0951}षामावायन्ता\accentmark{27}{\char"1CD2}वा\accentmark{22}{\char"0951}\mbox{॥ ४\hspace{0pt}॥} \\
मूर्धानंदीवोआ\accentmark{22}{\char"0951}रातिं\accentmark{20}{\char"1CF9}पृ\accentmark{22}{\char"0951}थिव्याः\accentmark{27}{\char"1CD2}।\ वैश्वा\accentmark{27}{\char"1CD2}\kern0.15emनारा\accentmark{27}{\char"1CD2}\kern0.15emमृतआ\accentmark{27}{\char"1CD2}\kern0.15emजाता\accentmark{27}{\char"1CD2}\kern0.15emम\accentmark{27}{\char"1CD2}ग्निम्।\ काविं\accentmark{27}{\char"1CD2}सम्म्राजमा\accentmark{20}{\char"1CF9}तीथि\accentmark{22}{\char"0951}ञ्ज\accentmark{22}{\char"0951}ना\accentmark{22}{\char"0951}नां।\ 
आ\accentmark{27}{\char"1CD2}\kern0.15emस\accentmark{27}{\char"1CD2}न्नापा\accentmark{22}{\char"0951}त्रञ्जनयन्तादेवाः\mbox{॥ ५\hspace{0pt}॥} \\
वित्वा\accentmark{27}{\char"1CD2}\kern0.15emदापो\accentmark{27}{\char"1CD2}\kern0.15emनापा\accentmark{27}{\char"1CD2}र्वातस्यापृष्ठात्।\ उक्थे\accentmark{27}{\char"1CD2}भी\accentmark{22}{\char"0951}रग्नेर्जनयन्तादे\accentmark{27}{\char"1CD2}\kern0.15emवाः\accentmark{27}{\char"1CD2}।\ तन्त्वा\accentmark{27}{\char"1CD2}गीरा\accentmark{22}{\char"0951}स्सू\accentmark{22}{\char"0951}ष्टू\accentmark{22}{\char"0951}तायो\accentmark{22}{\char"0951}र्वाजयन्ति।\ 
आ\accentmark{27}{\char"1CD2}\kern0.15emजि\accentmark{20}{\char"1CF9}न्नागीर्वावाहो\accentmark{22}{\char"0951}जीग्यूरा\accentmark{27}{\char"1CD2}श्वाः\mbox{॥ ६\hspace{0pt}॥} \\
आ\accentmark{27}{\char"1CD2}\kern0.15emवोरा\accentmark{27}{\char"1CD2}जानमध्वर\accentmark{20}{\char"1CF9}स्या\accentmark{22}{\char"0951}रूद्रं\accentmark{27}{\char"1CD2}\kern0.15em।\ हो\accentmark{27}{\char"1CD2}ता\accentmark{22}{\char"0951}रंसत्याया\accentmark{27}{\char"1CD2}\kern0.15emजंरो\accentmark{27}{\char"1CD2}द॑\accentmark{22}{\char"0951}स्योः।\ अग्निंपूरा\accentmark{27}{\char"1CD2}तानाइत्नो\accentmark{27}{\char"1CD2}राचिक्तात्।\ 
हिरण्यरूपामा\accentmark{27}{\char"1CD2}वा\accentmark{22}{\char"0951}सेकृणुध्वम्\mbox{॥ ७\hspace{0pt}॥} \\
इन्धेरा\accentmark{27}{\char"1CD2}जासामर्यो\accentmark{20}{\char"1CF9}नामो\accentmark{22}{\char"0951}भिः।\ यस्याप्रा\accentmark{20}{\char"1CF9}ती\accentmark{22}{\char"0951}कामा\accentmark{22}{\char"0951}हूतङ्घृतेना।\ नारो\accentmark{27}{\char"1CD2}हव्येभीरीडतेसाबाधाः।\ 
अ\accentmark{27}{\char"1CD2}\kern0.15emग्नी\accentmark{27}{\char"1CD2}राग्रामूषा\accentmark{27}{\char"1CD2}\kern0.15emसा\accentmark{27}{\char"1CD2}माशोचि\mbox{॥ ८\hspace{0pt}॥} \\
प्रा\accentmark{27}{\char"1CD2}केतुना\accentmark{22}{\char"0951}बृहाता\accentmark{20}{\char"1CF9}यात्य\accentmark{27}{\char"1CD2}\kern0.15emग्निः\accentmark{27}{\char"1CD2}\kern0.15em।\ आ\accentmark{27}{\char"1CD2}रो\accentmark{22}{\char"0951}दसी\accentmark{22}{\char"0951}वृषाभो\accentmark{27}{\char"1CD2}रो\accentmark{22}{\char"0951}रवीति।\ दीवा\accentmark{27}{\char"1CD2}\kern0.15emश्चीद\accentmark{27}{\char"1CD2}न्तादूपामा\accentmark{27}{\char"1CD2}\kern0.15emमू\accentmark{27}{\char"1CD2}दानाट्।\ 
आ\accentmark{27}{\char"1CD2}\kern0.15emपा\accentmark{27}{\char"1CD2}मूपास्थे\accentmark{22}{\char"0951}माहीषो\accentmark{27}{\char"1CD2}वावर्धा\mbox{॥ ९\hspace{0pt}॥} \\
अग्निन्नारो\accentmark{27}{\char"1CD2}दीधितिभिरारण्योः\accentmark{22}{\char"0951}\kern0.15em।\ ह\accentmark{27}{\char"1CD2}स्ता\accentmark{22}{\char"0951}च्युतञ्जनयतप्राशस्तम्।\ दूरेदृशङ्गृहा\accentmark{27}{\char"1CD2}पा\accentmark{22}{\char"0951}तीमाथव्युम्\mbox{॥ १०\hspace{0pt}॥} \\
\mbox{॥ इति सप्तमः खण्डः\hspace{0pt}॥} \\
आ\accentmark{27}{\char"1CD2}बो\accentmark{22}{\char"0951}ध्यग्निस्सा\accentmark{27}{\char"1CD2}\kern0.15emमीधाज\accentmark{27}{\char"1CD2}ना\accentmark{22}{\char"0951}नां।\ प्रा\accentmark{20}{\char"1CF8}तीधेनू\accentmark{27}{\char"1CD2}मीवायातीमू\accentmark{27}{\char"1CD2}\kern0.15emषा\accentmark{27}{\char"1CD2}सम्।\ यह्वा\accentmark{20}{\char"1CF8}इ\accentmark{20}{\char"1CF8}वप्रा\accentmark{27}{\char"1CD2}\kern0.15emवाया\accentmark{27}{\char"1CD2}मूज्जीहा\accentmark{22}{\char"0951}नाः।\ 
प्रा\accentmark{20}{\char"1CF8}भान्नावासस्राते\accentmark{27}{\char"1CD2}नाकामाच्छा\mbox{॥ १\hspace{0pt}॥} \\
प्राभूर्जय\accentmark{27}{\char"1CD2}न्तम्माहीं\accentmark{20}{\char"1CF9}वीपोधाम्।\ मूरै\accentmark{27}{\char"1CD2}\kern0.15emरा\accentmark{27}{\char"1CD2}\kern0.15emमू\accentmark{27}{\char"1CD2}रंपूरा\accentmark{27}{\char"1CD2}न्द\accentmark{22}{\char"0951}र्माण\accentmark{22}{\char"0951}म्।\ 
ना\accentmark{27}{\char"1CD2}यान्तंगी\accentmark{27}{\char"1CD2}र्भिर्वाना\accentmark{27}{\char"1CD2}धिय\accentmark{22}{\char"0951}न्धाः।\ हारी\accentmark{22}{\char"0951}श्माश्रून्नावर्मणाधानार्चि\accentmark{22}{\char"0951}म्\mbox{॥ २\hspace{0pt}॥} \\
शु\accentmark{27}{\char"1CD2}क्रन्ते\accentmark{22}{\char"0951}अन्य\accentmark{22}{\char"0951}द्याजतन्ते\accentmark{22}{\char"0951}अन्यात्।\ वि\accentmark{22}{\char"0951}षूरूपेआ\accentmark{20}{\char"1CF8}हा\accentmark{22}{\char"0951}निद्यौरीवा\accentmark{22}{\char"0951}सी।\ 
विश्वा\accentmark{20}{\char"1CF9}हीमाया\accentmark{27}{\char"1CD2}आवासिस्वधावन्।\ भ\accentmark{27}{\char"1CD2}\kern0.15emद्रा\accentmark{27}{\char"1CD2}तेपूषन्नीहारातीरस्तू\mbox{॥ ३\hspace{0pt}॥} \\
ई\accentmark{22}{\char"0951}ळामग्ने\accentmark{20}{\char"1CF9}पू\accentmark{22}{\char"0951}रूदं\accentmark{20}{\char"1CF8}सं\accentmark{20}{\char"1CF8}सा\accentmark{20}{\char"1CF8}निंगोः।\ श\accentmark{27}{\char"1CD2}श्वक्तामं\accentmark{27}{\char"1CD2}\kern0.15emहा\accentmark{27}{\char"1CD2}वा\accentmark{22}{\char"0951}मानायासा\accentmark{27}{\char"1CD2}धा:\accentmark{22}{\char"0951}।\ 
स्यानस्सूनुस्ताना\accentmark{22}{\char"0951}योर्वीजा\accentmark{27}{\char"1CD2}\kern0.15emवा।\ अ\accentmark{27}{\char"1CD2}ग्नेसाते\accentmark{22}{\char"0951}सूमात्य\accentmark{20}{\char"1CF9}र्भूत्वस्मे\mbox{॥ ४\hspace{0pt}॥} \\
प्रहो\accentmark{20}{\char"1CF8}ता\accentmark{22}{\char"0951}जातो\accentmark{22}{\char"0951}माहा\accentmark{20}{\char"1CF9}न्ना\accentmark{27}{\char"1CD2}भोवीत्।\ नृषात्मा\accentmark{22}{\char"0951}सीददपां\accentmark{20}{\char"1CF9}वीवार्ते\accentmark{22}{\char"0951}।\ 
दधद्यो\accentmark{22}{\char"0951}धायीसू\accentmark{22}{\char"0951}\kern0.15emते\accentmark{27}{\char"1CD2}\kern0.15emवा\accentmark{27}{\char"1CD2}यांसि\accentmark{22}{\char"0951}।\ यन्तावासूनीवीधाते\accentmark{20}{\char"1CF9}ता\accentmark{22}{\char"0951}नूपाः\mbox{॥ ५\hspace{0pt}॥} \\
प्रासम्राजमा\accentmark{27}{\char"1CD2}सू\accentmark{22}{\char"0951}रस्यप्रा\accentmark{22}{\char"0951}शस्तं।\ पुंसःकृष्टीना\accentmark{20}{\char"1CF9}मा\accentmark{22}{\char"0951}नूमाद्यास्या\accentmark{22}{\char"0951}।\ इन्द्रस्येवप्रा\accentmark{27}{\char"1CD2}\kern0.15emतावा\accentmark{20}{\char"1CF9}सःकृता\accentmark{27}{\char"1CD2}\kern0.15emनि।\ व\accentmark{27}{\char"1CD2}\kern0.15emन्द\accentmark{20}{\char"1CF9}द्वीरावन्दमानाविवष्टु\mbox{॥ ६\hspace{0pt}॥} \\
आरण्\accentmark{27}{\char"1CD2}योर्नीहीतोजातावे\accentmark{22}{\char"0951}दाः।\ ग\accentmark{20}{\char"1CF8}र्भा\accentmark{22}{\char"0951}यीवेत्सू\accentmark{20}{\char"1CF9}भृ\accentmark{22}{\char"0951}तोगर्भिणीभिः।\ दीवे\accentmark{20}{\char"1CF9}दीवाईश्वोजागृवा\accentmark{27}{\char"1CD2}त्मिः।\ हा\accentmark{27}{\char"1CD2}\kern0.15emवि\accentmark{27}{\char"1CD2}ष्मा\accentmark{22}{\char"0951}द्धि\accentmark{20}{\char"1CF9}र्मानुष्ये\accentmark{22}{\char"0951}भीरग्निः\accentmark{27}{\char"1CD2}\mbox{॥ ७\hspace{0pt}॥} \\
सा\accentmark{27}{\char"1CD2}नाद\accentmark{22}{\char"0951}अग्नेमृणसि\accentmark{22}{\char"0951}यतूधाना\accentmark{22}{\char"0951}\kern0.15emन्।\ न\accentmark{27}{\char"1CD2}त्वारक्षां\accentmark{20}{\char"1CF9}सीपृ\accentmark{27}{\char"1CD2}ता\accentmark{22}{\char"0951}नासुजिग्युः।\ आनू\accentmark{22}{\char"0951}दहासाहामूरान्काया\accentmark{27}{\char"1CD2}दाः।\ मा\accentmark{20}{\char"1CF8}ते\accentmark{22}{\char"0951}हेत्यामू\accentmark{22}{\char"0951}क्षातादै\accentmark{27}{\char"1CD2}व्या\accentmark{22}{\char"0951}याः\mbox{॥ ८\hspace{0pt}॥} \\
\mbox{॥ इति अष्ठम: खण्डः॥} \\
अग्ना\accentmark{20}{\char"1CF9}ओजीष्ठामा\accentmark{27}{\char"1CD2}भा\accentmark{22}{\char"0951}रा।\ धुम्नामस्माभ्या\accentmark{22}{\char"0951}मधिगो।\ प्रा\accentmark{20}{\char"1CF8}नो\accentmark{20}{\char"1CF8}रा\accentmark{22}{\char"0951}\kern0.15emयेपा\accentmark{27}{\char"1CD2}नीयसे।\ रत्सीवा\accentmark{20}{\char"1CF9}जायापन्थाम्\mbox{॥ १\hspace{0pt}॥} \\
या\accentmark{20}{\char"1CF8}दीवीरो\accentmark{22}{\char"0951}आनूष्यात्।\ अ\accentmark{27}{\char"1CD2}\kern0.15emग्नी\accentmark{27}{\char"1CD2}\kern0.15emमि\accentmark{27}{\char"1CD2}न्धीतामार्क्त्या:।\ आ(२)जुह्वद्धेर्व्यमानूषाक्।\ शर्माभक्षितादै(२)व्य\accentmark{22}{\char"0951}म्\mbox{॥ २\hspace{0pt}॥} \\
आबोत्वे\accentmark{27}{\char"1CD2}\kern0.15emष\accentmark{20}{\char"1CF9}स्ते\accentmark{22}{\char"0951}\kern0.15emधूमा\accentmark{27}{\char"1CD2}ऋण्वति।\ दिविसत्शुक्रआ\accentmark{27}{\char"1CD2}त\accentmark{22}{\char"0951}\kern0.15emतः।\ सू\accentmark{27}{\char"1CD2}रोनहिद्यूतात्वं\accentmark{27}{\char"1CD2}।\ कृपापावा\accentmark{27}{\char"1CD2}\kern0.15emकारो\accentmark{27}{\char"1CD2}चा\accentmark{22}{\char"0951}से\mbox{॥ ३\hspace{0pt}॥} \\
त्वं\accentmark{22}{\char"0951}हीक्षैता\accentmark{27}{\char"1CD2}\kern0.15emव\accentmark{27}{\char"1CD2}\kern0.15emद्या\accentmark{27}{\char"1CD2}\kern0.15emशाः।\ अ\accentmark{20}{\char"1CF8}ग्ने\accentmark{22}{\char"0951}मित्रो\accentmark{27}{\char"1CD2}\kern0.15emनापा\accentmark{27}{\char"1CD2}त्या\accentmark{22}{\char"0951}से।\ त्वंवि\accentmark{22}{\char"0951}चर्षा\accentmark{22}{\char"0951}णेश्रा\accentmark{27}{\char"1CD2}वाः\accentmark{22}{\char"0951}।\ वासो\accentmark{22}{\char"0951}पूष्टिन्ना\accentmark{27}{\char"1CD2}पूष्यसि\mbox{॥ ४\hspace{0pt}॥} \\
प्रा\accentmark{20}{\char"1CF8}ता\accentmark{22}{\char"0951}रग्निःपूरूप्रीयः\accentmark{27}{\char"1CD2}\kern0.15em।\ वीश\accentmark{20}{\char"1CF9}स्तावेता\accentmark{27}{\char"1CD2}तीथिः।\ विश्वेयास्मिन्ना\accentmark{27}{\char"1CD2}मार्ये।\ हव्यंमर्ता\accentmark{22}{\char"0951}साइन्धाते\mbox{॥ ५\hspace{0pt}॥} \\
य\accentmark{20}{\char"1CF8}द्वाही\accentmark{22}{\char"0951}ष्ठन्ता\accentmark{22}{\char"0951}\kern0.15emदग्ना\accentmark{27}{\char"1CD2}ये।\ बृहा\accentmark{27}{\char"1CD2}द\accentmark{22}{\char"0951}र्चविभा\accentmark{22}{\char"0951}वसो।\ माही\accentmark{22}{\char"0951}षीवत्वा\accentmark{27}{\char"1CD2}द्रायिः।\ त्व\accentmark{20}{\char"1CF8}द्वा\accentmark{22}{\char"0951}\kern0.15emजाऊ\accentmark{27}{\char"1CD2}दी\accentmark{22}{\char"0951}रते\mbox{॥ ६\hspace{0pt}॥} \\
वी\accentmark{27}{\char"1CD2}\kern0.15emशोवी\accentmark{27}{\char"1CD2}\kern0.15emशोवोआ\accentmark{27}{\char"1CD2}ती\accentmark{22}{\char"0951}थिं।\ वज\accentmark{27}{\char"1CD2}यन्ताःपूरूप्रीयम्।\ अ\accentmark{27}{\char"1CD2}\kern0.15emग्निं\accentmark{20}{\char"1CF9}वो\accentmark{22}{\char"0951}दुर्यंवाचाः।\ स्तूषेशुषास्याम\accentmark{27}{\char"1CD2}न्मा\accentmark{22}{\char"0951}भिः\mbox{॥ ७\hspace{0pt}॥} \\
बृ\accentmark{27}{\char"1CD2}\kern0.15emह\accentmark{20}{\char"1CF9}द्वा\accentmark{20}{\char"1CF9}योहीभा\accentmark{27}{\char"1CD2}\kern0.15emना\accentmark{27}{\char"1CD2}\kern0.15emवे।\ अ\accentmark{20}{\char"1CF8}र्चा\accentmark{22}{\char"0951}\kern0.15emदेवा\accentmark{27}{\char"1CD2}यअग्ना\accentmark{27}{\char"1CD2}ये।\ यंमित्र\accentmark{27}{\char"1CD2}न्नप्रा\accentmark{27}{\char"1CD2}शा\accentmark{22}{\char"0951}स्तये।\ म\accentmark{27}{\char"1CD2}र्क्ता\accentmark{22}{\char"0951}सोदधी\accentmark{20}{\char"1CF9}रे\accentmark{22}{\char"0951}पूरः\mbox{॥ ८\hspace{0pt}॥} \\
आ\accentmark{27}{\char"1CD2}ग\accentmark{22}{\char"0951}न्मवृत्राह\accentmark{27}{\char"1CD2}न्तामं।\ ज्ये\accentmark{20}{\char"1CF8}ष्ठा\accentmark{22}{\char"0951}मग्निमाना\accentmark{27}{\char"1CD2}वम्।\ यस्म\accentmark{22}{\char"0951}श्रूतार्वन्नाक्षे।\ बृहाद\accentmark{22}{\char"0951}नीका\accentmark{22}{\char"0951}इध्याते\mbox{॥ ९\hspace{0pt}॥} \\
जाताः\accentmark{20}{\char"1CF9}पा\accentmark{27}{\char"1CD2}रे\accentmark{22}{\char"0951}\kern0.15emणाध\accentmark{27}{\char"1CD2}र्माणा।\ यत्सा\accentmark{27}{\char"1CD2}\kern0.15emवृ\accentmark{20}{\char"1CF9}त्भी\accentmark{22}{\char"0951}स्साहा\accentmark{27}{\char"1CD2}भू\accentmark{22}{\char"0951}\kern0.15emवः।\ पी\accentmark{27}{\char"1CD2}\kern0.15emताय\accentmark{27}{\char"1CD2}त्काश्या\accentmark{27}{\char"1CD2}पा\accentmark{22}{\char"0951}स्याग्निः।\ श्रद्धा\accentmark{20}{\char"1CF9}माता\accentmark{22}{\char"0951}मानूःकाविः\mbox{॥ १०\hspace{0pt}॥} \\
\mbox{॥ इति नवमः खण्डः\hspace{0pt}॥} \\ 

सोमंरा\accentmark{20}{\char"1CF9}जा\accentmark{22}{\char"0951}\kern0.15emनंवा\accentmark{27}{\char"1CD2}रूणम्।\ अ\accentmark{27}{\char"1CD2}\kern0.15emग्नी\accentmark{27}{\char"1CD2}मन्वारा\accentmark{22}{\char"0951}भामहे।\ आ\accentmark{27}{\char"1CD2}दित्यंविष्णुंसू\accentmark{27}{\char"1CD2}र्य\accentmark{22}{\char"0951}म्।\ ब्र\accentmark{27}{\char"1CD2}\kern0.15emह्मा\accentmark{20}{\char"1CF9}ण\accentmark{22}{\char"0951}ञ्चाबृहस्पा\accentmark{27}{\char"1CD2}तिम्\mbox{॥ १\hspace{0pt}॥} \\ईता\accentmark{27}{\char"1CD2}\kern0.15emएता\accentmark{27}{\char"1CD2}ऊदारूहन्।\ दीवः\accentmark{27}{\char"1CD2}पृष्ठान्यारू\accentmark{22}{\char"0951}हन्।\ प्राभूर्जयो\accentmark{22}{\char"0951}या\accentmark{20}{\char"1CF9}थापाथा\accentmark{27}{\char"1CD2}\kern0.15em।\ उ\accentmark{27}{\char"1CD2}द्यामङ्गी\accentmark{22}{\char"0951}रसोययुः\mbox{॥ २\hspace{0pt}॥} \\
रा\accentmark{27}{\char"1CD2}\kern0.15emये\accentmark{27}{\char"1CD2}आग्नेमाहेत्वा।\ दा\accentmark{20}{\char"1CF9}ना\accentmark{22}{\char"0951}\kern0.15emयासा\accentmark{27}{\char"1CD2}मी\accentmark{22}{\char"0951}धीमहि।\ ई\accentmark{20}{\char"1CF8}ळिष्वाहीमा\accentmark{22}{\char"0951}हेवृषन्।\ द्या\accentmark{20}{\char"1CF8}वा\accentmark{22}{\char"0951}होत्रा\accentmark{22}{\char"0951}या\accentmark{22}{\char"0951}पृथिवी\accentmark{27}{\char"1CD2}\mbox{॥ ३\hspace{0pt}॥} \\
दधन्वे\accentmark{27}{\char"1CD2}\kern0.15emवाया\accentmark{27}{\char"1CD2}दीमानू।\ वो\accentmark{20}{\char"1CF9}चा\accentmark{22}{\char"0951}त्ब्रह्मे\accentmark{27}{\char"1CD2}\kern0.15emतीवे\accentmark{27}{\char"1CD2}रूतात्।\ पा\accentmark{27}{\char"1CD2}\kern0.15emरीवि\accentmark{20}{\char"1CF9}श्वा\accentmark{22}{\char"0951}\kern0.15emनीका\accentmark{27}{\char"1CD2}व्या\accentmark{22}{\char"0951}\kern0.15em।\ ने\accentmark{27}{\char"1CD2}मीश्चक्रा\accentmark{27}{\char"1CD2}मी\accentmark{22}{\char"0951}वाभुवात्\mbox{॥ ४\hspace{0pt}॥} \\त्वामा\accentmark{22}{\char"0951}ग्नेवा\accentmark{20}{\char"1CF9}सुंरिहा।\ रूद्रं\accentmark{20}{\char"1CF9}आ\accentmark{22}{\char"0951}दित्यंऊता।\ याजस्वाध्वार\accentmark{27}{\char"1CD2}\kern0.15emञ्ज\accentmark{27}{\char"1CD2}न\accentmark{22}{\char"0951}\kern0.15emम्।\ आ\accentmark{27}{\char"1CD2}नूजातङ्घृ\accentmark{20}{\char"1CF9}तप्रू\accentmark{27}{\char"1CD2}षम्\mbox{॥ ५\hspace{0pt}॥} \\
प्रत्य\accentmark{22}{\char"0951}ग्नेहा\accentmark{20}{\char"1CF9}रा\accentmark{22}{\char"0951}\kern0.15emसाहा\accentmark{27}{\char"1CD2}राः।\ शृणा\accentmark{27}{\char"1CD2}हीविश्वा\accentmark{27}{\char"1CD2}\kern0.15emतप्रा\accentmark{27}{\char"1CD2}\kern0.15emती।\ या\accentmark{27}{\char"1CD2}\kern0.15emतूधा\accentmark{27}{\char"1CD2}ना\accentmark{22}{\char"0951}स्या\accentmark{20}{\char"1CF8}रक्षासाः\accentmark{22}{\char"0951}।\ बलंन्युब्जावीर्य\accentmark{22}{\char"0951}म्\mbox{॥ ६\hspace{0pt}॥} \\
\mbox{॥ इति दशमः खण्डः\hspace{0pt}॥} \\ पूरू\accentmark{27}{\char"1CD2}त्वा\accentmark{22}{\char"0951}दाशीवं\accentmark{27}{\char"1CD2}वो\accentmark{22}{\char"0951}चे\accentmark{20}{\char"1CF8}त् ।\  आरीरग्नेता\accentmark{20}{\char"1CF9}वास्वीदा\accentmark{27}{\char"1CD2} ।\  तोद\accentmark{27}{\char"1CD2}\kern0.15emस्ये\accentmark{27}{\char"1CD2}वशारणआ\accentmark{22}{\char"0951}माहास्या\accentmark{22}{\char"0951}\mbox{॥ १\hspace{0pt}॥} \\ प्र\accentmark{20}{\char"1CF8}होत्रे\accentmark{22}{\char"0951}पूर्वीयंवाचाः\accentmark{22}{\char"0951} ।\  अग्नाये\accentmark{22}{\char"0951}भरताबृहा\accentmark{27}{\char"1CD2}त् ।\  वीपांज्योती\accentmark{22}{\char"0951}षिबि\accentmark{20}{\char"1CF9}भ्रा\accentmark{22}{\char"0951}\kern0.15em
तेना\accentmark{27}{\char"1CD2}वेधासे\accentmark{22}{\char"0951}\mbox{॥ २\hspace{0pt}॥} \\अग्नेवाज\accentmark{22}{\char"0951}स्यागोमा\accentmark{22}{\char"0951}\kern0.15emतः।\ ई\accentmark{27}{\char"1CD2}शा\accentmark{22}{\char"0951}नस्सहसोयहो।\ अ\accentmark{20}{\char"1CF8}स्मे\accentmark{27}{\char"1CD2}देहीजातवेदोमा\accentmark{27}{\char"1CD2}हिश्रावाः\mbox{॥ ३\hspace{0pt}॥} \\अ\accentmark{20}{\char"1CF8}ग्ने\accentmark{27}{\char"1CD2}\kern0.15emया\accentmark{27}{\char"1CD2}जीष्ठोआध्वारे।\ देवान्दे\accentmark{22}{\char"0951}वायातेयाजा।\ होता\accentmark{22}{\char"0951}मन्द्रो\accentmark{20}{\char"1CF9}वी\accentmark{22}{\char"0951}\kern0.15emराज\accentmark{27}{\char"1CD2}स्यातिसृधाः\mbox{॥ ४\hspace{0pt}॥} \\
जज्ञान\accentmark{27}{\char"1CD2}स्सप्ता\accentmark{27}{\char"1CD2}\kern0.15emमातृ\accentmark{27}{\char"1CD2}भिः।\ मे\accentmark{27}{\char"1CD2}धामाशा\accentmark{22}{\char"0951}सातश्रिये\accentmark{27}{\char"1CD2}\kern0.15em।\ आ\accentmark{20}{\char"1CF8}यन्ध्रुवोरायीणां\accentmark{20}{\char"1CF9}चीके\accentmark{22}{\char"0951}तादा\mbox{॥ ५\hspace{0pt}॥} \\
ऊत\accentmark{27}{\char"1CD2}\kern0.15emस्या\accentmark{27}{\char"1CD2}नोदीवामातिः।\ आदी\accentmark{22}{\char"0951}तीरूत्यगमात्।\ साश\accentmark{20}{\char"1CF8}न्तातामायास्कारादपश्री\accentmark{27}{\char"1CD2}धाः\mbox{॥ ६\hspace{0pt}॥} \\
ईळि\accentmark{22}{\char"0951}ष्वाहि\accentmark{20}{\char"1CF9}प्रा\accentmark{22}{\char"0951}तिव्यम्\accentmark{22}{\char"0951}\kern0.15em।\ या\accentmark{27}{\char"1CD2}ज\accentmark{22}{\char"0951}स्वाजातावे\accentmark{22}{\char"0951}दसं।\ चरिष्णू\accentmark{22}{\char"0951}धूमामागृभीतशोचिषम्\mbox{॥ ७\hspace{0pt}॥} \\नत\accentmark{27}{\char"1CD2}स्यामाया\accentmark{20}{\char"1CF9}याचाना\accentmark{27}{\char"1CD2}\kern0.15em।\ री\accentmark{27}{\char"1CD2}\kern0.15emपू\accentmark{27}{\char"1CD2}रीशीतामाक्र्त्याः।\ यो\accentmark{27}{\char"1CD2}\kern0.15emअग्ना\accentmark{20}{\char"1CF9}ये\accentmark{22}{\char"0951}ददा\accentmark{20}{\char"1CF9}शाहव्या\accentmark{27}{\char"1CD2}दा\accentmark{22}{\char"0951}तये\mbox{॥ ८\hspace{0pt}॥} \\अपत्यं\accentmark{20}{\char"1CF9}वृ\accentmark{22}{\char"0951}जीनंरिपुं\accentmark{27}{\char"1CD2}\kern0.15em।\ स्ते\accentmark{27}{\char"1CD2}\kern0.15emना\accentmark{27}{\char"1CD2}मा\accentmark{22}{\char"0951}ग्नेदूराध्यं।\ दे\accentmark{27}{\char"1CD2}वीष्ठमत्स्यस\accentmark{22}{\char"0951}त्पतेकृधीसूगम्\mbox{॥ ९\hspace{0pt}॥} \\श्रु\accentmark{27}{\char"1CD2}ष्ट्या\accentmark{27}{\char"1CD2}ग्नेना\accentmark{27}{\char"1CD2}वा\accentmark{22}{\char"0951}स्यमे।\ स्तोमा\accentmark{22}{\char"0951}स्यवीरविस्पते।\ नीमा\accentmark{27}{\char"1CD2}यिनस्तपा\accentmark{27}{\char"1CD2}सारक्षा\accentmark{27}{\char"1CD2}सो\accentmark{22}{\char"0951}दह\mbox{॥ १०\hspace{0pt}॥} \\
 \mbox{॥ इति एकादशः खण्डः\hspace{0pt}॥} \\ 
प्रमंही\accentmark{22}{\char"0951}ष्ठायगायता।\ ऋताव्नेबृहाते\accentmark{27}{\char"1CD2}शुक्राशो\accentmark{22}{\char"0951}चिषे।\ उ\accentmark{27}{\char"1CD2}\kern0.15emप\accentmark{27}{\char"1CD2}स्तूता\accentmark{20}{\char"1CF9}सो\accentmark{22}{\char"0951}अग्नाये\mbox{॥ १\hspace{0pt}॥} \\प्र\accentmark{20}{\char"1CF8}सोअग्नेतावोतीभीः।\ सू\accentmark{27}{\char"1CD2}वीरा\accentmark{22}{\char"0951}भिस्तारातीवा\accentmark{27}{\char"1CD2}जकर्म\accentmark{22}{\char"0951}भिः।\ या\accentmark{27}{\char"1CD2}स्यत्वं\accentmark{27}{\char"1CD2}सख्यमा\accentmark{27}{\char"1CD2}\kern0.15emवी\accentmark{27}{\char"1CD2}थ\mbox{॥ २\hspace{0pt}॥} \\तं\accentmark{27}{\char"1CD2}गूर्धायास्व\accentmark{27}{\char"1CD2}र्णारं।\ देवासो\accentmark{22}{\char"0951}देवा\accentmark{20}{\char"1CF9}मा\accentmark{22}{\char"0951}रातिन्द\accentmark{22}{\char"0951}धन्विरे।\ देवात्रा\accentmark{27}{\char"1CD2}\kern0.15emहव्या\accentmark{27}{\char"1CD2}मू\accentmark{22}{\char"0951}हिषे\mbox{॥ ३\hspace{0pt}॥} \\मानो\accentmark{22}{\char"0951}हृ\accentmark{22}{\char"0951}\kern0.15emणिथाआ\accentmark{27}{\char"1CD2}तीथिं।\ वा\accentmark{20}{\char"1CF9}सूरग्निः\accentmark{27}{\char"1CD2}पू\accentmark{22}{\char"0951}रूप्राशस्ता\accentmark{27}{\char"1CD2}एषः।\ या\accentmark{27}{\char"1CD2}सुहोता\accentmark{22}{\char"0951}सूआध्वारः\mbox{॥ ४\hspace{0pt}॥} \\
भद्रो\accentmark{20}{\char"1CF9}नो\accentmark{22}{\char"0951}\kern0.15emअग्नि\accentmark{27}{\char"1CD2}राहूतः।\ भ\accentmark{27}{\char"1CD2}\kern0.15emद्रा\accentmark{27}{\char"1CD2}रातिस्सू\accentmark{27}{\char"1CD2}भागभद्रो\accentmark{20}{\char"1CF9}आ\accentmark{22}{\char"0951}ध्वारः\accentmark{27}{\char"1CD2}।\ भद्रा\accentmark{20}{\char"1CF9}ऊतप्रा\accentmark{27}{\char"1CD2}शा\accentmark{22}{\char"0951}स्तयः\mbox{॥ ५\hspace{0pt}॥} \\या\accentmark{27}{\char"1CD2}जी\accentmark{22}{\char"0951}ष्ठं\accentmark{22}{\char"0951}त्वाववृमहे।\ देवन्दे\accentmark{22}{\char"0951}वत्रा\accentmark{20}{\char"1CF9}हो\accentmark{20}{\char"1CF9}ता\accentmark{22}{\char"0951}\kern0.15emरामा\accentmark{27}{\char"1CD2}मा\accentmark{22}{\char"0951}र्क्त्यम्।\ अस्या\accentmark{27}{\char"1CD2}\kern0.15emयज्ञा\accentmark{20}{\char"1CF9}स्या\accentmark{22}{\char"0951}सुक्रातु\accentmark{22}{\char"0951}म्\mbox{॥ ६\hspace{0pt}॥} \\ता\accentmark{27}{\char"1CD2}दग्नेद्युम्नामा\accentmark{27}{\char"1CD2}भारा।\ य\accentmark{27}{\char"1CD2}त्सासा\accentmark{27}{\char"1CD2}हासाद\accentmark{22}{\char"0951}नेकञ्ची\accentmark{22}{\char"0951}\kern0.15emदत्री\accentmark{27}{\char"1CD2}णं।\ मन्यु\accentmark{27}{\char"1CD2}ञ्जना\accentmark{22}{\char"0951}स्यादूढ्यम्\mbox{॥ ७\hspace{0pt}॥} \\य\accentmark{27}{\char"1CD2}\kern0.15emद्वा\accentmark{27}{\char"1CD2}ऊविश्पा\accentmark{20}{\char"1CF9}तीश्शितः।\ सुप्रीतो\accentmark{22}{\char"0951}\kern0.15emमा\accentmark{27}{\char"1CD2}नुषोवीशे।\ विश्वेदग्निप्रा\accentmark{27}{\char"1CD2}तिर\accentmark{22}{\char"0951}क्षांसिसेधति\mbox{॥ ८\hspace{0pt}॥} \\
आ\accentmark{27}{\char"1CD2}\kern0.15emया\accentmark{27}{\char"1CD2}\kern0.15emमग्नि\accentmark{27}{\char"1CD2}श्रेष्ठातमः।\ आ\accentmark{27}{\char"1CD2}\kern0.15emयं\accentmark{27}{\char"1CD2}वामाधू\accentmark{27}{\char"1CD2}मक्तमः।\ आयं\accentmark{27}{\char"1CD2}साहस्रासा\accentmark{27}{\char"1CD2}\kern0.15emतमः।\ अ\accentmark{27}{\char"1CD2}\kern0.15emस्मि\accentmark{27}{\char"1CD2}न्ना\accentmark{22}{\char"0951}स्तुसूवीर्यम्\mbox{॥ ९\hspace{0pt}॥} \\
आ\accentmark{27}{\char"1CD2}दित्यः\accentmark{27}{\char"1CD2}शुक्रा\accentmark{27}{\char"1CD2}\kern0.15emऊ\accentmark{27}{\char"1CD2}दगात्पूरास्तात्।\ ज्यो\accentmark{20}{\char"1CF8}तिःकृण्वन्वी\accentmark{27}{\char"1CD2}तामो\accentmark{22}{\char"0951}बाधामानः।\ आ\accentmark{27}{\char"1CD2}\kern0.15emभा\accentmark{27}{\char"1CD2}स\accentmark{22}{\char"0951}मानाप्रादीशो\accentmark{22}{\char"0951}\kern0.15emनूसा\accentmark{27}{\char"1CD2}र्वाः।\ भद्रस्या\accentmark{27}{\char"1CD2}कर्ता\accentmark{22}{\char"0951}रूचायन्नाआगात्\mbox{॥ १०\hspace{0pt}॥} \\
\mbox{॥ इति द्वादशः खण्डः\hspace{0pt}॥} \\ \mbox{॥ इति आग्नेयपाठः समाप्तः\hspace{0pt}॥} \\ \clearpage
\mbox{॥ अथ तद्वपाठः प्रारम्भः\hspace{0pt}॥} \\
त\accentmark{27}{\char"1CD2}\kern0.15emद्वो\accentmark{27}{\char"1CD2}गायासू\accentmark{27}{\char"1CD2}\kern0.15emतेसा\accentmark{27}{\char"1CD2}चा।\ पुरुहूता\accentmark{27}{\char"1CD2}यासात्वा\accentmark{22}{\char"0951}\kern0.15emने।\ शं\accentmark{27}{\char"1CD2}\kern0.15emय\accentmark{27}{\char"1CD2}\kern0.15emत्ग\accentmark{27}{\char"1CD2}वेनाशाकीने\mbox{॥ १\hspace{0pt}॥} \\
य\accentmark{20}{\char"1CF8}स्ते\accentmark{22}{\char"0951}नूनंशा\accentmark{22}{\char"0951}तकृतो\accentmark{22}{\char"0951}।\ इन्द्र\accentmark{22}{\char"0951}घुम्नीता\accentmark{22}{\char"0951}\kern0.15emमो\accentmark{27}{\char"1CD2}\kern0.15emमा\accentmark{27}{\char"1CD2}दाः\accentmark{22}{\char"0951}।\ ते\accentmark{20}{\char"1CF8}ना\accentmark{22}{\char"0951}नूनम्मादे\accentmark{22}{\char"0951}मदेः\mbox{॥ २\hspace{0pt}॥} \\
गा\accentmark{27}{\char"1CD2}\kern0.15emवाऊ\accentmark{27}{\char"1CD2}पा\accentmark{22}{\char"0951}वादावाटे\accentmark{27}{\char"1CD2}\kern0.15em।\ माही\accentmark{27}{\char"1CD2}यज्ञास्या\accentmark{20}{\char"1CF9}रप्सू\accentmark{22}{\char"0951}\kern0.15emदाः।\ ऊ\accentmark{27}{\char"1CD2}\kern0.15emभाक\accentmark{27}{\char"1CD2}र्णा\accentmark{22}{\char"0951}हीरण्या\accentmark{27}{\char"1CD2}या\mbox{॥ ३\hspace{0pt}॥} \\
आ\accentmark{27}{\char"1CD2}\kern0.15emराम\accentmark{20}{\char"1CF9}श्व\accentmark{22}{\char"0951}यगायता।\ श्रू\accentmark{20}{\char"1CF8}ता\accentmark{22}{\char"0951}\kern0.15emकक्षा\accentmark{27}{\char"1CD2}\kern0.15emरंग\accentmark{27}{\char"1CD2}\kern0.15emवे।\ आ\accentmark{27}{\char"1CD2}\kern0.15emरामि\accentmark{20}{\char"1CF9}न्द्रा\accentmark{22}{\char"0951}स्याधा\accentmark{27}{\char"1CD2}\kern0.15emम्ने\accentmark{20}{\char"1CF8}\mbox{॥ ४\hspace{0pt}॥} \\
त\accentmark{27}{\char"1CD2}\kern0.15emमि\accentmark{27}{\char"1CD2}न्द्रं\accentmark{22}{\char"0951}वाजयामसि।\ मा\accentmark{20}{\char"1CF8}हे\accentmark{22}{\char"0951}वृत्रायाह\accentmark{27}{\char"1CD2}न्ता\accentmark{22}{\char"0951}\kern0.15emवे।\ स\accentmark{27}{\char"1CD2}\kern0.15emवृ\accentmark{27}{\char"1CD2}षा\accentmark{22}{\char"0951}वृषाभो\accentmark{27}{\char"1CD2}भू\accentmark{22}{\char"0951}वात्\mbox{॥ ५\hspace{0pt}॥} \\
त्वा\accentmark{20}{\char"1CF8}मिन्द्राब\accentmark{27}{\char"1CD2}\kern0.15emलाद\accentmark{27}{\char"1CD2}धी\accentmark{22}{\char"0951}\kern0.15em।\ साहा\accentmark{27}{\char"1CD2}\kern0.15emसा\accentmark{27}{\char"1CD2}\kern0.15emजाता\accentmark{27}{\char"1CD2}\kern0.15emओ\accentmark{27}{\char"1CD2}ज\accentmark{22}{\char"0951}सा।\ त्वं\accentmark{20}{\char"1CF8}सन्वृष\accentmark{22}{\char"0951}न्वृषेद\accentmark{22}{\char"0951}सि\mbox{॥ ६\hspace{0pt}॥} \\
यज्ञाइ\accentmark{27}{\char"1CD2}न्द्रा\accentmark{22}{\char"0951}मवर्धयात्।\ य\accentmark{27}{\char"1CD2}त्भूमिंव्या\accentmark{27}{\char"1CD2}वा\accentmark{22}{\char"0951}र्क्तयात्।\ चक्रा\accentmark{20}{\char"1CF9}णाओपाश\accentmark{27}{\char"1CD2}न्दीवि\mbox{॥ ७\hspace{0pt}॥} \\
या\accentmark{20}{\char"1CF8}दिन्द्राहं\accentmark{20}{\char"1CF8}या\accentmark{27}{\char"1CD2}थात्वं\accentmark{27}{\char"1CD2}।\ ईशी\accentmark{22}{\char"0951}\kern0.15emयावा\accentmark{27}{\char"1CD2}\kern0.15emस्वाए\accentmark{27}{\char"1CD2}\kern0.15emकाई\accentmark{27}{\char"1CD2}त्।\ स्तोता\accentmark{27}{\char"1CD2}मेगोसाखास्यात्\mbox{॥ ८\hspace{0pt}॥} \\
प\accentmark{27}{\char"1CD2}न्यं\accentmark{22}{\char"0951}पन्यामि\accentmark{27}{\char"1CD2}स्तो\accentmark{22}{\char"0951}\kern0.15emतरः।\ आ\accentmark{27}{\char"1CD2}धा\accentmark{22}{\char"0951}वातामा\accentmark{27}{\char"1CD2}द्या\accentmark{22}{\char"0951}या।\ सो\accentmark{20}{\char"1CF8}मं\accentmark{22}{\char"0951}\kern0.15emवीरा\accentmark{27}{\char"1CD2}यासूरा\accentmark{22}{\char"0951}या\mbox{॥ ९\hspace{0pt}॥} \\
ईदं\accentmark{27}{\char"1CD2}\kern0.15em वासो\accentmark{27}{\char"1CD2} सूताम\accentmark{27}{\char"1CD2}न्धाः ।\  पीबा सूपू\accentmark{22}{\char"0951}र्णा\accentmark{22}{\char"0951}\kern0.15emमू\accentmark{27}{\char"1CD2}दरं\accentmark{22}{\char"0951} ।\  आनाभ इन्द्रा\accentmark{27}{\char"1CD2} रीमा ते \mbox{॥ १०\hspace{0pt}॥} \\ \mbox{॥ इति प्रथमः खण्डः\hspace{0pt}॥} \\
उ\accentmark{27}{\char"1CD2}त्घेदभी\accentmark{27}{\char"1CD2}श्रूता\accentmark{27}{\char"1CD2}मा\accentmark{22}{\char"0951}घं।\ वृषाभ\accentmark{27}{\char"1CD2}न्नर्या\accentmark{22}{\char"0951}पसं।\ अस्तारमेळिसूर्या\mbox{॥ १\hspace{0pt}॥} \\
यादद्याकाच्चावृत्रहन्।\ ऊदगाआभी\accentmark{27}{\char"1CD2}सूर्या\accentmark{27}{\char"1CD2}\kern0.15em।\ स\accentmark{27}{\char"1CD2}र्वन्ता\accentmark{27}{\char"1CD2}दिन्द्राते\accentmark{22}{\char"0951}\kern0.15emवा\accentmark{27}{\char"1CD2}शे\mbox{॥ २\hspace{0pt}॥} \\
य\accentmark{27}{\char"1CD2}\kern0.15emआ\accentmark{27}{\char"1CD2}ना\accentmark{22}{\char"0951}यत्पारावा\accentmark{27}{\char"1CD2}ताः।\ सू\accentmark{27}{\char"1CD2}नी\accentmark{22}{\char"0951}तीतूर्वा\accentmark{27}{\char"1CD2}\kern0.15emशं\accentmark{27}{\char"1CD2}\kern0.15emया\accentmark{27}{\char"1CD2}दुं\accentmark{22}{\char"0951}\kern0.15em।\ इ\accentmark{27}{\char"1CD2}न्द्रस्सा2नोयू\accentmark{22}{\char"0951}\kern0.15emवासा\accentmark{27}{\char"1CD2}खा\accentmark{22}{\char"0951}\mbox{॥ ३\hspace{0pt}॥} \\
मा\accentmark{27}{\char"1CD2}ना\accentmark{22}{\char"0951}इन्द्राभ्याटट्यदीशः\accentmark{22}{\char"0951}।\ सू\accentmark{20}{\char"1CF9}रोअत्तुष्वाया\accentmark{22}{\char"0951}मात्।\ त्वा\accentmark{27}{\char"1CD2}\kern0.15emयूजा\accentmark{27}{\char"1CD2}वानेमाता\accentmark{27}{\char"1CD2}त्\mbox{॥ ४\hspace{0pt}॥} \\
ए\accentmark{27}{\char"1CD2}न्द्रा\accentmark{22}{\char"0951}सानासिं\accentmark{27}{\char"1CD2}\kern0.15emरायिं\accentmark{27}{\char"1CD2}\kern0.15em।\ सा\accentmark{27}{\char"1CD2}\kern0.15emजी\accentmark{27}{\char"1CD2}त्वा\accentmark{22}{\char"0951}नंसादासा\accentmark{27}{\char"1CD2}हं।\ वर्षी\accentmark{22}{\char"0951}ष्ठामू\accentmark{27}{\char"1CD2}\kern0.15emताये\accentmark{27}{\char"1CD2}भरा\mbox{॥ ५\hspace{0pt}॥} \\
इ\accentmark{20}{\char"1CF8}न्द्रं\accentmark{27}{\char"1CD2}\kern0.15emवाय\accentmark{27}{\char"1CD2}म्माहाधाने\accentmark{27}{\char"1CD2}।\ इन्द्रा\accentmark{27}{\char"1CD2}मर्भे\accentmark{22}{\char"0951}हवामहे।\ यूजं\accentmark{22}{\char"0951}वृ\accentmark{20}{\char"1CF8}त्रे\accentmark{20}{\char"1CF9}षुव\accentmark{27}{\char"1CD2}ज्रीण\accentmark{22}{\char"0951}म्\mbox{॥ ६\hspace{0pt}॥} \\
आपी\accentmark{22}{\char"0951}बत्काद्रू\accentmark{20}{\char"1CF9}वा\accentmark{22}{\char"0951}स्सुतं\accentmark{27}{\char"1CD2}।\ इन्द्रा\accentmark{20}{\char"1CF8}सा\accentmark{22}{\char"0951}\kern0.15emहा\accentmark{27}{\char"1CD2}स्रा\accentmark{22}{\char"0951}वांभे।\ तत्राददी\accentmark{22}{\char"0951}ष्टापौंठस्यम्\mbox{॥ ७\hspace{0pt}॥} \\
वा\accentmark{27}{\char"1CD2}यामिन्द्रात्वाया\accentmark{27}{\char"1CD2}वाः\accentmark{22}{\char"0951}\kern0.15em।\ आभि\accentmark{27}{\char"1CD2}प्रानो\accentmark{22}{\char"0951}नुमोवृषन्।\ विद्धीत्वाटट्यस्या\accentmark{27}{\char"1CD2}नोवसो\mbox{॥ ८\hspace{0pt}॥} \\
आ\accentmark{27}{\char"1CD2}\kern0.15emघाये\accentmark{27}{\char"1CD2}\kern0.15emअग्नी\accentmark{27}{\char"1CD2}मिन्धा\accentmark{27}{\char"1CD2}ते।\ स्तृणन्तीबही\accentmark{20}{\char"1CF9}रानूषाक्।\ ए\accentmark{27}{\char"1CD2}\kern0.15emषा\accentmark{27}{\char"1CD2}\kern0.15emमि\accentmark{27}{\char"1CD2}न्द्रोयूवासाखा\accentmark{27}{\char"1CD2}\mbox{॥ ९\hspace{0pt}॥} \\
 भि\accentmark{27}{\char"1CD2}न्धि\accentmark{22}{\char"0951} विश्वा\accentmark{27}{\char"1CD2} आप द्वी\accentmark{27}{\char"1CD2}षाः ।\  पा\accentmark{22}{\char"0951}री बाधो\accentmark{20}{\char"1CF9} जही\accentmark{22}{\char"0951}\kern0.15em मृ\accentmark{27}{\char"1CD2}धाः\accentmark{22}{\char"0951} ।\  वा\accentmark{20}{\char"1CF8}सु\accentmark{22}{\char"0951} स्पार्हं तादा भा\accentmark{22}{\char"0951}रा \mbox{॥ १०\hspace{0pt}॥} \\ \mbox{॥ इति द्वितीयः खण्डः\hspace{0pt}॥} \\ 
ईहे\accentmark{27}{\char"1CD2}वा\accentmark{22}{\char"0951}शृण्वाएषां।\ काशाह\accentmark{20}{\char"1CF9}स्तेषूयद्वादा\accentmark{22}{\char"0951}न्।\ निया\accentmark{20}{\char"1CF8}मैःचित्रा\accentmark{27}{\char"1CD2}मृन्जते\mbox{॥ १\hspace{0pt}॥} \\
ईमा\accentmark{20}{\char"1CF9}ऊत्वावीचा\accentmark{22}{\char"0951}क्षते।\ सा\accentmark{27}{\char"1CD2}खा\accentmark{22}{\char"0951}याइन्द्रासोमीनाः\accentmark{22}{\char"0951}।\ पुष्टावन्तोया\accentmark{20}{\char"1CF9}था\accentmark{22}{\char"0951}\kern0.15emपाशु\accentmark{27}{\char"1CD2}म्\mbox{॥ २\hspace{0pt}॥} \\
सा\accentmark{27}{\char"1CD2}मा\accentmark{22}{\char"0951}स्यामन्या\accentmark{27}{\char"1CD2}वेवीशाः\accentmark{22}{\char"0951}।\ विश्वे\accentmark{22}{\char"0951}नमन्ताकृष्टा\accentmark{27}{\char"1CD2}याः।\ समुद्रा\accentmark{20}{\char"1CF9}ये\accentmark{22}{\char"0951}वासिन्धावः\mbox{॥ ३\hspace{0pt}॥} \\
देवा\accentmark{27}{\char"1CD2}\kern0.15emनामी\accentmark{20}{\char"1CF9}दावो\accentmark{22}{\char"0951}\kern0.15emमाहा\accentmark{27}{\char"1CD2}\kern0.15emत्।\ ता\accentmark{27}{\char"1CD2}\kern0.15emदा\accentmark{27}{\char"1CD2}वृणीमहेवायं।\ वृष्णा\accentmark{22}{\char"0951}मस्मा\accentmark{20}{\char"1CF9}भ्या\accentmark{22}{\char"0951}\kern0.15emमूता\accentmark{27}{\char"1CD2}ये\mbox{॥ ४\hspace{0pt}॥} \\
सोमानांस्वा\accentmark{27}{\char"1CD2}रा\accentmark{22}{\char"0951}णां।\ कृ\accentmark{27}{\char"1CD2}\kern0.15emणू\accentmark{27}{\char"1CD2}ही\accentmark{22}{\char"0951}ब्रह्मणस्पते।\ कक्षीवन्तंया\accentmark{20}{\char"1CF9}औशीजः\accentmark{27}{\char"1CD2}\mbox{॥ ५\hspace{0pt}॥} \\
बोध\accentmark{27}{\char"1CD2}न्माना\accentmark{27}{\char"1CD2}\kern0.15emइद\accentmark{27}{\char"1CD2}स्तुनः।\ वृत्राहा\accentmark{20}{\char"1CF8}भू\accentmark{27}{\char"1CD2}र्या\accentmark{22}{\char"0951}सुतिः।\ श्रृणो\accentmark{20}{\char"1CF9}तू\accentmark{22}{\char"0951}\kern0.15emशक्रा\accentmark{27}{\char"1CD2}\kern0.15emआशी\accentmark{27}{\char"1CD2}ष\accentmark{22}{\char"0951}म्\mbox{॥ ६\hspace{0pt}॥} \\
अ\accentmark{27}{\char"1CD2}\kern0.15emद्या\accentmark{27}{\char"1CD2}नो\accentmark{22}{\char"0951}देवसवितः।\ प्रा\accentmark{27}{\char"1CD2}\kern0.15emजा\accentmark{27}{\char"1CD2}वा\accentmark{22}{\char"0951}\kern0.15emत्सा\accentmark{27}{\char"1CD2}\kern0.15emवीसौ\accentmark{27}{\char"1CD2}भा\accentmark{22}{\char"0951}गं।\ पा\accentmark{20}{\char"1CF8}रा\accentmark{20}{\char"1CF8}दुष्वा\accentmark{27}{\char"1CD2}प्न्यिं\accentmark{22}{\char"0951}सुवा\mbox{॥ ७\hspace{0pt}॥} \\
क्वाट्ट्यस्या\accentmark{20}{\char"1CF9}वृ\accentmark{22}{\char"0951}षाभोयू\accentmark{22}{\char"0951}वा।\ तुविग्री\accentmark{27}{\char"1CD2}\kern0.15emवोआ\accentmark{27}{\char"1CD2}नानतः।\ ब्रह्मा\accentmark{27}{\char"1CD2}कस्तंसा\accentmark{27}{\char"1CD2}पर्यति\mbox{॥ ८\hspace{0pt}॥} \\
ऊपह्वारे\accentmark{20}{\char"1CF9}गी\accentmark{22}{\char"0951}\kern0.15emरीणां\accentmark{27}{\char"1CD2}।\ संगमे\accentmark{20}{\char"1CF9}चा\accentmark{22}{\char"0951}\kern0.15emनादी\accentmark{27}{\char"1CD2}नां\accentmark{22}{\char"0951}।\ धी\accentmark{22}{\char"0951}\kern0.15emया\accentmark{27}{\char"1CD2}विप्रो\accentmark{22}{\char"0951}अजायत\mbox{॥ ९\hspace{0pt}॥} \\
 प्रा\accentmark{27}{\char"1CD2} सम्म्रा\accentmark{27}{\char"1CD2}\kern0.15emजं\accentmark{27}{\char"1CD2} च ऋषाणीनां\accentmark{27}{\char"1CD2}\kern0.15em ।\  इ\accentmark{20}{\char"1CF8}न्द्रं\accentmark{22}{\char"0951} स्तोता न\accentmark{20}{\char"1CF9}व्यं गीर्भिः\accentmark{22}{\char"0951} ।\  ना\accentmark{20}{\char"1CF8}रं नृळा\accentmark{27}{\char"1CD2}हम्मंही\accentmark{22}{\char"0951}ष्ठम् \mbox{॥ १०\hspace{0pt}॥} \\ \mbox{॥ इति तृतीयः खण्डः\hspace{0pt}॥} \\ 
आपा\accentmark{22}{\char"0951}दूशिप्रि\accentmark{27}{\char"1CD2}यन्धा\accentmark{22}{\char"0951}साः।\ सूदक्षा\accentmark{22}{\char"0951}स्याप्राहोषीणाः\accentmark{22}{\char"0951}\kern0.15em।\ इ\accentmark{27}{\char"1CD2}न्दोरिन्द्रोया\accentmark{27}{\char"1CD2}वा\accentmark{22}{\char"0951}शिरः\mbox{॥ १\hspace{0pt}॥} \\
ईमा\accentmark{27}{\char"1CD2}उत्वापूरोवासो।\ आ\accentmark{27}{\char"1CD2}भिप्रा\accentmark{27}{\char"1CD2}नो\accentmark{22}{\char"0951}नावुत्गीराः\accentmark{22}{\char"0951}।\ गा\accentmark{20}{\char"1CF8}वो\accentmark{22}{\char"0951}\kern0.15emवत्स\accentmark{27}{\char"1CD2}\kern0.15emन्ना\accentmark{27}{\char"1CD2}\kern0.15emधेना\accentmark{27}{\char"1CD2}वाः\accentmark{22}{\char"0951}\kern0.15em\mbox{॥ २\hspace{0pt}॥} \\
अ\accentmark{27}{\char"1CD2}\kern0.15emत्रा\accentmark{27}{\char"1CD2}\kern0.15emहागो\accentmark{27}{\char"1CD2}रा\accentmark{22}{\char"0951}मन्वत।\ नामत्वष्टू\accentmark{22}{\char"0951}रा\accentmark{20}{\char"1CF9}पीठच्यं।\ इत्था\accentmark{27}{\char"1CD2}चन्द्रा\accentmark{27}{\char"1CD2}मा\accentmark{22}{\char"0951}\kern0.15emसोगृ\accentmark{27}{\char"1CD2}हे\mbox{॥ ३\hspace{0pt}॥} \\
य\accentmark{27}{\char"1CD2}\kern0.15emदि\accentmark{27}{\char"1CD2}न्द्रोआ\accentmark{20}{\char"1CF9}ना\accentmark{22}{\char"0951}यद्रीताः।\ माही\accentmark{27}{\char"1CD2}\kern0.15emरापो\accentmark{27}{\char"1CD2}\kern0.15emवृ\accentmark{27}{\char"1CD2}ष\accentmark{22}{\char"0951}न्तमः।\ तत्रा\accentmark{22}{\char"0951}पूषा\accentmark{20}{\char"1CF9}भू\accentmark{22}{\char"0951}\kern0.15emवत्सा\accentmark{27}{\char"1CD2}चा\mbox{॥ ४\hspace{0pt}॥} \\
गौ\accentmark{27}{\char"1CD2}र्धायतीमारुतां\accentmark{22}{\char"0951}।\ श्रवस्यु\accentmark{27}{\char"1CD2}र्मातामाघो\accentmark{27}{\char"1CD2}नां\accentmark{22}{\char"0951}।\ युक्ता\accentmark{27}{\char"1CD2}\kern0.15emव\accentmark{27}{\char"1CD2}\kern0.15emन्ही\accentmark{20}{\char"1CF8}रा\accentmark{27}{\char"1CD2}था\accentmark{22}{\char"0951}नाम्\mbox{॥ ५\hspace{0pt}॥} \\
ऊ\accentmark{27}{\char"1CD2}पा\accentmark{22}{\char"0951}\kern0.15emनोहा\accentmark{27}{\char"1CD2}रीभीस्सू\accentmark{27}{\char"1CD2}तं।\ याही\accentmark{27}{\char"1CD2}\kern0.15emमा\accentmark{27}{\char"1CD2}दानांपते\accentmark{22}{\char"0951}।\ ऊ\accentmark{20}{\char"1CF8}पानोहा\accentmark{27}{\char"1CD2}री\accentmark{22}{\char"0951}भीस्सू\accentmark{27}{\char"1CD2}तम्\mbox{॥ ६\hspace{0pt}॥} \\
इष्टाहोत्राअसृक्षत।\ इ\accentmark{20}{\char"1CF8}न्द्रं\accentmark{22}{\char"0951}वृधन्तो\accentmark{22}{\char"0951}आध्वा\accentmark{22}{\char"0951}\kern0.15emरे\accentmark{27}{\char"1CD2}।\ अच्छावाभृथमो\accentmark{27}{\char"1CD2}ज\accentmark{22}{\char"0951}सा\mbox{॥ ७\hspace{0pt}॥} \\
अ\accentmark{27}{\char"1CD2}\kern0.15emह\accentmark{27}{\char"1CD2}\kern0.15emमि\accentmark{27}{\char"1CD2}\kern0.15emद्धि\accentmark{27}{\char"1CD2}पीतुःपा\accentmark{22}{\char"0951}री।\ मेधा\accentmark{27}{\char"1CD2}\kern0.15emमृता\accentmark{20}{\char"1CF9}स्या\accentmark{22}{\char"0951}\kern0.15emजग्रा\accentmark{27}{\char"1CD2}हा\accentmark{22}{\char"0951}\kern0.15em।\ अहं\accentmark{27}{\char"1CD2}सू\accentmark{22}{\char"0951}\kern0.15emर्या\accentmark{27}{\char"1CD2}\kern0.15emइवा\accentmark{27}{\char"1CD2}जनी\mbox{॥ ८\hspace{0pt}॥} \\
रेवातीर्ण\accentmark{22}{\char"0951}स्साधामादे\accentmark{22}{\char"0951}।\ इन्द्रे\accentmark{22}{\char"0951}सन्तूतूवी\accentmark{27}{\char"1CD2}वा\accentmark{22}{\char"0951}जाः।\ क्षुम\accentmark{27}{\char"1CD2}न्तोयाभिर्मा\accentmark{22}{\char"0951}दे\accentmark{22}{\char"0951}मा\mbox{॥ ९\hspace{0pt}॥} \\
 सो\accentmark{20}{\char"1CF9}माः\accentmark{22}{\char"0951}\kern0.15emपूषा\accentmark{27}{\char"1CD2} चा\accentmark{22}{\char"0951}चेततुः।\ वि\accentmark{27}{\char"1CD2}\kern0.15emश्वा\accentmark{20}{\char"1CF8}सां सुक्षी\accentmark{27}{\char"1CD2}\kern0.15emतीनां\accentmark{27}{\char"1CD2}।\ देवात्रा\accentmark{27}{\char"1CD2} रथ्योरहितः\accentmark{22}{\char"0951}\mbox{॥ १०\hspace{0pt}॥} \\ \mbox{॥ इति चतुर्थः खण्डः॥} \\ 
पा\accentmark{27}{\char"1CD2}न्तामा\accentmark{27}{\char"1CD2}\kern0.15emवोअ\accentmark{27}{\char"1CD2}न्धा\accentmark{22}{\char"0951}सः।\ इ\accentmark{20}{\char"1CF8}न्द्रा\accentmark{22}{\char"0951}माभिप्रा\accentmark{27}{\char"1CD2}गा\accentmark{22}{\char"0951}यत।\ विश्वा\accentmark{27}{\char"1CD2}\kern0.15emसा\accentmark{20}{\char"1CF9}हं\accentmark{22}{\char"0951}\kern0.15emशाता\accentmark{27}{\char"1CD2}क्रातुं।\ मंहीं\accentmark{22}{\char"0951}ष्ठंचऋषाणीना\accentmark{27}{\char"1CD2}म्\mbox{॥ १\hspace{0pt}॥} \\
प्रा\accentmark{27}{\char"1CD2}\kern0.15emवाइ\accentmark{20}{\char"1CF9}न्द्रा\accentmark{22}{\char"0951}\kern0.15emयामा\accentmark{27}{\char"1CD2}दा\accentmark{22}{\char"0951}\kern0.15emनं।\ ह\accentmark{27}{\char"1CD2}रीया\accentmark{22}{\char"0951}श्वायगायता।\ साखा\accentmark{22}{\char"0951}यस्सोमापा\accentmark{22}{\char"0951}व्ने\mbox{॥ २\hspace{0pt}॥} \\
वाया\accentmark{27}{\char"1CD2}मू\accentmark{22}{\char"0951}त्वातादी\accentmark{22}{\char"0951}दर्थाः।\ इ\accentmark{20}{\char"1CF8}न्द्रत्वायन्तस्स\accentmark{22}{\char"0951}खायः।\ क\accentmark{20}{\char"1CF8}ण्वा\accentmark{22}{\char"0951}\kern0.15emउक्थे\accentmark{27}{\char"1CD2}\accentmark{22}{\char"0951}भी\accentmark{22}{\char"0951}\kern0.15emर्ज\accentmark{27}{\char"1CD2}रन्ते\mbox{॥ ३\hspace{0pt}॥} \\
इ\accentmark{20}{\char"1CF8}न्द्राया\accentmark{22}{\char"0951}माद्वा\accentmark{22}{\char"0951}नेसूतं।\ पा\accentmark{27}{\char"1CD2}री\accentmark{22}{\char"0951}ष्टोभन्तूनो\accentmark{27}{\char"1CD2}\kern0.15emगी\accentmark{27}{\char"1CD2}राः।\ अर्का\accentmark{27}{\char"1CD2}\kern0.15emमा\accentmark{27}{\char"1CD2}चन्तूका\accentmark{27}{\char"1CD2}\kern0.15emरा\accentmark{27}{\char"1CD2}वाः\mbox{॥ ४\hspace{0pt}॥} \\
आ\accentmark{27}{\char"1CD2}\kern0.15emय\accentmark{27}{\char"1CD2}न्ताइन्द्रासो\accentmark{27}{\char"1CD2}माः।\ नीपू\accentmark{22}{\char"0951}तोआ\accentmark{20}{\char"1CF9}धी\accentmark{22}{\char"0951}बर्हीषि।\ ए\accentmark{20}{\char"1CF8}ही\accentmark{22}{\char"0951}मास्या\accentmark{27}{\char"1CD2}द्रावा\accentmark{27}{\char"1CD2}\kern0.15emपी\accentmark{27}{\char"1CD2}बा\mbox{॥ ५\hspace{0pt}॥} \\
सुरूपकृ\accentmark{27}{\char"1CD2}त्नूमूता\accentmark{27}{\char"1CD2}ये।\ सूदूघा\accentmark{27}{\char"1CD2}मीवागोदूहे।\ जुहूमा\accentmark{27}{\char"1CD2}सी\accentmark{22}{\char"0951}द्यावी\accentmark{27}{\char"1CD2}द्यवि\mbox{॥ ६\hspace{0pt}॥} \\
आभित्वावृषभासूते\accentmark{27}{\char"1CD2}\kern0.15em।\ सूतं\accentmark{27}{\char"1CD2}सृजामीपीताये\accentmark{22}{\char"0951}\kern0.15em।\ तृ\accentmark{27}{\char"1CD2}\kern0.15emम्पा\accentmark{27}{\char"1CD2}व्याश्नूहीमा\accentmark{27}{\char"1CD2}\kern0.15emदम्\accentmark{27}{\char"1CD2}\mbox{॥ ७\hspace{0pt}॥} \\
या\accentmark{27}{\char"1CD2}\kern0.15emई\accentmark{27}{\char"1CD2}न्द्रचा\accentmark{27}{\char"1CD2}मा\accentmark{22}{\char"0951}स्सेष्वा।\ सो\accentmark{20}{\char"1CF8}मा\accentmark{22}{\char"0951}श्चामू\accentmark{27}{\char"1CD2}षू\accentmark{22}{\char"0951}तेसूतः।\ पीबे\accentmark{27}{\char"1CD2}द\accentmark{22}{\char"0951}स्यत्वा\accentmark{27}{\char"1CD2}मी\accentmark{22}{\char"0951}शिषे\mbox{॥ ८\hspace{0pt}॥} \\
यो\accentmark{27}{\char"1CD2}गे\accentmark{22}{\char"0951}योगेतावास्ता\accentmark{22}{\char"0951}रं।\ वाजेवा\accentmark{27}{\char"1CD2}जे\accentmark{22}{\char"0951}हवामाहे।\ साखा\accentmark{22}{\char"0951}याइन्द्रामूता\accentmark{27}{\char"1CD2}ये\mbox{॥ ९\hspace{0pt}॥} \\
 आ\accentmark{20}{\char"1CF8} त्वे\accentmark{22}{\char"0951}तानीषी\accentmark{22}{\char"0951}दत।\ इन्द्रा\accentmark{22}{\char"0951}माभि प्रा\accentmark{27}{\char"1CD2}गा\accentmark{22}{\char"0951}यता।\ सा\accentmark{20}{\char"1CF8}खा\accentmark{22}{\char"0951}य स्तो\accentmark{27}{\char"1CD2}मा\accentmark{22}{\char"0951}वाहसः \mbox{॥ १०\hspace{0pt}॥} \\ \mbox{॥ इति पञ्चमः खण्डः\hspace{0pt}॥} \\   
ईदं\accentmark{27}{\char"1CD2}ह्यन्वोज\accentmark{22}{\char"0951}सा।\ सूतं\accentmark{27}{\char"1CD2}रा\accentmark{22}{\char"0951}धानांपते।\ पीबा\accentmark{27}{\char"1CD2}त्वाटट्यस्या\accentmark{27}{\char"1CD2}गी\accentmark{22}{\char"0951}र्वणः\mbox{॥ १\hspace{0pt}॥} \\माहं\accentmark{20}{\char"1CF9}इ\accentmark{20}{\char"1CF9}न्द्रा\accentmark{22}{\char"0951}पूराश्चा\accentmark{22}{\char"0951}नः।\ महित्वामा\accentmark{27}{\char"1CD2}\kern0.15emस्तू\accentmark{27}{\char"1CD2}\kern0.15emवज्री\accentmark{27}{\char"1CD2}णे\accentmark{22}{\char"0951}।\ 
द्यौर्ना\accentmark{27}{\char"1CD2}\kern0.15emप्रा\accentmark{20}{\char"1CF9}थी\accentmark{22}{\char"0951}\kern0.15emनाशा\accentmark{27}{\char"1CD2}वाः\mbox{॥ २\hspace{0pt}॥} \\
आ\accentmark{27}{\char"1CD2}तूना\accentmark{22}{\char"0951}इन्द्राक्षूम\accentmark{27}{\char"1CD2}न्तं\accentmark{22}{\char"0951}।\ चित्रं\accentmark{27}{\char"1CD2}रा\accentmark{22}{\char"0951}\kern0.15emभंसं\accentmark{27}{\char"1CD2}गृ\accentmark{22}{\char"0951}भाय।\ माहा\accentmark{27}{\char"1CD2}हास्ती\accentmark{27}{\char"1CD2}\kern0.15emद\accentmark{27}{\char"1CD2}\kern0.15emक्षी\accentmark{27}{\char"1CD2}णेन\mbox{॥ ३\hspace{0pt}॥} \\
आभिप्रा\accentmark{20}{\char"1CF9}गो\accentmark{20}{\char"1CF9}पा\accentmark{22}{\char"0951}तिङ्गीराः\accentmark{22}{\char"0951}।\ इन्द्रामश्चाया\accentmark{20}{\char"1CF9}था\accentmark{22}{\char"0951}वीदे।\ 
सूनूंसत्या\accentmark{27}{\char"1CD2}स्यासा\accentmark{27}{\char"1CD2}त्पा\accentmark{22}{\char"0951}तिम्\mbox{॥ ४\hspace{0pt}॥} \\
का\accentmark{27}{\char"1CD2}या\accentmark{22}{\char"0951}नश्चित्रआ\accentmark{27}{\char"1CD2}भु\accentmark{22}{\char"0951}वात्।\ ऊतीस्सा\accentmark{20}{\char"1CF9}दावृ\accentmark{22}{\char"0951}\kern0.15emधस्सा\accentmark{27}{\char"1CD2}खा\accentmark{22}{\char"0951}\kern0.15em।\ का\accentmark{27}{\char"1CD2}\kern0.15emया\accentmark{27}{\char"1CD2}शाची\accentmark{22}{\char"0951}ष्ठयावृतः\mbox{॥ ५\hspace{0pt}॥} \\
त्या\accentmark{27}{\char"1CD2}मूवस्सत्रासा\accentmark{27}{\char"1CD2}हं\accentmark{22}{\char"0951}।\ विश्वा\accentmark{22}{\char"0951}सूगीरिष्वा\accentmark{27}{\char"1CD2}यातम्।\ 
आच्या\accentmark{22}{\char"0951}वयास्यूता\accentmark{27}{\char"1CD2}ये\mbox{॥ ६\hspace{0pt}॥} \\
साद\accentmark{22}{\char"0951}सस्पातीमा\accentmark{27}{\char"1CD2}त्भू\accentmark{22}{\char"0951}तम्।\ प्री\accentmark{20}{\char"1CF9}य\accentmark{20}{\char"1CF9}मि\accentmark{20}{\char"1CF9}न्द्रा\accentmark{22}{\char"0951}स्याकाम्यं\accentmark{22}{\char"0951}।\ 
सा\accentmark{22}{\char"0951}निंमेधा\accentmark{27}{\char"1CD2}मा\accentmark{22}{\char"0951}\kern0.15emया\accentmark{27}{\char"1CD2}सिषम्\mbox{॥ ७\hspace{0pt}॥} \\
एतेप\accentmark{20}{\char"1CF9}न्थाआथो\accentmark{22}{\char"0951}दीवः।\ एभीर्व्या\accentmark{27}{\char"1CD2}श्वामै\accentmark{27}{\char"1CD2}राया।\ ऊता\accentmark{27}{\char"1CD2}श्रो\accentmark{22}{\char"0951}षन्तूनोभुवाः\mbox{॥ ८\hspace{0pt}॥} \\
भ\accentmark{27}{\char"1CD2}\kern0.15emद्रं\accentmark{27}{\char"1CD2}भ\accentmark{22}{\char"0951}द्रन्नआभा\accentmark{22}{\char"0951}\kern0.15emरा।\ ई\accentmark{27}{\char"1CD2}ळामूर्जं\accentmark{22}{\char"0951}शतक्रतो।\ यादिन्द्रा\accentmark{22}{\char"0951}\kern0.15emमृढ\accentmark{27}{\char"1CD2}या\accentmark{22}{\char"0951}सिनः\mbox{॥ ९\hspace{0pt}॥} \\
अस्तीसो\accentmark{27}{\char"1CD2}\kern0.15emमोआयं\accentmark{27}{\char"1CD2}सूतः।\ पी\accentmark{27}{\char"1CD2}\kern0.15emब\accentmark{27}{\char"1CD2}न्त्यस्या\accentmark{27}{\char"1CD2}मारुताः\accentmark{20}{\char"1CF8}।\ उतस्वारो\accentmark{20}{\char"1CF9}जो\accentmark{22}{\char"0951}\kern0.15emअ\accentmark{27}{\char"1CD2}श्वीनाः\mbox{॥ १०\hspace{0pt}॥} \\
 \mbox{॥ इति षष्ठः खण्डः\hspace{0pt}॥} \\ 
ईं\accentmark{27}{\char"1CD2}\kern0.15emख्या\accentmark{27}{\char"1CD2}यन्तीरापस्यू\accentmark{27}{\char"1CD2}वाः\accentmark{22}{\char"0951}\kern0.15em।\ इ\accentmark{27}{\char"1CD2}न्द्रं\accentmark{22}{\char"0951}जातामू\accentmark{27}{\char"1CD2}पासते।\ वन्वाना\accentmark{20}{\char"1CF9}साः\accentmark{22}{\char"0951}\kern0.15emसू\accentmark{27}{\char"1CD2}वी\accentmark{22}{\char"0951}र्यम्\mbox{॥ १\hspace{0pt}॥} \\
ना\accentmark{27}{\char"1CD2}की\accentmark{22}{\char"0951}देवाइनीमसि।\ न\accentmark{27}{\char"1CD2}\kern0.15emक्या\accentmark{27}{\char"1CD2}यो\accentmark{22}{\char"0951}पयामसि।\ 
म\accentmark{27}{\char"1CD2}न्द्रश्रु\accentmark{27}{\char"1CD2}त्य\accentmark{22}{\char"0951}ङ्चरामसि\mbox{॥ २\hspace{0pt}॥} \\
दोषो\accentmark{20}{\char"1CF9}आ\accentmark{20}{\char"1CF9}गा\accentmark{22}{\char"0951}\kern0.15emत्ब\accentmark{27}{\char"1CD2}\kern0.15emहा\accentmark{27}{\char"1CD2}त्गा\accentmark{22}{\char"0951}या।\ द्धूमा\accentmark{22}{\char"0951}द्गमाथर्वणा।\ स्तू\accentmark{20}{\char"1CF9}हिदे\accentmark{22}{\char"0951}\kern0.15emवंसा\accentmark{27}{\char"1CD2}वीता\accentmark{22}{\char"0951}रम्\mbox{॥ ३\hspace{0pt}॥} \\
एषो\accentmark{27}{\char"1CD2}\kern0.15emऊषा\accentmark{27}{\char"1CD2}आ\accentmark{22}{\char"0951}पूर्व्या।\ व्यूंठ\accentmark{22}{\char"0951}च्छतीप्रीया\accentmark{27}{\char"1CD2}दीवः।\ स्तू\accentmark{27}{\char"1CD2}षेवा\accentmark{22}{\char"0951}मश्विनाबृहा\accentmark{27}{\char"1CD2}त्\mbox{॥ ४\hspace{0pt}॥} \\
इन्द्रो\accentmark{22}{\char"0951}\kern0.15emदधीचो\accentmark{27}{\char"1CD2}\kern0.15emअस्था\accentmark{27}{\char"1CD2}भिः\accentmark{22}{\char"0951}।\ वृत्रा\accentmark{27}{\char"1CD2}\kern0.15emण्य\accentmark{27}{\char"1CD2}\kern0.15emप्रा\accentmark{27}{\char"1CD2}तिष्कृतः।\ जघा\accentmark{27}{\char"1CD2}ना\accentmark{22}{\char"0951}नावा\accentmark{22}{\char"0951}तीर्नावा\accentmark{22}{\char"0951}\kern0.15em\mbox{॥ ५\hspace{0pt}॥} \\
इ\accentmark{27}{\char"1CD2}न्द्रे\accentmark{27}{\char"1CD2}\kern0.15emहीम\accentmark{27}{\char"1CD2}\kern0.15emत्स्य\accentmark{27}{\char"1CD2}न्धा\accentmark{22}{\char"0951}सः।\ विश्वेभिस्सोमापा\accentmark{27}{\char"1CD2}वा\accentmark{22}{\char"0951}भीः।\ मा\accentmark{20}{\char"1CF9}हं\accentmark{20}{\char"1CF9}आ\accentmark{22}{\char"0951}\kern0.15emभी\accentmark{27}{\char"1CD2}ष्ठीरोज\accentmark{22}{\char"0951}सा\mbox{॥ ६\hspace{0pt}॥} \\
आ\accentmark{27}{\char"1CD2}\kern0.15emतू\accentmark{27}{\char"1CD2}नो\accentmark{22}{\char"0951}इन्द्रवृत्रहन्।\ अस्मा\accentmark{27}{\char"1CD2}\kern0.15emकामा\accentmark{27}{\char"1CD2}र्धामाग\accentmark{22}{\char"0951}ही।\ माहा\accentmark{27}{\char"1CD2}\kern0.15emन्मा\accentmark{20}{\char"1CF9}ही\accentmark{20}{\char"1CF9}भी\accentmark{22}{\char"0951}रुतीभीः\accentmark{22}{\char"0951}\mbox{॥ ७\hspace{0pt}॥} \\
ओजस्ता\accentmark{27}{\char"1CD2}द\accentmark{22}{\char"0951}स्यतिद्विषे।\ ऊ\accentmark{27}{\char"1CD2}\kern0.15emभेया\accentmark{27}{\char"1CD2}त्सामा\accentmark{27}{\char"1CD2}वा\accentmark{22}{\char"0951}र्क्तयात्।\ इन्द्राश्चर्मेवा\accentmark{22}{\char"0951}\kern0.15emरो\accentmark{27}{\char"1CD2}द\accentmark{22}{\char"0951}सी\mbox{॥ ८\hspace{0pt}॥} \\
आया\accentmark{27}{\char"1CD2}मू\accentmark{22}{\char"0951}\kern0.15emतेसा\accentmark{27}{\char"1CD2}मा\accentmark{22}{\char"0951}तसि।\ का\accentmark{27}{\char"1CD2}पो\accentmark{22}{\char"0951}ताइवगर्भा\accentmark{22}{\char"0951}धिं।\ वा\accentmark{27}{\char"1CD2}\kern0.15emचस्ता\accentmark{27}{\char"1CD2}च्चिन्नओहसे\accentmark{27}{\char"1CD2}\mbox{॥ ९\hspace{0pt}॥} \\
वातआ\accentmark{27}{\char"1CD2}वा\accentmark{22}{\char"0951}तूभेषाजं\accentmark{27}{\char"1CD2}।\ शंभूमा\accentmark{27}{\char"1CD2}योभूनोहृदे।\ प्राणआयूंषि\accentmark{22}{\char"0951}तारि\accentmark{22}{\char"0951}षात्\mbox{॥ १०\hspace{0pt}॥} \\
 \mbox{॥ इति सप्तमः खण्डः\hspace{0pt}॥} \\ 
यं\accentmark{20}{\char"1CF8}रा\accentmark{22}{\char"0951}क्ष\accentmark{20}{\char"1CF9}न्तिप्रा\accentmark{27}{\char"1CD2}चे\accentmark{22}{\char"0951}\kern0.15emतसःवा\accentmark{27}{\char"1CD2}रू\accentmark{22}{\char"0951}णोमित्रो\accentmark{20}{\char"1CF9}अर्यामा\accentmark{27}{\char"1CD2}\kern0.15em।\ ना\accentmark{27}{\char"1CD2}कीस्सा\accentmark{27}{\char"1CD2}दभ्यातेज\accentmark{27}{\char"1CD2}नाः\accentmark{22}{\char"0951}\kern0.15em\mbox{॥ १\hspace{0pt}॥} \\
ग\accentmark{27}{\char"1CD2}व्योषू\accentmark{27}{\char"1CD2}णो\accentmark{22}{\char"0951}या\accentmark{20}{\char"1CF9}था\accentmark{22}{\char"0951}पूरा।\ अश्वायो\accentmark{20}{\char"1CF9}तारा\accentmark{22}{\char"0951}\kern0.15emथाया\accentmark{27}{\char"1CD2}।\ वरिवास्या\accentmark{27}{\char"1CD2}\kern0.15emमाहो\accentmark{27}{\char"1CD2}ना\accentmark{22}{\char"0951}म्\mbox{॥ २\hspace{0pt}॥} \\
ईमा\accentmark{27}{\char"1CD2}स्ता\accentmark{22}{\char"0951}इन्द्रापृ\accentmark{27}{\char"1CD2}श्ना\accentmark{22}{\char"0951}यः।\ घृतन्दू\accentmark{22}{\char"0951}\kern0.15emहाता\accentmark{27}{\char"1CD2}\kern0.15emआशी\accentmark{27}{\char"1CD2}रं।\ एनामृ\accentmark{20}{\char"1CF9}ता\accentmark{20}{\char"1CF9}स्या\accentmark{22}{\char"0951}पिष्यूषीः\mbox{॥ ३\hspace{0pt}॥} \\
आ\accentmark{27}{\char"1CD2}\kern0.15emया\accentmark{27}{\char"1CD2}\kern0.15emधीया\accentmark{27}{\char"1CD2}चर्चा\accentmark{22}{\char"0951}गव्याया\accentmark{27}{\char"1CD2}\kern0.15em।\ पू\accentmark{27}{\char"1CD2}रू\accentmark{22}{\char"0951}नामन्पुरूष्ठूत।\ यत्सो\accentmark{27}{\char"1CD2}मे\accentmark{22}{\char"0951}\kern0.15emसोमआ\accentmark{27}{\char"1CD2}भू\accentmark{22}{\char"0951}वः\mbox{॥ ४\hspace{0pt}॥} \\
पवाका\accentmark{27}{\char"1CD2}\kern0.15emनस्सा\accentmark{27}{\char"1CD2}\kern0.15emरा\accentmark{27}{\char"1CD2}स्वती।\ वा\accentmark{27}{\char"1CD2}जे\accentmark{22}{\char"0951}भिर्वाजि\accentmark{22}{\char"0951}नीवति।\ यज्ञां\accentmark{27}{\char"1CD2}वा\accentmark{22}{\char"0951}ष्टूधीया\accentmark{27}{\char"1CD2}वा\accentmark{22}{\char"0951}सुः\mbox{॥ ५\hspace{0pt}॥} \\
का\accentmark{27}{\char"1CD2}\kern0.15emईम\accentmark{20}{\char"1CF9}न्ना\accentmark{20}{\char"1CF9}हू\accentmark{22}{\char"0951}षीष्वा।\ इ\accentmark{27}{\char"1CD2}न्द्रंसो\accentmark{27}{\char"1CD2}\kern0.15emमा\accentmark{27}{\char"1CD2}स्यतर्पयात्।\ सा\accentmark{27}{\char"1CD2}नो\accentmark{22}{\char"0951}वासून्या\accentmark{27}{\char"1CD2}भा\accentmark{22}{\char"0951}रात्\mbox{॥ ६\hspace{0pt}॥} \\
आया\accentmark{22}{\char"0951}हीसूषूमा\accentmark{27}{\char"1CD2}ही\accentmark{22}{\char"0951}\kern0.15emते\accentmark{27}{\char"1CD2}\kern0.15em।\ इ\accentmark{27}{\char"1CD2}न्द्रासो\accentmark{20}{\char"1CF9}मंपीबा\accentmark{27}{\char"1CD2}इमं।\ 
ए\accentmark{27}{\char"1CD2}दंबर्हि\accentmark{20}{\char"1CF9}स्सादोमा\accentmark{27}{\char"1CD2}मा\mbox{॥ ७\hspace{0pt}॥} \\
माहि\accentmark{22}{\char"0951}त्रीणा\accentmark{27}{\char"1CD2}\kern0.15emमा\accentmark{27}{\char"1CD2}वा\accentmark{22}{\char"0951}रस्तु।\ धुक्षंमित्र\accentmark{20}{\char"1CF9}स्या\accentmark{22}{\char"0951}र्यम्णः।\ दूराध\accentmark{27}{\char"1CD2}\kern0.15emऋषंवा\accentmark{27}{\char"1CD2}रूणस्या\mbox{॥ ८\hspace{0pt}॥} \\
त्वा\accentmark{27}{\char"1CD2}वा\accentmark{22}{\char"0951}तःपुरोवसो।\ वा\accentmark{27}{\char"1CD2}\kern0.15emया\accentmark{27}{\char"1CD2}मिन्द्रः\accentmark{22}{\char"0951}प्रणेता।\ अस्मासी\accentmark{22}{\char"0951}स्थातर्हरीणाम्\mbox{॥ ९\hspace{0pt}॥} \\
\mbox{॥ इति अष्ठमः खण्डः\hspace{0pt}॥} \\ 
ऊत्वा\accentmark{22}{\char"0951}मन्ततूसो\accentmark{27}{\char"1CD2}माः।\ कृणुष्वाराथोअद्रिवः।\ आ\accentmark{27}{\char"1CD2}व\accentmark{22}{\char"0951}ब्रह्मद्दी\accentmark{27}{\char"1CD2}षो\accentmark{22}{\char"0951}जहि\mbox{॥ १\hspace{0pt}॥} \\
गिर्वा\accentmark{22}{\char"0951}णःपाही\accentmark{20}{\char"1CF9}ना\accentmark{22}{\char"0951}स्सूतं\accentmark{27}{\char"1CD2}।\ माधोर्धारा\accentmark{22}{\char"0951}भीरज्यसे।\ इन्द्रत्वा\accentmark{20}{\char"1CF9}दा\accentmark{22}{\char"0951}तामिद्या\accentmark{20}{\char"1CF9}शाः\accentmark{22}{\char"0951}\mbox{॥ २\hspace{0pt}॥} \\
सादा\accentmark{20}{\char"1CF9}वा\accentmark{22}{\char"0951}इन्द्रश्चकृ\accentmark{27}{\char"1CD2}षात्।\ ऊ\accentmark{27}{\char"1CD2}\kern0.15emपोनू\accentmark{20}{\char"1CF9}सा\accentmark{20}{\char"1CF9}सा\accentmark{22}{\char"0951}पर्यन्।\ ना\accentmark{27}{\char"1CD2}\kern0.15emदे\accentmark{27}{\char"1CD2}\kern0.15emवा\accentmark{27}{\char"1CD2}ऋतश्शूराइन्द्राः\mbox{॥ ३\hspace{0pt}॥} \\आ\accentmark{27}{\char"1CD2}त्वा\accentmark{22}{\char"0951}वीशन्तू\accentmark{22}{\char"0951}विन्दा\accentmark{22}{\char"0951}वः।\ समुद्रा\accentmark{20}{\char"1CF9}मी\accentmark{22}{\char"0951}वासिन्धा\accentmark{22}{\char"0951}वः।\ 
नत्वा\accentmark{22}{\char"0951}मिन्द्राति\accentmark{22}{\char"0951}रिच्यते\mbox{॥ ४\hspace{0pt}॥} \\
इ\accentmark{27}{\char"1CD2}न्द्रा\accentmark{20}{\char"1CF8}मित्गा\accentmark{22}{\char"0951}थीनो\accentmark{22}{\char"0951}बृहात्।\ इ\accentmark{20}{\char"1CF8}न्द्रा\accentmark{22}{\char"0951}मर्केभीरर्कीणाः\accentmark{22}{\char"0951}\kern0.15em।\ इ\accentmark{27}{\char"1CD2}न्द्रंवाणी\accentmark{22}{\char"0951}रनूषत\mbox{॥ ५\hspace{0pt}॥} \\
इन्द्राईषे\accentmark{27}{\char"1CD2}द\accentmark{22}{\char"0951}दातुनः।\ ऋभुक्षा\accentmark{20}{\char"1CF9}णा\accentmark{22}{\char"0951}म्भुंरायिं\accentmark{27}{\char"1CD2}।\ 
वाजीद\accentmark{22}{\char"0951}दातूवाजी\accentmark{27}{\char"1CD2}नम्\mbox{॥ ६\hspace{0pt}॥} \\
इन्द्रोअङ्गा\accentmark{27}{\char"1CD2}\kern0.15emमाहा\accentmark{27}{\char"1CD2}\kern0.15emत्भाय\accentmark{27}{\char"1CD2}म्।\ आभीषाद\accentmark{27}{\char"1CD2}पा\accentmark{22}{\char"0951}चुच्यवा\accentmark{22}{\char"0951}त्।\ सहिस्थीरोवीच\accentmark{22}{\char"0951}र्षणिः\mbox{॥ ७\hspace{0pt}॥} \\
ईमा\accentmark{27}{\char"1CD2}ऊ\accentmark{22}{\char"0951}त्वासूते\accentmark{27}{\char"1CD2}सू\accentmark{22}{\char"0951}ते।\ नक्ष\accentmark{22}{\char"0951}न्ते\accentmark{22}{\char"0951}गिर्वा\accentmark{27}{\char"1CD2}णोगीराः\accentmark{27}{\char"1CD2}\kern0.15em।\ गा\accentmark{20}{\char"1CF8}वो\accentmark{22}{\char"0951}वत्सन्नाधे\accentmark{27}{\char"1CD2}नावाः\accentmark{22}{\char"0951}\mbox{॥ ८\hspace{0pt}॥} \\
इन्द्रानूपूषा\accentmark{20}{\char"1CF9}णा\accentmark{22}{\char"0951}वायं।\ सख्या\accentmark{20}{\char"1CF9}या\accentmark{22}{\char"0951}स्वस्ताये।\ 
हूवे\accentmark{27}{\char"1CD2}\kern0.15emमावा\accentmark{27}{\char"1CD2}ज\accentmark{22}{\char"0951}सातये\mbox{॥ ९\hspace{0pt}॥} \\
ना\accentmark{20}{\char"1CF8}कीन्द्रात्वा\accentmark{27}{\char"1CD2}\kern0.15emदू\accentmark{27}{\char"1CD2}क्ता\accentmark{22}{\char"0951}\kern0.15emरं।\ न\accentmark{27}{\char"1CD2}\kern0.15emज्या\accentmark{27}{\char"1CD2}यो\accentmark{22}{\char"0951}अस्तिवृत्रहन्।\ न\accentmark{27}{\char"1CD2}क्येवंयाथात्व\accentmark{27}{\char"1CD2}म्\mbox{॥ १०\hspace{0pt}॥} \\
 \mbox{॥ इति नवमः खण्डः\hspace{0pt}॥} \\ 
ता\accentmark{20}{\char"1CF9}रा\accentmark{20}{\char"1CF9}णिवोज\accentmark{27}{\char"1CD2}ना\accentmark{22}{\char"0951}नां।\ त्रादं\accentmark{20}{\char"1CF9}वा\accentmark{20}{\char"1CF9}ज\accentmark{22}{\char"0951}स्यागो\accentmark{27}{\char"1CD2}मा\accentmark{22}{\char"0951}तः।\ समाना\accentmark{27}{\char"1CD2}मुप्रा\accentmark{27}{\char"1CD2}शंसिषः\mbox{॥ १\hspace{0pt}॥} \\
आसृ\accentmark{22}{\char"0951}ग्रमिन्द्राते\accentmark{27}{\char"1CD2}गीराः\accentmark{22}{\char"0951}\kern0.15em।\ प्रा\accentmark{27}{\char"1CD2}तीत्वा\accentmark{27}{\char"1CD2}\kern0.15emमू\accentmark{27}{\char"1CD2}दा\accentmark{22}{\char"0951}हासता।\ सा\accentmark{27}{\char"1CD2}\kern0.15emजो\accentmark{27}{\char"1CD2}षा\accentmark{22}{\char"0951}वृषा\accentmark{20}{\char"1CF8}भं\accentmark{27}{\char"1CD2}\kern0.15emपा\accentmark{27}{\char"1CD2}तिम्\mbox{॥ २\hspace{0pt}॥} \\
सुनीथोघासामार्त्याः\accentmark{22}{\char"0951}\kern0.15em।\ य\accentmark{27}{\char"1CD2}\kern0.15emम्मा\accentmark{20}{\char"1CF8}रूतोया\accentmark{20}{\char"1CF9}मर्यामा।\ मित्रःपा\accentmark{27}{\char"1CD2}न्त्यद्रू\accentmark{27}{\char"1CD2}\kern0.15emहोआ\accentmark{27}{\char"1CD2}तीद्वीषाः\accentmark{22}{\char"0951}\kern0.15em\mbox{॥ ३\hspace{0pt}॥} \\य\accentmark{27}{\char"1CD2}द्वीळा\accentmark{20}{\char"1CF9}वि\accentmark{22}{\char"0951}न्द्राया\accentmark{27}{\char"1CD2}त्स्थि\accentmark{22}{\char"0951}रे।\ य\accentmark{20}{\char"1CF8}त्प\accentmark{20}{\char"1CF8}र्शा\accentmark{22}{\char"0951}\kern0.15emनेपा\accentmark{27}{\char"1CD2}रा\accentmark{22}{\char"0951}भृतं।\ वा\accentmark{20}{\char"1CF8}सु\accentmark{22}{\char"0951}स्पार्हं\accentmark{22}{\char"0951}\kern0.15emता\accentmark{27}{\char"1CD2}\kern0.15emदा\accentmark{27}{\char"1CD2}भा\accentmark{22}{\char"0951}रा\mbox{॥ ४\hspace{0pt}॥} \\
श्रूतं\accentmark{27}{\char"1CD2}वो\accentmark{22}{\char"0951}वृत्रा\accentmark{22}{\char"0951}\kern0.15emह\accentmark{27}{\char"1CD2}न्ता\accentmark{22}{\char"0951}मं।\ प्र\accentmark{27}{\char"1CD2}\kern0.15emश\accentmark{27}{\char"1CD2}र्धञ्चर्षो\accentmark{22}{\char"0951}णीनां।\ आ\accentmark{27}{\char"1CD2}\kern0.15emशी\accentmark{27}{\char"1CD2}\kern0.15emषेरा\accentmark{27}{\char"1CD2}धा\accentmark{22}{\char"0951}सेमाहे\mbox{॥ ५\hspace{0pt}॥} \\
आर\accentmark{22}{\char"0951}न्तइन्द्राश्रा\accentmark{27}{\char"1CD2}वा\accentmark{22}{\char"0951}से।\ गमे\accentmark{27}{\char"1CD2}मा\accentmark{22}{\char"0951}शूरत्वा\accentmark{27}{\char"1CD2}वातः।\ 
आ\accentmark{27}{\char"1CD2}रं\accentmark{22}{\char"0951}शाक्रापा\accentmark{27}{\char"1CD2}रेमणि\mbox{॥ ६\hspace{0pt}॥} \\
धानाव\accentmark{22}{\char"0951}न्तंकारंभीणम्।\ आ\accentmark{27}{\char"1CD2}\kern0.15emपू\accentmark{27}{\char"1CD2}पाव\accentmark{22}{\char"0951}न्तामुक्थी\accentmark{27}{\char"1CD2}न\accentmark{22}{\char"0951}म्।\ इ\accentmark{20}{\char"1CF8}न्द्राप्राता\accentmark{27}{\char"1CD2}र्जु\accentmark{22}{\char"0951}षस्वनः\mbox{॥ ७\hspace{0pt}॥} \\
आ\accentmark{27}{\char"1CD2}\kern0.15emपांफे\accentmark{20}{\char"1CF9}ने\accentmark{22}{\char"0951}नानामू\accentmark{22}{\char"0951}चेः।\ शी\accentmark{22}{\char"0951}रा\accentmark{22}{\char"0951}इ\accentmark{22}{\char"0951}न्द्रोद\accentmark{22}{\char"0951}वर्क्तया।\ वि\accentmark{27}{\char"1CD2}\kern0.15emश्वा\accentmark{27}{\char"1CD2}\kern0.15emया\accentmark{20}{\char"1CF9}दजयस्पृधाः\accentmark{22}{\char"0951}\mbox{॥ ८\hspace{0pt}॥} \\
ईमे\accentmark{27}{\char"1CD2}ताइन्द्रासो\accentmark{27}{\char"1CD2}माः।\ सूता\accentmark{27}{\char"1CD2}सोयेचासो\accentmark{27}{\char"1CD2}त्वाः\accentmark{22}{\char"0951}।\ तेषांमत्स्वप्रभूवसो\mbox{॥ ९\hspace{0pt}॥} \\
तुभ्यंसूतासस्सोमाः\accentmark{22}{\char"0951}।\ स्तीर्ण\accentmark{22}{\char"0951}म्बर्ही\accentmark{22}{\char"0951}र्वी\accentmark{22}{\char"0951}भवसो।\ स्तो\accentmark{27}{\char"1CD2}तृभ्या\accentmark{22}{\char"0951}इन्द्रमृडया\mbox{॥ १०\hspace{0pt}॥} \\
\mbox{॥ इति दशमः खण्डः\hspace{0pt}॥} \\ 
आ\accentmark{27}{\char"1CD2}वाइन्द्रङ्कृवींया\accentmark{27}{\char"1CD2}था\accentmark{22}{\char"0951}\kern0.15em।\ वजय\accentmark{27}{\char"1CD2}न्ताश्शा\accentmark{22}{\char"0951}ताक्रातुं।\ मंहीं\accentmark{22}{\char"0951}ष्ठंसिञ्चाइन्दुभिः\mbox{॥ १\hspace{0pt}॥} \\
आताश्चिदिन्द्राना\accentmark{27}{\char"1CD2}\kern0.15emऊ\accentmark{27}{\char"1CD2}पा\accentmark{22}{\char"0951}\kern0.15em।\ आ\accentmark{27}{\char"1CD2}या\accentmark{22}{\char"0951}हीसातावा\accentmark{22}{\char"0951}जया।\ इळा\accentmark{27}{\char"1CD2}\kern0.15emसाहा\accentmark{27}{\char"1CD2}स्रा\accentmark{22}{\char"0951}वाजया\mbox{॥ २\hspace{0pt}॥} \\
आबुन्दं\accentmark{20}{\char"1CF9}वृ\accentmark{20}{\char"1CF9}त्राहाददे।\ जातः\accentmark{20}{\char"1CF9}प्र\accentmark{22}{\char"0951}च्छद्वीमाता\accentmark{27}{\char"1CD2}\kern0.15emरं।\ का\accentmark{27}{\char"1CD2}उग्रःकेहा\accentmark{22}{\char"0951}शृण्विरे\accentmark{27}{\char"1CD2}\mbox{॥ ३\hspace{0pt}॥} \\
बृबदू\accentmark{22}{\char"0951}क्थंहवा\accentmark{22}{\char"0951}महे।\ सृप्रा\accentmark{27}{\char"1CD2}का\accentmark{22}{\char"0951}रस्नामू\accentmark{27}{\char"1CD2}\kern0.15emता\accentmark{27}{\char"1CD2}\kern0.15emये।\ सा\accentmark{20}{\char"1CF8}धाः\accentmark{22}{\char"0951}\kern0.15emकृण्व\accentmark{27}{\char"1CD2}\kern0.15emन्ता\accentmark{27}{\char"1CD2}\kern0.15emमा\accentmark{27}{\char"1CD2}वा\accentmark{22}{\char"0951}से\mbox{॥ ४\hspace{0pt}॥} \\
ऋजु\accentmark{22}{\char"0951}नीतीनोवारू\accentmark{22}{\char"0951}\kern0.15emणः।\ मि\accentmark{27}{\char"1CD2}\kern0.15emत्रो\accentmark{27}{\char"1CD2}\kern0.15emनायति\accentmark{27}{\char"1CD2}विद्वांसः।\ अर्यामा\accentmark{22}{\char"0951}\kern0.15emदेवै\accentmark{27}{\char"1CD2}स्साजो\accentmark{27}{\char"1CD2}षाः\mbox{॥ ५\hspace{0pt}॥} \\
दूरा\accentmark{27}{\char"1CD2}\kern0.15emदीहे\accentmark{27}{\char"1CD2}वायात्सातः।\ आरूणा\accentmark{27}{\char"1CD2}स्पूरा\accentmark{27}{\char"1CD2}शिश्वीतात्।\ विभा\accentmark{20}{\char"1CF9}नूंविश्वाथातनात्\mbox{॥ ६\hspace{0pt}॥} \\
आनो\accentmark{22}{\char"0951}मित्रावरुणा।\ घृतैः\accentmark{27}{\char"1CD2}गव्यूतिमुक्षतं।\ मध्वारा\accentmark{27}{\char"1CD2}जां\accentmark{22}{\char"0951}सिसुक्रतू\mbox{॥ ७\hspace{0pt}॥} \\
ऊ\accentmark{27}{\char"1CD2}दूत्ये\accentmark{27}{\char"1CD2}\kern0.15emसूना\accentmark{27}{\char"1CD2}\kern0.15emवोगी\accentmark{27}{\char"1CD2}राः\accentmark{22}{\char"0951}।\ काष्ठार्यज्ञेष्वात्नता।\ वा\accentmark{27}{\char"1CD2}\kern0.15emश्रा\accentmark{20}{\char"1CF9}आ\accentmark{22}{\char"0951}भीजुया\accentmark{27}{\char"1CD2}ता\accentmark{22}{\char"0951}वे\mbox{॥ ८\hspace{0pt}॥} \\
ईदंविष्णुर्विचा\accentmark{22}{\char"0951}क्रमे।\ त्रेधा\accentmark{27}{\char"1CD2}\kern0.15emनी\accentmark{27}{\char"1CD2}दे\accentmark{22}{\char"0951}थेपादंसा\accentmark{27}{\char"1CD2}मू\accentmark{22}{\char"0951}ढमस्यपांसूले\accentmark{27}{\char"1CD2}\mbox{॥ ९\hspace{0pt}॥} \\
 \mbox{॥ इति एकादशः खण्डः\hspace{0pt}॥} \\ 
आतीहीमन्यूषावीणं।\ सूषूवां\accentmark{27}{\char"1CD2}\kern0.15emसामू\accentmark{27}{\char"1CD2}पै\accentmark{22}{\char"0951}रय।\ अस्या\accentmark{22}{\char"0951}रा\accentmark{20}{\char"1CF9}तौ\accentmark{22}{\char"0951}सूतंपी\accentmark{22}{\char"0951}बा\mbox{॥ १\hspace{0pt}॥} \\
का\accentmark{27}{\char"1CD2}दूप्रा\accentmark{27}{\char"1CD2}चे\accentmark{22}{\char"0951}तसेमहे\accentmark{27}{\char"1CD2}\kern0.15em।\ वा\accentmark{20}{\char"1CF8}चो\accentmark{22}{\char"0951}देवाया\accentmark{22}{\char"0951}शस्यते।\ तदी\accentmark{20}{\char"1CF9}द्ध्या\accentmark{22}{\char"0951}स्यावार्धा\accentmark{27}{\char"1CD2}नम्\mbox{॥ २\hspace{0pt}॥} \\उक्थञ्चाना\accentmark{27}{\char"1CD2}\kern0.15emशस्या\accentmark{27}{\char"1CD2}मा\accentmark{22}{\char"0951}नं।\ ना\accentmark{20}{\char"1CF8}गो\accentmark{22}{\char"0951}\kern0.15emरा\accentmark{27}{\char"1CD2}\kern0.15emयिरा\accentmark{27}{\char"1CD2}ची\accentmark{22}{\char"0951}केता।\ ना\accentmark{20}{\char"1CF8}गा\accentmark{27}{\char"1CD2}यत्रं\accentmark{22}{\char"0951}गीयामानम्\mbox{॥ ३\hspace{0pt}॥} \\
इ\accentmark{20}{\char"1CF8}न्द्रा\accentmark{22}{\char"0951}\kern0.15emउक्थे\accentmark{27}{\char"1CD2}भिर्म\accentmark{22}{\char"0951}न्दीष्ठः।\ वा\accentmark{27}{\char"1CD2}जा\accentmark{22}{\char"0951}नांचावा\accentmark{22}{\char"0951}\kern0.15emज\accentmark{27}{\char"1CD2}प\accentmark{22}{\char"0951}तिः।\ हा\accentmark{27}{\char"1CD2}रीवा\accentmark{22}{\char"0951}न्सूतानांसाखा\accentmark{22}{\char"0951}\mbox{॥ ४\hspace{0pt}॥} \\
ब्र\accentmark{27}{\char"1CD2}ह्मा\accentmark{22}{\char"0951}णादीन्द्रारा\accentmark{22}{\char"0951}धा\accentmark{22}{\char"0951}साः।\ पी\accentmark{27}{\char"1CD2}\kern0.15emबा\accentmark{20}{\char"1CF9}सो\accentmark{20}{\char"1CF9}मा\accentmark{22}{\char"0951}मृतूंरा\accentmark{27}{\char"1CD2}नू।\ तवे\accentmark{20}{\char"1CF9}दु\accentmark{22}{\char"0951}\kern0.15emसख्या\accentmark{27}{\char"1CD2}मस्तृतम्\mbox{॥ ५\hspace{0pt}॥} \\
वा\accentmark{20}{\char"1CF8}यङ्घा\accentmark{22}{\char"0951}तेआपी\accentmark{22}{\char"0951}स्मसि।\ स्तोता\accentmark{27}{\char"1CD2}र\accentmark{22}{\char"0951}इन्द्र\accentmark{20}{\char"1CF8}गिर्वणः।\ त्व\accentmark{27}{\char"1CD2}न्नोजिन्वसोमपः\mbox{॥ ६\hspace{0pt}॥} \\
आ\accentmark{22}{\char"0951}याह्यू\accentmark{27}{\char"1CD2}पा\accentmark{22}{\char"0951}नास्सू\accentmark{27}{\char"1CD2}\kern0.15emतं।\ वा\accentmark{20}{\char"1CF8}जेभिर्मा\accentmark{22}{\char"0951}हृ\accentmark{22}{\char"0951}णीयथा।\ माहं\accentmark{20}{\char"1CF9}ईवा\accentmark{22}{\char"0951}\kern0.15emयूवा\accentmark{27}{\char"1CD2}जानि\mbox{॥ ७\hspace{0pt}॥} \\
का\accentmark{20}{\char"1CF8}दा\accentmark{27}{\char"1CD2}वासोस्तोत्रं\accentmark{20}{\char"1CF9}ह\accentmark{20}{\char"1CF9}र्याताआ।\ आव\accentmark{22}{\char"0951}श्मशा\accentmark{20}{\char"1CF9}रूधाद्वाः\accentmark{27}{\char"1CD2}।\ दीर्घं\accentmark{27}{\char"1CD2}\kern0.15emसूतं\accentmark{27}{\char"1CD2}\kern0.15emवा\accentmark{27}{\char"1CD2}\kern0.15emता\accentmark{27}{\char"1CD2}प्या\accentmark{22}{\char"0951}या\mbox{॥ ८\hspace{0pt}॥} \\
ए\accentmark{20}{\char"1CF8}न्द्रा\accentmark{22}{\char"0951}पृक्षूका\accentmark{22}{\char"0951}सूचीत्।\ नृम्णं\accentmark{27}{\char"1CD2}\kern0.15emतानू\accentmark{27}{\char"1CD2}षू\accentmark{22}{\char"0951}धेहिनः।\ स\accentmark{27}{\char"1CD2}\kern0.15emत्रा\accentmark{27}{\char"1CD2}जिदुग्रापौंठ\accentmark{22}{\char"0951}स्यम्\mbox{॥ ९\hspace{0pt}॥} \\
आ\accentmark{27}{\char"1CD2}यामेनंसचतां\accentmark{27}{\char"1CD2}\kern0.15emसुतः\accentmark{27}{\char"1CD2}।\ मन्दी\accentmark{27}{\char"1CD2}मिन्द्रा\accentmark{22}{\char"0951}यामन्दी\accentmark{27}{\char"1CD2}\kern0.15emने।\ च\accentmark{27}{\char"1CD2}क्रिंवि\accentmark{22}{\char"0951}श्वा\accentmark{22}{\char"0951}नी\accentmark{22}{\char"0951}चाक्रा\accentmark{22}{\char"0951}ये\mbox{॥ १०\hspace{0pt}॥} \\
\mbox{॥ इति द्वादशः खण्डः\hspace{0pt}॥} \\   
\mbox{॥ इति तद्वपाठः समाप्तः\hspace{0pt}॥} \\ \clearpage
\mbox{॥ अथ बृहतीपाठः\hspace{0pt}॥} \\
आभित्वा\accentmark{27}{\char"1CD2}\kern0.15emशू\accentmark{27}{\char"1CD2}र\accentmark{22}{\char"0951}नोनुमः।\ आदू\accentmark{20}{\char"1CF8}ग्धाइवाधेना\accentmark{27}{\char"1CD2}\kern0.15emवाः।\ ई\accentmark{27}{\char"1CD2}शा\accentmark{22}{\char"0951}\kern0.15emना\accentmark{27}{\char"1CD2}मा\accentmark{22}{\char"0951}स्याजग\accentmark{22}{\char"0951}तःस्वदृशं।\ ईशा\accentmark{22}{\char"0951}नमिन्द्रातस्थू\accentmark{27}{\char"1CD2}षाः\accentmark{22}{\char"0951}\mbox{॥ १\hspace{0pt}॥} \\त्वा\accentmark{27}{\char"1CD2}\kern0.15emमि\accentmark{27}{\char"1CD2}\kern0.15emद्धि\accentmark{27}{\char"1CD2}\kern0.15emहा\accentmark{27}{\char"1CD2}वा\accentmark{22}{\char"0951}महे\accentmark{22}{\char"0951}\kern0.15em।\ सा\accentmark{27}{\char"1CD2}\kern0.15emतौवा\accentmark{27}{\char"1CD2}ज\accentmark{22}{\char"0951}स्याकारा\accentmark{27}{\char"1CD2}वाः\accentmark{22}{\char"0951}।\ त्वांवृ\accentmark{22}{\char"0951}\kern0.15emत्रे\accentmark{27}{\char"1CD2}ष्विन्द्रा\accentmark{22}{\char"0951}सात्पातिन्ना\accentmark{27}{\char"1CD2}राः\accentmark{22}{\char"0951}।\ त्वांकाष्ठास्व\accentmark{22}{\char"0951}र्वातः\mbox{॥ २\hspace{0pt}॥} \\
आभी\accentmark{20}{\char"1CF9}प्रा\accentmark{20}{\char"1CF9}वास्सूरा\accentmark{27}{\char"1CD2}धा\accentmark{22}{\char"0951}सम्।\ इन्द्रा\accentmark{22}{\char"0951}मर्चाया\accentmark{20}{\char"1CF9}था\accentmark{22}{\char"0951}वीदे।\ योज\accentmark{22}{\char"0951}रितृ\accentmark{20}{\char"1CF9}भ्यो\accentmark{22}{\char"0951}\kern0.15emमाघा\accentmark{27}{\char"1CD2}वा\accentmark{22}{\char"0951}पूरो\accentmark{20}{\char"1CF8}वासूः\accentmark{20}{\char"1CF9}।\ सा\accentmark{27}{\char"1CD2}हा\accentmark{22}{\char"0951}स्रेणेवाशी\accentmark{27}{\char"1CD2}क्षा\accentmark{22}{\char"0951}ति\mbox{॥ ३\hspace{0pt}॥} \\
तंवो\accentmark{22}{\char"0951}दस्मा\accentmark{20}{\char"1CF9}मृ\accentmark{22}{\char"0951}तीषाहम्।\ वा\accentmark{27}{\char"1CD2}सो\accentmark{22}{\char"0951}र्मन्दानमन्धा\accentmark{22}{\char"0951}\kern0.15emसा।\ आ\accentmark{27}{\char"1CD2}\kern0.15emभी\accentmark{27}{\char"1CD2}\kern0.15emवत्स\accentmark{27}{\char"1CD2}न्नस्वा\accentmark{27}{\char"1CD2}सारेषुधेना\accentmark{27}{\char"1CD2}वाः\accentmark{22}{\char"0951}।\ इ\accentmark{20}{\char"1CF8}न्द्रं\accentmark{22}{\char"0951}गीर्भि\accentmark{22}{\char"0951}र्ना\accentmark{22}{\char"0951}वामहे\mbox{॥ ४\hspace{0pt}॥} \\
ता\accentmark{27}{\char"1CD2}रो\accentmark{22}{\char"0951}भिर्वोवीद\accentmark{27}{\char"1CD2}द्वा\accentmark{22}{\char"0951}सुम्।\ इन्द्रंसाबा\accentmark{20}{\char"1CF9}धा\accentmark{22}{\char"0951}\kern0.15emऊता\accentmark{27}{\char"1CD2}ये।\ बृह\accentmark{27}{\char"1CD2}\kern0.15emत्गा\accentmark{27}{\char"1CD2}यन्तास्सूता\accentmark{27}{\char"1CD2}\kern0.15emसो\accentmark{27}{\char"1CD2}मेआध्वारे।\ हूवे\accentmark{27}{\char"1CD2}भारन्नाकारीण\accentmark{22}{\char"0951}म्\mbox{॥ ५\hspace{0pt}॥} \\ता\accentmark{27}{\char"1CD2}\kern0.15emरा\accentmark{27}{\char"1CD2}णीरित्सी\accentmark{27}{\char"1CD2}षासती।\ वाजं\accentmark{27}{\char"1CD2}\kern0.15emपू\accentmark{27}{\char"1CD2}रन्ध्यायूजा।\ आ\accentmark{27}{\char"1CD2}वाइन्द्रं\accentmark{22}{\char"0951}पूरुहूत\accentmark{27}{\char"1CD2}न्ना\accentmark{22}{\char"0951}मेगीरा\accentmark{27}{\char"1CD2}\kern0.15em।\ नेमि\accentmark{27}{\char"1CD2}न्तष्टे\accentmark{22}{\char"0951}वासुद्रू
व\accentmark{22}{\char"0951}म्\mbox{॥ ६\hspace{0pt}॥} \\
पीबा\accentmark{20}{\char"1CF8}सू\accentmark{20}{\char"1CF9}तस्या\accentmark{22}{\char"0951}\kern0.15emरासी\accentmark{27}{\char"1CD2}नाः।\ मत्स्वा\accentmark{22}{\char"0951}नइन्द्रागो\accentmark{27}{\char"1CD2}मा\accentmark{22}{\char"0951}तः।\ आपिर्नोबोधीसा\accentmark{27}{\char"1CD2}धा\accentmark{22}{\char"0951}मा\accentmark{20}{\char"1CF9}द्ये\accentmark{22}{\char"0951}\kern0.15emवृधे\accentmark{27}{\char"1CD2}।\ अस्मंआ\accentmark{22}{\char"0951}वन्तूते\accentmark{27}{\char"1CD2}\kern0.15emधी\accentmark{27}{\char"1CD2}याः\accentmark{22}{\char"0951}\mbox{॥ ७\hspace{0pt}॥} \\त्वं\accentmark{27}{\char"1CD2}\kern0.15emह्ये\accentmark{27}{\char"1CD2}हीचेरा\accentmark{27}{\char"1CD2}वे।\ वीदा\accentmark{20}{\char"1CF9}भा\accentmark{22}{\char"0951}\kern0.15emगंवा\accentmark{27}{\char"1CD2}सू\accentmark{22}{\char"0951}क्तये।\ उद्वा\accentmark{22}{\char"0951}वृष\accentmark{22}{\char"0951}स्वमा\accentmark{22}{\char"0951}घावन्ग\accentmark{22}{\char"0951}\kern0.15emवी\accentmark{27}{\char"1CD2}ष्टये।\ उदि\accentmark{27}{\char"1CD2}न्द्रो\accentmark{22}{\char"0951}अश्वा\accentmark{22}{\char"0951}मिष्ठ\accentmark{22}{\char"0951}ये\mbox{॥ ८\hspace{0pt}॥} \\
ना\accentmark{27}{\char"1CD2}\kern0.15emही\accentmark{27}{\char"1CD2}वा\accentmark{22}{\char"0951}श्वाराम\accentmark{27}{\char"1CD2}ञ्चाना\accentmark{27}{\char"1CD2}।\ वासी\accentmark{22}{\char"0951}ष्ठःपारीमं\accentmark{27}{\char"1CD2}सा\accentmark{22}{\char"0951}ते।\ अस्माका\accentmark{22}{\char"0951}माद्यामारूतस्सूते\accentmark{27}{\char"1CD2}\kern0.15emसा\accentmark{27}{\char"1CD2}चा\accentmark{22}{\char"0951}।\ विश्वे\accentmark{22}{\char"0951}पिबन्तूका\accentmark{27}{\char"1CD2}\kern0.15emमी\accentmark{27}{\char"1CD2}नाः\accentmark{22}{\char"0951}\mbox{॥ ९\hspace{0pt}॥} \\माची\accentmark{22}{\char"0951}दन्यद्वी\accentmark{27}{\char"1CD2}शंसता।\ सा\accentmark{20}{\char"1CF8}खायोमा\accentmark{27}{\char"1CD2}रि\accentmark{22}{\char"0951}षण्यत।\ इ\accentmark{20}{\char"1CF8}न्द्रामित्स्तो\accentmark{20}{\char"1CF8}ता\accentmark{20}{\char"1CF8}वृषाणंसा\accentmark{20}{\char"1CF9}चा\accentmark{22}{\char"0951}\kern0.15emसूते\accentmark{27}{\char"1CD2}।\ मुहू\accentmark{22}{\char"0951}रुत्थाचाशंसत\mbox{॥ १०\hspace{0pt}॥} \\
\mbox{॥ इति प्रथमः खण्डः\hspace{0pt}॥} \\ ना\accentmark{20}{\char"1CF8}की\accentmark{20}{\char"1CF8}ष्टंकर्माणानशात् ।\  यश्चाका\accentmark{20}{\char"1CF8}रा\accentmark{22}{\char"0951}\kern0.15emस्सा\accentmark{27}{\char"1CD2}दावृधं ।\  इ\accentmark{27}{\char"1CD2}न्द्रन्ना य\accentmark{20}{\char"1CF9}ज्ञैर्वि\accentmark{20}{\char"1CF9}श्वा\accentmark{20}{\char"1CF9}र्गू\accentmark{22}{\char"0951}र्तामृ\accentmark{22}{\char"0951}भ्वा\accentmark{22}{\char"0951}सम् ।\  आधृष्ट\accentmark{27}{\char"1CD2}न्धृष्णु\accentmark{27}{\char"1CD2}मोज\accentmark{22}{\char"0951}सा \mbox{॥ १\hspace{0pt}॥} \\ या\accentmark{22}{\char"0951}\kern0.15emऋते\accentmark{27}{\char"1CD2}चीदभिश्री\accentmark{27}{\char"1CD2}षाः\accentmark{22}{\char"0951} ।\  पूरा\accentmark{27}{\char"1CD2} जत्रु\accentmark{20}{\char"1CF9}भ्या\accentmark{22}{\char"0951}\kern0.15emआतृ\accentmark{27}{\char"1CD2}दाः\accentmark{22}{\char"0951}\kern0.15em ।\  
स\accentmark{27}{\char"1CD2}न्धा\accentmark{22}{\char"0951}तासंधींमा\accentmark{22}{\char"0951}घावा\accentmark{22}{\char"0951}पूरोवा\accentmark{27}{\char"1CD2}सूः।\ निष्क\accentmark{22}{\char"0951}र्ता\accentmark{22}{\char"0951}वी\accentmark{20}{\char"1CF9}ह्रूतंपू\accentmark{27}{\char"1CD2}नाः\mbox{॥ २\hspace{0pt}॥} \\
आ\accentmark{20}{\char"1CF8}त्वासाहस्रा\accentmark{22}{\char"0951}\kern0.15emमाशात\accentmark{27}{\char"1CD2}म्।\ युक्ता\accentmark{27}{\char"1CD2}राथेहीरण्याये\accentmark{22}{\char"0951}।\ 
ब्रह्मायू\accentmark{27}{\char"1CD2}\kern0.15emजोहा\accentmark{27}{\char"1CD2}रा\accentmark{22}{\char"0951}यिन्द्राकेशी\accentmark{27}{\char"1CD2}नाः।\ वाहन्तूसो\accentmark{27}{\char"1CD2}मापीतये\mbox{॥ ३\hspace{0pt}॥} \\आमन्द्रै\accentmark{20}{\char"1CF9}रि\accentmark{22}{\char"0951}न्द्राहा\accentmark{27}{\char"1CD2}री\accentmark{22}{\char"0951}भिः।\ याही\accentmark{27}{\char"1CD2}मा\accentmark{22}{\char"0951}यूरा\accentmark{22}{\char"0951}\kern0.15emरोम\accentmark{27}{\char"1CD2}भिः।\ मा\accentmark{27}{\char"1CD2}त्वा\accentmark{22}{\char"0951}केचिन्नी\accentmark{20}{\char"1CF9}ये\accentmark{22}{\char"0951}मुर्वि\accentmark{27}{\char"1CD2}\kern0.15emन्ना\accentmark{27}{\char"1CD2}\kern0.15emपाशी\accentmark{27}{\char"1CD2}नाः\accentmark{22}{\char"0951}।\ आतीधन्वेवतंई\accentmark{22}{\char"0951}हि\mbox{॥ ४\hspace{0pt}॥} \\
त्वा\accentmark{27}{\char"1CD2}मंगप्रा\accentmark{27}{\char"1CD2}शं\accentmark{22}{\char"0951}सिषः।\ देवा\accentmark{27}{\char"1CD2}श्शा\accentmark{22}{\char"0951}वीष्ठमा\accentmark{27}{\char"1CD2}र्क्त्यम्।\ न\accentmark{20}{\char"1CF8}त्वा\accentmark{22}{\char"0951}\kern0.15emदन्यो\accentmark{27}{\char"1CD2}मा\accentmark{22}{\char"0951}घावन्नस्तिमार्डीत।\ इन्द्रब्रा\accentmark{27}{\char"1CD2}वी\accentmark{22}{\char"0951}मीतेवा\accentmark{27}{\char"1CD2}चाः\accentmark{22}{\char"0951}\mbox{॥ ५\hspace{0pt}॥} \\त्वा\accentmark{27}{\char"1CD2}\kern0.15emमि\accentmark{27}{\char"1CD2}न्द्रायाशा\accentmark{27}{\char"1CD2}आ\accentmark{22}{\char"0951}सि।\ ऋजीषीशावासस्पा\accentmark{27}{\char"1CD2}तिः।\ त्वंवृत्रा\accentmark{27}{\char"1CD2}णी\accentmark{22}{\char"0951}हंस्यप्रातीन्ये\accentmark{27}{\char"1CD2}\kern0.15emकाई\accentmark{27}{\char"1CD2}त्पूरूः\accentmark{27}{\char"1CD2}।\ आनूक्तस्चर्षाणिधृ\accentmark{27}{\char"1CD2}त्तिः\mbox{॥ ६\hspace{0pt}॥} \\इ\accentmark{27}{\char"1CD2}न्द्रामीद्देवातातये।\ इन्द्रं\accentmark{22}{\char"0951}प्रायत्याध्वारे\accentmark{27}{\char"1CD2}।\ 
इन्द्रंसामीके\accentmark{27}{\char"1CD2}\kern0.15emवानी\accentmark{27}{\char"1CD2}नो\accentmark{22}{\char"0951}हवामहे।\ इ\accentmark{27}{\char"1CD2}न्द्र\accentmark{22}{\char"0951}न्धानास्यासा\accentmark{27}{\char"1CD2}\kern0.15emता\accentmark{27}{\char"1CD2}ये\accentmark{22}{\char"0951}\mbox{॥ ७\hspace{0pt}॥} \\ईमा\accentmark{27}{\char"1CD2}ऊ\accentmark{22}{\char"0951}त्वापूरोवसो।\ गी\accentmark{27}{\char"1CD2}रो\accentmark{22}{\char"0951}वर्धन्तूया\accentmark{27}{\char"1CD2}\kern0.15emमा\accentmark{27}{\char"1CD2}मा\accentmark{22}{\char"0951}।\ पवाका\accentmark{20}{\char"1CF9}व\accentmark{22}{\char"0951}र्णश्शू\accentmark{27}{\char"1CD2}\kern0.15emचा\accentmark{27}{\char"1CD2}\kern0.15emयोवीप\accentmark{27}{\char"1CD2}\kern0.15emश्ची\accentmark{27}{\char"1CD2}\kern0.15emताः\accentmark{27}{\char"1CD2}\kern0.15em।\ आभी\accentmark{27}{\char"1CD2}स्तोमै\accentmark{22}{\char"0951}रनूषत\mbox{॥ ८\hspace{0pt}॥} \\
ऊ\accentmark{27}{\char"1CD2}दूत्येमाधू\accentmark{22}{\char"0951}मक्तमः।\ गीरस्तोमा\accentmark{22}{\char"0951}सैरते।\ स\accentmark{27}{\char"1CD2}त्राजितो\accentmark{22}{\char"0951}धानासाआक्षी\accentmark{22}{\char"0951}तोतयः।\ वजय\accentmark{22}{\char"0951}न्तोरा\accentmark{22}{\char"0951}था\accentmark{22}{\char"0951}इवा\mbox{॥ ९\hspace{0pt}॥} \\
या\accentmark{20}{\char"1CF8}था\accentmark{22}{\char"0951}\kern0.15emगौरो\accentmark{27}{\char"1CD2}\kern0.15emआपा\accentmark{27}{\char"1CD2}कृ\accentmark{22}{\char"0951}\kern0.15emत\accentmark{27}{\char"1CD2}\kern0.15emम्।\ तृ\accentmark{27}{\char"1CD2}ष्यन्ने\accentmark{27}{\char"1CD2}\kern0.15emत्यवे\accentmark{27}{\char"1CD2}\kern0.15emरि\accentmark{27}{\char"1CD2}णम्।\ आपि\accentmark{27}{\char"1CD2}त्वे\accentmark{22}{\char"0951}नप्रापित्वेतू\accentmark{27}{\char"1CD2}\kern0.15emया\accentmark{27}{\char"1CD2}\kern0.15emमा\accentmark{27}{\char"1CD2}ग\accentmark{22}{\char"0951}हि।\ क\accentmark{20}{\char"1CF8}ण्वेषूसू\accentmark{27}{\char"1CD2}\kern0.15emसाचा\accentmark{27}{\char"1CD2}\kern0.15emपी\accentmark{27}{\char"1CD2}बा\mbox{॥ १०\hspace{0pt}॥} \\
\mbox{॥ इति द्वीतीयः खण्डः\hspace{0pt}॥} \\ 
श\accentmark{27}{\char"1CD2}ग्धूटट्यषूशा\accentmark{22}{\char"0951}चीपते।\ इन्द्रावि\accentmark{22}{\char"0951}\kern0.15emश्वा\accentmark{27}{\char"1CD2}\kern0.15emभीरू\accentmark{20}{\char"1CF8}तीभिः।\ भा\accentmark{27}{\char"1CD2}\kern0.15emगन्न\accentmark{20}{\char"1CF9}हित्वा\accentmark{22}{\char"0951}\kern0.15emयाशा\accentmark{27}{\char"1CD2}सं\accentmark{22}{\char"0951}वासूवी\accentmark{27}{\char"1CD2}दम्।\ आनू\accentmark{22}{\char"0951}शूराचा\accentmark{27}{\char"1CD2}रा\accentmark{22}{\char"0951}मसि\mbox{॥ १\hspace{0pt}॥} \\
या\accentmark{20}{\char"1CF8}इ\accentmark{22}{\char"0951}न्द्राभू\accentmark{27}{\char"1CD2}\kern0.15emजआ\accentmark{27}{\char"1CD2}भा\accentmark{22}{\char"0951}रा।\ 
स्वठ\accentmark{22}{\char"0951}र्वंआसू\accentmark{22}{\char"0951}रेभ्यः।\ स्तोता\accentmark{27}{\char"1CD2}रामिन्मा\accentmark{22}{\char"0951}घवन्नस्यावा\accentmark{22}{\char"0951}र्धया।\ ये\accentmark{27}{\char"1CD2}चत्वे\accentmark{22}{\char"0951}वृक्ता\accentmark{27}{\char"1CD2}ब\accentmark{22}{\char"0951}र्हीषः\mbox{॥ २\hspace{0pt}॥} \\प्रा\accentmark{27}{\char"1CD2}मित्रा\accentmark{27}{\char"1CD2}यप्रार्यम्णे\accentmark{22}{\char"0951}\kern0.15em।\ सा\accentmark{27}{\char"1CD2}चात्थ्याठ\accentmark{22}{\char"0951}मृतावसो।\ वरु\accentmark{27}{\char"1CD2}त्थ्येटट्य\accentmark{27}{\char"1CD2}\kern0.15emवा\accentmark{20}{\char"1CF9}रूणेछन्द्यंवा\accentmark{27}{\char"1CD2}चाः\accentmark{22}{\char"0951}।\ स्तोत्रं\accentmark{22}{\char"0951}\kern0.15emरा\accentmark{27}{\char"1CD2}ज\accentmark{22}{\char"0951}सुगायता\mbox{॥ ३\hspace{0pt}॥} \\
आ\accentmark{27}{\char"1CD2}\kern0.15emभी\accentmark{20}{\char"1CF9}त्वा\accentmark{22}{\char"0951}पूर्वापी\accentmark{22}{\char"0951}\kern0.15emतये।\ इ\accentmark{27}{\char"1CD2}न्द्रास्तो\accentmark{22}{\char"0951}मेभीरायावाः।\ समी\accentmark{27}{\char"1CD2}\kern0.15emचीना\accentmark{20}{\char"1CF9}साऋभा\accentmark{27}{\char"1CD2}\kern0.15emवस्सा\accentmark{27}{\char"1CD2}\kern0.15emमा\accentmark{27}{\char"1CD2}स्वरन्।\ रुद्रागृ\accentmark{22}{\char"0951}णन्तापूर्व्यम्\mbox{॥ ४\hspace{0pt}॥} \\प्रा\accentmark{27}{\char"1CD2}\kern0.15emवाइ\accentmark{27}{\char"1CD2}न्द्रा\accentmark{22}{\char"0951}यबृहा\accentmark{22}{\char"0951}ते।\ मारूतोब्र\accentmark{27}{\char"1CD2}ह्मा\accentmark{22}{\char"0951}र्च्चता।\ वृत्रंहा\accentmark{27}{\char"1CD2}नातिवृत्रा\accentmark{27}{\char"1CD2}हाशाता\accentmark{27}{\char"1CD2}क्रा\accentmark{22}{\char"0951}तुः\accentmark{22}{\char"0951}\kern0.15em।\ व\accentmark{27}{\char"1CD2}ज्ज्रे\accentmark{22}{\char"0951}णाशाता\accentmark{27}{\char"1CD2}पा\accentmark{22}{\char"0951}र्वणा\mbox{॥ ५\hspace{0pt}॥} \\
बृहा\accentmark{27}{\char"1CD2}दिन्द्रा\accentmark{22}{\char"0951}यगायत।\ मारू\accentmark{22}{\char"0951}\kern0.15emतोवृ\accentmark{27}{\char"1CD2}त्राहन्ता\accentmark{22}{\char"0951}\kern0.15emमं।\ ये\accentmark{27}{\char"1CD2}\kern0.15emनज्यो\accentmark{27}{\char"1CD2}\kern0.15emतीरा\accentmark{27}{\char"1CD2}ज\accentmark{22}{\char"0951}नयन्नृता\accentmark{27}{\char"1CD2}वृधाः\accentmark{22}{\char"0951}\kern0.15em।\ देव\accentmark{27}{\char"1CD2}न्देवा\accentmark{27}{\char"1CD2}\kern0.15emयाजा\accentmark{27}{\char"1CD2}ग्रवीम्\mbox{॥ ६\hspace{0pt}॥} \\
इन्द्राक्रा\accentmark{20}{\char"1CF9}तुं\accentmark{22}{\char"0951}\kern0.15emनआ\accentmark{27}{\char"1CD2}भा\accentmark{22}{\char"0951}रा।\ पिता\accentmark{27}{\char"1CD2}पुत्रे\accentmark{27}{\char"1CD2}भ्योया\accentmark{27}{\char"1CD2}था।\ शिक्षा\accentmark{22}{\char"0951}णोअस्मिन्पु\accentmark{22}{\char"0951}रुहूताया\accentmark{27}{\char"1CD2}मा\accentmark{22}{\char"0951}\kern0.15emनी।\ जी\accentmark{27}{\char"1CD2}वाज्ज्योती\accentmark{22}{\char"0951}रशीमहि\mbox{॥ ७\hspace{0pt}॥} \\
माना\accentmark{22}{\char"0951}इन्द्रापारा\accentmark{22}{\char"0951}वृणात्।\ भा\accentmark{27}{\char"1CD2}वा\accentmark{22}{\char"0951}नस्साधामाद्याः\accentmark{22}{\char"0951}।\ त्व\accentmark{20}{\char"1CF9}न्ना\accentmark{22}{\char"0951}ऊतीस्त्वामिन्ना\accentmark{22}{\char"0951}आप्यं\accentmark{22}{\char"0951}।\ मानाइन्द्रापा\accentmark{27}{\char"1CD2}रा\accentmark{22}{\char"0951}वृणात्\mbox{॥ ८\hspace{0pt}॥} \\
वा\accentmark{22}{\char"0951}\kern0.15emय\accentmark{27}{\char"1CD2}ङ्घात्वा\accentmark{22}{\char"0951}\kern0.15emसूता\accentmark{27}{\char"1CD2}वन्तः।\ आ\accentmark{27}{\char"1CD2}\kern0.15emपोना\accentmark{27}{\char"1CD2}वृक्ताब\accentmark{22}{\char"0951}र्हिषः।\ पा\accentmark{27}{\char"1CD2}\kern0.15emवी\accentmark{27}{\char"1CD2}त्रास्याप्रस्रा\accentmark{27}{\char"1CD2}वा\accentmark{22}{\char"0951}णेषुवृत्रहन्।\ पा\accentmark{20}{\char"1CF8}रीस्तोता\accentmark{27}{\char"1CD2}रा\accentmark{22}{\char"0951}आसते\mbox{॥ ९\hspace{0pt}॥} \\
यादिन्द्रा\accentmark{22}{\char"0951}ना\accentmark{20}{\char"1CF9}हू\accentmark{22}{\char"0951}षीष्वा।\ ओजो\accentmark{22}{\char"0951}निम्णं\accentmark{20}{\char"1CF9}चा\accentmark{22}{\char"0951}कृष्टीषु\accentmark{22}{\char"0951}\kern0.15em।\ य\accentmark{27}{\char"1CD2}\kern0.15emद्वाप\accentmark{27}{\char"1CD2}ञ्चा\accentmark{22}{\char"0951}क्षीतीनां\accentmark{27}{\char"1CD2}द्युम्नमाभा\accentmark{22}{\char"0951}रा।\ सत्रा\accentmark{20}{\char"1CF9}वि\accentmark{20}{\char"1CF9}श्वा\accentmark{22}{\char"0951}\kern0.15emनीपौं\accentmark{27}{\char"1CD2}ठ\accentmark{22}{\char"0951}स्यम्\mbox{॥ १०\hspace{0pt}॥} \\
\mbox{॥ इति तृतीयः खण्डः\hspace{0pt}॥} \\ 
सत्य\accentmark{27}{\char"1CD2}मित्थावृषेद\accentmark{22}{\char"0951}\kern0.15emसि।\ वृ\accentmark{27}{\char"1CD2}र्षा\accentmark{22}{\char"0951}जूतिर्नोवीता।\ 
वृ\accentmark{27}{\char"1CD2}षाह्युठे\accentmark{22}{\char"0951}ग्रश्रृण्वीषे\accentmark{20}{\char"1CF8}पा\accentmark{22}{\char"0951}रावा\accentmark{22}{\char"0951}ती।\ वृषोअर्वावा\accentmark{20}{\char"1CF9}ती\accentmark{22}{\char"0951}श्रुताः\accentmark{20}{\char"1CF8}\mbox{॥ १\hspace{0pt}॥} \\
या\accentmark{27}{\char"1CD2}च्चाक्रासी\accentmark{22}{\char"0951}पा\accentmark{22}{\char"0951}\kern0.15emरावा\accentmark{27}{\char"1CD2}ति।\ याद\accentmark{22}{\char"0951}र्वावा\accentmark{27}{\char"1CD2}तीवृत्रहन्।\ आ\accentmark{27}{\char"1CD2}ता\accentmark{22}{\char"0951}स्त्वागीर्भिद्युगदिन्द्राकेशी\accentmark{27}{\char"1CD2}भीः।\ सूता\accentmark{27}{\char"1CD2}\kern0.15emवंआ\accentmark{27}{\char"1CD2}\kern0.15emवी\accentmark{27}{\char"1CD2}वा\accentmark{22}{\char"0951}सति\mbox{॥ २\hspace{0pt}॥} \\
आभीवो\accentmark{22}{\char"0951}वीर\accentmark{20}{\char"1CF9}म\accentmark{27}{\char"1CD2}न्धा\accentmark{22}{\char"0951}सामादेषुगाया।\ गीरा\accentmark{27}{\char"1CD2}माहावी\accentmark{27}{\char"1CD2}चे\accentmark{22}{\char"0951}\kern0.15emतसं।\ इ\accentmark{27}{\char"1CD2}न्द्रन्नामश्रू\accentmark{20}{\char"1CF9}त्यंशाकीनं\accentmark{20}{\char"1CF9}वाचोयाथा\mbox{॥ ३\hspace{0pt}॥} \\
इ\accentmark{20}{\char"1CF8}न्द्रा\accentmark{22}{\char"0951}त्रीधा\accentmark{27}{\char"1CD2}तू\accentmark{22}{\char"0951}शाराणं\accentmark{27}{\char"1CD2}।\ त्रीवा\accentmark{27}{\char"1CD2}\kern0.15emरू\accentmark{27}{\char"1CD2}थंस्वस्ताये\accentmark{20}{\char"1CF9}।\ छर्दिया\accentmark{22}{\char"0951}च्छामाघा\accentmark{22}{\char"0951}वा\accentmark{22}{\char"0951}भ्या\accentmark{22}{\char"0951}श्चामंह्य\accentmark{22}{\char"0951}न्च।\ यावा\accentmark{20}{\char"1CF9}यादिद्यू\accentmark{27}{\char"1CD2}मे\accentmark{22}{\char"0951}भ्यः\mbox{॥ ४\hspace{0pt}॥} \\
श्रा\accentmark{27}{\char"1CD2}यन्ता\accentmark{22}{\char"0951}ईवासूर्य\accentmark{22}{\char"0951}।\ विश्वे\accentmark{27}{\char"1CD2}दिन्द्रा\accentmark{22}{\char"0951}स्यभक्षत।\ 
वासू\accentmark{27}{\char"1CD2}नीजातो\accentmark{20}{\char"1CF9}ज\accentmark{22}{\char"0951}निमान्यो\accentmark{27}{\char"1CD2}ज\accentmark{22}{\char"0951}सा।\ प्रा\accentmark{20}{\char"1CF8}ती\accentmark{22}{\char"0951}भा\accentmark{22}{\char"0951}\kern0.15emग\accentmark{27}{\char"1CD2}\kern0.15emन्ना\accentmark{27}{\char"1CD2}दी\accentmark{22}{\char"0951}धीमः \mbox{॥ ५\hspace{0pt}॥} \\
ना\accentmark{27}{\char"1CD2}सीमादे\accentmark{22}{\char"0951}वआपाता\accentmark{27}{\char"1CD2}त्।\ ईळन्दी\accentmark{22}{\char"0951}र्घा\accentmark{22}{\char"0951}योमर्क्त्याः।\ एता\accentmark{22}{\char"0951}ग्वाचिद्या\accentmark{27}{\char"1CD2}एता\accentmark{22}{\char"0951}शोयूयो
\accentmark{27}{\char"1CD2}जाते।\ इन्द्रोहा\accentmark{20}{\char"1CF9}रीयूयो\accentmark{27}{\char"1CD2}जा\accentmark{22}{\char"0951}ते\mbox{॥ ६\hspace{0pt}॥} \\
आ\accentmark{27}{\char"1CD2}\kern0.15emनोवि\accentmark{20}{\char"1CF8}श्वासुहा\accentmark{27}{\char"1CD2}व्यं\accentmark{22}{\char"0951}।\ इ\accentmark{20}{\char"1CF8}न्द्रं\accentmark{22}{\char"0951}सामात्सू\accentmark{22}{\char"0951}भूषत।\ 
ऊ\accentmark{27}{\char"1CD2}\kern0.15emपाब्र\accentmark{20}{\char"1CF9}ह्णा\accentmark{22}{\char"0951}\kern0.15emणीसा\accentmark{27}{\char"1CD2}वा\accentmark{22}{\char"0951}नानिवृत्रहन्।\ परमाज्यऋचीषमा \mbox{॥ ७\hspace{0pt}॥} \\
त\accentmark{20}{\char"1CF8}वेदि\accentmark{22}{\char"0951}न्द्रावा\accentmark{27}{\char"1CD2}\kern0.15emमंवा\accentmark{27}{\char"1CD2}सू।\ त्वंपू\accentmark{22}{\char"0951}ष्यसिमाध्यामं।\ सत्रा\accentmark{27}{\char"1CD2}विश्वास्यापारामास्या\accentmark{22}{\char"0951}राजसि।\ ना\accentmark{20}{\char"1CF8}कीष्ट्वा\accentmark{20}{\char"1CF8}गोषू\accentmark{22}{\char"0951}वृण्वते\mbox{॥ ८\hspace{0pt}॥} \\
क्वे\accentmark{20}{\char"1CF8}या\accentmark{22}{\char"0951}\kern0.15emथक्वे\accentmark{27}{\char"1CD2}दसि।\ पुरूत्रा\accentmark{27}{\char"1CD2}चिद्धितेमानाः।\ आलाऋषीयुध्मखजकृत्पुरन्दरा।\ प्रा\accentmark{20}{\char"1CF8}गा\accentmark{22}{\char"0951}यत्राआ\accentmark{22}{\char"0951}गासिषुः\mbox{॥ ९\hspace{0pt}॥} \\
वायामे\accentmark{22}{\char"0951}नामीदाह्यः\accentmark{22}{\char"0951}\kern0.15em।\ आ\accentmark{27}{\char"1CD2}पी\accentmark{22}{\char"0951}पेमेहा\accentmark{27}{\char"1CD2}वज्री\accentmark{22}{\char"0951}णं।\ 
तस्मा\accentmark{22}{\char"0951}ऊआद्या\accentmark{27}{\char"1CD2}\kern0.15emसा\accentmark{27}{\char"1CD2}वानेसूतं\accentmark{27}{\char"1CD2}भा\accentmark{22}{\char"0951}रा।\ आनू\accentmark{27}{\char"1CD2}नंभूषताश्रूते\accentmark{27}{\char"1CD2} \mbox{॥ १०\hspace{0pt}॥} \\
\mbox{॥ इति चतुर्थः खण्डः\hspace{0pt}॥} \\ 
योरा\accentmark{27}{\char"1CD2}जा\accentmark{22}{\char"0951}चर्षा\accentmark{22}{\char"0951}णीनां।\ या\accentmark{27}{\char"1CD2}\kern0.15emतारा\accentmark{20}{\char"1CF9}थेभीरा\accentmark{27}{\char"1CD2}ध्री\accentmark{22}{\char"0951}गुः।\ वि\accentmark{27}{\char"1CD2}श्वा\accentmark{22}{\char"0951}सान्तारु\accentmark{27}{\char"1CD2}\kern0.15emतापृ\accentmark{27}{\char"1CD2}तानानां।\ जे\accentmark{27}{\char"1CD2}ष्ठंयो\accentmark{20}{\char"1CF9}वृत्राहा\accentmark{27}{\char"1CD2}गृणे \mbox{॥ १\hspace{0pt}॥} \\
याता\accentmark{22}{\char"0951}इन्द्राभाया\accentmark{22}{\char"0951}महे।\ ता\accentmark{20}{\char"1CF8}तो\accentmark{22}{\char"0951}\kern0.15emनोआ\accentmark{27}{\char"1CD2}भा\accentmark{22}{\char"0951}यंकृ\accentmark{22}{\char"0951}\kern0.15emधि।\ मा\accentmark{27}{\char"1CD2}घा\accentmark{22}{\char"0951}वन्शग्धिता\accentmark{20}{\char"1CF9}वा\accentmark{22}{\char"0951}तन्नाऊता\accentmark{27}{\char"1CD2}ये\accentmark{22}{\char"0951}\kern0.15em।\ वि\accentmark{27}{\char"1CD2}\kern0.15emद्वि\accentmark{27}{\char"1CD2}षोविमृधो\accentmark{22}{\char"0951}जहि\mbox{॥ २\hspace{0pt}॥} \\
वास्तो\accentmark{22}{\char"0951}ष्पतेध्रूवा\accentmark{27}{\char"1CD2}स्थूणां।\ अंसत्रं\accentmark{27}{\char"1CD2}सौ\accentmark{22}{\char"0951}\kern0.15emम्या\accentmark{27}{\char"1CD2}नाम्।\ 
द्रप्सः\accentmark{27}{\char"1CD2}\kern0.15emपूरां\accentmark{27}{\char"1CD2}भेक्ताशश्वा\accentmark{22}{\char"0951}तीनाम्।\ इ\accentmark{27}{\char"1CD2}न्द्रो\accentmark{27}{\char"1CD2}\kern0.15emमू\accentmark{20}{\char"1CF9}नीनां\accentmark{22}{\char"0951}\kern0.15emसा\accentmark{27}{\char"1CD2}खा\accentmark{22}{\char"0951} \mbox{॥ ३\hspace{0pt}॥} \\
ब\accentmark{27}{\char"1CD2}ण्माहं\accentmark{27}{\char"1CD2}आ\accentmark{22}{\char"0951}सीसूर्या।\ बा\accentmark{27}{\char"1CD2}ळा\accentmark{22}{\char"0951}दित्यामाहंआसि।\ 
माह\accentmark{27}{\char"1CD2}स्ते\accentmark{22}{\char"0951}सातो\accentmark{20}{\char"1CF9}मा\accentmark{22}{\char"0951}\kern0.15emहीमा\accentmark{27}{\char"1CD2}पा\accentmark{22}{\char"0951}निष्ठमा।\ मन्हा\accentmark{27}{\char"1CD2}दे\accentmark{22}{\char"0951}वामाहं\accentmark{27}{\char"1CD2}आसि \mbox{॥ ४\hspace{0pt}॥} \\
या\accentmark{20}{\char"1CF8}दि\accentmark{22}{\char"0951}न्द्राप्रा\accentmark{27}{\char"1CD2}गपागूद\accentmark{22}{\char"0951}त्।\ न्या\accentmark{22}{\char"0951}ग्वाहूयासेनृभिः\accentmark{22}{\char"0951}।\ सीमा\accentmark{22}{\char"0951}पूरूनृ\accentmark{27}{\char"1CD2}षू\accentmark{22}{\char"0951}तोअस्या\accentmark{27}{\char"1CD2}\kern0.15emना\accentmark{20}{\char"1CF8}वे।\ आ\accentmark{27}{\char"1CD2}सी\accentmark{22}{\char"0951}प्रशार्धातुर्वा\accentmark{27}{\char"1CD2}शे \mbox{॥ ५\hspace{0pt}॥} \\
क\accentmark{27}{\char"1CD2}स्तामि\accentmark{22}{\char"0951}न्द्रत्वावसो।\ आ\accentmark{27}{\char"1CD2}\kern0.15emमा\accentmark{27}{\char"1CD2}र्क्त्यो\accentmark{22}{\char"0951}दधर्षते।\ 
श्र\accentmark{27}{\char"1CD2}द्धाही\accentmark{27}{\char"1CD2}तेमाघा\accentmark{27}{\char"1CD2}\kern0.15emवन्पा\accentmark{20}{\char"1CF9}र्येदीवी।\ वाजी\accentmark{27}{\char"1CD2}वाजंसिषासति \mbox{॥ ६\hspace{0pt}॥} \\
अश्वीराथीसुरूपायुः\accentmark{27}{\char"1CD2}\kern0.15em।\ गोम\accentmark{27}{\char"1CD2}य्यादिन्द्राते\accentmark{27}{\char"1CD2}साखा।\ श्वात्राभा\accentmark{27}{\char"1CD2}\kern0.15emजावा\accentmark{27}{\char"1CD2}या\accentmark{22}{\char"0951}सासचाते\accentmark{27}{\char"1CD2}सादा\accentmark{22}{\char"0951}।\ चन्द्रै\accentmark{27}{\char"1CD2}र्यातीसाभा\accentmark{27}{\char"1CD2}\kern0.15emऊ\accentmark{27}{\char"1CD2}पा\accentmark{22}{\char"0951}\kern0.15em\mbox{॥ ७\hspace{0pt}॥} \\
य\accentmark{27}{\char"1CD2}द्यावा\accentmark{27}{\char"1CD2}इन्द्रतेशातं\accentmark{27}{\char"1CD2}\kern0.15em।\ शातं\accentmark{20}{\char"1CF9}भूमी\accentmark{22}{\char"0951}रूतास्युः।\ 
न\accentmark{27}{\char"1CD2}त्वावज्ज्रिन्साहा\accentmark{27}{\char"1CD2}स्रंसू\accentmark{27}{\char"1CD2}र्याआनू\accentmark{22}{\char"0951}\kern0.15em।\ ना\accentmark{27}{\char"1CD2}\kern0.15emजाता\accentmark{20}{\char"1CF9}माष्टारोद\accentmark{22}{\char"0951}सी \mbox{॥ ८\hspace{0pt}॥} \\
इन्द्रा\accentmark{22}{\char"0951}ग्नीआपा\accentmark{27}{\char"1CD2}\kern0.15emदीय\accentmark{27}{\char"1CD2}म्।\ पूर्वागा\accentmark{20}{\char"1CF8}त्पद्वा\accentmark{27}{\char"1CD2}ती\accentmark{22}{\char"0951}भ्यः।\ हित्वा\accentmark{20}{\char"1CF9}शि\accentmark{20}{\char"1CF9}रो\accentmark{22}{\char"0951}जिह्वा\accentmark{27}{\char"1CD2}\kern0.15emया\accentmark{20}{\char"1CF9}रा\accentmark{22}{\char"0951}रापच्चारात्।\ त्रिंशा\accentmark{27}{\char"1CD2}त्पादा\accentmark{27}{\char"1CD2}न्या\accentmark{22}{\char"0951}क्रमीत्॥९इ\accentmark{27}{\char"1CD2}न्द्राने\accentmark{20}{\char"1CF9}दी\accentmark{27}{\char"1CD2}\kern0.15emयाए\accentmark{27}{\char"1CD2}दिहि।\ मीता\accentmark{27}{\char"1CD2}\kern0.15emमे\accentmark{27}{\char"1CD2}धाभीरू\accentmark{27}{\char"1CD2}\kern0.15emती\accentmark{27}{\char"1CD2}भीः\accentmark{22}{\char"0951}\kern0.15em।\ 
आ\accentmark{27}{\char"1CD2}श\accentmark{22}{\char"0951}न्तामाश\accentmark{27}{\char"1CD2}न्ता\accentmark{22}{\char"0951}\kern0.15emमाभी\accentmark{27}{\char"1CD2}राभीष्टीभिः।\ आस्वा\accentmark{27}{\char"1CD2}पेस्वापि\accentmark{27}{\char"1CD2}भिः\accentmark{22}{\char"0951} \mbox{॥ १०\hspace{0pt}॥} \\
 \mbox{॥ इति पञ्चमः खण्डः\hspace{0pt}॥} \\
ई\accentmark{27}{\char"1CD2}\kern0.15emताऊती\accentmark{20}{\char"1CF9}वो\accentmark{22}{\char"0951}\kern0.15emआज\accentmark{27}{\char"1CD2}रं\accentmark{22}{\char"0951}\kern0.15em।\ प्रे\accentmark{27}{\char"1CD2}\kern0.15emहेता\accentmark{27}{\char"1CD2}\kern0.15emरामा\accentmark{27}{\char"1CD2}प्रा\accentmark{22}{\char"0951}हितं।\ 
आशू\accentmark{20}{\char"1CF9}ञ्जे\accentmark{20}{\char"1CF9}ता\accentmark{22}{\char"0951}रंहोता\accentmark{22}{\char"0951}रंराथी\accentmark{27}{\char"1CD2}तामं।\ आ\accentmark{27}{\char"1CD2}तूर्तन्तूग्रीयावृ\accentmark{27}{\char"1CD2}धम् \mbox{॥ १\hspace{0pt}॥} \\
मोषु\accentmark{27}{\char"1CD2}\kern0.15emत्वा\accentmark{27}{\char"1CD2}\kern0.15emवाघा\accentmark{20}{\char"1CF9}ताः\accentmark{22}{\char"0951}चाना।\ आ\accentmark{27}{\char"1CD2}\kern0.15emरे\accentmark{27}{\char"1CD2}आस्मिन्नी\accentmark{27}{\char"1CD2}री\accentmark{22}{\char"0951}रमन्।\ आ\accentmark{27}{\char"1CD2}\kern0.15emरा\accentmark{27}{\char"1CD2}\kern0.15emत्ता\accentmark{27}{\char"1CD2}द्वासाधा\accentmark{22}{\char"0951}मा\accentmark{20}{\char"1CF9}द\accentmark{22}{\char"0951}\kern0.15emन्नाआ\accentmark{27}{\char"1CD2}गहि।\ ईहा\accentmark{27}{\char"1CD2}वासन्नूपा\accentmark{22}{\char"0951}श्रुधि\mbox{॥ २\hspace{0pt}॥} \\
सू\accentmark{27}{\char"1CD2}नोता\accentmark{22}{\char"0951}\kern0.15emसो\accentmark{27}{\char"1CD2}\kern0.15emमा\accentmark{27}{\char"1CD2}पाव्ने।\ सो\accentmark{27}{\char"1CD2}\kern0.15emमामि\accentmark{27}{\char"1CD2}न्द्रा\accentmark{22}{\char"0951}यावज्रीणे।\ पा\accentmark{27}{\char"1CD2}\kern0.15emचा\accentmark{27}{\char"1CD2}तापक्तिरावा\accentmark{22}{\char"0951}सेकृ\accentmark{22}{\char"0951}णुध्वमित्।\ पृ\accentmark{20}{\char"1CF9}णन्नि\accentmark{20}{\char"1CF9}त्पृणातेमायाः\accentmark{22}{\char"0951} \mbox{॥ ३\hspace{0pt}॥} \\
य\accentmark{27}{\char"1CD2}स्सा\accentmark{22}{\char"0951}त्राहा\accentmark{27}{\char"1CD2}\kern0.15emवीच\accentmark{27}{\char"1CD2}ऋषणिः।\ इ\accentmark{27}{\char"1CD2}न्द्रन्तंहू\accentmark{20}{\char"1CF8}माहेवायं\accentmark{27}{\char"1CD2}\kern0.15em।\ सा\accentmark{27}{\char"1CD2}हा\accentmark{22}{\char"0951}स्रमन्योतुविनिर्म\accentmark{22}{\char"0951}णसत्पते।\ भा\accentmark{20}{\char"1CF8}वा\accentmark{22}{\char"0951}सामात्सू\accentmark{22}{\char"0951}नोवृधे\accentmark{27}{\char"1CD2}\mbox{॥ ४\hspace{0pt}॥} \\
शा\accentmark{27}{\char"1CD2}चिभिर्नश्शची\accentmark{27}{\char"1CD2}वसू।\ दीवान\accentmark{27}{\char"1CD2}क्तंदिशस्यतं।\ 
मा\accentmark{20}{\char"1CF8}वां\accentmark{22}{\char"0951}रातिरूपा\accentmark{22}{\char"0951}\kern0.15emद\accentmark{27}{\char"1CD2}सात्कादा\accentmark{27}{\char"1CD2}\kern0.15emचा\accentmark{27}{\char"1CD2}ना।\ अस्म\accentmark{27}{\char"1CD2}\kern0.15emद्रा\accentmark{27}{\char"1CD2}तीःकादाचाना\accentmark{27}{\char"1CD2}\mbox{॥ ५\hspace{0pt}॥} \\
यादाकादा\accentmark{20}{\char"1CF9}चामीढूषे\accentmark{22}{\char"0951}।\ स्तोता\accentmark{27}{\char"1CD2}जरेतमार्क्त्याः\accentmark{22}{\char"0951}।\ आदीद्व\accentmark{27}{\char"1CD2}\kern0.15emन्दे\accentmark{27}{\char"1CD2}तावारुणंवीपा\accentmark{27}{\char"1CD2}गीरा\accentmark{22}{\char"0951}।\ धक्ता\accentmark{27}{\char"1CD2}\kern0.15emरंवि\accentmark{27}{\char"1CD2}\kern0.15emप्रा\accentmark{27}{\char"1CD2}तानाम् \mbox{॥ ६\hspace{0pt}॥} \\
पाहि\accentmark{27}{\char"1CD2}\kern0.15emगाया\accentmark{20}{\char"1CF9}न्धा\accentmark{22}{\char"0951}सामा\accentmark{22}{\char"0951}\kern0.15emदे।\ इ\accentmark{27}{\char"1CD2}न्द्रा\accentmark{20}{\char"1CF8}यमेधियातिथे।\ 
य\accentmark{20}{\char"1CF8}स\accentmark{20}{\char"1CF8}म्मिश्लोहर्यो\accentmark{22}{\char"0951}र्यो\accentmark{22}{\char"0951}हीरण्या\accentmark{27}{\char"1CD2}या।\ इन्द्रोवज्री\accentmark{20}{\char"1CF9}हीरण्या\accentmark{27}{\char"1CD2}या \mbox{॥ ७\hspace{0pt}॥} \\
ऊभा\accentmark{20}{\char"1CF9}यं\accentmark{27}{\char"1CD2}श्रृणा\accentmark{27}{\char"1CD2}वा\accentmark{22}{\char"0951}चनः।\ इ\accentmark{20}{\char"1CF8}न्द्रो\accentmark{22}{\char"0951}\kern0.15emअर्वा\accentmark{27}{\char"1CD2}\kern0.15emगीदं\accentmark{27}{\char"1CD2}\kern0.15emवा\accentmark{27}{\char"1CD2}चाः।\ सत्रा\accentmark{20}{\char"1CF9}च्या\accentmark{20}{\char"1CF9}मा\accentmark{22}{\char"0951}\kern0.15emघावा\accentmark{27}{\char"1CD2}\kern0.15emसो\accentmark{27}{\char"1CD2}मा\accentmark{22}{\char"0951}पीतये।\ धीया\accentmark{20}{\char"1CF9}श्शावी\accentmark{22}{\char"0951}ष्ठआग\accentmark{22}{\char"0951}मात्\mbox{॥ ८\hspace{0pt}॥} \\
माहे\accentmark{27}{\char"1CD2}\kern0.15emचा\accentmark{27}{\char"1CD2}\kern0.15emना\accentmark{27}{\char"1CD2}त्वा\accentmark{22}{\char"0951}आद्रीवः।\ पा\accentmark{20}{\char"1CF9}रा\accentmark{22}{\char"0951}शुल्काया\accentmark{22}{\char"0951}दीयसे।\ 
ना\accentmark{27}{\char"1CD2}\kern0.15emसाह\accentmark{20}{\char"1CF9}स्रा\accentmark{22}{\char"0951}यानायू\accentmark{27}{\char"1CD2}ता\accentmark{22}{\char"0951}वज्रिवः।\ ना\accentmark{27}{\char"1CD2}\kern0.15emशा\accentmark{27}{\char"1CD2}ता\accentmark{22}{\char"0951}याशतामघ \mbox{॥ ९\hspace{0pt}॥} \\
व\accentmark{27}{\char"1CD2}स्यठ\accentmark{22}{\char"0951}इन्द्रासिमेपीतूः\accentmark{27}{\char"1CD2}।\ ऊताभ्रा\accentmark{27}{\char"1CD2}\kern0.15emतूरा\accentmark{27}{\char"1CD2}भु\accentmark{22}{\char"0951}ञ्जतः।\ मा\accentmark{27}{\char"1CD2}\kern0.15emता\accentmark{27}{\char"1CD2}चा\accentmark{22}{\char"0951}मेछदयथास्सामा\accentmark{27}{\char"1CD2}वासो।\ व\accentmark{27}{\char"1CD2}सुत्वाना\accentmark{27}{\char"1CD2}\kern0.15emयारा\accentmark{27}{\char"1CD2}धा\accentmark{22}{\char"0951}से\mbox{॥ १०\hspace{0pt}॥} \\
\mbox{॥ इति षष्ठः खण्डः\hspace{0pt}॥} \\ 
इमइ\accentmark{27}{\char"1CD2}न्द्रायसुन्विरे।\ सो\accentmark{20}{\char"1CF8}मा\accentmark{22}{\char"0951}सोदध्या\accentmark{27}{\char"1CD2}शिरः।\ 
तं\accentmark{27}{\char"1CD2}\kern0.15emआ\accentmark{27}{\char"1CD2}\kern0.15emमा\accentmark{27}{\char"1CD2}दा\accentmark{22}{\char"0951}यवज्रहास्तापीता\accentmark{27}{\char"1CD2}\kern0.15emये।\ हा\accentmark{27}{\char"1CD2}रीभ्याय्याह्योकाआ \mbox{॥ १\hspace{0pt}॥} \\
ईमा\accentmark{20}{\char"1CF9}इ\accentmark{22}{\char"0951}न्द्रामा\accentmark{27}{\char"1CD2}दा\accentmark{22}{\char"0951}यते।\ सो\accentmark{27}{\char"1CD2}मा\accentmark{22}{\char"0951}श्चिकीत्राउक्थी\accentmark{27}{\char"1CD2}नाः\accentmark{22}{\char"0951}।\ 
माधोपावाना\accentmark{20}{\char"1CF9}ऊ\accentmark{20}{\char"1CF9}पा\accentmark{22}{\char"0951}\kern0.15emनोगी\accentmark{27}{\char"1CD2}राः\accentmark{22}{\char"0951}श्रुणु।\ रा\accentmark{20}{\char"1CF8}स्वा\accentmark{22}{\char"0951}स्तोत्रा\accentmark{27}{\char"1CD2}या\accentmark{22}{\char"0951}गिर्वणः\mbox{॥ २\hspace{0pt}॥} \\
आत्वाटट्यद्या\accentmark{20}{\char"1CF9}सा\accentmark{22}{\char"0951}बर्दूघां\accentmark{22}{\char"0951}।\ हूवे\accentmark{20}{\char"1CF9}गा\accentmark{22}{\char"0951}यत्रावे\accentmark{22}{\char"0951}पसम्।\ इ\accentmark{20}{\char"1CF8}न्द्रं\accentmark{22}{\char"0951}\kern0.15emधेनुं\accentmark{27}{\char"1CD2}\kern0.15emसूदू\accentmark{27}{\char"1CD2}घा\accentmark{22}{\char"0951}म\accentmark{22}{\char"0951}न्यामीष\accentmark{22}{\char"0951}म्।\ ऊरू\accentmark{27}{\char"1CD2}धारामारंकृ\accentmark{22}{\char"0951}तम् \mbox{॥ ३\hspace{0pt}॥} \\
नं\accentmark{20}{\char"1CF8}त्वा\accentmark{22}{\char"0951}\kern0.15emबृह\accentmark{27}{\char"1CD2}\kern0.15emन्तोअ\accentmark{27}{\char"1CD2}द्रा\accentmark{22}{\char"0951}\kern0.15emयः।\ वा\accentmark{27}{\char"1CD2}रन्त\accentmark{22}{\char"0951}इन्द्रावीळावाः\accentmark{22}{\char"0951}\kern0.15em।\ य\accentmark{27}{\char"1CD2}\kern0.15emच्छि\accentmark{27}{\char"1CD2}क्षा\accentmark{22}{\char"0951}सिस्तूवाते\accentmark{27}{\char"1CD2}मावातेवा\accentmark{27}{\char"1CD2}सू\accentmark{22}{\char"0951}।\ नाकीष्ट\accentmark{27}{\char"1CD2}\kern0.15emदा\accentmark{27}{\char"1CD2}मी\accentmark{22}{\char"0951}नातिते\mbox{॥ ४\hspace{0pt}॥} \\
का\accentmark{27}{\char"1CD2}यीं\accentmark{22}{\char"0951}\kern0.15emवे\accentmark{27}{\char"1CD2}दस्सूते\accentmark{27}{\char"1CD2}\kern0.15emसा\accentmark{27}{\char"1CD2}\kern0.15emचा\accentmark{27}{\char"1CD2}\kern0.15em।\ पी\accentmark{20}{\char"1CF8}बन्तःकद्वायो\accentmark{22}{\char"0951}दधुः।\ 
आयं\accentmark{27}{\char"1CD2}यःपू\accentmark{22}{\char"0951}रो\accentmark{22}{\char"0951}वीभीनत्यो\accentmark{27}{\char"1CD2}ज\accentmark{22}{\char"0951}सा।\ मन्दा\accentmark{27}{\char"1CD2}\kern0.15emनाः\accentmark{27}{\char"1CD2}शिप्री\accentmark{27}{\char"1CD2}\kern0.15emय\accentmark{27}{\char"1CD2}न्धा\accentmark{22}{\char"0951}सा\mbox{॥ ५\hspace{0pt}॥} \\
यादिन्द्राशासो\accentmark{22}{\char"0951}आव्रातं\accentmark{27}{\char"1CD2}।\ च्यावा\accentmark{27}{\char"1CD2}\kern0.15emयासा\accentmark{20}{\char"1CF9}द\accentmark{22}{\char"0951}सस्पारी\accentmark{22}{\char"0951}।\ अस्मा\accentmark{20}{\char"1CF9}का\accentmark{22}{\char"0951}मंशुर्म्मा\accentmark{22}{\char"0951}धावन्पूरूस्पृहं\accentmark{27}{\char"1CD2}।\ वसव्येटट्यआ\accentmark{27}{\char"1CD2}धी\accentmark{22}{\char"0951}बर्हय\mbox{॥ ६\hspace{0pt}॥} \\
त्व\accentmark{20}{\char"1CF8}ष्टानोदै\accentmark{27}{\char"1CD2}व्यंवा\accentmark{27}{\char"1CD2}चाः\accentmark{22}{\char"0951}\kern0.15em।\ पर्ज\accentmark{27}{\char"1CD2}न्योब्र\accentmark{20}{\char"1CF9}ह्मा\accentmark{22}{\char"0951}\kern0.15emण\accentmark{27}{\char"1CD2}\kern0.15emस्पा\accentmark{27}{\char"1CD2}तीः।\ पुत्रै\accentmark{20}{\char"1CF9}भ्रा\accentmark{20}{\char"1CF9}तृ\accentmark{22}{\char"0951}भिरा\accentmark{20}{\char"1CF9}दीतीर्नूपातु\accentmark{22}{\char"0951}नः।\ दुष्टा\accentmark{22}{\char"0951}रन्द्रा\accentmark{20}{\char"1CF9}मा\accentmark{22}{\char"0951}\kern0.15emणंवा\accentmark{27}{\char"1CD2}चाः \mbox{॥ ७\hspace{0pt}॥} \\
कादा\accentmark{27}{\char"1CD2}\kern0.15emचाना\accentmark{27}{\char"1CD2}स्तारी\accentmark{27}{\char"1CD2}रा\accentmark{22}{\char"0951}सी।\ नेद्रं\accentmark{22}{\char"0951}सश्वसीदाशू\accentmark{27}{\char"1CD2}\kern0.15emषे।\ ऊ\accentmark{20}{\char"1CF8}पो\accentmark{22}{\char"0951}पेन्नूमाघावन्भू\accentmark{27}{\char"1CD2}\kern0.15emयाइ\accentmark{27}{\char"1CD2}न्नूते।\ दा\accentmark{20}{\char"1CF8}नन्देवा\accentmark{27}{\char"1CD2}स्या\accentmark{22}{\char"0951}प्रच्यते\mbox{॥ ८\hspace{0pt}॥} \\
युंक्ष्वा\accentmark{27}{\char"1CD2}हीवृ\accentmark{22}{\char"0951}त्रहन्तमः।\ हारी\accentmark{22}{\char"0951}इन्द्रापारावा\accentmark{27}{\char"1CD2}ताः\accentmark{22}{\char"0951}।\ अर्वाची\accentmark{27}{\char"1CD2}\kern0.15emनो\accentmark{27}{\char"1CD2}मा\accentmark{22}{\char"0951}घावन्सो\accentmark{27}{\char"1CD2}मा\accentmark{22}{\char"0951}पीतये।\ उग्रा\accentmark{20}{\char"1CF9}रि\accentmark{20}{\char"1CF9}ष्वाभीरा\accentmark{27}{\char"1CD2}ग\accentmark{22}{\char"0951}हि\mbox{॥ ९\hspace{0pt}॥} \\
त्वा\accentmark{27}{\char"1CD2}\kern0.15emमीदा\accentmark{27}{\char"1CD2}ह्योना\accentmark{27}{\char"1CD2}\kern0.15emराः।\ आ\accentmark{27}{\char"1CD2}पीप्यन्वज्ज्रिन्भू\accentmark{22}{\char"0951}र्णा\accentmark{22}{\char"0951}यः।\ 
सा\accentmark{20}{\char"1CF8}इ\accentmark{22}{\char"0951}न्द्रस्तोमा\accentmark{22}{\char"0951}वाहसाई\accentmark{27}{\char"1CD2}\kern0.15emहा\accentmark{27}{\char"1CD2}श्रू\accentmark{22}{\char"0951}धिः।\ ऊपा\accentmark{20}{\char"1CF9}स्वा\accentmark{20}{\char"1CF9}सा\accentmark{22}{\char"0951}\kern0.15emरामा\accentmark{27}{\char"1CD2}ग\accentmark{22}{\char"0951}हि\mbox{॥ १०\hspace{0pt}॥} \\
\mbox{॥ इति सप्तमः खण्डः\hspace{0pt}॥} \\ 
प्रा\accentmark{27}{\char"1CD2}ती\accentmark{22}{\char"0951}वोदृश्याया\accentmark{27}{\char"1CD2}ती।\ उच्छन्ती\accentmark{22}{\char"0951}दूहीता\accentmark{27}{\char"1CD2}दीवः।\ 
आ\accentmark{20}{\char"1CF8}पो\accentmark{22}{\char"0951}माही\accentmark{22}{\char"0951}वृणूतेचा\accentmark{20}{\char"1CF9}क्षू\accentmark{22}{\char"0951}\kern0.15emषाता\accentmark{27}{\char"1CD2}माः।\ ज्यो\accentmark{27}{\char"1CD2}तीःकृ\accentmark{22}{\char"0951}णोतीसूना\accentmark{27}{\char"1CD2}री \mbox{॥ १\hspace{0pt}॥} \\
इमा\accentmark{20}{\char"1CF9}ऊ\accentmark{22}{\char"0951}\kern0.15emवा\accentmark{27}{\char"1CD2}न्दिविष्टयः।\ उस्रा\accentmark{27}{\char"1CD2}\kern0.15emहा\accentmark{27}{\char"1CD2}वन्तेअश्विना।\ 
आ\accentmark{20}{\char"1CF8}यं\accentmark{27}{\char"1CD2}वा\accentmark{22}{\char"0951}मंह्वाया\accentmark{27}{\char"1CD2}वा\accentmark{22}{\char"0951}सेशचीवसू।\ वीशं\accentmark{22}{\char"0951}वीशंही\accentmark{27}{\char"1CD2}\kern0.15emग\accentmark{27}{\char"1CD2}च्छा\accentmark{22}{\char"0951}थः \mbox{॥ २\hspace{0pt}॥} \\
कु\accentmark{27}{\char"1CD2}ष्ठःकोवा\accentmark{22}{\char"0951}मश्विना।\ तपानो\accentmark{20}{\char"1CF9}दे\accentmark{22}{\char"0951}वामार्क्त्याः\accentmark{22}{\char"0951}।\ 
घ्नाता\accentmark{27}{\char"1CD2}\kern0.15emवा\accentmark{27}{\char"1CD2}मश्नायाक्षा\accentmark{22}{\char"0951}पा\accentmark{22}{\char"0951}माणोंशू\accentmark{27}{\char"1CD2}\kern0.15emना।\ इ\accentmark{27}{\char"1CD2}\kern0.15emत्था\accentmark{27}{\char"1CD2}\kern0.15emमूवा\accentmark{27}{\char"1CD2}द्वन्या\accentmark{27}{\char"1CD2}था\accentmark{22}{\char"0951}\mbox{॥ ३\hspace{0pt}॥} \\
आयं\accentmark{27}{\char"1CD2}\kern0.15emवामा\accentmark{27}{\char"1CD2}धूमक्तमः।\ सू\accentmark{27}{\char"1CD2}\kern0.15emतस्सो\accentmark{27}{\char"1CD2}मोदीवि\accentmark{22}{\char"0951}ष्टिषु।\ तामा\accentmark{22}{\char"0951}श्विनापीबतन्तीरो\accentmark{27}{\char"1CD2}अन्हं।\ धक्तं\accentmark{22}{\char"0951}\kern0.15emर\accentmark{27}{\char"1CD2}त्ना\accentmark{22}{\char"0951}नीदाशू\accentmark{22}{\char"0951}षे \mbox{॥ ४\hspace{0pt}॥} \\
आत्वासो\accentmark{20}{\char"1CF9}मा\accentmark{22}{\char"0951}\kern0.15emस्याग\accentmark{27}{\char"1CD2}ल्दा\accentmark{22}{\char"0951}\kern0.15emया।\ सा\accentmark{27}{\char"1CD2}\kern0.15emदाया\accentmark{20}{\char"1CF9}चन्नाहं\accentmark{27}{\char"1CD2}ज्या।\ भू\accentmark{27}{\char"1CD2}\kern0.15emर्णिं\accentmark{27}{\char"1CD2}\kern0.15emमृग\accentmark{27}{\char"1CD2}न्नासा\accentmark{27}{\char"1CD2}वा\accentmark{22}{\char"0951}नेषुचुक्रुधं।\ कै\accentmark{20}{\char"1CF8}शा\accentmark{22}{\char"0951}नन्नयाचिषात् \mbox{॥ ५\hspace{0pt}॥} \\
अध्वर्यो\accentmark{20}{\char"1CF9}आ\accentmark{22}{\char"0951}द्रावा\accentmark{27}{\char"1CD2}यात्वं\accentmark{27}{\char"1CD2}\kern0.15em।\ सो\accentmark{27}{\char"1CD2}मामिन्द्राः\accentmark{22}{\char"0951}पिपासति।\ ऊ\accentmark{20}{\char"1CF8}पो\accentmark{22}{\char"0951}\kern0.15emनूनं\accentmark{27}{\char"1CD2}यूयूजे\accentmark{22}{\char"0951}वृ\accentmark{20}{\char"1CF9}षाणाहा\accentmark{27}{\char"1CD2}रि।\ आचा\accentmark{27}{\char"1CD2}\kern0.15emजगा\accentmark{27}{\char"1CD2}मा\accentmark{22}{\char"0951}वृत्राहा\accentmark{27}{\char"1CD2} \mbox{॥ ६\hspace{0pt}॥} \\
आ\accentmark{27}{\char"1CD2}भीषादस्ता\accentmark{27}{\char"1CD2}दा\accentmark{22}{\char"0951}भा\accentmark{22}{\char"0951}\kern0.15emरा।\ इ\accentmark{27}{\char"1CD2}न्द्रज्यायःकानी\accentmark{22}{\char"0951}यसः।\ पुरोवा\accentmark{27}{\char"1CD2}सूहिमाघावन्बभू\accentmark{27}{\char"1CD2}वी\accentmark{22}{\char"0951}था।\ भारेभरेचाहा\accentmark{27}{\char"1CD2}व्याः\accentmark{22}{\char"0951} \mbox{॥ ७\hspace{0pt}॥} \\
यादिन्द्राया\accentmark{20}{\char"1CF8}वा\accentmark{22}{\char"0951}तस्त्वं।\ एता\accentmark{20}{\char"1CF9}वा\accentmark{22}{\char"0951}\kern0.15emदाह\accentmark{27}{\char"1CD2}\kern0.15emमीशी\accentmark{27}{\char"1CD2}या।\ स्तोतारामी\accentmark{27}{\char"1CD2}द्दधिषेरदावसो।\ ना\accentmark{20}{\char"1CF8}पा\accentmark{22}{\char"0951}पात्वा\accentmark{27}{\char"1CD2}यारंसिषः\mbox{॥ ८\hspace{0pt}॥} \\
त्वा\accentmark{20}{\char"1CF8}मि\accentmark{22}{\char"0951}न्द्राप्रातू\accentmark{22}{\char"0951}र्तिषु।\ आ\accentmark{27}{\char"1CD2}भीविश्वा\accentmark{22}{\char"0951}आसिस्पृर्धाः\accentmark{22}{\char"0951}।\ आशस्ती\accentmark{20}{\char"1CF9}हाज\accentmark{22}{\char"0951}नीता\accentmark{20}{\char"1CF9}वृ\accentmark{22}{\char"0951}त्रातूरा\accentmark{22}{\char"0951}सी।\ त्व\accentmark{27}{\char"1CD2}न्तू\accentmark{22}{\char"0951}र्यतरूष्यातः \mbox{॥ ९\hspace{0pt}॥} \\
प्र\accentmark{20}{\char"1CF8}योरीरिक्षाओ\accentmark{27}{\char"1CD2}ज\accentmark{22}{\char"0951}सा।\ दीवः\accentmark{20}{\char"1CF9}स\accentmark{20}{\char"1CF9}दो\accentmark{22}{\char"0951}ठ\accentmark{22}{\char"0951}भ्यस्पा\accentmark{27}{\char"1CD2}\kern0.15emरी।\ न\accentmark{27}{\char"1CD2}त्वा\accentmark{22}{\char"0951}विव्या\accentmark{27}{\char"1CD2}\kern0.15emचारा\accentmark{27}{\char"1CD2}जाइन्द्रापार्थी\accentmark{22}{\char"0951}\kern0.15emवम्।\ आ\accentmark{27}{\char"1CD2}तीविश्वववक्षिथ\mbox{॥ १०\hspace{0pt}॥} \\
\mbox{॥ इति अष्टमः खण्डः\hspace{0pt}॥} \\  
\mbox{॥ इति बृहती पाठः समाप्तः\hspace{0pt}॥} \\ \clearpage
\mbox{॥ अथ असावी पाठः\hspace{0pt}॥} \\
आ\accentmark{27}{\char"1CD2}सा\accentmark{22}{\char"0951}वीदेवङ्गो\accentmark{22}{\char"0951}ऋ\accentmark{20}{\char"1CF8}जीका\accentmark{20}{\char"1CF8}मन्धाः\accentmark{22}{\char"0951}\kern0.15em।\ न्य\accentmark{27}{\char"1CD2}स्मिन्नि\accentmark{27}{\char"1CD2}न्द्रोजनु\accentmark{27}{\char"1CD2}षेमूवोच।\ बो\accentmark{27}{\char"1CD2}\kern0.15emधा\accentmark{27}{\char"1CD2}मसित्वाहर्यश्वार्यज्ञैः।\ 
बो\accentmark{27}{\char"1CD2}धा\accentmark{22}{\char"0951}नस्तोमाम\accentmark{20}{\char"1CF9}न्धा\accentmark{22}{\char"0951}\kern0.15emसामा\accentmark{27}{\char"1CD2}दे\accentmark{22}{\char"0951}षु \mbox{॥ १\hspace{0pt}॥} \\
यो\accentmark{27}{\char"1CD2}नीष्टइन्द्रासा\accentmark{27}{\char"1CD2}द\accentmark{22}{\char"0951}नेयकारी।\ ता\accentmark{27}{\char"1CD2}\kern0.15emमा\accentmark{27}{\char"1CD2}\kern0.15emनृभीः\accentmark{20}{\char"1CF8}पू\accentmark{20}{\char"1CF8}रूहूत\accentmark{27}{\char"1CD2}प्राया\accentmark{22}{\char"0951}\kern0.15emहि।\ आ\accentmark{27}{\char"1CD2}सोयाथा\accentmark{27}{\char"1CD2}नोवीता\accentmark{27}{\char"1CD2}वृधा\accentmark{22}{\char"0951}श्चीत्।\ ददो\accentmark{22}{\char"0951}\kern0.15emवा\accentmark{27}{\char"1CD2}सू\accentmark{22}{\char"0951}नीमामा\accentmark{20}{\char"1CF9}दश्चासो\accentmark{27}{\char"1CD2}मैः \mbox{॥ २\hspace{0pt}॥} \\
आ\accentmark{20}{\char"1CF8}द\accentmark{22}{\char"0951}दरुत्सामा\accentmark{20}{\char"1CF9}सृ\accentmark{22}{\char"0951}जोविखानि।\ त्वा\accentmark{27}{\char"1CD2}मर्णावान्बल्बधानंआराम्णः।\ मा\accentmark{27}{\char"1CD2}\kern0.15emहा\accentmark{27}{\char"1CD2}न्ता\accentmark{22}{\char"0951}मिन्द्राप\accentmark{20}{\char"1CF9}र्वातंवीयद्वाः।\ सृ\accentmark{27}{\char"1CD2}\kern0.15emज\accentmark{27}{\char"1CD2}द्धाराआ\accentmark{27}{\char"1CD2}\kern0.15emवाय\accentmark{20}{\char"1CF9}द\accentmark{22}{\char"0951}दानावान्हन्\mbox{॥ ३\hspace{0pt}॥} \\
सुष्वाणा\accentmark{27}{\char"1CD2}सा\accentmark{22}{\char"0951}इन्द्रा\accentmark{20}{\char"1CF8}स्तुमा\accentmark{27}{\char"1CD2}सीत्वा।\ सनिष्य\accentmark{27}{\char"1CD2}न्ता\accentmark{22}{\char"0951}श्चीत्तुविनृम्णावा\accentmark{22}{\char"0951}जं।\ आनो\accentmark{22}{\char"0951}भरासूवीतं\accentmark{20}{\char"1CF9}य\accentmark{20}{\char"1CF9}स्याकोना\accentmark{27}{\char"1CD2}\kern0.15em।\ ताना\accentmark{27}{\char"1CD2}त्मनासह्यामात्वो\accentmark{27}{\char"1CD2}ताः\accentmark{22}{\char"0951}\kern0.15em\mbox{॥ ४\hspace{0pt}॥} \\
ज\accentmark{27}{\char"1CD2}\kern0.15emगृ\accentmark{27}{\char"1CD2}ह्यातेद\accentmark{27}{\char"1CD2}क्षीणमिन्द्राहा\accentmark{27}{\char"1CD2}स्तं\accentmark{22}{\char"0951}।\ वसूया\accentmark{27}{\char"1CD2}वो\accentmark{22}{\char"0951}वसूपातेवा\accentmark{27}{\char"1CD2}सूनां।\ वि\accentmark{27}{\char"1CD2}\kern0.15emद्या\accentmark{27}{\char"1CD2}\kern0.15emहि\accentmark{27}{\char"1CD2}त्वागो\accentmark{27}{\char"1CD2}\kern0.15emपा\accentmark{27}{\char"1CD2}तिंसूरागो\accentmark{27}{\char"1CD2}नां।\ अस्म\accentmark{20}{\char"1CF9}भ्य\accentmark{22}{\char"0951}ञ्चित्रं\accentmark{27}{\char"1CD2}\kern0.15emवृ\accentmark{27}{\char"1CD2}षाणंरायि\accentmark{27}{\char"1CD2}\kern0.15emन्दाः\accentmark{27}{\char"1CD2} \mbox{॥ ५\hspace{0pt}॥} \\
इ\accentmark{27}{\char"1CD2}न्द्रन्ना\accentmark{20}{\char"1CF9}रो\accentmark{22}{\char"0951}\kern0.15emनेमा\accentmark{27}{\char"1CD2}धी\accentmark{22}{\char"0951}ताहवन्ते।\ यत्पार्यायू\accentmark{22}{\char"0951}ना\accentmark{20}{\char"1CF9}ज\accentmark{22}{\char"0951}न्तेधीयस्ताः\accentmark{27}{\char"1CD2}\kern0.15em।\ शू\accentmark{27}{\char"1CD2}\kern0.15emरोनृ\accentmark{20}{\char"1CF9}षा\accentmark{22}{\char"0951}ताश्रावा\accentmark{22}{\char"0951}साश्चका\accentmark{27}{\char"1CD2}\kern0.15emमे।\ 
आ\accentmark{27}{\char"1CD2}\kern0.15emगो\accentmark{27}{\char"1CD2}मा\accentmark{22}{\char"0951}तीव्रा\accentmark{20}{\char"1CF8}जे\accentmark{20}{\char"1CF9}भाजात्व\accentmark{27}{\char"1CD2}न्नाः\accentmark{22}{\char"0951} \mbox{॥ ६\hspace{0pt}॥} \\
वायाःसूपर्णाऊपासे\accentmark{27}{\char"1CD2}दूरिन्द्रं।\ प्रीयामेधा\accentmark{27}{\char"1CD2}र्षायोना\accentmark{27}{\char"1CD2}धा\accentmark{22}{\char"0951}मा\accentmark{20}{\char"1CF9}नः।\ आपध्वान्तामूर्णहिपूर्धी\accentmark{22}{\char"0951}चाक्षुः।\ मुमूग्धाटट्यस्मान्नीधाये\accentmark{22}{\char"0951}वाबदध्नान् \mbox{॥ ७\hspace{0pt}॥} \\
नाकेसूपर्णा\accentmark{22}{\char"0951}मूपा\accentmark{22}{\char"0951}यत्पा\accentmark{22}{\char"0951}त\accentmark{22}{\char"0951}न्तं।\ हृ\accentmark{27}{\char"1CD2}\kern0.15emदा\accentmark{27}{\char"1CD2}वेन\accentmark{22}{\char"0951}न्तोअभ्या\accentmark{27}{\char"1CD2}चा\accentmark{22}{\char"0951}क्षतत्वा।\ हिर\accentmark{22}{\char"0951}ण्यपा\accentmark{20}{\char"1CF8}क्षंवारूणस्यादूतं\accentmark{27}{\char"1CD2}।\ यामास्यायो\accentmark{27}{\char"1CD2}नू\accentmark{22}{\char"0951}शाकूनं\accentmark{20}{\char"1CF9}भूरण्यूम् \mbox{॥ ८\hspace{0pt}॥} \\
ब्र\accentmark{27}{\char"1CD2}ह्मा\accentmark{22}{\char"0951}जज्ञानं\accentmark{20}{\char"1CF9}प्राथामंपुरास्ता\accentmark{22}{\char"0951}त्।\ वीसीमातस्सूरुचोवेनआ\accentmark{22}{\char"0951}वात्।\ सा\accentmark{27}{\char"1CD2}बुध्न्याऊ\accentmark{22}{\char"0951}पामाआस्याविष्ठाः।\ 
साताश्चा\accentmark{20}{\char"1CF9}योनी\accentmark{27}{\char"1CD2}\kern0.15emमा\accentmark{27}{\char"1CD2}साताश्चावी\accentmark{27}{\char"1CD2}वाः \mbox{॥ ९\hspace{0pt}॥} \\
आपू\accentmark{22}{\char"0951}र्व्यापूरूता\accentmark{20}{\char"1CF9}मा\accentmark{22}{\char"0951}न्यस्मै।\ माहे\accentmark{27}{\char"1CD2}वीराया\accentmark{22}{\char"0951}तावा\accentmark{20}{\char"1CF9}सेतूराया\accentmark{22}{\char"0951}।\ वीरप्सीनेवज्रीणेशन्ता\accentmark{22}{\char"0951}मानी।\ वा\accentmark{27}{\char"1CD2}चां\accentmark{22}{\char"0951}स्यास्मयस्थावी\accentmark{22}{\char"0951}रायतक्षुः \mbox{॥ १०\hspace{0pt}॥} \\
 \mbox{॥ इति प्रथमः खण्डः\hspace{0pt}॥} \\ 
आ\accentmark{27}{\char"1CD2}व\accentmark{22}{\char"0951}द्रप्सो\accentmark{20}{\char"1CF9}अं\accentmark{22}{\char"0951}\kern0.15emशूमा\accentmark{27}{\char"1CD2}तीमतिष्ठात्।\ इया\accentmark{27}{\char"1CD2}\kern0.15emनः\accentmark{27}{\char"1CD2}कृष्णो\accentmark{27}{\char"1CD2}\kern0.15emदशा\accentmark{20}{\char"1CF9}भीस्साहा\accentmark{27}{\char"1CD2}स्रैः\accentmark{22}{\char"0951}।\ आवक्ता\accentmark{27}{\char"1CD2}मिन्द्र\accentmark{22}{\char"0951}श्शच्या\accentmark{27}{\char"1CD2}धामन्तं।\ आ\accentmark{27}{\char"1CD2}पस्नीही\accentmark{22}{\char"0951}तिंनृमा\accentmark{27}{\char"1CD2}णा\accentmark{22}{\char"0951}आधद्राः\accentmark{27}{\char"1CD2} \mbox{॥ १\hspace{0pt}॥} \\
वृत्रास्या\accentmark{27}{\char"1CD2}त्वश्वासाथादीषामाणः।\ विश्वेदेवा\accentmark{27}{\char"1CD2}आ\accentmark{22}{\char"0951}जाहुर्ये\accentmark{27}{\char"1CD2}साखायः।\ मारू\accentmark{27}{\char"1CD2}द्धी\accentmark{22}{\char"0951}रिन्द्रासख्य\accentmark{27}{\char"1CD2}न्ते\accentmark{22}{\char"0951}अस्तु।\ आ\accentmark{27}{\char"1CD2}\kern0.15emथेमा\accentmark{27}{\char"1CD2}विश्वाःपृ\accentmark{27}{\char"1CD2}नाजयासि \mbox{॥ २\hspace{0pt}॥} \\
विधु\accentmark{20}{\char"1CF9}न्दद्रा\accentmark{22}{\char"0951}\kern0.15emणंसा\accentmark{27}{\char"1CD2}\kern0.15emमा\accentmark{27}{\char"1CD2}नेबहूनां।\ यू\accentmark{27}{\char"1CD2}\kern0.15emवा\accentmark{27}{\char"1CD2}\kern0.15emनंस\accentmark{27}{\char"1CD2}न्तं\accentmark{22}{\char"0951}पालीतो\accentmark{27}{\char"1CD2}ज\accentmark{22}{\char"0951}गारा।\ दे\accentmark{27}{\char"1CD2}\kern0.15emवा\accentmark{27}{\char"1CD2}स्या\accentmark{22}{\char"0951}पाश्या\accentmark{22}{\char"0951}\kern0.15emका\accentmark{27}{\char"1CD2}व्यं\accentmark{22}{\char"0951}मा\accentmark{22}{\char"0951}हित्वा\accentmark{22}{\char"0951}\kern0.15em।\ 
अ\accentmark{27}{\char"1CD2}\kern0.15emद्या\accentmark{27}{\char"1CD2}\kern0.15emमा\accentmark{27}{\char"1CD2}मारास्स\accentmark{27}{\char"1CD2}\kern0.15emह्य\accentmark{27}{\char"1CD2}स्सामा\accentmark{27}{\char"1CD2}ना \mbox{॥ ३\hspace{0pt}॥} \\
त्वंहत्यस्सप्त\accentmark{27}{\char"1CD2}भ्योजा\accentmark{27}{\char"1CD2}या\accentmark{22}{\char"0951}मानः।\ अ\accentmark{27}{\char"1CD2}\kern0.15emश\accentmark{27}{\char"1CD2}त्रुभ्योअभावश्श\accentmark{27}{\char"1CD2}त्रू\accentmark{22}{\char"0951}रिन्द्रा।\ गू\accentmark{27}{\char"1CD2}\kern0.15emढे\accentmark{27}{\char"1CD2}\kern0.15emद्या\accentmark{27}{\char"1CD2}वा\accentmark{22}{\char"0951}पृथिवीअ\accentmark{27}{\char"1CD2}\kern0.15emन्वा\accentmark{27}{\char"1CD2}विन्दः।\ वि\accentmark{27}{\char"1CD2}\kern0.15emभूमा\accentmark{27}{\char"1CD2}द्भ्योभू\accentmark{27}{\char"1CD2}वा\accentmark{22}{\char"0951}नेभ्योरा\accentmark{27}{\char"1CD2}\kern0.15emण\accentmark{27}{\char"1CD2}न्धाः\accentmark{22}{\char"0951} \mbox{॥ ४\hspace{0pt}॥} \\
मेळि\accentmark{20}{\char"1CF9}न्न\accentmark{20}{\char"1CF9}त्वा\accentmark{22}{\char"0951}वज्रीणं\accentmark{22}{\char"0951}भृष्टीम\accentmark{27}{\char"1CD2}न्तं।\ पु\accentmark{27}{\char"1CD2}\kern0.15emरूधा\accentmark{27}{\char"1CD2}स्मानं\accentmark{22}{\char"0951}वृषाभंस्थीरा\accentmark{27}{\char"1CD2}प्सु\accentmark{22}{\char"0951}नुं।\ का\accentmark{27}{\char"1CD2}रो\accentmark{22}{\char"0951}ष्यर्यस्तारुषीर्दूवस्यूः।\ डन्दइन्द्र\accentmark{22}{\char"0951}द्युक्षंवृ\accentmark{22}{\char"0951}त्राहा\accentmark{27}{\char"1CD2}णंगृणीषे \mbox{॥ ५\hspace{0pt}॥} \\
प्रावो\accentmark{22}{\char"0951}माहे\accentmark{20}{\char"1CF9}मा\accentmark{22}{\char"0951}\kern0.15emहेवृ\accentmark{27}{\char"1CD2}\kern0.15emधे\accentmark{27}{\char"1CD2}भरध्वं।\ प्रा\accentmark{27}{\char"1CD2}चे\accentmark{22}{\char"0951}तासेप्रासू\accentmark{20}{\char"1CF8}मातिंकृ\accentmark{22}{\char"0951}णुध्वम्।\ वीशःपूर्वीप्रा\accentmark{27}{\char"1CD2}चा\accentmark{22}{\char"0951}राचर्षाणिप्राः\accentmark{22}{\char"0951} \mbox{॥ ६\hspace{0pt}॥} \\
शू\accentmark{27}{\char"1CD2}\kern0.15emनं\accentmark{27}{\char"1CD2}हूवेमामाघा\accentmark{20}{\char"1CF9}वानामिन्द्रं।\ अ\accentmark{27}{\char"1CD2}स्मिन्भा\accentmark{27}{\char"1CD2}\kern0.15emरेनृ\accentmark{20}{\char"1CF9}तामंवा\accentmark{27}{\char"1CD2}ज\accentmark{22}{\char"0951}सातौ।\ श्रृण्वन्ता\accentmark{27}{\char"1CD2}मुग्रा\accentmark{27}{\char"1CD2}\kern0.15emमूता\accentmark{20}{\char"1CF9}येसामात्सु\accentmark{27}{\char"1CD2}।\ घ्नन्तं\accentmark{22}{\char"0951}वृत्राणि\accentmark{22}{\char"0951}\kern0.15emसंची\accentmark{27}{\char"1CD2}\kern0.15emतन्धा\accentmark{27}{\char"1CD2}ना\accentmark{22}{\char"0951}नी \mbox{॥ ७\hspace{0pt}॥} \\
ऊदुब्र\accentmark{22}{\char"0951}ह्माणैरतश्रवस्या\accentmark{27}{\char"1CD2}।\ इन्द्रंसामर्येमाहयावसिष्ठा।\ आ\accentmark{20}{\char"1CF8}यो\accentmark{20}{\char"1CF8}र्विश्वा\accentmark{22}{\char"0951}नीश्रा\accentmark{27}{\char"1CD2}वासताताना\accentmark{22}{\char"0951}।\ ऊपश्रो\accentmark{27}{\char"1CD2}\kern0.15emता\accentmark{20}{\char"1CF9}मा\accentmark{27}{\char"1CD2}\kern0.15emइवा\accentmark{27}{\char"1CD2}\kern0.15emतोवा\accentmark{27}{\char"1CD2}चां\accentmark{22}{\char"0951}सि \mbox{॥ ८\hspace{0pt}॥} \\
च\accentmark{27}{\char"1CD2}\kern0.15emक्रं\accentmark{20}{\char"1CF9}याद\accentmark{22}{\char"0951}स्यप्सू\accentmark{27}{\char"1CD2}आनिषाक्तं।\ ऊतो\accentmark{20}{\char"1CF9}ता\accentmark{20}{\char"1CF9}द\accentmark{22}{\char"0951}\kern0.15emस्मै\accentmark{27}{\char"1CD2}\kern0.15emम\accentmark{27}{\char"1CD2}\kern0.15emध्वी\accentmark{27}{\char"1CD2}चा\accentmark{22}{\char"0951}च्छद्यात्।\ पृ\accentmark{27}{\char"1CD2}\kern0.15emथि\accentmark{27}{\char"1CD2}व्या\accentmark{22}{\char"0951}\kern0.15emमा\accentmark{27}{\char"1CD2}तीषितंया\accentmark{27}{\char"1CD2}दूधाः।\ पायोगो\accentmark{20}{\char"1CF9}ष्वा\accentmark{20}{\char"1CF9}द\accentmark{22}{\char"0951}\kern0.15emधाओ\accentmark{27}{\char"1CD2}षाधीषु \mbox{॥ ९\hspace{0pt}॥} \\
 \mbox{॥ इति द्वितीयः खण्डः\hspace{0pt}॥} \\ 
त्या\accentmark{27}{\char"1CD2}मूषुवाजी\accentmark{27}{\char"1CD2}नन्देवा\accentmark{27}{\char"1CD2}जू\accentmark{22}{\char"0951}\kern0.15emतं।\ स\accentmark{27}{\char"1CD2}\kern0.15emहोवा\accentmark{27}{\char"1CD2}नं\accentmark{22}{\char"0951}\kern0.15emता\accentmark{27}{\char"1CD2}\kern0.15emरूता\accentmark{27}{\char"1CD2}रंराथा\accentmark{22}{\char"0951}नाम्।\ आ\accentmark{27}{\char"1CD2}री\accentmark{22}{\char"0951}ष्टनेमिंपृतना\accentmark{20}{\char"1CF9}ज\accentmark{22}{\char"0951}\kern0.15emमाशू\accentmark{27}{\char"1CD2}म्।\ 
स्व\accentmark{27}{\char"1CD2}स्ताये\accentmark{27}{\char"1CD2}\kern0.15emता\accentmark{20}{\char"1CF9}क्ष्यामीहाहू\accentmark{22}{\char"0951}वेम \mbox{॥ १\hspace{0pt}॥} \\
त्रा\accentmark{27}{\char"1CD2}\kern0.15emता\accentmark{27}{\char"1CD2}\kern0.15emरमि\accentmark{27}{\char"1CD2}न्द्रामा\accentmark{27}{\char"1CD2}\kern0.15emवीता\accentmark{27}{\char"1CD2}\kern0.15emरामि\accentmark{27}{\char"1CD2}न्द्रं\accentmark{22}{\char"0951}\kern0.15em।\ हा\accentmark{27}{\char"1CD2}वे\accentmark{22}{\char"0951}हवेसूहावंशू\accentmark{27}{\char"1CD2}रामि\accentmark{22}{\char"0951}न्द्रम्।\ हूवे\accentmark{27}{\char"1CD2}नूशक्रं\accentmark{27}{\char"1CD2}पू\accentmark{22}{\char"0951}रुहूतामिन्द्रं।\ ई\accentmark{27}{\char"1CD2}\kern0.15emदंहा\accentmark{27}{\char"1CD2}\kern0.15emवि\accentmark{27}{\char"1CD2}र्मा\accentmark{22}{\char"0951}घा\accentmark{20}{\char"1CF9}वावे\accentmark{22}{\char"0951}त्विन्द्राः\accentmark{22}{\char"0951} \mbox{॥ २\hspace{0pt}॥} \\
या\accentmark{27}{\char"1CD2}\kern0.15emजा\accentmark{27}{\char"1CD2}\kern0.15emमाहाइ\accentmark{27}{\char"1CD2}न्द्रंव\accentmark{27}{\char"1CD2}ज्रा\accentmark{22}{\char"0951}दक्षिणं।\ हा\accentmark{27}{\char"1CD2}री\accentmark{22}{\char"0951}णांरथ्यांटट्यवि\accentmark{27}{\char"1CD2}व्रा\accentmark{22}{\char"0951}तानाम्।\ प्र\accentmark{20}{\char"1CF8}श्म\accentmark{20}{\char"1CF8}श्रू\accentmark{22}{\char"0951}भिर्दोधुवदू\accentmark{27}{\char"1CD2}र्ध्वाधा\accentmark{22}{\char"0951}भुवात्।\ वी\accentmark{20}{\char"1CF8}से\accentmark{20}{\char"1CF8}ना\accentmark{22}{\char"0951}भिर्भा\accentmark{27}{\char"1CD2}या\accentmark{22}{\char"0951}मानोवीरा\accentmark{27}{\char"1CD2}धा\accentmark{22}{\char"0951}सा \mbox{॥ ३\hspace{0pt}॥} \\
सत्रा\accentmark{27}{\char"1CD2}हा\accentmark{22}{\char"0951}णंदाधृ\accentmark{22}{\char"0951}षिन्तूम्रामिन्द्रं\accentmark{22}{\char"0951}\kern0.15em।\ मा\accentmark{27}{\char"1CD2}हामा\accentmark{22}{\char"0951}पारं\accentmark{20}{\char"1CF9}वृ\accentmark{22}{\char"0951}\kern0.15emषाभं\accentmark{27}{\char"1CD2}\kern0.15emसूवा\accentmark{27}{\char"1CD2}ज्र\accentmark{22}{\char"0951}म्।\ हन्तायो\accentmark{27}{\char"1CD2}वृत्रं\accentmark{20}{\char"1CF9}सानीतोता\accentmark{20}{\char"1CF9}वा\accentmark{27}{\char"1CD2}जं\accentmark{22}{\char"0951}।\ 
दातामाघा\accentmark{20}{\char"1CF9}नीमाघा\accentmark{20}{\char"1CF9}वा\accentmark{22}{\char"0951}\kern0.15emसूरा\accentmark{27}{\char"1CD2}धाः\accentmark{22}{\char"0951} \mbox{॥ ४\hspace{0pt}॥} \\
यो\accentmark{27}{\char"1CD2}नो\accentmark{22}{\char"0951}वनूष्य\accentmark{20}{\char"1CF9}न्ना\accentmark{27}{\char"1CD2}\kern0.15emभीदा\accentmark{27}{\char"1CD2}\kern0.15emतीमा\accentmark{27}{\char"1CD2}र्क्त्याः\accentmark{22}{\char"0951}।\ ऊगणावा\accentmark{20}{\char"1CF9}म\accentmark{27}{\char"1CD2}न्यामानास्तुरो\accentmark{27}{\char"1CD2}वा।\ क्षी\accentmark{27}{\char"1CD2}धीयूधा\accentmark{27}{\char"1CD2}शावा\accentmark{22}{\char"0951}सा
वा\accentmark{20}{\char"1CF9}ता\accentmark{22}{\char"0951}मिन्द्राः\accentmark{27}{\char"1CD2}\kern0.15em।\ आ\accentmark{27}{\char"1CD2}\kern0.15emभि\accentmark{27}{\char"1CD2}ष्या\accentmark{22}{\char"0951}मवृषामाणस्त्वोताः \mbox{॥ ५\hspace{0pt}॥} \\
यं\accentmark{27}{\char"1CD2}वृत्रे\accentmark{20}{\char"1CF9}षु\accentmark{22}{\char"0951}क्षीता\accentmark{27}{\char"1CD2}\kern0.15emयास्प\accentmark{27}{\char"1CD2}\kern0.15emर्धा\accentmark{27}{\char"1CD2}मानः।\ यं\accentmark{27}{\char"1CD2}युक्तेषु\accentmark{22}{\char"0951}तूरा\accentmark{20}{\char"1CF9}य॑\accentmark{22}{\char"0951}न्तोहा\accentmark{27}{\char"1CD2}व\accentmark{22}{\char"0951}न्ते।\ यं\accentmark{27}{\char"1CD2}\kern0.15emशू\accentmark{27}{\char"1CD2}रा\accentmark{22}{\char"0951}सातौयामापा\accentmark{22}{\char"0951}मूपा\accentmark{22}{\char"0951}ज्मन्।\ 
यंवि\accentmark{27}{\char"1CD2}प्रा\accentmark{22}{\char"0951}सो\accentmark{22}{\char"0951}वाज\accentmark{20}{\char"1CF9}यन्तेसइन्द्राः\accentmark{22}{\char"0951} \mbox{॥ ६\hspace{0pt}॥} \\
इन्द्रा\accentmark{22}{\char"0951}पर्वताबृहा\accentmark{27}{\char"1CD2}\kern0.15emतारा\accentmark{27}{\char"1CD2}थेना।\ वा\accentmark{27}{\char"1CD2}\kern0.15emमी\accentmark{27}{\char"1CD2}\kern0.15emरीळआ\accentmark{27}{\char"1CD2}वाहतंसूवी\accentmark{27}{\char"1CD2}राः।\ वी\accentmark{27}{\char"1CD2}\kern0.15emतं\accentmark{27}{\char"1CD2}हाव्या\accentmark{20}{\char"1CF9}न्याध्वारेषुदेवाः\accentmark{27}{\char"1CD2}।\ वर्धेथांगीर्भि\accentmark{20}{\char"1CF9}रिळा\accentmark{22}{\char"0951}यामाद\accentmark{22}{\char"0951}न्ताः \mbox{॥ ७\hspace{0pt}॥} \\
इ\accentmark{20}{\char"1CF8}न्द्रा\accentmark{22}{\char"0951}यागीरोआनी\accentmark{22}{\char"0951}शीतसर्गः\accentmark{22}{\char"0951}\kern0.15em।\ आ\accentmark{27}{\char"1CD2}\kern0.15emपप्रै\accentmark{20}{\char"1CF9}रा\accentmark{22}{\char"0951}\kern0.15emयत्सा\accentmark{27}{\char"1CD2}ग\accentmark{22}{\char"0951}रस्याबुध्ना\accentmark{27}{\char"1CD2}\kern0.15emत्।\ यो\accentmark{27}{\char"1CD2}\kern0.15emआ\accentmark{27}{\char"1CD2}क्षेणेवाचा\accentmark{22}{\char"0951}क्रीयौशा\accentmark{27}{\char"1CD2}ची\accentmark{22}{\char"0951}भिः।\ वि\accentmark{20}{\char"1CF8}ष्वात्तस्तंभा\accentmark{22}{\char"0951}\kern0.15emपृथि\accentmark{27}{\char"1CD2}\kern0.15emवी\accentmark{27}{\char"1CD2}मूतद्याम् \mbox{॥ ८\hspace{0pt}॥} \\
आ\accentmark{27}{\char"1CD2}त्वासा\accentmark{27}{\char"1CD2}खा\accentmark{22}{\char"0951}यासख्यावा\accentmark{22}{\char"0951}वृत्युः\accentmark{22}{\char"0951}।\ ती\accentmark{20}{\char"1CF8}राः\accentmark{27}{\char"1CD2}\kern0.15emपूरू\accentmark{27}{\char"1CD2}ची\accentmark{22}{\char"0951}दर्णावा\accentmark{27}{\char"1CD2}ञ्ज\accentmark{22}{\char"0951}गम्या।\ पि\accentmark{27}{\char"1CD2}\kern0.15emतु\accentmark{20}{\char"1CF9}र्नापातामा\accentmark{27}{\char"1CD2}दधीतावेधाः।\ अस्मि\accentmark{27}{\char"1CD2}न्क्षायेप्रातारा\accentmark{27}{\char"1CD2}न्दीद्यानः \mbox{॥ ९\hspace{0pt}॥} \\
को\accentmark{27}{\char"1CD2}अद्यायुक्तेधुरिगा\accentmark{27}{\char"1CD2}\kern0.15emऋता\accentmark{27}{\char"1CD2}स्य।\ शी\accentmark{27}{\char"1CD2}मी\accentmark{22}{\char"0951}वतोभामीनो\accentmark{22}{\char"0951}दूर्ह्रणायुन्।\ आ\accentmark{27}{\char"1CD2}\kern0.15emस\accentmark{27}{\char"1CD2}न्नेषामप्सूवा\accentmark{27}{\char"1CD2}होमायोभून्।\ 
या\accentmark{27}{\char"1CD2}ए\accentmark{22}{\char"0951}षांभृत्या\accentmark{27}{\char"1CD2}\kern0.15emमृणा\accentmark{27}{\char"1CD2}\kern0.15emधत्सा\accentmark{27}{\char"1CD2}जी\accentmark{22}{\char"0951}वान् \mbox{॥ १०\hspace{0pt}॥} \\
\mbox{॥ इति तृतीयः खण्डः\hspace{0pt}॥} \\  
गा\accentmark{27}{\char"1CD2}यन्तित्वागायत्रीणाः\accentmark{22}{\char"0951}।\ अ\accentmark{20}{\char"1CF8}र्चन्त्यर्का\accentmark{27}{\char"1CD2}मर्कीणाः।\ ब्र\accentmark{27}{\char"1CD2}\kern0.15emह्मा\accentmark{27}{\char"1CD2}णस्त्वाशतक्रतो।\ उ\accentmark{27}{\char"1CD2}द्वं\accentmark{22}{\char"0951}शामीवयेमिरे \mbox{॥ १\hspace{0pt}॥} \\
इ\accentmark{27}{\char"1CD2}न्द्रंविश्वा\accentmark{22}{\char"0951}अविवृधन्।\ समूद्रा\accentmark{27}{\char"1CD2}व्या\accentmark{22}{\char"0951}चासंगीराः\accentmark{22}{\char"0951}\kern0.15em।\ रा\accentmark{27}{\char"1CD2}थीता\accentmark{22}{\char"0951}मंराथी\accentmark{27}{\char"1CD2}नां\accentmark{22}{\char"0951}।\ वाजानां\accentmark{27}{\char"1CD2}सात्पा\accentmark{22}{\char"0951}\kern0.15emतिंपा\accentmark{27}{\char"1CD2}तिम् \mbox{॥ २\hspace{0pt}॥} \\
ईमा\accentmark{27}{\char"1CD2}मिन्द्रासूतंपी\accentmark{27}{\char"1CD2}बा।\ ज्ये\accentmark{22}{\char"0951}ष्ठमा\accentmark{20}{\char"1CF9}मार्त्य\accentmark{22}{\char"0951}म्मादम्\accentmark{27}{\char"1CD2}।\ शुक्र\accentmark{20}{\char"1CF9}स्या\accentmark{22}{\char"0951}त्वाभ्याठ\accentmark{22}{\char"0951}क्ष\accentmark{20}{\char"1CF9}रन्।\ धा\accentmark{20}{\char"1CF9}रा\accentmark{22}{\char"0951}ऋतस्यासाद\accentmark{22}{\char"0951}ने \mbox{॥ ३\hspace{0pt}॥} \\
या\accentmark{27}{\char"1CD2}दि\accentmark{22}{\char"0951}न्द्रचित्रमाह्याना\accentmark{27}{\char"1CD2}\kern0.15em।\ अ\accentmark{27}{\char"1CD2}\kern0.15emस्ति\accentmark{20}{\char"1CF8}त्वादा\accentmark{22}{\char"0951}तमद्रिवः।\ रा\accentmark{27}{\char"1CD2}\kern0.15emधास्त\accentmark{27}{\char"1CD2}न्नोवि\accentmark{27}{\char"1CD2}दद्वसो।\ उ\accentmark{27}{\char"1CD2}\kern0.15emभ\accentmark{27}{\char"1CD2}याहस्त्याभा\accentmark{22}{\char"0951}रा \mbox{॥ ४\hspace{0pt}॥} \\
श्रू\accentmark{27}{\char"1CD2}\kern0.15emधीहा\accentmark{27}{\char"1CD2}व\accentmark{22}{\char"0951}न्तीरश्याः\accentmark{22}{\char"0951}।\ इन्द्राय\accentmark{27}{\char"1CD2}स्त्वासापर्या\accentmark{27}{\char"1CD2}\kern0.15emति।\ सू\accentmark{27}{\char"1CD2}\kern0.15emर्वी\accentmark{20}{\char"1CF9}र्या\accentmark{22}{\char"0951}स्यागोमा\accentmark{22}{\char"0951}\kern0.15emतः।\ रा\accentmark{27}{\char"1CD2}\kern0.15emया\accentmark{27}{\char"1CD2}स्पूर्धी\accentmark{22}{\char"0951}\kern0.15emमाहं\accentmark{27}{\char"1CD2}आ\accentmark{22}{\char"0951}सि \mbox{॥ ५\hspace{0pt}॥} \\
आ\accentmark{20}{\char"1CF8}सा\accentmark{22}{\char"0951}\kern0.15emवीसो\accentmark{27}{\char"1CD2}\kern0.15emमा\accentmark{27}{\char"1CD2}इन्द्रते।\ शा\accentmark{27}{\char"1CD2}वी\accentmark{22}{\char"0951}ष्ठधृष्णा\accentmark{27}{\char"1CD2}वाग\accentmark{22}{\char"0951}हि।\ 
आत्वा॑\accentmark{22}{\char"0951}प्रणत्विन्द्रियं।\ रा\accentmark{27}{\char"1CD2}\kern0.15emजस्सू\accentmark{27}{\char"1CD2}र्यो\accentmark{22}{\char"0951}नारश्मी\accentmark{27}{\char"1CD2}भीः \mbox{॥ ६\hspace{0pt}॥} \\
ए\accentmark{20}{\char"1CF8}न्द्रायाहिहारी\accentmark{22}{\char"0951}\kern0.15emभिः।\ ऊ\accentmark{27}{\char"1CD2}\kern0.15emपा\accentmark{27}{\char"1CD2}कण्वा\accentmark{22}{\char"0951}स्यसूष्ट्रतिं।\ 
देवा\accentmark{27}{\char"1CD2}\kern0.15emआमू\accentmark{27}{\char"1CD2}ष्याशासातः।\ दीवं\accentmark{27}{\char"1CD2}\kern0.15emयाया\accentmark{27}{\char"1CD2}दिवावसो \mbox{॥ ७\hspace{0pt}॥} \\
आत्वागी\accentmark{20}{\char"1CF9}रो\accentmark{22}{\char"0951}राथीरिव।\ अ\accentmark{20}{\char"1CF8}स्तु\accentmark{22}{\char"0951}स्सूते\accentmark{27}{\char"1CD2}षू\accentmark{22}{\char"0951}गिर्वणः।\ 
आभीत्वासामानूषत।\ गा\accentmark{20}{\char"1CF8}वो\accentmark{22}{\char"0951}वत्सन्ना\accentmark{27}{\char"1CD2}\kern0.15emधेना\accentmark{27}{\char"1CD2}वाः \mbox{॥ ८\hspace{0pt}॥} \\
ए\accentmark{27}{\char"1CD2}तोन्वी\accentmark{22}{\char"0951}न्द्रंस्ता\accentmark{27}{\char"1CD2}वामा।\ शुद्धं\accentmark{27}{\char"1CD2}शुद्धे\accentmark{22}{\char"0951}नासाम्ना।\ शुद्धै\accentmark{22}{\char"0951}रुक्थै\accentmark{20}{\char"1CF9}र्वावृध्वां\accentmark{27}{\char"1CD2}सं।\ शुद्धैराशी\accentmark{22}{\char"0951}र्वान्ममक्तु \mbox{॥ ९\hspace{0pt}॥} \\
योरा\accentmark{27}{\char"1CD2}\kern0.15emयिं\accentmark{20}{\char"1CF9}वोरायिं\accentmark{22}{\char"0951}तामः।\ योद्यु\accentmark{27}{\char"1CD2}म्नैद्यु\accentmark{27}{\char"1CD2}\kern0.15emम्ना\accentmark{27}{\char"1CD2}वाक्तमः।\ सोम\accentmark{22}{\char"0951}स्सूतस्सा\accentmark{27}{\char"1CD2}इ\accentmark{22}{\char"0951}न्द्रते।\ अस्ति\accentmark{27}{\char"1CD2}स्वाधा\accentmark{22}{\char"0951}पातेमा\accentmark{27}{\char"1CD2}दाः \mbox{॥ १०\hspace{0pt}॥} \\
\mbox{॥ इति चतुर्थः खण्डः\hspace{0pt}॥} \\ 
प्र\accentmark{20}{\char"1CF8}त्य\accentmark{22}{\char"0951}स्मैपी\accentmark{27}{\char"1CD2}पी\accentmark{22}{\char"0951}षते।\ वि\accentmark{27}{\char"1CD2}श्वा\accentmark{22}{\char"0951}नीविदू\accentmark{27}{\char"1CD2}षे\accentmark{22}{\char"0951}भरा।\ 
अ\accentmark{20}{\char"1CF8}रं\accentmark{22}{\char"0951}\kern0.15emगमा\accentmark{27}{\char"1CD2}\kern0.15emयाज\accentmark{27}{\char"1CD2}ग्मा\accentmark{22}{\char"0951}\kern0.15emये।\ आ\accentmark{27}{\char"1CD2}पा\accentmark{22}{\char"0951}श्चाद्दाघ्वाने\accentmark{27}{\char"1CD2}नाराः \mbox{॥ १\hspace{0pt}॥} \\
आ\accentmark{27}{\char"1CD2}नोवयोवयाश्शायं\accentmark{27}{\char"1CD2}।\ माहा\accentmark{22}{\char"0951}न्तंगंभारेष्ठाम्।\ 
माहा\accentmark{27}{\char"1CD2}न्तं\accentmark{22}{\char"0951}पूर्वी\accentmark{22}{\char"0951}नेष्टां\accentmark{27}{\char"1CD2}।\ उग्रं\accentmark{27}{\char"1CD2}\kern0.15emवाचोआ\accentmark{27}{\char"1CD2}पा\accentmark{22}{\char"0951}वधीः \mbox{॥ २\hspace{0pt}॥} \\
आ\accentmark{27}{\char"1CD2}त्वाराथं\accentmark{22}{\char"0951}याथोता\accentmark{27}{\char"1CD2}ये।\ सुम्नाया\accentmark{22}{\char"0951}वर्क्तया\accentmark{22}{\char"0951}मसि।\ तु\accentmark{27}{\char"1CD2}विकूर्मि\accentmark{22}{\char"0951}मृतीषाहं।\ इ\accentmark{27}{\char"1CD2}न्द्रं\accentmark{22}{\char"0951}शवीष्ठासा\accentmark{27}{\char"1CD2}त्पा\accentmark{22}{\char"0951}तिम् \mbox{॥ ३\hspace{0pt}॥} \\
सा\accentmark{27}{\char"1CD2}पूर्व्यो\accentmark{27}{\char"1CD2}\kern0.15emमाहो\accentmark{27}{\char"1CD2}नां।\ वेनःक्रा\accentmark{27}{\char"1CD2}तूभिरानजे।\ 
यस्या\accentmark{27}{\char"1CD2}द्वारामा\accentmark{20}{\char"1CF9}नूः\accentmark{22}{\char"0951}\kern0.15emपीता\accentmark{27}{\char"1CD2}\kern0.15em।\ देवे\accentmark{27}{\char"1CD2}\kern0.15emषूधी\accentmark{27}{\char"1CD2}याआनाजे \mbox{॥ ४\hspace{0pt}॥} \\
या\accentmark{27}{\char"1CD2}\kern0.15emदीवा\accentmark{20}{\char"1CF9}ह\accentmark{22}{\char"0951}न्त्याशा\accentmark{27}{\char"1CD2}वाः।\ भ्राज\accentmark{22}{\char"0951}मानाराथेष्वा\accentmark{27}{\char"1CD2}।\ 
पीब\accentmark{22}{\char"0951}न्तोमादीरम्माधू\accentmark{22}{\char"0951}।\ तत्रश्रा\accentmark{27}{\char"1CD2}वां\accentmark{22}{\char"0951}सिकृण्वते \mbox{॥ ५\hspace{0pt}॥} \\
त्या\accentmark{20}{\char"1CF8}मू\accentmark{22}{\char"0951}वोअप्रा\accentmark{22}{\char"0951}हणं।\ गृणीषे\accentmark{20}{\char"1CF9}शा\accentmark{20}{\char"1CF9}वासस्पा\accentmark{27}{\char"1CD2}तिम्।\ इन्द्रं\accentmark{22}{\char"0951}विश्वासाहन्नारं।\ शाची\accentmark{22}{\char"0951}ष्ठंविश्वा\accentmark{22}{\char"0951}वे\accentmark{22}{\char"0951}दसम् \mbox{॥ ६\hspace{0pt}॥} \\
दधिक्रा\accentmark{27}{\char"1CD2}विण्णो\accentmark{22}{\char"0951}अकार्षं।\ जिष्णोरश्वा\accentmark{22}{\char"0951}स्यावाजीनाः\accentmark{22}{\char"0951}।\ सूराभी\accentmark{27}{\char"1CD2}\kern0.15emनोमू\accentmark{27}{\char"1CD2}खा\accentmark{22}{\char"0951}करात्।\ प्राणआयूंषितारिषात् \mbox{॥ ७\hspace{0pt}॥} \\
पूरांभिन्दू\accentmark{27}{\char"1CD2}रियूवाः\accentmark{27}{\char"1CD2}कविः।\ आमी\accentmark{22}{\char"0951}तौजाअजायत।\ इ\accentmark{27}{\char"1CD2}न्द्रो\accentmark{20}{\char"1CF9}विश्वास्याक\accentmark{27}{\char"1CD2}र्मा\accentmark{22}{\char"0951}णः।\ धर्ता\accentmark{22}{\char"0951}वज्रीपू\accentmark{22}{\char"0951}रूष्ट्रतः \mbox{॥ ८\hspace{0pt}॥} \\
 \mbox{॥ इति पञ्चमः खण्डः\hspace{0pt}॥} \\ 
प्रप्रा\accentmark{22}{\char"0951}वास्त्रष्टू\accentmark{27}{\char"1CD2}भामीषं।\ वन्द\accentmark{20}{\char"1CF9}द्वी\accentmark{22}{\char"0951}\kern0.15emराये\accentmark{27}{\char"1CD2}न्दा\accentmark{22}{\char"0951}वे।\ 
धी\accentmark{27}{\char"1CD2}\kern0.15emया\accentmark{20}{\char"1CF9}वो\accentmark{22}{\char"0951}\kern0.15emमेधा\accentmark{27}{\char"1CD2}सातये।\ पूरन्ध्या\accentmark{22}{\char"0951}वीवासती \mbox{॥ १\hspace{0pt}॥} \\
कश्यापा\accentmark{22}{\char"0951}स्यास्वर्वीदाः।\ या\accentmark{27}{\char"1CD2}आहुस्सायूजावी\accentmark{22}{\char"0951}\kern0.15emति।\ या\accentmark{27}{\char"1CD2}योर्वीश्वामा\accentmark{20}{\char"1CF9}पी\accentmark{22}{\char"0951}व्राते।\ यज्ञ\accentmark{20}{\char"1CF9}न्धी\accentmark{20}{\char"1CF9}रो\accentmark{22}{\char"0951}\kern0.15emनीचा\accentmark{27}{\char"1CD2}य्या \mbox{॥ २\hspace{0pt}॥} \\
विश्वा\accentmark{27}{\char"1CD2}नरा\accentmark{22}{\char"0951}स्यवास्पातिं।\ आनानतास्याशा\accentmark{27}{\char"1CD2}वासः।\ एवै\accentmark{22}{\char"0951}श्वचर्षा\accentmark{22}{\char"0951}\kern0.15emणीनां\accentmark{27}{\char"1CD2}।\ ऊतीहू\accentmark{22}{\char"0951}वेराथानाम् \mbox{॥ ३\hspace{0pt}॥} \\
सा\accentmark{27}{\char"1CD2}\kern0.15emखाय\accentmark{20}{\char"1CF9}स्ते\accentmark{22}{\char"0951}दीवोना\accentmark{22}{\char"0951}राः।\ धीया\accentmark{20}{\char"1CF9}मर्तास्याश\accentmark{27}{\char"1CD2}र्मा\accentmark{22}{\char"0951}\kern0.15emतः।\ ऊ\accentmark{27}{\char"1CD2}\kern0.15emतीसा\accentmark{20}{\char"1CF9}बृ\accentmark{20}{\char"1CF9}हातो\accentmark{27}{\char"1CD2}दीवः।\ द्वीषो\accentmark{27}{\char"1CD2}अंहोना\accentmark{27}{\char"1CD2}ता\accentmark{22}{\char"0951}रती \mbox{॥ ४\hspace{0pt}॥} \\
अ\accentmark{27}{\char"1CD2}र्चातप्रा\accentmark{27}{\char"1CD2}र्चातानरः।\ प्री\accentmark{27}{\char"1CD2}यामे\accentmark{22}{\char"0951}\kern0.15emधासोअ\accentmark{27}{\char"1CD2}र्चातः।\ 
अर्चन्तूपूत्राका\accentmark{27}{\char"1CD2}ऊता।\ पू\accentmark{27}{\char"1CD2}रामिधृष्णुंठ\accentmark{22}{\char"0951}वार्चत \mbox{॥ ५\hspace{0pt}॥} \\
उक्थ\accentmark{20}{\char"1CF9}मिन्द्रा\accentmark{22}{\char"0951}\kern0.15emयाशं\accentmark{27}{\char"1CD2}स्यं\accentmark{22}{\char"0951}\kern0.15em।\ व\accentmark{27}{\char"1CD2}र्धानंपूरूनिष्षिधे।\ 
शक्रो\accentmark{20}{\char"1CF9}या\accentmark{20}{\char"1CF9}था\accentmark{22}{\char"0951}सूते\accentmark{22}{\char"0951}षू\accentmark{22}{\char"0951}णः।\ रारा\accentmark{20}{\char"1CF9}णात्सख्ये\accentmark{27}{\char"1CD2}षु\accentmark{22}{\char"0951}च \mbox{॥ ६\hspace{0pt}॥} \\
विभोष्टइन्द्ररा\accentmark{27}{\char"1CD2}धा\accentmark{22}{\char"0951}सः।\ विभ्वीरातीश्शातक्रतो।\ 
अ\accentmark{27}{\char"1CD2}था\accentmark{22}{\char"0951}नोविश्वचर्ष\accentmark{22}{\char"0951}णे।\ द्युम्नं\accentmark{27}{\char"1CD2}सूदत्रमंहया \mbox{॥ ७\hspace{0pt}॥} \\
वा\accentmark{27}{\char"1CD2}या\accentmark{22}{\char"0951}श्चीक्तेपा\accentmark{27}{\char"1CD2}तत्रीणाः\accentmark{22}{\char"0951}।\ द्वीपाश्चा\accentmark{27}{\char"1CD2}तू\accentmark{22}{\char"0951}ष्पदर्जूनी।\ 
ऊषप्रा\accentmark{20}{\char"1CF9}र\accentmark{22}{\char"0951}न्मृतूंरा\accentmark{27}{\char"1CD2}नू\accentmark{22}{\char"0951}।\ दीवो\accentmark{20}{\char"1CF9}अ\accentmark{22}{\char"0951}न्तभ्यस्पारी\accentmark{22}{\char"0951} \mbox{॥ ८\hspace{0pt}॥} \\
आ\accentmark{27}{\char"1CD2}\kern0.15emमीये\accentmark{27}{\char"1CD2}\kern0.15emदेवा\accentmark{27}{\char"1CD2}\kern0.15emस्था\accentmark{27}{\char"1CD2}ना\accentmark{22}{\char"0951}।\ मध्येये\accentmark{20}{\char"1CF9}रो\accentmark{20}{\char"1CF9}चा\accentmark{22}{\char"0951}नेदीवः।\ 
कद्व\accentmark{20}{\char"1CF8}ऋ\accentmark{22}{\char"0951}तंकादमृ\accentmark{27}{\char"1CD2}\kern0.15emतं।\ का\accentmark{27}{\char"1CD2}प्रत्ना\accentmark{27}{\char"1CD2}आहू\accentmark{22}{\char"0951}तिः \mbox{॥ ९\hspace{0pt}॥} \\
ऋ\accentmark{27}{\char"1CD2}\kern0.15emचं\accentmark{27}{\char"1CD2}\kern0.15emसा\accentmark{27}{\char"1CD2}मायजामहे।\ याम्यांकर्माणीकृ\accentmark{22}{\char"0951}\kern0.15emण्वा\accentmark{27}{\char"1CD2}ते।\ 
वी\accentmark{27}{\char"1CD2}तेसाद\accentmark{22}{\char"0951}सिराजतः।\ य\accentmark{27}{\char"1CD2}\kern0.15emज्ञं\accentmark{27}{\char"1CD2}\kern0.15emदेवे\accentmark{27}{\char"1CD2}षु\accentmark{22}{\char"0951}वक्ष\accentmark{22}{\char"0951}तः \mbox{॥ १०\hspace{0pt}॥} \\
\mbox{॥ इति षष्ठः खण्डः\hspace{0pt}॥} \\  
\mbox{॥ इति असावी पाठः समाप्तः\hspace{0pt}॥} \\ \clearpage
\mbox{॥ अथ ऐन्द्र पाठः\hspace{0pt}॥} \\
विश्वाः\accentmark{27}{\char"1CD2}\kern0.15emपृ\accentmark{27}{\char"1CD2}ता\accentmark{22}{\char"0951}\kern0.15emना\accentmark{27}{\char"1CD2}\kern0.15emआभीभू\accentmark{20}{\char"1CF9}ता\accentmark{22}{\char"0951}रन्नाराः।\ साजू\accentmark{27}{\char"1CD2}स्ताता\accentmark{22}{\char"0951}क्षुरिन्द्रं\accentmark{22}{\char"0951}जजनु\accentmark{20}{\char"1CF9}श्चा\accentmark{22}{\char"0951}राजसे।\ क्र\accentmark{27}{\char"1CD2}त्वेवा\accentmark{20}{\char"1CF9}रेस्थेमन्यामू\accentmark{20}{\char"1CF9}री\accentmark{22}{\char"0951}\kern0.15emमू\accentmark{27}{\char"1CD2}\kern0.15emता।\ उ\accentmark{27}{\char"1CD2}\kern0.15emग्र\accentmark{27}{\char"1CD2}\kern0.15emमो\accentmark{27}{\char"1CD2}जी\accentmark{22}{\char"0951}ष्ठन्तारा\accentmark{27}{\char"1CD2}स\accentmark{22}{\char"0951}न्तारस्वीनम् \mbox{॥ १\hspace{0pt}॥} \\
श्र\accentmark{27}{\char"1CD2}क्ते\accentmark{22}{\char"0951}दधामिप्राथामा\accentmark{20}{\char"1CF9}या\accentmark{22}{\char"0951}\kern0.15emमन्या\accentmark{27}{\char"1CD2}\kern0.15emवे।\ आ\accentmark{27}{\char"1CD2}\kern0.15emह\accentmark{27}{\char"1CD2}न्यद्द\accentmark{27}{\char"1CD2}स्युन्न\accentmark{20}{\char"1CF9}र्यं\accentmark{22}{\char"0951}\kern0.15emवीवे\accentmark{27}{\char"1CD2}रापः।\ ऊ\accentmark{27}{\char"1CD2}भेयत्वा\accentmark{27}{\char"1CD2}रोद\accentmark{22}{\char"0951}सीधावा\accentmark{22}{\char"0951}\kern0.15emतामा\accentmark{27}{\char"1CD2}नू\accentmark{22}{\char"0951}।\ 
भ्या\accentmark{20}{\char"1CF8}सा\accentmark{22}{\char"0951}क्तेशू\accentmark{27}{\char"1CD2}ष्मात्पृथि\accentmark{27}{\char"1CD2}वी\accentmark{22}{\char"0951}ची\accentmark{22}{\char"0951}दद्रीवः \mbox{॥ २\hspace{0pt}॥} \\
सामेताविश्वा\accentmark{20}{\char"1CF9}ओज\accentmark{22}{\char"0951}सापा\accentmark{20}{\char"1CF9}तिन्दीवः\accentmark{27}{\char"1CD2}\kern0.15em।\ य\accentmark{27}{\char"1CD2}\kern0.15emए\accentmark{27}{\char"1CD2}\kern0.15emका\accentmark{20}{\char"1CF9}इ\accentmark{20}{\char"1CF8}त्भू\accentmark{22}{\char"0951}\kern0.15emरा\accentmark{27}{\char"1CD2}तीथिर्जना\accentmark{22}{\char"0951}नाम्।\ सा\accentmark{27}{\char"1CD2}पूर्व्यो\accentmark{27}{\char"1CD2}नूता\accentmark{22}{\char"0951}नामाजी\accentmark{27}{\char"1CD2}गीष\accentmark{22}{\char"0951}न्
तंवार्क्तानीरानूवावृताएकाईत् \mbox{॥ ३\hspace{0pt}॥} \\
ईमेताइन्द्रातेवा\accentmark{22}{\char"0951}यंपूरूष्टुत।\ येत्वाराभ्याचारामसिप्रभूवसो।\ नाहित्वादन्योगीर्वाणोगीरस्साघात्।\ क्षोणीरीवप्रातीताद्धरीयानो\accentmark{27}{\char"1CD2}\kern0.15emवा\accentmark{27}{\char"1CD2}चाः\accentmark{22}{\char"0951} \mbox{॥ ४\hspace{0pt}॥} \\
चऋष\accentmark{22}{\char"0951}णीधृ\accentmark{20}{\char"1CF9}तंम्माघावानाठ\accentmark{22}{\char"0951}मुक्त्यं।\ इ\accentmark{27}{\char"1CD2}न्द्रंगीरोबृहातीरभ्यानूषत।\ वा\accentmark{27}{\char"1CD2}\kern0.15emवृधा\accentmark{27}{\char"1CD2}नंपू\accentmark{22}{\char"0951}रूहूतं\accentmark{20}{\char"1CF9}सूवृक्ती\accentmark{22}{\char"0951}भीः।\ आ\accentmark{20}{\char"1CF9}मा\accentmark{22}{\char"0951}र्क्त्यंजरा\accentmark{22}{\char"0951}माणन्दीवे\accentmark{27}{\char"1CD2}दिवे \mbox{॥ ५\hspace{0pt}॥} \\
अच्छावाइन्द्रम्माता\accentmark{20}{\char"1CF9}यास्वर्यू\accentmark{27}{\char"1CD2}वाः\accentmark{22}{\char"0951}\kern0.15em।\ स\accentmark{27}{\char"1CD2}\kern0.15emध्री\accentmark{27}{\char"1CD2}चीर्विश्वा\accentmark{22}{\char"0951}\kern0.15emऊशाती\accentmark{27}{\char"1CD2}रा\accentmark{22}{\char"0951}नूषत्।\ पा\accentmark{27}{\char"1CD2}री\accentmark{22}{\char"0951}ष्वजन्ताज\accentmark{20}{\char"1CF9}ना\accentmark{20}{\char"1CF9}यो\accentmark{22}{\char"0951}याथापातिं।\ म\accentmark{27}{\char"1CD2}र्यान्ना\accentmark{27}{\char"1CD2}शुन्ध्यु\accentmark{27}{\char"1CD2}मा॑\accentmark{22}{\char"0951}घा\accentmark{20}{\char"1CF8}वा\accentmark{22}{\char"0951}नामूता\accentmark{27}{\char"1CD2}ये \mbox{॥ ६\hspace{0pt}॥} \\
आ\accentmark{27}{\char"1CD2}\kern0.15emभि\accentmark{27}{\char"1CD2}\kern0.15emत्यं\accentmark{27}{\char"1CD2}मेषंपू\accentmark{22}{\char"0951}रुहूता\accentmark{27}{\char"1CD2}मृग्मी\accentmark{27}{\char"1CD2}यं।\ इन्द्रं\accentmark{22}{\char"0951}गीर्भि\accentmark{22}{\char"0951}र्मादतावस्वोअर्णा\accentmark{22}{\char"0951}\kern0.15emवम्।\ य\accentmark{27}{\char"1CD2}\kern0.15emस्या\accentmark{27}{\char"1CD2}द्यावोना\accentmark{27}{\char"1CD2}\kern0.15emवीचा\accentmark{20}{\char"1CF9}र\accentmark{22}{\char"0951}न्तीमानू\accentmark{22}{\char"0951}\kern0.15emषं।\ भु\accentmark{27}{\char"1CD2}\kern0.15emजे\accentmark{27}{\char"1CD2}\kern0.15emमं\accentmark{27}{\char"1CD2}हीष्ठामाभी\accentmark{27}{\char"1CD2}विप्रा\accentmark{22}{\char"0951}मर्चत \mbox{॥ ७\hspace{0pt}॥} \\
त्यं\accentmark{27}{\char"1CD2}\kern0.15emसूमेष\accentmark{27}{\char"1CD2}म्मा\accentmark{22}{\char"0951}हयास्व\accentmark{27}{\char"1CD2}\kern0.15emर्वी\accentmark{27}{\char"1CD2}दं\accentmark{22}{\char"0951}\kern0.15em।\ शा\accentmark{27}{\char"1CD2}\kern0.15emतं\accentmark{20}{\char"1CF9}यस्या\accentmark{22}{\char"0951}सूभूवाठ\accentmark{22}{\char"0951}\kern0.15emसाक\accentmark{27}{\char"1CD2}मीरा\accentmark{22}{\char"0951}\kern0.15emते।\ अ\accentmark{27}{\char"1CD2}त्यन्ना\accentmark{27}{\char"1CD2}\kern0.15emवा\accentmark{27}{\char"1CD2}जं\accentmark{22}{\char"0951}हवानस्यादंरा\accentmark{27}{\char"1CD2}थं\accentmark{22}{\char"0951}\kern0.15em।\ ए\accentmark{27}{\char"1CD2}न्द्रं\accentmark{20}{\char"1CF8}ववृत्यामा\accentmark{27}{\char"1CD2}वा\accentmark{22}{\char"0951}\kern0.15emसेसू\accentmark{27}{\char"1CD2}वृक्तीभीः\accentmark{22}{\char"0951} \mbox{॥ ८\hspace{0pt}॥} \\
घृ\accentmark{27}{\char"1CD2}\kern0.15emता\accentmark{20}{\char"1CF9}वा\accentmark{22}{\char"0951}\kern0.15emतीभू\accentmark{27}{\char"1CD2}वानानामाभि\accentmark{27}{\char"1CD2}श्रीयाः।\ उ\accentmark{27}{\char"1CD2}र्वी\accentmark{22}{\char"0951}पृथ्वी\accentmark{20}{\char"1CF9}माधूदु\accentmark{20}{\char"1CF9}घे\accentmark{22}{\char"0951}\kern0.15emसुपे\accentmark{27}{\char"1CD2}शा\accentmark{22}{\char"0951}सा।\ द्या\accentmark{27}{\char"1CD2}वा\accentmark{22}{\char"0951}पृथिवी\accentmark{27}{\char"1CD2}वारू\accentmark{22}{\char"0951}णस्याध\accentmark{27}{\char"1CD2}र्मा\accentmark{22}{\char"0951}णा।\ विष्का\accentmark{22}{\char"0951}भितेआजरेभू\accentmark{27}{\char"1CD2}रिरेतसा \mbox{॥ ९\hspace{0pt}॥} \\
ऊर्भे\accentmark{20}{\char"1CF9}या\accentmark{20}{\char"1CF9}दिन्द्रारो\accentmark{27}{\char"1CD2}द\accentmark{22}{\char"0951}सि।\ आपःप्रा\accentmark{27}{\char"1CD2}\kern0.15emथोषा\accentmark{27}{\char"1CD2}ई\accentmark{22}{\char"0951}वा।\ माहा\accentmark{27}{\char"1CD2}न्त\accentmark{22}{\char"0951}न्त्वामाही\accentmark{27}{\char"1CD2}नां\accentmark{22}{\char"0951}\kern0.15em।\ स\accentmark{27}{\char"1CD2}म्रा\accentmark{22}{\char"0951}जन्वर्षाणीनां\accentmark{27}{\char"1CD2}\kern0.15em।\ 
दे\accentmark{27}{\char"1CD2}\kern0.15emवी\accentmark{27}{\char"1CD2}जनीत्र\accentmark{22}{\char"0951}यजीजनात्।\ भ\accentmark{27}{\char"1CD2}\kern0.15emद्रा\accentmark{27}{\char"1CD2}\kern0.15emज\accentmark{27}{\char"1CD2}नीत्रयजीजना\accentmark{27}{\char"1CD2}त् \mbox{॥ १०\hspace{0pt}॥} \\
प्रा\accentmark{27}{\char"1CD2}मन्दीने\accentmark{22}{\char"0951}पीतूमाद॑\accentmark{22}{\char"0951}र्चातावा\accentmark{27}{\char"1CD2}चाः\accentmark{22}{\char"0951}।\ याःकृष्णागर्भा\accentmark{22}{\char"0951}णीरा\accentmark{20}{\char"1CF9}ह\accentmark{22}{\char"0951}न्नृजीश्वा\accentmark{22}{\char"0951}\kern0.15emना।\ अव\accentmark{27}{\char"1CD2}स्यावो\accentmark{27}{\char"1CD2}\kern0.15emवृ\accentmark{27}{\char"1CD2}षा\accentmark{22}{\char"0951}\kern0.15emणंव\accentmark{27}{\char"1CD2}ज्रा\accentmark{22}{\char"0951}दक्षिणं।\ मा\accentmark{27}{\char"1CD2}रूत्व\accentmark{22}{\char"0951}न्तंसख्या\accentmark{27}{\char"1CD2}या\accentmark{22}{\char"0951}हुवेमही \mbox{॥ ११\hspace{0pt}॥} \\
\mbox{॥ इति प्रथमः खण्डः\hspace{0pt}॥} \\ 
स्वादो\accentmark{27}{\char"1CD2}रित्था\accentmark{20}{\char"1CF9}वी\accentmark{22}{\char"0951}\kern0.15emषूवा\accentmark{27}{\char"1CD2}ताः।\ मा\accentmark{27}{\char"1CD2}धोः\accentmark{22}{\char"0951}पिबन्तिगौर्याःठ\accentmark{22}{\char"0951}\kern0.15em।\ या\accentmark{27}{\char"1CD2}\kern0.15emइ\accentmark{27}{\char"1CD2}न्द्रे\accentmark{22}{\char"0951}\kern0.15emणासा\accentmark{27}{\char"1CD2}याववा\accentmark{22}{\char"0951}रिः।\ वृष्णामाद\accentmark{22}{\char"0951}न्तीशो\accentmark{27}{\char"1CD2}\kern0.15emभा\accentmark{27}{\char"1CD2}\kern0.15emथा।\ व\accentmark{27}{\char"1CD2}स्वीरा\accentmark{20}{\char"1CF9}नू\accentmark{22}{\char"0951}स्वारा\accentmark{27}{\char"1CD2}ज्य\accentmark{22}{\char"0951}म्\mbox{॥ १\hspace{0pt}॥} \\
इत्था\accentmark{27}{\char"1CD2}\kern0.15emहीसो\accentmark{27}{\char"1CD2}\kern0.15emमाइ\accentmark{27}{\char"1CD2}\kern0.15emन्मा\accentmark{27}{\char"1CD2}दाः\accentmark{22}{\char"0951}।\ ब्र\accentmark{20}{\char"1CF8}ह्मा\accentmark{22}{\char"0951}\kern0.15emचाका\accentmark{27}{\char"1CD2}\kern0.15emरावा\accentmark{27}{\char"1CD2}र्धा\accentmark{22}{\char"0951}नं।\ शावी\accentmark{22}{\char"0951}ष्ठवाज्रिन्नो\accentmark{22}{\char"0951}ज\accentmark{22}{\char"0951}सा।\ पृ\accentmark{20}{\char"1CF8}थिव्या\accentmark{20}{\char"1CF9}नि\accentmark{20}{\char"1CF9}श्शा\accentmark{22}{\char"0951}\kern0.15emशाआ\accentmark{27}{\char"1CD2}हिं।\ अर्चन्ना\accentmark{20}{\char"1CF9}नूस्वारा\accentmark{27}{\char"1CD2}ज्य\accentmark{22}{\char"0951}म्\mbox{॥ २\hspace{0pt}॥} \\
इ\accentmark{27}{\char"1CD2}न्द्रोमा\accentmark{27}{\char"1CD2}दायवावृधे।\ शावा\accentmark{22}{\char"0951}सेवृत्रा\accentmark{20}{\char"1CF8}हा\accentmark{27}{\char"1CD2}\kern0.15emनृ\accentmark{27}{\char"1CD2}र्भीः\accentmark{22}{\char"0951}\kern0.15em।\ त\accentmark{27}{\char"1CD2}मिन्माहा\accentmark{27}{\char"1CD2}त्स्वाजीषु।\ ऊतीमर्भेहवामहे।\ 
सा\accentmark{20}{\char"1CF8}वा\accentmark{20}{\char"1CF8}जे\accentmark{22}{\char"0951}षुप्रा\accentmark{27}{\char"1CD2}नो\accentmark{22}{\char"0951}विषात्\mbox{॥ ३\hspace{0pt}॥} \\
इन्द्रातू\accentmark{27}{\char"1CD2}\kern0.15emभ्या\accentmark{27}{\char"1CD2}मीद\accentmark{22}{\char"0951}द्रीवः।\ आ\accentmark{27}{\char"1CD2}नु\accentmark{22}{\char"0951}क्तंवज्रिन्वीर्यं\accentmark{22}{\char"0951}।\ 
यद्धत्यं\accentmark{27}{\char"1CD2}\kern0.15emमायी\accentmark{20}{\char"1CF9}नं\accentmark{22}{\char"0951}मृगम्।\ ता\accentmark{27}{\char"1CD2}\kern0.15emवत्य\accentmark{27}{\char"1CD2}म्माययावा\accentmark{27}{\char"1CD2}धीः।\ अर्चन्ना\accentmark{20}{\char"1CF9}नू\accentmark{22}{\char"0951}स्वारा\accentmark{27}{\char"1CD2}ज्य\accentmark{22}{\char"0951}म्\mbox{॥ ४\hspace{0pt}॥} \\
प्रे\accentmark{27}{\char"1CD2}ह्याभीठ\accentmark{22}{\char"0951}हीधृष्णूहि\accentmark{27}{\char"1CD2}।\ नातेवाज्रोनी\accentmark{27}{\char"1CD2}यंसते।\ 
इन्द्रानृम्णं\accentmark{20}{\char"1CF9}ही\accentmark{22}{\char"0951}तेशावाः\accentmark{22}{\char"0951}।\ हानो\accentmark{22}{\char"0951}वृत्र\accentmark{20}{\char"1CF9}ञ्ज\accentmark{20}{\char"1CF9}या\accentmark{22}{\char"0951}आ\accentmark{22}{\char"0951}पः।\ अर्चन्नानू\accentmark{22}{\char"0951}स्वारा\accentmark{27}{\char"1CD2}ज्य\accentmark{22}{\char"0951}म्\mbox{॥ ५\hspace{0pt}॥} \\
या\accentmark{27}{\char"1CD2}दूदीराताआज\accentmark{27}{\char"1CD2}याः\accentmark{22}{\char"0951}\kern0.15em।\ धृ\accentmark{27}{\char"1CD2}\kern0.15emष्णा\accentmark{27}{\char"1CD2}वे\accentmark{22}{\char"0951}धीयातेधानं\accentmark{22}{\char"0951}\kern0.15em।\ युं\accentmark{27}{\char"1CD2}क्ष्वा\accentmark{20}{\char"1CF9}मा\accentmark{22}{\char"0951}\kern0.15emदच्यू\accentmark{27}{\char"1CD2}\kern0.15emताहा\accentmark{27}{\char"1CD2}री।\ कंहा\accentmark{27}{\char"1CD2}\kern0.15emनःकं\accentmark{27}{\char"1CD2}\kern0.15emवा\accentmark{27}{\char"1CD2}सौदधः।\ 
अस्मं\accentmark{20}{\char"1CF9}इ\accentmark{22}{\char"0951}न्द्रावा\accentmark{27}{\char"1CD2}सौ\accentmark{22}{\char"0951}दधः\mbox{॥ ६\hspace{0pt}॥} \\
अ\accentmark{27}{\char"1CD2}\kern0.15emक्ष\accentmark{27}{\char"1CD2}\kern0.15emन्ना\accentmark{27}{\char"1CD2}मी\accentmark{22}{\char"0951}मादन्ताही\accentmark{27}{\char"1CD2}\kern0.15em।\ आ\accentmark{20}{\char"1CF8}वा\accentmark{22}{\char"0951}प्रियाआधूषत।\ अ\accentmark{27}{\char"1CD2}स्तो\accentmark{22}{\char"0951}षातस्वाभा\accentmark{22}{\char"0951}नवः।\ त्रिप्रान्नावीष्ठयामति।\ यो\accentmark{27}{\char"1CD2}जान्विन्द्रातेहारी\accentmark{22}{\char"0951}\kern0.15em\mbox{॥ ७\hspace{0pt}॥} \\
ऊ\accentmark{27}{\char"1CD2}\kern0.15emपोषू\accentmark{20}{\char"1CF9}श्रूणूहीगीराः\accentmark{22}{\char"0951}।\ मा\accentmark{20}{\char"1CF8}घा\accentmark{22}{\char"0951}वन्माताथाइव।\ 
कादा\accentmark{20}{\char"1CF9}ना\accentmark{22}{\char"0951}स्सू\accentmark{22}{\char"0951}नृता\accentmark{22}{\char"0951}वतः।\ काराई\accentmark{27}{\char"1CD2}दा\accentmark{22}{\char"0951}र्थार्यासाइत्।\ यो\accentmark{27}{\char"1CD2}जान्वि\accentmark{22}{\char"0951}न्द्राते\accentmark{27}{\char"1CD2}\kern0.15emहा\accentmark{27}{\char"1CD2}री\accentmark{22}{\char"0951}\mbox{॥ ८\hspace{0pt}॥} \\
चन्द्रा\accentmark{20}{\char"1CF9}मा\accentmark{22}{\char"0951}प्सूआ\accentmark{22}{\char"0951}टट्यन्तारा।\ सूपर्णोधा\accentmark{22}{\char"0951}वतेदीवि\accentmark{27}{\char"1CD2}।\ 
नावो\accentmark{22}{\char"0951}हीरण्यनेमयः।\ पा\accentmark{27}{\char"1CD2}\kern0.15emदं\accentmark{27}{\char"1CD2}\kern0.15emवि\accentmark{27}{\char"1CD2}न्द\accentmark{22}{\char"0951}न्तीवीद्यूताः।\ विक्त\accentmark{20}{\char"1CF9}म्मेअस्य\accentmark{27}{\char"1CD2}\kern0.15emरो\accentmark{27}{\char"1CD2}दसि\mbox{॥ ९\hspace{0pt}॥} \\
प्रा\accentmark{20}{\char"1CF8}तिप्री\accentmark{22}{\char"0951}यातामंरा\accentmark{27}{\char"1CD2}थं\accentmark{22}{\char"0951}।\ वृषाणंवासूवाहा\accentmark{22}{\char"0951}नां।\ स्तोता\accentmark{27}{\char"1CD2}वा\accentmark{22}{\char"0951}मश्विनावृ\accentmark{27}{\char"1CD2}षीः\accentmark{22}{\char"0951}।\ स्तोमेभिर्भूषातिप्राती।\ मध्वीमा\accentmark{27}{\char"1CD2}मा\accentmark{22}{\char"0951}श्रुतंहा\accentmark{27}{\char"1CD2}वम्\mbox{॥ १०\hspace{0pt}॥} \\
\mbox{॥ इति द्वितीयः खण्डः\hspace{0pt}॥} \\ 
आ\accentmark{27}{\char"1CD2}तेअग्नेइधीमही\accentmark{27}{\char"1CD2}।\ घूमन्तं\accentmark{22}{\char"0951}देवाआज\accentmark{27}{\char"1CD2}रं\accentmark{22}{\char"0951}।\ यद्धस्या\accentmark{27}{\char"1CD2}तेपानी\accentmark{22}{\char"0951}यसी।\ सामी\accentmark{27}{\char"1CD2}द्दीदया\accentmark{22}{\char"0951}तिद्याविईळंस्तोतृ\accentmark{27}{\char"1CD2}\kern0.15emभ्यआ\accentmark{27}{\char"1CD2}भारा\mbox{॥ १\hspace{0pt}॥} \\
अग्निन्नस्वा\accentmark{27}{\char"1CD2}वृ\accentmark{22}{\char"0951}क्तीभिः।\ हो\accentmark{27}{\char"1CD2}ता\accentmark{22}{\char"0951}रन्त्वावृणीमहे।\ शीरं\accentmark{20}{\char"1CF9}पा\accentmark{22}{\char"0951}वाका\accentmark{22}{\char"0951}\kern0.15emशोची\accentmark{27}{\char"1CD2}षंवीवोमादे।\ यज्ञे\accentmark{20}{\char"1CF9}षु\accentmark{22}{\char"0951}स्तीर्णाब\accentmark{22}{\char"0951}ऋहीषंवीवाक्षसे\mbox{॥ २\hspace{0pt}॥} \\
माहे\accentmark{20}{\char"1CF9}नो\accentmark{22}{\char"0951}अद्याबो\accentmark{22}{\char"0951}धय।\ ऊ\accentmark{20}{\char"1CF8}षो\accentmark{22}{\char"0951}\kern0.15emराये\accentmark{27}{\char"1CD2}\kern0.15emदिवी\accentmark{27}{\char"1CD2}द्माती।\ या\accentmark{27}{\char"1CD2}\kern0.15emथाचि\accentmark{27}{\char"1CD2}\kern0.15emन्नोआ\accentmark{27}{\char"1CD2}बो\accentmark{22}{\char"0951}धयः।\ सत्या\accentmark{27}{\char"1CD2}श्रावसीवाय्ये\accentmark{27}{\char"1CD2}\kern0.15em।\ 
सू\accentmark{20}{\char"1CF8}जा\accentmark{22}{\char"0951}तेअश्वा\accentmark{22}{\char"0951}सूनृते\mbox{॥ ३ \clearpage
\mbox{॥} \\ भद्र\accentmark{27}{\char"1CD2}\kern0.15emन्नो\accentmark{27}{\char"1CD2}आपी\accentmark{22}{\char"0951}वातय ।\ मानोद\accentmark{20}{\char"1CF9}क्षा\accentmark{22}{\char"0951}\kern0.15emमूत\accentmark{27}{\char"1CD2}क्रातुं\accentmark{22}{\char"0951} ।\ अथा\accentmark{27}{\char"1CD2}तेसख्ये\accentmark{20}{\char"1CF9}अ\accentmark{22}{\char"0951}न्धासावी\accentmark{27}{\char"1CD2}वोमादे ।\ रा\accentmark{27}{\char"1CD2}\kern0.15emणागा\accentmark{27}{\char"1CD2}\kern0.15emवोनया\accentmark{20}{\char"1CF9}वा\accentmark{22}{\char"0951}सेवीवा\accentmark{22}{\char"0951}क्षसे \mbox{॥} \\ ४\hspace{0pt}॥} 
कृ\accentmark{20}{\char"1CF8}त्वामाहं\accentmark{27}{\char"1CD2}आनूष्वाधः\accentmark{27}{\char"1CD2}\kern0.15em।\ भीम\accentmark{27}{\char"1CD2}\kern0.15emआ\accentmark{27}{\char"1CD2}वा\accentmark{22}{\char"0951}वृतेशा\accentmark{27}{\char"1CD2}वाः\accentmark{22}{\char"0951}।\ श्रीय्या\accentmark{20}{\char"1CF9}रिष्वा\accentmark{20}{\char"1CF9}उपाका\accentmark{27}{\char"1CD2}योः।\ नी\accentmark{27}{\char"1CD2}शीप्रीहा\accentmark{27}{\char"1CD2}रीवान्दधे।\ ह\accentmark{20}{\char"1CF8}स्ता\accentmark{27}{\char"1CD2}योर्वज्रा\accentmark{22}{\char"0951}मायासम्\mbox{॥ ५\hspace{0pt}॥} \\
साघातँ\accentmark{20}{\char"1CF9}वृ\accentmark{22}{\char"0951}षाणंरा\accentmark{27}{\char"1CD2}थं\accentmark{22}{\char"0951}।\ आधीतिष्ठाती\accentmark{22}{\char"0951}गोवी\accentmark{22}{\char"0951}दं।\ यःपात्रं\accentmark{22}{\char"0951}हार्यो\accentmark{27}{\char"1CD2}जान\accentmark{22}{\char"0951}म्।\ पूर्णा\accentmark{20}{\char"1CF9}मि\accentmark{22}{\char"0951}न्द्राचि\accentmark{27}{\char"1CD2}के\accentmark{22}{\char"0951}ततिः।\ यो\accentmark{27}{\char"1CD2}जान्विन्द्रातेहा\accentmark{27}{\char"1CD2}री\mbox{॥ ६\hspace{0pt}॥} \\
अ\accentmark{27}{\char"1CD2}ग्निन्त\accentmark{20}{\char"1CF9}म्मन्ये\accentmark{27}{\char"1CD2}योवासूः\accentmark{22}{\char"0951}।\ अस्तंयय्य\accentmark{20}{\char"1CF9}न्ती\accentmark{22}{\char"0951}\kern0.15emधेना\accentmark{27}{\char"1CD2}\kern0.15emवाः।\ अ\accentmark{27}{\char"1CD2}स्ता\accentmark{22}{\char"0951}\kern0.15emम\accentmark{27}{\char"1CD2}र्वतोआशा\accentmark{27}{\char"1CD2}वाः\accentmark{22}{\char"0951}।\ अस्तन्नित्या\accentmark{22}{\char"0951}सोवाजी\accentmark{27}{\char"1CD2}\kern0.15emनाः।\ ई\accentmark{20}{\char"1CF8}षं\accentmark{22}{\char"0951}स्तोतृभ्यआ\accentmark{27}{\char"1CD2}भा\accentmark{22}{\char"0951}रा\mbox{॥ ७\hspace{0pt}॥} \\
नतमं\accentmark{27}{\char"1CD2}हो\accentmark{22}{\char"0951}ना\accentmark{20}{\char"1CF9}दू\accentmark{22}{\char"0951}रीतं।\ देवा\accentmark{22}{\char"0951}सोआष्टामा\accentmark{27}{\char"1CD2}र्क्त्यं।\ 
सा\accentmark{27}{\char"1CD2}\kern0.15emजो\accentmark{27}{\char"1CD2}षा\accentmark{22}{\char"0951}सोयामर्यामा।\ मित्रोनाय\accentmark{22}{\char"0951}न्तीवा\accentmark{20}{\char"1CF9}रूणोआ\accentmark{27}{\char"1CD2}तिद्वी\accentmark{27}{\char"1CD2}षाः \mbox{॥ ८\hspace{0pt}॥} \\
\mbox{॥ इति तृतीयः खण्डः\hspace{0pt}॥} \\
पा\accentmark{27}{\char"1CD2}रिप्राध\accentmark{22}{\char"0951}न्वाइन्द्रा\accentmark{22}{\char"0951}\kern0.15emयसोमा\accentmark{27}{\char"1CD2}।\ स्वादु\accentmark{27}{\char"1CD2}र्मित्रा\accentmark{27}{\char"1CD2}\kern0.15emय।\ पू\accentmark{27}{\char"1CD2}ष्णेभा\accentmark{27}{\char"1CD2}गा\accentmark{22}{\char"0951}य\mbox{॥ १\hspace{0pt}॥} \\
पर्यू\accentmark{20}{\char"1CF9}षु\accentmark{20}{\char"1CF9}प्रा\accentmark{20}{\char"1CF9}ध\accentmark{22}{\char"0951}\kern0.15emन्वा\accentmark{27}{\char"1CD2}\kern0.15emवा\accentmark{27}{\char"1CD2}ज\accentmark{22}{\char"0951}सातये।\ पारी\accentmark{22}{\char"0951}वृ\accentmark{22}{\char"0951}त्रा\accentmark{20}{\char"1CF9}णी\accentmark{22}{\char"0951}\kern0.15emसक्षा\accentmark{27}{\char"1CD2}णीः\accentmark{22}{\char"0951}।\ द्वीष\accentmark{20}{\char"1CF9}स्ता\accentmark{22}{\char"0951}\kern0.15emराधी\accentmark{27}{\char"1CD2}\kern0.15emया\accentmark{20}{\char"1CF8}ऋणाया\accentmark{27}{\char"1CD2}ना\accentmark{22}{\char"0951}ईरसे\mbox{॥ २\hspace{0pt}॥} \\
पा\accentmark{27}{\char"1CD2}वा\accentmark{22}{\char"0951}स्वसोमा\accentmark{27}{\char"1CD2}\kern0.15em।\ माहा\accentmark{20}{\char"1CF9}न्सा\accentmark{22}{\char"0951}मुद्रः।\ पीता\accentmark{27}{\char"1CD2}दे\accentmark{22}{\char"0951}वानां\accentmark{22}{\char"0951}।\ विश्वाभि\accentmark{27}{\char"1CD2}धामा\mbox{॥ ३\hspace{0pt}॥} \\
पा\accentmark{27}{\char"1CD2}वा\accentmark{22}{\char"0951}स्वसोमा\accentmark{27}{\char"1CD2}\kern0.15em।\ माहे\accentmark{27}{\char"1CD2}दक्षा\accentmark{22}{\char"0951}\kern0.15emय।\ अ\accentmark{27}{\char"1CD2}\kern0.15emश्वो\accentmark{27}{\char"1CD2}\kern0.15emना\accentmark{27}{\char"1CD2}निक्तः।\ 
वाजी\accentmark{27}{\char"1CD2}\kern0.15emधा\accentmark{27}{\char"1CD2}नाय\mbox{॥ ४\hspace{0pt}॥} \\
इ\accentmark{27}{\char"1CD2}न्दूः\accentmark{22}{\char"0951}पविष्टा।\ चारूर्मा\accentmark{22}{\char"0951}दा\accentmark{22}{\char"0951}\kern0.15emय।\ आपा\accentmark{27}{\char"1CD2}\kern0.15emमूपा\accentmark{27}{\char"1CD2}स्थे\accentmark{22}{\char"0951}\kern0.15em।\ का\accentmark{27}{\char"1CD2}विर्भा\accentmark{22}{\char"0951}गा\accentmark{22}{\char"0951}य\mbox{॥ ५\hspace{0pt}॥} \\
आ\accentmark{27}{\char"1CD2}नूहित्वा\accentmark{20}{\char"1CF8}सूतं\accentmark{20}{\char"1CF9}सो\accentmark{22}{\char"0951}\kern0.15emमा\accentmark{27}{\char"1CD2}\kern0.15emमा\accentmark{27}{\char"1CD2}दा\accentmark{22}{\char"0951}मसि।\ मा\accentmark{27}{\char"1CD2}\kern0.15emहं\accentmark{27}{\char"1CD2}सा\accentmark{22}{\char"0951}मरियारा\accentmark{27}{\char"1CD2}ज्ये\accentmark{22}{\char"0951}।\ वा\accentmark{20}{\char"1CF8}जं\accentmark{22}{\char"0951}\kern0.15emआभि\accentmark{27}{\char"1CD2}पा\accentmark{22}{\char"0951}वमानप्रागा\accentmark{22}{\char"0951}हसे\mbox{॥ ६\hspace{0pt}॥} \\
का\accentmark{27}{\char"1CD2}ईंव्याठ\accentmark{22}{\char"0951}क्ताः।\ ना\accentmark{27}{\char"1CD2}\kern0.15emरस्सा\accentmark{27}{\char"1CD2}नी\accentmark{22}{\char"0951}ळाः।\ रु\accentmark{27}{\char"1CD2}\kern0.15emद्रा\accentmark{27}{\char"1CD2}\kern0.15emस्साम\accentmark{27}{\char"1CD2}र्याः।\ या\accentmark{27}{\char"1CD2}\kern0.15emथास्व\accentmark{27}{\char"1CD2}श्वाः\accentmark{22}{\char"0951}\kern0.15em\mbox{॥ ७\hspace{0pt}॥} \\
अ\accentmark{27}{\char"1CD2}ग्नेता\accentmark{27}{\char"1CD2}\kern0.15emमद्या\accentmark{27}{\char"1CD2}।\ अश्वन्न\accentmark{27}{\char"1CD2}स्तोमैः\accentmark{22}{\char"0951}\kern0.15em।\ क्रा\accentmark{27}{\char"1CD2}तून्ना\accentmark{27}{\char"1CD2}\kern0.15emभद्रं\accentmark{27}{\char"1CD2}।\ हृदिस्पृ\accentmark{27}{\char"1CD2}शं\accentmark{22}{\char"0951}।\ ऋद्ध्या\accentmark{20}{\char"1CF9}मा\accentmark{22}{\char"0951}\kern0.15emताओ\accentmark{27}{\char"1CD2}हैः\accentmark{22}{\char"0951}\mbox{॥ ८\hspace{0pt}॥} \\
आवि\accentmark{20}{\char"1CF8}र्म\accentmark{22}{\char"0951}र्याआवा\accentmark{22}{\char"0951}जं\accentmark{22}{\char"0951}\kern0.15emवाजी\accentmark{27}{\char"1CD2}नो\accentmark{22}{\char"0951}ग्मन्।\ देवा\accentmark{27}{\char"1CD2}स्या\accentmark{22}{\char"0951}सावीतु\accentmark{27}{\char"1CD2}स्सा\accentmark{22}{\char"0951}वे।\ स्वर्गा\accentmark{22}{\char"0951}अ\accentmark{20}{\char"1CF9}र्वन्तोजा\accentmark{27}{\char"1CD2}यर्या\accentmark{22}{\char"0951}ता\mbox{॥ ९\hspace{0pt}॥} \\
पावा\accentmark{22}{\char"0951}स्वसोम।\ द्युम्नी\accentmark{20}{\char"1CF9}सूधारः\accentmark{22}{\char"0951}\kern0.15em।\ माही\accentmark{27}{\char"1CD2}ना\accentmark{22}{\char"0951}मानूपूर्व्याः\accentmark{27}{\char"1CD2}\mbox{॥ १०\hspace{0pt}॥} \\
\mbox{॥ इति चतुर्थः खण्डः\hspace{0pt}॥} \\ 
इन्द्रा\accentmark{22}{\char"0951}\kern0.15emसूते\accentmark{27}{\char"1CD2}\kern0.15emषुसो\accentmark{27}{\char"1CD2}मे\accentmark{22}{\char"0951}षु।\ क्रातुं\accentmark{22}{\char"0951}पुनीषाउक्त्थं।\ वीदे\accentmark{27}{\char"1CD2}\kern0.15emवृधा\accentmark{27}{\char"1CD2}\kern0.15emस्याद\accentmark{27}{\char"1CD2}क्षा\accentmark{22}{\char"0951}स्यामाहं\accentmark{27}{\char"1CD2}\kern0.15emहिषः\accentmark{27}{\char"1CD2}\mbox{॥ १\hspace{0pt}॥} \\
तमुवांठ\accentmark{22}{\char"0951}भिप्रा\accentmark{27}{\char"1CD2}गा\accentmark{22}{\char"0951}यत।\ पू\accentmark{22}{\char"0951}रुहूतं\accentmark{27}{\char"1CD2}पू\accentmark{22}{\char"0951}रूष्टुतं\accentmark{27}{\char"1CD2}।\ इन्द्रं\accentmark{22}{\char"0951}गी\accentmark{20}{\char"1CF9}र्भीस्ता\accentmark{22}{\char"0951}वीषामा\accentmark{27}{\char"1CD2}वी\accentmark{22}{\char"0951}वासता\mbox{॥ २\hspace{0pt}॥} \\
तन्तेमा\accentmark{27}{\char"1CD2}दं\accentmark{22}{\char"0951}गृणीमसि।\ वृ\accentmark{27}{\char"1CD2}षा\accentmark{22}{\char"0951}णंप्रक्षु\accentmark{20}{\char"1CF9}सासा\accentmark{22}{\char"0951}हिं।\ 
ऊलोकाकूत्नू\accentmark{27}{\char"1CD2}माद्रीवो\accentmark{27}{\char"1CD2}हारिश्रीय\accentmark{22}{\char"0951}म्\mbox{॥ ३\hspace{0pt}॥} \\य\accentmark{27}{\char"1CD2}\kern0.15emत्सो\accentmark{27}{\char"1CD2}मा\accentmark{22}{\char"0951}मिन्द्रावि\accentmark{27}{\char"1CD2}ष्णा\accentmark{22}{\char"0951}वे।\ यद्वा\accentmark{22}{\char"0951}घात्रीता\accentmark{27}{\char"1CD2}आप्त्ये\accentmark{27}{\char"1CD2}।\ 
यद्वा\accentmark{20}{\char"1CF9}मा\accentmark{22}{\char"0951}रूत्सूम\accentmark{20}{\char"1CF9}न्दसेसमिन्दुभिः\mbox{॥ ४\hspace{0pt}॥} \\
ए\accentmark{27}{\char"1CD2}\kern0.15emदुमा\accentmark{20}{\char"1CF9}धोर्मा\accentmark{22}{\char"0951}दिन्ता\accentmark{22}{\char"0951}\kern0.15emरं।\ सि\accentmark{27}{\char"1CD2}\kern0.15emञ्चा\accentmark{20}{\char"1CF9}ध्व\accentmark{22}{\char"0951}\kern0.15emर्योअ\accentmark{27}{\char"1CD2}न्धा\accentmark{22}{\char"0951}\kern0.15emसा।\ 
ए\accentmark{27}{\char"1CD2}\kern0.15emवा\accentmark{27}{\char"1CD2}\kern0.15emहीवीर\accentmark{27}{\char"1CD2}\kern0.15emस्ता\accentmark{27}{\char"1CD2}वा\accentmark{22}{\char"0951}तेसादा\accentmark{27}{\char"1CD2}वृ\accentmark{22}{\char"0951}धः\mbox{॥ ५\hspace{0pt}॥} \\
ए\accentmark{27}{\char"1CD2}न्दूमिन्द्रा\accentmark{22}{\char"0951}यसिञ्चत।\ पी\accentmark{27}{\char"1CD2}बातासोम्यम्माधू\accentmark{22}{\char"0951}।\ 
प्रारा\accentmark{27}{\char"1CD2}धां\accentmark{22}{\char"0951}सिचोदयतामहीत्वाना\mbox{॥ ६\hspace{0pt}॥} \\
ए\accentmark{27}{\char"1CD2}तोन्विन्द्रंस्ता\accentmark{27}{\char"1CD2}वा\accentmark{22}{\char"0951}मा।\ सा\accentmark{20}{\char"1CF8}खा\accentmark{22}{\char"0951}यस्तोम्यन्ना\accentmark{27}{\char"1CD2}रं\accentmark{22}{\char"0951}।\ कृष्टिर्यो\accentmark{20}{\char"1CF9}विश्वा\accentmark{22}{\char"0951}अभ्यास्ते\accentmark{27}{\char"1CD2}\kern0.15emकाई\accentmark{27}{\char"1CD2}त्\mbox{॥ ७\hspace{0pt}॥} \\
इ\accentmark{20}{\char"1CF8}न्द्रा\accentmark{22}{\char"0951}यासामा\accentmark{22}{\char"0951}\kern0.15emगाय\accentmark{27}{\char"1CD2}\kern0.15emता।\ वि\accentmark{27}{\char"1CD2}प्रा\accentmark{22}{\char"0951}याबृहातेबृहा\accentmark{27}{\char"1CD2}त्।\ ब्र\accentmark{27}{\char"1CD2}ह्माकृ\accentmark{27}{\char"1CD2}ते\accentmark{22}{\char"0951}वीपश्ची\accentmark{27}{\char"1CD2}तेपानस्या\accentmark{27}{\char"1CD2}वे\mbox{॥ ८\hspace{0pt}॥} \\
यए\accentmark{27}{\char"1CD2}\kern0.15emकाई\accentmark{27}{\char"1CD2}\kern0.15emद्वीद\accentmark{27}{\char"1CD2}या\accentmark{22}{\char"0951}\kern0.15emते।\ वा\accentmark{27}{\char"1CD2}सूमर्क्ता\accentmark{22}{\char"0951}यादाशू\accentmark{27}{\char"1CD2}षे\accentmark{22}{\char"0951}।\ 
ई\accentmark{20}{\char"1CF8}शा\accentmark{22}{\char"0951}\kern0.15emनोअ\accentmark{27}{\char"1CD2}प्रातिष्कुता।\ इन्द्रो\accentmark{20}{\char"1CF9}अङ्गा\mbox{॥ ९\hspace{0pt}॥} \\
सा\accentmark{20}{\char"1CF8}खा\accentmark{22}{\char"0951}\kern0.15emय\accentmark{27}{\char"1CD2}\kern0.15emआ\accentmark{27}{\char"1CD2}शी\accentmark{22}{\char"0951}षामहे।\ ब्रह्मे\accentmark{27}{\char"1CD2}न्द्रा\accentmark{22}{\char"0951}यावज्रीणे।\ स्तूषा\accentmark{27}{\char"1CD2}ऊ\accentmark{22}{\char"0951}षूवोनृतामा\accentmark{22}{\char"0951}याधृष्णा\accentmark{22}{\char"0951}वे\mbox{॥ १०\hspace{0pt}॥} \\
 \mbox{॥ इति पञ्चमः खण्डः\hspace{0pt}॥} \\ 
ए\accentmark{27}{\char"1CD2}न्द्रा\accentmark{22}{\char"0951}नोगधिप्रिया।\ सत्रा\accentmark{22}{\char"0951}जितगोह्या।\ गीरि\accentmark{27}{\char"1CD2}र्न्ना\accentmark{22}{\char"0951}विश्वातः\accentmark{22}{\char"0951}पृथूःपा\accentmark{22}{\char"0951}तीर्दीवः\mbox{॥ १\hspace{0pt}॥} \\
यस्यत्य\accentmark{20}{\char"1CF9}च्छां\accentmark{20}{\char"1CF9}बरम्मा\accentmark{27}{\char"1CD2}दे\accentmark{22}{\char"0951}\kern0.15em।\ दीवो\accentmark{27}{\char"1CD2}\kern0.15emदा\accentmark{27}{\char"1CD2}सा\accentmark{22}{\char"0951}यारन्धा\accentmark{27}{\char"1CD2}यन्।\ आयं\accentmark{27}{\char"1CD2}\kern0.15emससो\accentmark{27}{\char"1CD2}मा\accentmark{22}{\char"0951}इन्द्रतेसूतःपीबा\mbox{॥ २\hspace{0pt}॥} \\
गृणेता\accentmark{27}{\char"1CD2}दिन्द्रा\accentmark{22}{\char"0951}\kern0.15emतेशा\accentmark{27}{\char"1CD2}वाः\accentmark{22}{\char"0951}\kern0.15em।\ ऊपा\accentmark{27}{\char"1CD2}मान्देवा\accentmark{27}{\char"1CD2}ता\accentmark{22}{\char"0951}तये।\ 
यद्धं\accentmark{20}{\char"1CF9}सी\accentmark{22}{\char"0951}\kern0.15emवृत्र\accentmark{27}{\char"1CD2}मोज\accentmark{22}{\char"0951}\kern0.15emसा\accentmark{27}{\char"1CD2}\kern0.15emशा\accentmark{27}{\char"1CD2}चीपते\mbox{॥ ३\hspace{0pt}॥} \\
याइ\accentmark{27}{\char"1CD2}न्द्रासोमापाता\accentmark{22}{\char"0951}\kern0.15emमः।\ मा\accentmark{27}{\char"1CD2}द\accentmark{22}{\char"0951}श्शवीष्ठाचे\accentmark{27}{\char"1CD2}ताती।\ 
ये\accentmark{27}{\char"1CD2}नाहंसिन्याटट्यत्रीणंता\accentmark{27}{\char"1CD2}मीमाहे\mbox{॥ ४\hspace{0pt}॥} \\
तूचेतूनाः\accentmark{20}{\char"1CF9}यात\accentmark{27}{\char"1CD2}त्सू\accentmark{22}{\char"0951}नाः।\ द्राघी\accentmark{22}{\char"0951}याआ\accentmark{20}{\char"1CF9}यू\accentmark{22}{\char"0951}र्जीवा\accentmark{27}{\char"1CD2}से\accentmark{22}{\char"0951}।\ आ\accentmark{22}{\char"0951}दि\accentmark{22}{\char"0951}त्यासस्सम्महसाःकृणो\accentmark{27}{\char"1CD2}ता\accentmark{22}{\char"0951}नः\mbox{॥ ५\hspace{0pt}॥} \\
वे\accentmark{27}{\char"1CD2}त्थाहि\accentmark{27}{\char"1CD2}\kern0.15emनि\accentmark{27}{\char"1CD2}रिऋ\accentmark{22}{\char"0951}तीनां।\ वज्रा\accentmark{22}{\char"0951}हस्तापारिव्रा\accentmark{27}{\char"1CD2}जं\accentmark{22}{\char"0951}।\ आहा\accentmark{22}{\char"0951}रहश्शुन्धूः\accentmark{20}{\char"1CF9}पा\accentmark{22}{\char"0951}रीपादा\accentmark{22}{\char"0951}मिवा\mbox{॥ ६\hspace{0pt}॥} \\आ\accentmark{20}{\char"1CF8}पा\accentmark{20}{\char"1CF8}मी\accentmark{22}{\char"0951}\kern0.15emवा\accentmark{27}{\char"1CD2}\kern0.15emमा\accentmark{27}{\char"1CD2}\kern0.15emपस्री\accentmark{27}{\char"1CD2}\kern0.15emधं।\ आ\accentmark{27}{\char"1CD2}पा\accentmark{22}{\char"0951}सेधतदुर्मा\accentmark{22}{\char"0951}तिं।\ आदी\accentmark{22}{\char"0951}त्यासोयूयोतानानो\accentmark{27}{\char"1CD2}\kern0.15emअं\accentmark{27}{\char"1CD2}हा\accentmark{22}{\char"0951}सः\mbox{॥ ७\hspace{0pt}॥} \\
पीबासो\accentmark{27}{\char"1CD2}मा\accentmark{22}{\char"0951}मिन्द्रा\accentmark{27}{\char"1CD2}मन्द\accentmark{22}{\char"0951}तुत्वा।\ य\accentmark{20}{\char"1CF8}न्ते\accentmark{22}{\char"0951}\kern0.15emसूषा\accentmark{27}{\char"1CD2}वाहरिय्य\accentmark{27}{\char"1CD2}श्वाद्रिः\accentmark{27}{\char"1CD2}\kern0.15em।\ सोतु\accentmark{27}{\char"1CD2}र्बा\accentmark{22}{\char"0951}हुभ्यांसू\accentmark{20}{\char"1CF9}यातोना\accentmark{27}{\char"1CD2}र्वा\mbox{॥ ८\hspace{0pt}॥} \\
\mbox{॥ इति षष्ठःखण्डः॥} \\
अभ्रा\accentmark{27}{\char"1CD2}\kern0.15emतृ\accentmark{27}{\char"1CD2}व्योआना\accentmark{27}{\char"1CD2}\kern0.15emत्वं\accentmark{27}{\char"1CD2}।\ आनापिरिन्द्राजनु\accentmark{20}{\char"1CF9}षा\accentmark{22}{\char"0951}\kern0.15emसाना\accentmark{27}{\char"1CD2}द\accentmark{22}{\char"0951}सि।\ यू\accentmark{20}{\char"1CF8}थे\accentmark{20}{\char"1CF9}दापित्वामी\accentmark{22}{\char"0951}च्छासे\mbox{॥ १\hspace{0pt}॥} \\
यो\accentmark{20}{\char"1CF8}ना\accentmark{22}{\char"0951}ईदामी\accentmark{22}{\char"0951}दंपूरा।\ प्रव\accentmark{27}{\char"1CD2}स्याआ\accentmark{22}{\char"0951}\kern0.15emनीना\accentmark{27}{\char"1CD2}याता\accentmark{22}{\char"0951}मू\accentmark{22}{\char"0951}वस्तूषेसाखा\accentmark{27}{\char"1CD2}\kern0.15emयाइ\accentmark{20}{\char"1CF9}न्दामूताये\mbox{॥ २\hspace{0pt}॥} \\
आ\accentmark{20}{\char"1CF8}ग\accentmark{22}{\char"0951}न्तामारिषण्यत।\ प्रस्था\accentmark{22}{\char"0951}वानोमापा\accentmark{22}{\char"0951}स्थातासमन्यवः।\ दृढा\accentmark{27}{\char"1CD2}श्चिद्यामायिष्णावाः\mbox{॥ ३\hspace{0pt}॥} \\
आ\accentmark{20}{\char"1CF8}या\accentmark{22}{\char"0951}ह्यायामिन्दा\accentmark{22}{\char"0951}वे।\ आ\accentmark{22}{\char"0951}श्वा\accentmark{22}{\char"0951}\kern0.15emपा\accentmark{27}{\char"1CD2}\kern0.15emतेगो\accentmark{20}{\char"1CF9}पा\accentmark{22}{\char"0951}ताउर्वारापते।\ सो\accentmark{27}{\char"1CD2}मं\accentmark{22}{\char"0951}सोमपतेपीबा\accentmark{22}{\char"0951} \mbox{॥ ४\hspace{0pt}॥} \\
त्वाया\accentmark{22}{\char"0951}हस्वीद्युजावायं\accentmark{27}{\char"1CD2}।\ प्रातिश्व\accentmark{27}{\char"1CD2}स\accentmark{22}{\char"0951}न्तंवृषभब्रूवीमही\accentmark{22}{\char"0951}।\ संस्थे\accentmark{20}{\char"1CF9}ज\accentmark{20}{\char"1CF9}नास्यागोमा\accentmark{22}{\char"0951}तः \mbox{॥ ५\hspace{0pt}॥} \\
गावा\accentmark{27}{\char"1CD2}श्चित्घासमन्यवः।\ सा\accentmark{27}{\char"1CD2}जात्येनामारूतस्साब\accentmark{22}{\char"0951}न्धवः।\ रिहाते\accentmark{22}{\char"0951}काकू\accentmark{20}{\char"1CF9}भो\accentmark{22}{\char"0951}मीथः\mbox{॥ ६\hspace{0pt}॥} \\
त्व\accentmark{20}{\char"1CF8}न्ना\accentmark{22}{\char"0951}इन्द्रा\accentmark{27}{\char"1CD2}भा\accentmark{22}{\char"0951}रा।\ ओजो\accentmark{22}{\char"0951}निर्म्णं\accentmark{27}{\char"1CD2}शाताक्रतोविचर्षणे।\ आ\accentmark{27}{\char"1CD2}\kern0.15emवीरं\accentmark{27}{\char"1CD2}पृ\accentmark{22}{\char"0951}\kern0.15emतनासा\accentmark{27}{\char"1CD2}हम्\mbox{॥ ७\hspace{0pt}॥} \\
आ\accentmark{27}{\char"1CD2}थाहीन्द्रगिर्वणः।\ ऊपा\accentmark{22}{\char"0951}त्वाका\accentmark{20}{\char"1CF9}मा\accentmark{22}{\char"0951}\kern0.15emईमा\accentmark{27}{\char"1CD2}हे\accentmark{22}{\char"0951}सासृग्माहेऊदेवग्म\accentmark{20}{\char"1CF9}न्ता\accentmark{22}{\char"0951}\kern0.15emऊद\accentmark{27}{\char"1CD2}भीः\mbox{॥ ८\hspace{0pt}॥} \\
सी\accentmark{27}{\char"1CD2}दन्त\accentmark{22}{\char"0951}स्ते\accentmark{20}{\char"1CF9}वायोया\accentmark{27}{\char"1CD2}\kern0.15emथा।\ गो\accentmark{27}{\char"1CD2}\kern0.15emश्री\accentmark{27}{\char"1CD2}स्तेमा\accentmark{27}{\char"1CD2}धोर्मा\accentmark{27}{\char"1CD2}दीरोवी\accentmark{22}{\char"0951}वक्षणः।\ आ\accentmark{27}{\char"1CD2}भित्वा\accentmark{27}{\char"1CD2}मिन्द्रनोनुमः\mbox{॥ ९\hspace{0pt}॥} \\
वा\accentmark{27}{\char"1CD2}\kern0.15emया\accentmark{27}{\char"1CD2}मुत्वामा\accentmark{22}{\char"0951}पूर्व्या\accentmark{22}{\char"0951}।\ स्थूरन्न\accentmark{27}{\char"1CD2}\kern0.15emका\accentmark{27}{\char"1CD2}च्चिद्भा\accentmark{27}{\char"1CD2}र\accentmark{22}{\char"0951}न्तोवस्या\accentmark{22}{\char"0951}वाः\accentmark{20}{\char"1CF8}।\ वज्रि\accentmark{27}{\char"1CD2}ञ्चित्रंहा\accentmark{22}{\char"0951}वामहे\mbox{॥ १०\hspace{0pt}॥} \\
\mbox{॥ इति षष्ठः खण्डः\hspace{0pt}॥} \\
  \mbox{॥ इति सप्तमः खण्डः\hspace{0pt}॥} \\ 
विश्वा\accentmark{22}{\char"0951}\kern0.15emतोदाव\accentmark{27}{\char"1CD2}न्विश्वा\accentmark{20}{\char"1CF9}तो\accentmark{22}{\char"0951}नआभा\accentmark{22}{\char"0951}\kern0.15emरा।\ य\accentmark{27}{\char"1CD2}न्त्वाशा\accentmark{20}{\char"1CF9}विष्ठामी\accentmark{27}{\char"1CD2}मा\accentmark{22}{\char"0951}\kern0.15emहेएष\accentmark{27}{\char"1CD2}ब्रह्मा\accentmark{27}{\char"1CD2}याऋत्वी\accentmark{27}{\char"1CD2}याः।\ इन्द्रो\accentmark{27}{\char"1CD2}\kern0.15emना\accentmark{27}{\char"1CD2}माश्रू\accentmark{20}{\char"1CF9}तो\accentmark{22}{\char"0951}\kern0.15emगृणे\accentmark{27}{\char"1CD2} \mbox{॥ १\hspace{0pt}॥} \\
ब्रह्माणाइ\accentmark{22}{\char"0951}न्द्रं\accentmark{22}{\char"0951}मा\accentmark{22}{\char"0951}\kern0.15emहा\accentmark{27}{\char"1CD2}य\accentmark{22}{\char"0951}न्तोअर्कैः।\ आ\accentmark{27}{\char"1CD2}\kern0.15emवा\accentmark{27}{\char"1CD2}र्धायन्ना\accentmark{20}{\char"1CF9}हा\accentmark{22}{\char"0951}\kern0.15emयेहं\accentmark{27}{\char"1CD2}\kern0.15emतावा\accentmark{27}{\char"1CD2}ऊ\accentmark{22}{\char"0951}\kern0.15em\mbox{॥ २\hspace{0pt}॥} \\
आ\accentmark{27}{\char"1CD2}नावस्ते\accentmark{27}{\char"1CD2}राथामश्वा\accentmark{27}{\char"1CD2}यतक्षुः।\ त्वष्टव\accentmark{27}{\char"1CD2}ज्रंपूरुहूताद्युम\accentmark{27}{\char"1CD2}न्तम्\mbox{॥ ३\hspace{0pt}॥} \\
शं\accentmark{27}{\char"1CD2}पादम्माघं\accentmark{20}{\char"1CF9}रायी\accentmark{22}{\char"0951}षीणे।\ नाका\accentmark{27}{\char"1CD2}मामाव्रातो\accentmark{27}{\char"1CD2}ही\accentmark{22}{\char"0951}नोतिना\accentmark{27}{\char"1CD2}स्पृशाद्रायिम्\mbox{॥ ४\hspace{0pt}॥} \\
सादागावश्शूचायोविश्वा\accentmark{27}{\char"1CD2}धा\accentmark{22}{\char"0951}यसः।\ सादा\accentmark{22}{\char"0951}देवा\accentmark{20}{\char"1CF9}आरे\accentmark{27}{\char"1CD2}पासाः\accentmark{22}{\char"0951}\kern0.15em\mbox{॥ ५\hspace{0pt}॥} \\
आ\accentmark{27}{\char"1CD2}याहीवा\accentmark{27}{\char"1CD2}नासासाहा\accentmark{27}{\char"1CD2}\kern0.15em।\ गा\accentmark{27}{\char"1CD2}वा\accentmark{22}{\char"0951}स्सचन्तवा\accentmark{22}{\char"0951}र्तानींयादू\accentmark{27}{\char"1CD2}धा\accentmark{22}{\char"0951}भिः\mbox{॥ ६\hspace{0pt}॥} \\
उपप्रक्षेमा\accentmark{27}{\char"1CD2}धू\accentmark{22}{\char"0951}मतिक्षी\accentmark{22}{\char"0951}यन्ताः।\ पु\accentmark{27}{\char"1CD2}ष्ये\accentmark{22}{\char"0951}मारायिन्धीमा\accentmark{27}{\char"1CD2}हे\accentmark{22}{\char"0951}
तइन्द्र\mbox{॥ ७\hspace{0pt}॥} \\
अ\accentmark{20}{\char"1CF8}र्चन्त्यर्कम्मारू\accentmark{20}{\char"1CF9}ता\accentmark{22}{\char"0951}स्त्वर्कः।\ आ\accentmark{27}{\char"1CD2}स्तोभतीश्रूतो\accentmark{27}{\char"1CD2}यूवा
सइ\accentmark{27}{\char"1CD2}न्द्राः\mbox{॥ ८\hspace{0pt}॥} \\
प्रा\accentmark{27}{\char"1CD2}\kern0.15emवाइ\accentmark{27}{\char"1CD2}न्द्रायवृत्राह\accentmark{27}{\char"1CD2}न्ता\accentmark{22}{\char"0951}माया।\ वि\accentmark{27}{\char"1CD2}प्रा\accentmark{22}{\char"0951}\kern0.15emया\accentmark{27}{\char"1CD2}गा\accentmark{22}{\char"0951}थंगायाताय\accentmark{27}{\char"1CD2}ञ्जुजो\accentmark{27}{\char"1CD2}षाते\mbox{॥ ९\hspace{0pt}॥} \\
\mbox{॥ इति अष्ठमः खण्डः\hspace{0pt}॥} \\ 
आ\accentmark{20}{\char"1CF8}चे\accentmark{22}{\char"0951}त्याग्नी\accentmark{27}{\char"1CD2}\kern0.15emश्ची\accentmark{27}{\char"1CD2}की\accentmark{22}{\char"0951}\kern0.15emतिः।\ ह\accentmark{27}{\char"1CD2}व्यावाण्णासू\accentmark{20}{\char"1CF8}मा\accentmark{27}{\char"1CD2}द्रा\accentmark{22}{\char"0951}थः \mbox{॥ १\hspace{0pt}॥} \\
अ\accentmark{27}{\char"1CD2}ग्नेत्व\accentmark{27}{\char"1CD2}न्नो\accentmark{22}{\char"0951}अन्ता\accentmark{22}{\char"0951}मः।\ ऊता\accentmark{27}{\char"1CD2}त्राता\accentmark{27}{\char"1CD2}\kern0.15emशीवो\accentmark{27}{\char"1CD2}भू\accentmark{22}{\char"0951}वोवारूत्थ्याःठ\accentmark{22}{\char"0951}\mbox{॥ २\hspace{0pt}॥} \\
भा\accentmark{27}{\char"1CD2}गोनाचित्रः।\ अग्निर्मा\accentmark{27}{\char"1CD2}\kern0.15emहो\accentmark{27}{\char"1CD2}\kern0.15emनां।\ द\accentmark{20}{\char"1CF8}धा\accentmark{22}{\char"0951}\kern0.15emतीरा\accentmark{27}{\char"1CD2}त्नम् \mbox{॥ ३\hspace{0pt}॥} \\
वि\accentmark{20}{\char"1CF8}श्व\accentmark{22}{\char"0951}स्यप्रस्तो\accentmark{22}{\char"0951}भा।\ पूरो\accentmark{27}{\char"1CD2}वासन्या\accentmark{20}{\char"1CF9}दीवेहा\accentmark{27}{\char"1CD2}नूनम् \mbox{॥ ४\hspace{0pt}॥} \\
ऊषा\accentmark{27}{\char"1CD2}\kern0.15emआपा\accentmark{27}{\char"1CD2}स्वासुष्टा\accentmark{27}{\char"1CD2}माः।\ सं\accentmark{27}{\char"1CD2}वा\accentmark{22}{\char"0951}र्तयतिवार्क्तानिं\accentmark{20}{\char"1CF9}सूजाता\accentmark{27}{\char"1CD2}ता\accentmark{22}{\char"0951}\mbox{॥ ५\hspace{0pt}॥} \\
इमा\accentmark{22}{\char"0951}नूकं\accentmark{22}{\char"0951}\kern0.15emभू\accentmark{27}{\char"1CD2}वा\accentmark{22}{\char"0951}नासिषधेम।\ इन्द्रा\accentmark{22}{\char"0951}श्वावि\accentmark{27}{\char"1CD2}श्वे\accentmark{22}{\char"0951}चादेवाः\accentmark{27}{\char"1CD2}\mbox{॥ ६\hspace{0pt}॥} \\
ऊर्जा\accentmark{27}{\char"1CD2}मित्रोवारू\accentmark{22}{\char"0951}णोपिन्वा\accentmark{27}{\char"1CD2}ताईळाठ\accentmark{22}{\char"0951}।\ पी\accentmark{20}{\char"1CF8}वा\accentmark{22}{\char"0951}रीमीषं\accentmark{22}{\char"0951}कृणू\accentmark{22}{\char"0951}हीना\accentmark{22}{\char"0951}इन्द्रा\mbox{॥ ७\hspace{0pt}॥} \\
विस्रूतायो\accentmark{20}{\char"1CF9}या\accentmark{20}{\char"1CF9}था\accentmark{22}{\char"0951}\kern0.15emपा\accentmark{27}{\char"1CD2}था\accentmark{22}{\char"0951}।\ 
इन्द्रत्वद्य\accentmark{22}{\char"0951}न्तूरातायाः\accentmark{22}{\char"0951}\mbox{॥ ८\hspace{0pt}॥} \\
आया\accentmark{20}{\char"1CF9}वा\accentmark{20}{\char"1CF9}जं\accentmark{22}{\char"0951}\kern0.15emदेवा\accentmark{27}{\char"1CD2}ही\accentmark{22}{\char"0951}तंसनेम\accentmark{27}{\char"1CD2}।\ मा\accentmark{22}{\char"0951}देमाश्शा\accentmark{27}{\char"1CD2}\kern0.15emता\accentmark{27}{\char"1CD2}हीमासूवी\accentmark{27}{\char"1CD2}राः\accentmark{22}{\char"0951}\mbox{॥ ९\hspace{0pt}॥} \\
इन्द्रोवि\accentmark{27}{\char"1CD2}श्वास्यराजति\mbox{॥ १०\hspace{0pt}॥} \\
\mbox{॥ इति नवमः खण्डः\hspace{0pt}॥} \\ 
त्रीका\accentmark{27}{\char"1CD2}दूकेषुमा\accentmark{27}{\char"1CD2}\kern0.15emहीषो\accentmark{27}{\char"1CD2}\kern0.15emया\accentmark{27}{\char"1CD2}वा\accentmark{22}{\char"0951}शीर\accentmark{22}{\char"0951}न्तूवी\accentmark{27}{\char"1CD2}शूष्मः।\ त्रम्पत्सो\accentmark{27}{\char"1CD2}मामपीबद्वीष्णू\accentmark{22}{\char"0951}\kern0.15emना\accentmark{27}{\char"1CD2}सूतंय्या\accentmark{22}{\char"0951}थावाशं।\ 
साईं\accentmark{22}{\char"0951}ममादमाहीकर्मा\accentmark{22}{\char"0951}\kern0.15emका\accentmark{27}{\char"1CD2}र्तावेमाहा\accentmark{27}{\char"1CD2}\kern0.15emमूरुं\accentmark{27}{\char"1CD2}\kern0.15em।\ सै\accentmark{27}{\char"1CD2}नं\accentmark{22}{\char"0951}सश्चाद्देवो\accentmark{27}{\char"1CD2}देवंसत्य\accentmark{20}{\char"1CF9}इ\accentmark{20}{\char"1CF9}न्दूः\accentmark{22}{\char"0951}\kern0.15emसत्या\accentmark{27}{\char"1CD2}मिन्द्राः\mbox{॥ १\hspace{0pt}॥} \\आ\accentmark{27}{\char"1CD2}\kern0.15emयं\accentmark{27}{\char"1CD2}सा\accentmark{22}{\char"0951}हस्रामान\accentmark{22}{\char"0951}वः।\ दृशाः\accentmark{20}{\char"1CF9}कावीना\accentmark{27}{\char"1CD2}म्मातिज्योतिर्वीध\accentmark{22}{\char"0951}र्म।\ बुध्नस्सामीचीरूषा\accentmark{27}{\char"1CD2}\kern0.15emसस्सा\accentmark{27}{\char"1CD2}मी\accentmark{22}{\char"0951}रयात्।\ आरेपा\accentmark{27}{\char"1CD2}सस्साचे\accentmark{22}{\char"0951}तसः।\ स्व\accentmark{27}{\char"1CD2}सा\accentmark{22}{\char"0951}रेम\accentmark{20}{\char"1CF9}न्यूमन्ता\accentmark{22}{\char"0951}\kern0.15emश्चि\accentmark{27}{\char"1CD2}तागोः\accentmark{22}{\char"0951}\mbox{॥ २\hspace{0pt}॥} \\
ए\accentmark{20}{\char"1CF8}न्द्रा\accentmark{22}{\char"0951}याह्यू\accentmark{27}{\char"1CD2}पा\accentmark{22}{\char"0951}नःपारावा\accentmark{27}{\char"1CD2}ताः\accentmark{22}{\char"0951}।\ ना\accentmark{20}{\char"1CF8}यमा\accentmark{20}{\char"1CF8}च्छा\accentmark{22}{\char"0951}\kern0.15emवीद\accentmark{27}{\char"1CD2}धा\accentmark{22}{\char"0951}नीवासात्पा\accentmark{22}{\char"0951}तिः।\ अस्तारा\accentmark{20}{\char"1CF9}जे\accentmark{22}{\char"0951}वासात्पा\accentmark{22}{\char"0951}तिः।\ हावा\accentmark{22}{\char"0951}महेत्वाप्राया\accentmark{22}{\char"0951}स्वन्तस्सू\accentmark{27}{\char"1CD2}तेष्वा\accentmark{22}{\char"0951}\kern0.15em।\ पु\accentmark{27}{\char"1CD2}त्रासो\accentmark{22}{\char"0951}\kern0.15emना\accentmark{27}{\char"1CD2}\kern0.15emपीता\accentmark{27}{\char"1CD2}\kern0.15emरंवा\accentmark{27}{\char"1CD2}ज\accentmark{22}{\char"0951}सातये।\ मंहीष्ठंवाजसातये\mbox{॥ ३\hspace{0pt}॥} \\
त\accentmark{27}{\char"1CD2}\kern0.15emमि\accentmark{27}{\char"1CD2}न्द्रं\accentmark{22}{\char"0951}ज्योहवीमिमाघा\accentmark{27}{\char"1CD2}वा\accentmark{22}{\char"0951}नामुग्रं।\ सत्रा\accentmark{20}{\char"1CF9}दधा\accentmark{22}{\char"0951}\kern0.15emनामा\accentmark{27}{\char"1CD2}प्रातिष्कृतं।\ श्रा\accentmark{20}{\char"1CF9}वां\accentmark{22}{\char"0951}\kern0.15emसिभू\accentmark{27}{\char"1CD2}रिं।\ मंही\accentmark{22}{\char"0951}ष्ठोगीर्भि\accentmark{27}{\char"1CD2}राचायाज्ञीयो\accentmark{22}{\char"0951}\kern0.15emवा\accentmark{27}{\char"1CD2}\kern0.15emवा\accentmark{27}{\char"1CD2}र्क्ता\accentmark{22}{\char"0951}\kern0.15em।\ रा\accentmark{27}{\char"1CD2}\kern0.15emयेनो\accentmark{20}{\char"1CF9}विश्वासूपाथाकृ\accentmark{22}{\char"0951}\kern0.15emणो\accentmark{27}{\char"1CD2}तूवज्री\accentmark{27}{\char"1CD2}\kern0.15em\mbox{॥ ४\hspace{0pt}॥} \\
आ\accentmark{27}{\char"1CD2}\kern0.15emभित्य\accentmark{27}{\char"1CD2}न्देवंसा\accentmark{22}{\char"0951}वीता\accentmark{20}{\char"1CF9}रा\accentmark{20}{\char"1CF9}मोण्योः\accentmark{27}{\char"1CD2}\kern0.15emका\accentmark{27}{\char"1CD2}विक्रा\accentmark{22}{\char"0951}तुं।\ अर्चा\accentmark{22}{\char"0951}मीसत्या\accentmark{27}{\char"1CD2}सावंरात्नाधा\accentmark{27}{\char"1CD2}\kern0.15emमा\accentmark{27}{\char"1CD2}भीप्रीयम्मातिम्।\ 
ऊर्ध्वा\accentmark{27}{\char"1CD2}\kern0.15emय\accentmark{27}{\char"1CD2}स्याआमा\accentmark{22}{\char"0951}\kern0.15emती\accentmark{27}{\char"1CD2}र्भाआतिद्युत\accentmark{27}{\char"1CD2}त्सावि\accentmark{22}{\char"0951}मनी।\ हिर\accentmark{22}{\char"0951}ण्यपाणीरमीमीता\accentmark{27}{\char"1CD2}\kern0.15emसू\accentmark{20}{\char"1CF8}क्रा\accentmark{20}{\char"1CF9}तू\accentmark{22}{\char"0951}\kern0.15emकृपा\accentmark{27}{\char"1CD2}स्वाः\accentmark{22}{\char"0951}\mbox{॥ ५\hspace{0pt}॥} \\
ता\accentmark{27}{\char"1CD2}\kern0.15emवत्य\accentmark{20}{\char"1CF9}न्न\accentmark{20}{\char"1CF9}री\accentmark{22}{\char"0951}\kern0.15emयनृतो\accentmark{27}{\char"1CD2}पा\accentmark{22}{\char"0951}इन्द्रा।\ प्राथामंपूर्व्य\accentmark{27}{\char"1CD2}न्ती\accentmark{22}{\char"0951}विप्रावा\accentmark{20}{\char"1CF9}च्यंकृत\accentmark{27}{\char"1CD2}\kern0.15emम्।\ यो\accentmark{27}{\char"1CD2}\kern0.15emदेव\accentmark{27}{\char"1CD2}स्याशा\accentmark{27}{\char"1CD2}वासप्रारी\accentmark{22}{\char"0951}णाया\accentmark{20}{\char"1CF9}सुं\accentmark{22}{\char"0951}रीणन्नापाः\accentmark{27}{\char"1CD2}\kern0.15em।\ भु\accentmark{27}{\char"1CD2}वोविश्वा\accentmark{22}{\char"0951}माभ्या\accentmark{20}{\char"1CF9}दे\accentmark{22}{\char"0951}\kern0.15emवामो\accentmark{27}{\char"1CD2}ज\accentmark{22}{\char"0951}सा।\ वीर्द\accentmark{20}{\char"1CF9}दू\accentmark{20}{\char"1CF9}र्जंशा\accentmark{27}{\char"1CD2}ता\accentmark{22}{\char"0951}क्रातूर्वीदे\accentmark{27}{\char"1CD2}दीषम्\mbox{॥ ६\hspace{0pt}॥} \\
अस्तुश्रौ\accentmark{27}{\char"1CD2}षा\accentmark{22}{\char"0951}ट्पूरो\accentmark{27}{\char"1CD2}आ\accentmark{22}{\char"0951}ग्नीन्धीया\accentmark{27}{\char"1CD2}\kern0.15emदधे।\ आ\accentmark{20}{\char"1CF8}नूत्य\accentmark{20}{\char"1CF8}च्छर्धोदिव्यं\accentmark{27}{\char"1CD2}वृणीमहे।\ इन्द्रं\accentmark{22}{\char"0951}\kern0.15emवायू\accentmark{27}{\char"1CD2}वृ\accentmark{22}{\char"0951}णीमहे।\ य\accentmark{20}{\char"1CF8}द्धा\accentmark{22}{\char"0951}क्राणा\accentmark{27}{\char"1CD2}\kern0.15emवी\accentmark{27}{\char"1CD2}\kern0.15emवा\accentmark{27}{\char"1CD2}स्वा\accentmark{22}{\char"0951}ते।\ ना\accentmark{20}{\char"1CF8}भा\accentmark{22}{\char"0951}सन्दायाना\accentmark{27}{\char"1CD2}\kern0.15emव्या\accentmark{20}{\char"1CF8}से\mbox{॥ ७\hspace{0pt}॥} \\
आ\accentmark{27}{\char"1CD2}धःप्रा\accentmark{27}{\char"1CD2}\kern0.15emनूना\accentmark{27}{\char"1CD2}मूपा\accentmark{22}{\char"0951}यन्तीधीता\accentmark{27}{\char"1CD2}याः\accentmark{22}{\char"0951}\kern0.15em।\ दे\accentmark{27}{\char"1CD2}\kern0.15emवं\accentmark{27}{\char"1CD2}\kern0.15emअ\accentmark{27}{\char"1CD2}च्छाना\accentmark{27}{\char"1CD2}\kern0.15emधीता\accentmark{27}{\char"1CD2}याः।\ प्र\accentmark{20}{\char"1CF8}वोमाहे\accentmark{27}{\char"1CD2}\kern0.15emमाता\accentmark{27}{\char"1CD2}यो\accentmark{22}{\char"0951}यन्तूवी\accentmark{27}{\char"1CD2}ष्णा\accentmark{22}{\char"0951}वे।\ मारू\accentmark{27}{\char"1CD2}त्वा\accentmark{22}{\char"0951}तेगीरीजा\accentmark{20}{\char"1CF9}ये\accentmark{22}{\char"0951}वायामा\accentmark{22}{\char"0951}रूत्।\ प्र\accentmark{20}{\char"1CF8}शर्धायाप्रयज्यावेसूखाद\accentmark{27}{\char"1CD2}ये\accentmark{22}{\char"0951}\kern0.15em।\ ता\accentmark{27}{\char"1CD2}\kern0.15emवा\accentmark{20}{\char"1CF8}सेभन्द\accentmark{27}{\char"1CD2}दिष्टये।\ धू\accentmark{27}{\char"1CD2}नीव्रतायाशा\accentmark{27}{\char"1CD2}वासे \mbox{॥ ८\hspace{0pt}॥} \\
आ\accentmark{27}{\char"1CD2}\kern0.15emया\accentmark{27}{\char"1CD2}\kern0.15emरूचा\accentmark{27}{\char"1CD2}\kern0.15emहा\accentmark{27}{\char"1CD2}रिण्यापूना\accentmark{27}{\char"1CD2}\kern0.15emनः\accentmark{27}{\char"1CD2}\kern0.15em।\ 
वि\accentmark{27}{\char"1CD2}\kern0.15emश्वा\accentmark{27}{\char"1CD2}द्वेषांसितरतीसायूग्वा\accentmark{22}{\char"0951}भिः।\ सूरोना\accentmark{27}{\char"1CD2}सायूग्वा\accentmark{22}{\char"0951}भिः।\ धा\accentmark{20}{\char"1CF8}रा\accentmark{22}{\char"0951}पृष्ठास्या\accentmark{22}{\char"0951}रोचते।\ पु\accentmark{27}{\char"1CD2}\kern0.15emनानो\accentmark{20}{\char"1CF9}आ\accentmark{22}{\char"0951}रुषोहा\accentmark{27}{\char"1CD2}रिः\mbox{॥ ९\hspace{0pt}॥} \\
वि\accentmark{27}{\char"1CD2}श्वा\accentmark{22}{\char"0951}यद्रूपा\accentmark{20}{\char"1CF9}पा\accentmark{22}{\char"0951}\kern0.15emर्या\accentmark{27}{\char"1CD2}स्यऋत्वा\accentmark{22}{\char"0951}भिः।\ सप्ता\accentmark{20}{\char"1CF9}स्ये\accentmark{22}{\char"0951}\kern0.15emभिऋ\accentmark{27}{\char"1CD2}त्वा\accentmark{22}{\char"0951}भिः॥अग्निंहो\accentmark{27}{\char"1CD2}ता\accentmark{22}{\char"0951}रम्मन्येदा\accentmark{27}{\char"1CD2}स्व\accentmark{22}{\char"0951}न्तं।\ वा\accentmark{20}{\char"1CF8}सो\accentmark{22}{\char"0951}स्सूनु\accentmark{22}{\char"0951}\kern0.15emसा\accentmark{27}{\char"1CD2}हा\accentmark{22}{\char"0951}साजाता\accentmark{27}{\char"1CD2}वेदसं।\ विप्रान्ना\accentmark{27}{\char"1CD2}\kern0.15emजाता\accentmark{27}{\char"1CD2}वे\accentmark{22}{\char"0951}दसं।\ या\accentmark{22}{\char"0951}ऊर्ध्वायायास्वाध्वारः\accentmark{27}{\char"1CD2}।\ देवोदेवा\accentmark{20}{\char"1CF9}च्य\accentmark{22}{\char"0951}कृ\accentmark{22}{\char"0951}पा\accentmark{22}{\char"0951}\kern0.15em।\ घृ\accentmark{27}{\char"1CD2}\kern0.15emत\accentmark{27}{\char"1CD2}स्या\accentmark{22}{\char"0951}बिभ्र\accentmark{22}{\char"0951}ष्टीमा\accentmark{20}{\char"1CF9}नू\accentmark{22}{\char"0951}शुक्रा\accentmark{27}{\char"1CD2}शो\accentmark{22}{\char"0951}चिषः।\ आ\accentmark{27}{\char"1CD2}\kern0.15emजुं\accentmark{27}{\char"1CD2}भ्ह्वा\accentmark{22}{\char"0951}नास्यासर्पीषाः\mbox{॥ १०\hspace{0pt}॥} \\
 \mbox{॥ इति ऐन्द्र पाठः समाप्तः\hspace{0pt}॥} \\ \clearpage
\mbox{॥ अथ पवमान पाठः\hspace{0pt}॥} \\
उच्चा\accentmark{22}{\char"0951}ते\accentmark{22}{\char"0951}जातमन्धा\accentmark{20}{\char"1CF8}सा\accentmark{27}{\char"1CD2}।\ दिविसा\accentmark{27}{\char"1CD2}त्भूम्याद\accentmark{22}{\char"0951}दे।\ 
उग्रं\accentmark{27}{\char"1CD2}शर्म्मा\accentmark{27}{\char"1CD2}माहिश्रा\accentmark{27}{\char"1CD2}वाः\mbox{॥ १\hspace{0pt}॥} \\
स्वा\accentmark{22}{\char"0951}दी\accentmark{22}{\char"0951}ष्ठायामा\accentmark{27}{\char"1CD2}दी\accentmark{22}{\char"0951}ष्ठाया।\ पा\accentmark{27}{\char"1CD2}वा\accentmark{22}{\char"0951}स्वसोमाधारा\accentmark{22}{\char"0951}या।\ इ\accentmark{20}{\char"1CF8}न्द्रा\accentmark{22}{\char"0951}\kern0.15emयापा\accentmark{27}{\char"1CD2}तावेसूतः\accentmark{27}{\char"1CD2}\mbox{॥ २\hspace{0pt}॥} \\
वृ\accentmark{27}{\char"1CD2}षा\accentmark{22}{\char"0951}पवास्वधा\accentmark{27}{\char"1CD2}रा\accentmark{22}{\char"0951}\kern0.15emया।\ मा\accentmark{27}{\char"1CD2}रूत्वा\accentmark{22}{\char"0951}तेचमात्सारः\accentmark{27}{\char"1CD2}।\ विश्वा\accentmark{22}{\char"0951}दधा\accentmark{22}{\char"0951}\kern0.15emनओज\accentmark{27}{\char"1CD2}सा\mbox{॥ ३\hspace{0pt}॥} \\
यस्तेमा\accentmark{27}{\char"1CD2}\kern0.15emदोवा\accentmark{27}{\char"1CD2}रे\accentmark{22}{\char"0951}ण्यः।\ तेना\accentmark{22}{\char"0951}पावस्वा\accentmark{27}{\char"1CD2}न्ध\accentmark{22}{\char"0951}सा।\ देवावीरा\accentmark{22}{\char"0951}घशंसाहा\accentmark{27}{\char"1CD2}\mbox{॥ ४\hspace{0pt}॥} \\
ति\accentmark{20}{\char"1CF8}स्रो\accentmark{27}{\char"1CD2}वाचाऊदीरते।\ 
गा\accentmark{27}{\char"1CD2}वो\accentmark{22}{\char"0951}मिमन्तीधेना\accentmark{27}{\char"1CD2}वाः\accentmark{22}{\char"0951}\kern0.15em।\ 
हा\accentmark{27}{\char"1CD2}री\accentmark{22}{\char"0951}रेतीका\accentmark{27}{\char"1CD2}नीक्रदत्\mbox{॥ ५\hspace{0pt}॥} \\
इन्द्रा\accentmark{22}{\char"0951}येन्दोमा\accentmark{20}{\char"1CF9}रू\accentmark{27}{\char"1CD2}त्वा\accentmark{22}{\char"0951}ते।\ 
पा\accentmark{20}{\char"1CF8}वास्वामा\accentmark{27}{\char"1CD2}धूमक्तमः।\ 
अर्का\accentmark{27}{\char"1CD2}स्यायो\accentmark{27}{\char"1CD2}\kern0.15emनीमा\accentmark{27}{\char"1CD2}सादः\accentmark{22}{\char"0951}\mbox{॥ ६\hspace{0pt}॥} \\
अ\accentmark{20}{\char"1CF8}स\accentmark{22}{\char"0951}व्यंशुर्मा\accentmark{27}{\char"1CD2}दाय।\ 
अ\accentmark{20}{\char"1CF8}प्सु\accentmark{27}{\char"1CD2}\kern0.15emद\accentmark{27}{\char"1CD2}क्षोगीरीष्ठाः।\ 
श्ये\accentmark{27}{\char"1CD2}\kern0.15emनो\accentmark{27}{\char"1CD2}नयोनीमा\accentmark{27}{\char"1CD2}सा\accentmark{22}{\char"0951}दत्\mbox{॥ ७\hspace{0pt}॥} \\
पा\accentmark{20}{\char"1CF8}वा\accentmark{22}{\char"0951}स्वदक्षासाधा\accentmark{22}{\char"0951}नः।\ 
देवे\accentmark{20}{\char"1CF9}भ्यः\accentmark{22}{\char"0951}\kern0.15emपीता\accentmark{27}{\char"1CD2}ये\accentmark{22}{\char"0951}हरे।\ 
मारू\accentmark{20}{\char"1CF9}त्भ्योवाया\accentmark{27}{\char"1CD2}\kern0.15emवेमा\accentmark{27}{\char"1CD2}दाः\accentmark{22}{\char"0951}\mbox{॥ ८\hspace{0pt}॥} \\
पारीस्वानो\accentmark{20}{\char"1CF9}गीरीष्ठाः\accentmark{27}{\char"1CD2}।\ 
पावी\accentmark{27}{\char"1CD2}त्रेसोमो\accentmark{22}{\char"0951}अक्षरात्।\ 
मा\accentmark{27}{\char"1CD2}दे\accentmark{22}{\char"0951}षुसार्वाधा\accentmark{27}{\char"1CD2}आसि\mbox{॥ ९\hspace{0pt}॥} \\
पा\accentmark{20}{\char"1CF8}रि\accentmark{22}{\char"0951}प्रीया\accentmark{27}{\char"1CD2}\kern0.15emदीवाः\accentmark{27}{\char"1CD2}\kern0.15emकावीः\accentmark{27}{\char"1CD2}\kern0.15em।\ 
वा\accentmark{27}{\char"1CD2}यां\accentmark{22}{\char"0951}सिनप्त्यो\accentmark{27}{\char"1CD2}र्हितः।\ स्वानै\accentmark{27}{\char"1CD2}र्यातीःकावी\accentmark{27}{\char"1CD2}क्रातुः\mbox{॥ १०\hspace{0pt}॥} \\
\mbox{॥ इति प्रथमः खण्डः\hspace{0pt}॥} \\ 
प्र\accentmark{20}{\char"1CF8}सो\accentmark{20}{\char"1CF8}मासोमादच्यू\accentmark{27}{\char"1CD2}ताः।\ 
श्रा\accentmark{27}{\char"1CD2}वा\accentmark{22}{\char"0951}सेनोमा\accentmark{22}{\char"0951}\kern0.15emघो\accentmark{27}{\char"1CD2}नां\accentmark{22}{\char"0951}\kern0.15em।\ 
सू\accentmark{27}{\char"1CD2}\kern0.15emतावीद\accentmark{27}{\char"1CD2}थेअक्रमुः\mbox{॥ १\hspace{0pt}॥} \\
प्र\accentmark{27}{\char"1CD2}\kern0.15emसो\accentmark{27}{\char"1CD2}मा\accentmark{22}{\char"0951}सोवीपश्ची\accentmark{27}{\char"1CD2}ताः\accentmark{22}{\char"0951}\kern0.15em।\ 
आ\accentmark{27}{\char"1CD2}पो\accentmark{22}{\char"0951}नायन्ताऊर्म\accentmark{22}{\char"0951}याः\accentmark{22}{\char"0951}।\ 
वाना\accentmark{27}{\char"1CD2}नी\accentmark{22}{\char"0951}माहीषा\accentmark{22}{\char"0951}ईवा\mbox{॥ २\hspace{0pt}॥} \\
पा\accentmark{27}{\char"1CD2}वा\accentmark{22}{\char"0951}स्वेन्दोवृ\accentmark{20}{\char"1CF9}षा\accentmark{22}{\char"0951}सूतः।\ 
कृ\accentmark{27}{\char"1CD2}\kern0.15emधी\accentmark{20}{\char"1CF9}नोयाशासोजने।\ 
विश्वा\accentmark{27}{\char"1CD2}\kern0.15emआप\accentmark{27}{\char"1CD2}\kern0.15emद्वी\accentmark{27}{\char"1CD2}षो\accentmark{22}{\char"0951}जहि\mbox{॥ ३\hspace{0pt}॥} \\
वृषाह्या\accentmark{20}{\char"1CF9}सी\accentmark{22}{\char"0951}\kern0.15emभानु\accentmark{27}{\char"1CD2}ना।\ 
द्यु\accentmark{27}{\char"1CD2}मन्तं\accentmark{22}{\char"0951}त्वाहवामहे।\ 
पा\accentmark{27}{\char"1CD2}वा\accentmark{22}{\char"0951}मानास्वर्दृ\accentmark{27}{\char"1CD2}श\accentmark{22}{\char"0951}म्\mbox{॥ ४\hspace{0pt}॥} \\
इ\accentmark{27}{\char"1CD2}न्दूः\accentmark{22}{\char"0951}पविष्टाचे\accentmark{27}{\char"1CD2}तानः।\ 
प्रीयः\accentmark{20}{\char"1CF9}का\accentmark{22}{\char"0951}\kern0.15emवीना\accentmark{27}{\char"1CD2}म्मातिः।\ 
सृजद\accentmark{20}{\char"1CF9}श्वं\accentmark{22}{\char"0951}राथिरिव\mbox{॥ ५\hspace{0pt}॥} \\
आ\accentmark{27}{\char"1CD2}सृक्षातप्रावाजी\accentmark{27}{\char"1CD2}नाः।\ 
गव्या\accentmark{27}{\char"1CD2}सोमा\accentmark{22}{\char"0951}सोआश्वाया\accentmark{27}{\char"1CD2}।\ 
शुक्रा\accentmark{27}{\char"1CD2}सो\accentmark{22}{\char"0951}वीराया\accentmark{27}{\char"1CD2}\kern0.15emशा\accentmark{27}{\char"1CD2}\kern0.15emवाः\accentmark{27}{\char"1CD2}\mbox{॥ ६\hspace{0pt}॥} \\
पा\accentmark{27}{\char"1CD2}वा\accentmark{22}{\char"0951}स्वादेवा\accentmark{20}{\char"1CF9}आयूषा\accentmark{22}{\char"0951}त्।\ 
इन्द्रं\accentmark{22}{\char"0951}गच्छतूतेमादाः\accentmark{22}{\char"0951}।\ 
वायुमा\accentmark{20}{\char"1CF9}रो\accentmark{22}{\char"0951}\kern0.15emहाध\accentmark{27}{\char"1CD2}र्मा\accentmark{22}{\char"0951}णा\mbox{॥ ७\hspace{0pt}॥} \\
पा\accentmark{27}{\char"1CD2}वा\accentmark{22}{\char"0951}मानोअजीजनात्।\ दीव\accentmark{27}{\char"1CD2}\kern0.15emश्चि\accentmark{27}{\char"1CD2}त्रन्ना\accentmark{20}{\char"1CF9}तन्या\accentmark{22}{\char"0951}\kern0.15emतुं\accentmark{27}{\char"1CD2}।\ ज्योतिर्वैश्वानारं\accentmark{27}{\char"1CD2}\kern0.15emबृहा\accentmark{27}{\char"1CD2}त्\mbox{॥ ८\hspace{0pt}॥} \\
पा\accentmark{20}{\char"1CF8}री\accentmark{22}{\char"0951}स्वाना\accentmark{27}{\char"1CD2}साइन्दा\accentmark{22}{\char"0951}\kern0.15emवः।\ मा\accentmark{27}{\char"1CD2}दा\accentmark{22}{\char"0951}याबर्ह\accentmark{20}{\char"1CF9}णा\accentmark{22}{\char"0951}गीरा।\ 
मा\accentmark{27}{\char"1CD2}धो\accentmark{22}{\char"0951}अर्षन्तीधारा\accentmark{22}{\char"0951}\kern0.15emया\accentmark{27}{\char"1CD2}\mbox{॥ ९\hspace{0pt}॥} \\
पा\accentmark{27}{\char"1CD2}रीप्रासी\accentmark{22}{\char"0951}ष्यदत्काविः।\ सिन्धो\accentmark{22}{\char"0951}रूर्म\accentmark{20}{\char"1CF9}वा\accentmark{27}{\char"1CD2}धि\accentmark{22}{\char"0951}श्रीताः\accentmark{27}{\char"1CD2}।\ 
कारूं\accentmark{27}{\char"1CD2}\kern0.15emबि\accentmark{27}{\char"1CD2}भ्राट्‌यूरूस्पृ\accentmark{27}{\char"1CD2}ह\accentmark{22}{\char"0951}म्\mbox{॥ १०\hspace{0pt}॥} \\
 \mbox{॥ इति द्वितीयः खण्डः\hspace{0pt}॥} \\ 
ऊ\accentmark{27}{\char"1CD2}\kern0.15emपोषू\accentmark{27}{\char"1CD2}\kern0.15emजा\accentmark{27}{\char"1CD2}तामप्तू\accentmark{27}{\char"1CD2}रम्।\ 
गो\accentmark{20}{\char"1CF8}भि\accentmark{22}{\char"0951}र्भङ्ग\accentmark{27}{\char"1CD2}म्पारी\accentmark{22}{\char"0951}ष्कृतम्।\ 
इ\accentmark{20}{\char"1CF8}न्दु\accentmark{22}{\char"0951}न्देवा\accentmark{27}{\char"1CD2}आ\accentmark{22}{\char"0951}यासिषुः\mbox{॥ १\hspace{0pt}॥} \\
पुनानो\accentmark{27}{\char"1CD2}आ\accentmark{22}{\char"0951}क्रामीदाभिः\accentmark{27}{\char"1CD2}\kern0.15em।\ 
वि\accentmark{27}{\char"1CD2}श्वामृधोवी\accentmark{27}{\char"1CD2}\kern0.15emचऋ\accentmark{27}{\char"1CD2}षणिः।\ 
शुम्भ\accentmark{27}{\char"1CD2}न्तीवी\accentmark{20}{\char"1CF9}प्र\accentmark{22}{\char"0951}न्धीतीभीः\accentmark{22}{\char"0951}\mbox{॥ २\hspace{0pt}॥} \\
आवीश\accentmark{27}{\char"1CD2}न्काला\accentmark{20}{\char"1CF9}शं\accentmark{22}{\char"0951}सू\accentmark{22}{\char"0951}\kern0.15emतः\accentmark{27}{\char"1CD2}।\ 
विश्वाअ\accentmark{20}{\char"1CF9}र्ष\accentmark{22}{\char"0951}न्ताभि\accentmark{27}{\char"1CD2}\kern0.15emश्री\accentmark{27}{\char"1CD2}याः\accentmark{22}{\char"0951}।\ इन्दूरि\accentmark{27}{\char"1CD2}न्द्रा\accentmark{22}{\char"0951}यधीयते\mbox{॥ ३\hspace{0pt}॥} \\
आ\accentmark{20}{\char"1CF8}स\accentmark{22}{\char"0951}र्जिरथ्यो\accentmark{27}{\char"1CD2}\kern0.15emया\accentmark{27}{\char"1CD2}था\accentmark{22}{\char"0951}।\ 
पावी\accentmark{20}{\char"1CF9}त्रेचमूवोठ\accentmark{22}{\char"0951}सूतः।\ का\accentmark{20}{\char"1CF9}र्ष्ण\accentmark{22}{\char"0951}न्वा\accentmark{22}{\char"0951}जीन्या\accentmark{22}{\char"0951}क्रमीत्\mbox{॥ ४\hspace{0pt}॥} \\
प्र()य\accentmark{20}{\char"1CF8}त्गा\accentmark{22}{\char"0951}\kern0.15emवोनभू\accentmark{27}{\char"1CD2}र्णायः।\ 
त्वेषा\accentmark{27}{\char"1CD2}\kern0.15emआया\accentmark{27}{\char"1CD2}\kern0.15emसोआ\accentmark{27}{\char"1CD2}\kern0.15emक्रा\accentmark{27}{\char"1CD2}मुः।\ 
घ्न\accentmark{20}{\char"1CF8}न्तः\accentmark{22}{\char"0951}कृष्णा\accentmark{27}{\char"1CD2}मापत्वा\accentmark{27}{\char"1CD2}च\accentmark{22}{\char"0951}म्\mbox{॥ ५\hspace{0pt}॥} \\
अपध्नन्पा\accentmark{22}{\char"0951}वासेमृ\accentmark{27}{\char"1CD2}धाः।\ ऋतूवित्सो\accentmark{22}{\char"0951}\kern0.15emमा\accentmark{27}{\char"1CD2}मात्सारः\accentmark{27}{\char"1CD2}।\ नुतस्वा\accentmark{27}{\char"1CD2}दे\accentmark{22}{\char"0951}वायुञ्ज\accentmark{27}{\char"1CD2}न\accentmark{22}{\char"0951}म्\mbox{॥ ६\hspace{0pt}॥} \\
आया\accentmark{27}{\char"1CD2}पा\accentmark{22}{\char"0951}वास्वाधा\accentmark{27}{\char"1CD2}रा\accentmark{22}{\char"0951}या।\ 
या\accentmark{27}{\char"1CD2}\kern0.15emया\accentmark{27}{\char"1CD2}\kern0.15emसू\accentmark{27}{\char"1CD2}र्यामरो\accentmark{22}{\char"0951}चयः।\ 
हिन्वानो\accentmark{27}{\char"1CD2}मानू\accentmark{22}{\char"0951}षीरापाः\accentmark{27}{\char"1CD2}\mbox{॥ ७\hspace{0pt}॥} \\
सापा\accentmark{22}{\char"0951}वास्वायआ\accentmark{22}{\char"0951}वी\accentmark{22}{\char"0951}था।\ इ\accentmark{20}{\char"1CF8}न्द्रं\accentmark{20}{\char"1CF8}वृत्राया\accentmark{27}{\char"1CD2}\kern0.15emह\accentmark{27}{\char"1CD2}न्ता\accentmark{22}{\char"0951}वे।\ वब्रीवां\accentmark{20}{\char"1CF9}स\accentmark{22}{\char"0951}म्माहीरापः\accentmark{27}{\char"1CD2}\mbox{॥ ८\hspace{0pt}॥} \\
आया\accentmark{27}{\char"1CD2}वीतीपा\accentmark{27}{\char"1CD2}रिस्रावा।\ यस्ताइन्दो\accentmark{22}{\char"0951}मातेष्वा।\ आवा\accentmark{27}{\char"1CD2}ह\accentmark{22}{\char"0951}न्नावातीर्ना\accentmark{27}{\char"1CD2}वा\mbox{॥ ९\hspace{0pt}॥} \\
पारी\accentmark{22}{\char"0951}धुक्षं\accentmark{20}{\char"1CF9}सा\accentmark{20}{\char"1CF8}ना\accentmark{22}{\char"0951}द्रायि\accentmark{27}{\char"1CD2}म्।\ भारद्वा\accentmark{20}{\char"1CF9}ज\accentmark{22}{\char"0951}\kern0.15emन्नोअ\accentmark{27}{\char"1CD2}न्धा\accentmark{22}{\char"0951}सा।\ 
स्वानोअ\accentmark{22}{\char"0951}र्षपावी\accentmark{27}{\char"1CD2}\kern0.15emत्राआ\accentmark{27}{\char"1CD2}\mbox{॥ १०\hspace{0pt}॥} \\
\mbox{॥ इति तृतीयः खण्डः\hspace{0pt}॥} \\ 
आ\accentmark{27}{\char"1CD2}चिक्रादत्वृषाहा\accentmark{27}{\char"1CD2}रीः\accentmark{22}{\char"0951}\kern0.15em।\ माहा\accentmark{27}{\char"1CD2}न्मित्रो\accentmark{20}{\char"1CF9}ना\accentmark{20}{\char"1CF9}द\accentmark{22}{\char"0951}र्शातः।\ 
संसू\accentmark{27}{\char"1CD2}र्येणदिद्युते\mbox{॥ १\hspace{0pt}॥} \\
आ\accentmark{27}{\char"1CD2}तेदक्षं\accentmark{22}{\char"0951}मायोभू\accentmark{27}{\char"1CD2}व\accentmark{22}{\char"0951}\kern0.15emम्।\ व\accentmark{27}{\char"1CD2}\kern0.15emन्ही\accentmark{27}{\char"1CD2}\kern0.15emमद्या\accentmark{27}{\char"1CD2}वृ\accentmark{22}{\char"0951}\kern0.15emणीमहे\accentmark{27}{\char"1CD2}\kern0.15em।\ 
पा\accentmark{27}{\char"1CD2}न्तामा\accentmark{20}{\char"1CF9}पू\accentmark{22}{\char"0951}रूस्पृह\accentmark{22}{\char"0951}म्\mbox{॥ २\hspace{0pt}॥} \\
अ\accentmark{20}{\char"1CF8}ध्व\accentmark{22}{\char"0951}\kern0.15emर्योआ\accentmark{27}{\char"1CD2}द्री\accentmark{22}{\char"0951}भीस्सूतं।\ 
सो\accentmark{20}{\char"1CF8}मं\accentmark{22}{\char"0951}\kern0.15emपावी\accentmark{27}{\char"1CD2}\kern0.15emत्रा\accentmark{27}{\char"1CD2}\kern0.15emआ\accentmark{27}{\char"1CD2}नाया।\ पुनाही\accentmark{20}{\char"1CF9}न्द्रा\accentmark{22}{\char"0951}\kern0.15emयापा\accentmark{27}{\char"1CD2}ता\accentmark{22}{\char"0951}वे\mbox{॥ ३\hspace{0pt}॥} \\
ता\accentmark{27}{\char"1CD2}\kern0.15emरत्सा\accentmark{27}{\char"1CD2}मन्दीधा\accentmark{22}{\char"0951}वती।\ 
धा\accentmark{20}{\char"1CF8}रासूतस्या\accentmark{27}{\char"1CD2}न्धा\accentmark{22}{\char"0951}सः।\ 
ता\accentmark{27}{\char"1CD2}\kern0.15emरत्सा\accentmark{27}{\char"1CD2}मन्दीधावती\mbox{॥ ४\hspace{0pt}॥} \\
आ\accentmark{27}{\char"1CD2}पा\accentmark{22}{\char"0951}\kern0.15emवस्वा\accentmark{27}{\char"1CD2}साहास्री\accentmark{27}{\char"1CD2}णं।\ 
रा\accentmark{27}{\char"1CD2}यिंसो\accentmark{22}{\char"0951}मासूवी\accentmark{27}{\char"1CD2}र्यम्।\ 
अस्मेश्रा\accentmark{27}{\char"1CD2}\kern0.15emवां\accentmark{20}{\char"1CF8}सिधारया\mbox{॥ ५\hspace{0pt}॥} \\
आनूप्रत्ना\accentmark{20}{\char"1CF9}सा\accentmark{22}{\char"0951}\kern0.15emआया\accentmark{27}{\char"1CD2}वाः\accentmark{22}{\char"0951}।\ पादन्ना\accentmark{27}{\char"1CD2}वी\accentmark{22}{\char"0951}योअक्रमुः।\ 
रुचे\accentmark{27}{\char"1CD2}ज\accentmark{22}{\char"0951}नान्तासूर्य\accentmark{22}{\char"0951}म्\mbox{॥ ६\hspace{0pt}॥} \\
अ\accentmark{22}{\char"0951}र्षासोमा\accentmark{27}{\char"1CD2}धूमात्ता\accentmark{22}{\char"0951}मः।\ 
आभीद्रो\accentmark{20}{\char"1CF9}णा\accentmark{22}{\char"0951}\kern0.15emनीरो\accentmark{27}{\char"1CD2}रूवात्।\ सी\accentmark{27}{\char"1CD2}\kern0.15emदन्यो\accentmark{27}{\char"1CD2}\kern0.15emनौवा\accentmark{27}{\char"1CD2}नेष्वा\mbox{॥ ७\hspace{0pt}॥} \\
वृ\accentmark{27}{\char"1CD2}षा\accentmark{22}{\char"0951}सोमाधूमंआ\accentmark{22}{\char"0951}सी।\ 
वृषा\accentmark{27}{\char"1CD2}देवावृ\accentmark{27}{\char"1CD2}षाव्रा\accentmark{22}{\char"0951}तः।\ 
वृषाध\accentmark{27}{\char"1CD2}र्मा\accentmark{22}{\char"0951}णिदध्री\accentmark{27}{\char"1CD2}\kern0.15emषे\accentmark{27}{\char"1CD2}\mbox{॥ ८\hspace{0pt}॥} \\
ईषे\accentmark{27}{\char"1CD2}पा\accentmark{22}{\char"0951}वा\accentmark{20}{\char"1CF8}स्वाधा\accentmark{27}{\char"1CD2}रा\accentmark{22}{\char"0951}या।\ मृज्या\accentmark{27}{\char"1CD2}मा\accentmark{22}{\char"0951}नोमानीषी\accentmark{27}{\char"1CD2}भिः\accentmark{22}{\char"0951}।\ 
इ\accentmark{20}{\char"1CF8}न्दो\accentmark{22}{\char"0951}\kern0.15emरूचा\accentmark{27}{\char"1CD2}\kern0.15emभीगा\accentmark{27}{\char"1CD2}ईहि\mbox{॥ ९\hspace{0pt}॥} \\
मन्द्राया\accentmark{22}{\char"0951}सोमाधा\accentmark{27}{\char"1CD2}रा\accentmark{22}{\char"0951}य।\ 
वृषा\accentmark{22}{\char"0951}पवस्वदे\accentmark{27}{\char"1CD2}\kern0.15emवा\accentmark{27}{\char"1CD2}युः।\ 
अव्यावा\accentmark{27}{\char"1CD2}रे\accentmark{22}{\char"0951}भिरास्मायुः\mbox{॥ १०\hspace{0pt}॥} \\
आया\accentmark{27}{\char"1CD2}सोमसूकृत्या\accentmark{27}{\char"1CD2}या।\ माहा\accentmark{27}{\char"1CD2}न्सन्नाभ्या\accentmark{22}{\char"0951}वर्धथाः।\ म\accentmark{20}{\char"1CF8}न्दा\accentmark{22}{\char"0951}नाइ\accentmark{22}{\char"0951}द्वृषायसे\mbox{॥ ११\hspace{0pt}॥} \\
आया\accentmark{27}{\char"1CD2}वीचर्षणीर्हितः\accentmark{27}{\char"1CD2}।\ पावा\accentmark{22}{\char"0951}मानस्साचे\accentmark{22}{\char"0951}तति।\ 
हिन्वाना\accentmark{20}{\char"1CF9}आ\accentmark{20}{\char"1CF8}प्यंबृहात्\mbox{॥ १२\hspace{0pt}॥} \\
प्राणा\accentmark{22}{\char"0951}इन्दोमाहेतू\accentmark{27}{\char"1CD2}नाः\accentmark{22}{\char"0951}।\ ऊर्म्मि\accentmark{22}{\char"0951}न्नाबिभ्रादर्ष\accentmark{22}{\char"0951}सि।\ 
आभिदेवं\accentmark{27}{\char"1CD2}आयास्याः\accentmark{22}{\char"0951}\mbox{॥ १३\hspace{0pt}॥} \\
अपध्नन्पा\accentmark{22}{\char"0951}\kern0.15emवाते\accentmark{27}{\char"1CD2}मृधाः\accentmark{22}{\char"0951}\kern0.15em।\ 
आ\accentmark{27}{\char"1CD2}\kern0.15emप\accentmark{20}{\char"1CF8}सो\accentmark{27}{\char"1CD2}\kern0.15emमोआ\accentmark{27}{\char"1CD2}राब्णः\accentmark{27}{\char"1CD2}।\ गच्छन्निन्द्रा\accentmark{22}{\char"0951}स्यनिष्कृतम्\mbox{॥ १४\hspace{0pt}॥} \\
 \mbox{॥ इति चतुर्थः खण्डः\hspace{0pt}॥} \\ 
पुना\accentmark{20}{\char"1CF9}नस्सो\accentmark{22}{\char"0951}\kern0.15emमाधा\accentmark{27}{\char"1CD2}रा\accentmark{22}{\char"0951}या।\ 
आपो\accentmark{22}{\char"0951}\kern0.15emवा\accentmark{27}{\char"1CD2}सा\accentmark{22}{\char"0951}नोअर्ष\accentmark{22}{\char"0951}सि।\ 
आर\accentmark{22}{\char"0951}त्नाधा\accentmark{20}{\char"1CF9}यो\accentmark{20}{\char"1CF9}नीमृतास्या\accentmark{22}{\char"0951}सीदसि।\ उत्सोदेवो\accentmark{20}{\char"1CF9}ही\accentmark{22}{\char"0951}रण्या\accentmark{22}{\char"0951}या\mbox{॥ १\hspace{0pt}॥} \\
पा\accentmark{27}{\char"1CD2}रीतोषी\accentmark{22}{\char"0951}ञ्चतासू\accentmark{27}{\char"1CD2}तं।\ 
सोमोयाऊ\accentmark{22}{\char"0951}त्तामंहा\accentmark{27}{\char"1CD2}विः।\ दधन्वायो\accentmark{20}{\char"1CF9}न\accentmark{20}{\char"1CF9}र्योअप्सू\accentmark{27}{\char"1CD2}आट्यन्तारा\accentmark{27}{\char"1CD2}\kern0.15em।\ सूषा\accentmark{27}{\char"1CD2}\kern0.15emवासो\accentmark{27}{\char"1CD2}\kern0.15emमामा\accentmark{27}{\char"1CD2}द्री\accentmark{22}{\char"0951}भिः\mbox{॥ २\hspace{0pt}॥} \\
आ\accentmark{27}{\char"1CD2}सो\accentmark{22}{\char"0951}मस्वानो\accentmark{27}{\char"1CD2}अद्री\accentmark{22}{\char"0951}भिः।\ 
तीरोवा\accentmark{22}{\char"0951}राण्यव्या\accentmark{22}{\char"0951}या।\ 
जनोना\accentmark{27}{\char"1CD2}पूरीचम्मुवोठ\accentmark{22}{\char"0951}र्वीशद्धा\accentmark{27}{\char"1CD2}रिः।\ सादावा\accentmark{27}{\char"1CD2}नेषुद॑\accentmark{22}{\char"0951}ध्रीषे\mbox{॥ ३\hspace{0pt}॥} \\
प्रा\accentmark{20}{\char"1CF8}सो\accentmark{22}{\char"0951}मादेवा\accentmark{27}{\char"1CD2}वी\accentmark{22}{\char"0951}तये।\ 
सिन्धुर्न्ना\accentmark{20}{\char"1CF9}पी\accentmark{22}{\char"0951}\kern0.15emप्येअ\accentmark{27}{\char"1CD2}र्णासा।\ 
अं\accentmark{20}{\char"1CF8}शोःपा\accentmark{27}{\char"1CD2}या\accentmark{22}{\char"0951}सामादीरो\accentmark{27}{\char"1CD2}\kern0.15emना\accentmark{27}{\char"1CD2}\kern0.15emजा\accentmark{27}{\char"1CD2}गृविः।\ अच्छाको\accentmark{27}{\char"1CD2}श\accentmark{22}{\char"0951}म्माधुश्चू\accentmark{27}{\char"1CD2}त\accentmark{22}{\char"0951}म् \mbox{॥ ४\hspace{0pt}॥} \\
सोमा\accentmark{27}{\char"1CD2}उष्वाण\accentmark{27}{\char"1CD2}स्सोतृ\accentmark{27}{\char"1CD2}\kern0.15emभीः\accentmark{27}{\char"1CD2}।\ 
आधीष्णू\accentmark{27}{\char"1CD2}भीरावी\accentmark{22}{\char"0951}नां।\ 
अश्वा\accentmark{27}{\char"1CD2}ये\accentmark{22}{\char"0951}वाहारी\accentmark{27}{\char"1CD2}ता\accentmark{22}{\char"0951}यातीधा\accentmark{27}{\char"1CD2}रा\accentmark{22}{\char"0951}या।\ मन्द्रा\accentmark{27}{\char"1CD2}या\accentmark{22}{\char"0951}यातीधा\accentmark{27}{\char"1CD2}रा\accentmark{22}{\char"0951}या\mbox{॥ ५\hspace{0pt}॥} \\
तावाहंसो\accentmark{20}{\char"1CF8}मरारणा।\ 
सख्या\accentmark{27}{\char"1CD2}इ\accentmark{22}{\char"0951}न्दोदीवेदीवे।\ 
पूरू\accentmark{27}{\char"1CD2}णी\accentmark{22}{\char"0951}बभ्रोनीचा\accentmark{27}{\char"1CD2}र\accentmark{22}{\char"0951}न्तीमा\accentmark{27}{\char"1CD2}\kern0.15emमा\accentmark{27}{\char"1CD2}वा।\ पारीधीं\accentmark{27}{\char"1CD2}\kern0.15emरा\accentmark{27}{\char"1CD2}\kern0.15emतीतं\accentmark{27}{\char"1CD2}ई\accentmark{22}{\char"0951}हि\mbox{॥ ६\hspace{0pt}॥} \\
मृ\accentmark{27}{\char"1CD2}\kern0.15emज्या\accentmark{27}{\char"1CD2}मा\accentmark{22}{\char"0951}नस्सूहस्त्या\accentmark{27}{\char"1CD2}।\ 
समुद्रेवा\accentmark{27}{\char"1CD2}चा\accentmark{22}{\char"0951}मिन्वसि।\ रायिंपी\accentmark{27}{\char"1CD2}शं\accentmark{22}{\char"0951}गंबहूलं\accentmark{20}{\char"1CF9}पू\accentmark{22}{\char"0951}रूस्पृ\accentmark{22}{\char"0951}हं।\ पावामानाभ्यर्षसि\mbox{॥ ७\hspace{0pt}॥} \\
अभीसो\accentmark{27}{\char"1CD2}मा\accentmark{22}{\char"0951}\kern0.15emसाआया\accentmark{27}{\char"1CD2}वाः\accentmark{22}{\char"0951}।\ 
पा\accentmark{20}{\char"1CF8}वन्तेमा\accentmark{27}{\char"1CD2}द्यम्मादं\accentmark{22}{\char"0951}।\ समुद्रा\accentmark{20}{\char"1CF9}स्याधीविष्टापेमानीषीणाः\accentmark{22}{\char"0951}।\ मत्सा\accentmark{27}{\char"1CD2}रासोमादच्यू\accentmark{27}{\char"1CD2}ताः \mbox{॥ ८\hspace{0pt}॥} \\
पुनान\accentmark{20}{\char"1CF9}स्सोमाजाग्र\accentmark{22}{\char"0951}विः।\ 
अव्या\accentmark{27}{\char"1CD2}\kern0.15emवारैः\accentmark{27}{\char"1CD2}\kern0.15emपा\accentmark{20}{\char"1CF9}री\accentmark{22}{\char"0951}प्रीयः।\ 
त्वांवि\accentmark{27}{\char"1CD2}प्रो\accentmark{22}{\char"0951}अभवोअंगिरस्तमः।\ मध्वा\accentmark{22}{\char"0951}\kern0.15emय\accentmark{27}{\char"1CD2}\kern0.15emज्ञं\accentmark{27}{\char"1CD2}मीमीक्षणः\mbox{॥ ९\hspace{0pt}॥} \\
पा\accentmark{27}{\char"1CD2}वामानाअसृक्षता\accentmark{22}{\char"0951}\kern0.15em।\ पावी\accentmark{27}{\char"1CD2}त्रामाती\accentmark{27}{\char"1CD2}\kern0.15emधा\accentmark{27}{\char"1CD2}रा\accentmark{22}{\char"0951}या।\ मारू\accentmark{27}{\char"1CD2}त्व\accentmark{22}{\char"0951}न्तोमत्सारा\accentmark{20}{\char"1CF9}इन्द्रीया\accentmark{27}{\char"1CD2}\kern0.15emहायाः\accentmark{27}{\char"1CD2}\kern0.15em।\ मेधा\accentmark{27}{\char"1CD2}\kern0.15emमाभि\accentmark{27}{\char"1CD2}\kern0.15emप्रा\accentmark{27}{\char"1CD2}यां\accentmark{22}{\char"0951}\kern0.15emसिच\accentmark{27}{\char"1CD2} \mbox{॥ १०\hspace{0pt}॥} \\
पा\accentmark{27}{\char"1CD2}वा\accentmark{22}{\char"0951}स्ववाजसा\accentmark{27}{\char"1CD2}ता\accentmark{22}{\char"0951}\kern0.15emमः।\ 
आ\accentmark{27}{\char"1CD2}\kern0.15emभी\accentmark{20}{\char"1CF8}वि\accentmark{20}{\char"1CF9}श्वा\accentmark{22}{\char"0951}नीवार्याः।\ 
त्वंसा\accentmark{22}{\char"0951}मुद्रःप्रा\accentmark{22}{\char"0951}थामेवि\accentmark{27}{\char"1CD2}धर्मन्।\ देवे\accentmark{27}{\char"1CD2}भ्यासोममात्सा\accentmark{27}{\char"1CD2}रः\mbox{॥ ११\hspace{0pt}॥} \\
इन्द्रा\accentmark{22}{\char"0951}यापावातेमा\accentmark{27}{\char"1CD2}दाः\accentmark{22}{\char"0951}।\ 
सो\accentmark{20}{\char"1CF8}मो\accentmark{22}{\char"0951}\kern0.15emमारू\accentmark{27}{\char"1CD2}त्वा\accentmark{22}{\char"0951}तेसूतः\accentmark{27}{\char"1CD2}\kern0.15em।\ साहा\accentmark{27}{\char"1CD2}स्रा\accentmark{22}{\char"0951}धारोअत्या\accentmark{27}{\char"1CD2}\kern0.15emव्याम\accentmark{27}{\char"1CD2}र्ष\accentmark{22}{\char"0951}ति।\ तामि\accentmark{22}{\char"0951}मृजन्त्यायावाः\mbox{॥ १२\hspace{0pt}॥} \\
 \mbox{॥ इति पञ्चमः खण्डः\hspace{0pt}॥} \\ 
प्र\accentmark{20}{\char"1CF8}तूद्रा\accentmark{27}{\char"1CD2}\kern0.15emवापा\accentmark{27}{\char"1CD2}\kern0.15emरीकोश\accentmark{27}{\char"1CD2}न्नीषी\accentmark{22}{\char"0951}दा।\ नृभीः\accentmark{22}{\char"0951}पूनानोआभीवाज\accentmark{22}{\char"0951}मर्ष\accentmark{22}{\char"0951}।\ अश्वन्न\accentmark{20}{\char"1CF9}त्वा\accentmark{22}{\char"0951}वाजीन\accentmark{22}{\char"0951}र्म्म\accentmark{22}{\char"0951}र्जयन्ते।\ अच्छा\accentmark{22}{\char"0951}बर्ही\accentmark{20}{\char"1CF9}रा\accentmark{22}{\char"0951}\kern0.15emशाना\accentmark{27}{\char"1CD2}भिर्न्न\accentmark{22}{\char"0951}यन्ते \mbox{॥ १\hspace{0pt}॥} \\
प्राकाव्यामूशानेवब्रूवाणः।\ 
देवोदेवानाञ्जनीमाविवक्ति।\ माहीव्रातश्शूचीबन्धुःपावाकाः।\ पादावाराहोअभ्येतीरेभन्\mbox{॥ २\hspace{0pt}॥} \\
तिस्रोवाचा\accentmark{22}{\char"0951}ईरयातीप्रा\accentmark{27}{\char"1CD2}वह्रीः।\ ऋतस्या\accentmark{22}{\char"0951}धीतिंब्रह्मा\accentmark{22}{\char"0951}णोमानीषाम्।\ गावो\accentmark{22}{\char"0951}यन्तीगो\accentmark{27}{\char"1CD2}पातिंपृच्छामानाः।\ सो\accentmark{27}{\char"1CD2}मंयन्तिमा\accentmark{27}{\char"1CD2}\kern0.15emता\accentmark{27}{\char"1CD2}यो\accentmark{22}{\char"0951}वावाशानाः\accentmark{27}{\char"1CD2} \mbox{॥ ३\hspace{0pt}॥} \\
अस्य\accentmark{27}{\char"1CD2}प्रेषा\accentmark{27}{\char"1CD2}\kern0.15emहेमा\accentmark{20}{\char"1CF9}ना\accentmark{22}{\char"0951}\kern0.15emपूया\accentmark{27}{\char"1CD2}मा\accentmark{22}{\char"0951}नः।\ देवो\accentmark{27}{\char"1CD2}\kern0.15emदेवे\accentmark{27}{\char"1CD2}भिस्सा\accentmark{27}{\char"1CD2}मा\accentmark{22}{\char"0951}पृक्तारा\accentmark{27}{\char"1CD2}स\accentmark{22}{\char"0951}\kern0.15emम्।\ सू\accentmark{27}{\char"1CD2}\kern0.15emतःपा\accentmark{20}{\char"1CF9}वित्रंपर्येतीरे\accentmark{27}{\char"1CD2}भ\accentmark{22}{\char"0951}न्।\ मीतेवासा\accentmark{27}{\char"1CD2}द्मा\accentmark{22}{\char"0951}पाशूम\accentmark{22}{\char"0951}न्तीहो\accentmark{27}{\char"1CD2}ता\accentmark{22}{\char"0951} \mbox{॥ ४\hspace{0pt}॥} \\
अक्रा\accentmark{27}{\char"1CD2}न्सामुद्रः\accentmark{20}{\char"1CF9}प्रा\accentmark{22}{\char"0951}\kern0.15emथामे\accentmark{27}{\char"1CD2}\kern0.15emवि\accentmark{27}{\char"1CD2}\kern0.15emध\accentmark{27}{\char"1CD2}र्म्म\accentmark{22}{\char"0951}न्।\ जना\accentmark{20}{\char"1CF9}य\accentmark{22}{\char"0951}न्प्राजा\accentmark{27}{\char"1CD2}\kern0.15emभु\accentmark{27}{\char"1CD2}वानास्यागोपाः\accentmark{27}{\char"1CD2}।\ वृषा\accentmark{22}{\char"0951}\kern0.15emपावी\accentmark{27}{\char"1CD2}\kern0.15emत्रेआ\accentmark{27}{\char"1CD2}\kern0.15emधीसा\accentmark{27}{\char"1CD2}नोअव्ये\accentmark{22}{\char"0951}\kern0.15em।\ बृ\accentmark{27}{\char"1CD2}\kern0.15emह\accentmark{27}{\char"1CD2}\kern0.15emत्सो\accentmark{27}{\char"1CD2}मो\accentmark{22}{\char"0951}वावृधेस्वानो\accentmark{27}{\char"1CD2}\kern0.15emअ\accentmark{27}{\char"1CD2}द्रीः \mbox{॥ ५\hspace{0pt}॥} \\
का\accentmark{27}{\char"1CD2}नी\accentmark{22}{\char"0951}क्रन्तीहा\accentmark{27}{\char"1CD2}\kern0.15emरी\accentmark{20}{\char"1CF8}रासृज्या\accentmark{27}{\char"1CD2}मा\accentmark{22}{\char"0951}\kern0.15emनः।\ सी\accentmark{27}{\char"1CD2}\kern0.15emदन्वा\accentmark{27}{\char"1CD2}ना\accentmark{22}{\char"0951}स्याजठा\accentmark{22}{\char"0951}रेपूनानः।\ नृभी\accentmark{20}{\char"1CF9}रीयाताःकृणुतेनिर्न्नी\accentmark{22}{\char"0951}जङ्गाम्।\ आ\accentmark{20}{\char"1CF8}तो\accentmark{22}{\char"0951}मातीञ्जनायातास्वाधा\accentmark{27}{\char"1CD2}भिः \mbox{॥ ६\hspace{0pt}॥} \\
ए\accentmark{27}{\char"1CD2}\kern0.15emषस्या\accentmark{27}{\char"1CD2}\kern0.15emतेमा\accentmark{27}{\char"1CD2}धू\accentmark{22}{\char"0951}मंइन्द्रासोमाः\accentmark{22}{\char"0951}\kern0.15em।\ वृ\accentmark{27}{\char"1CD2}षावृष्णःपा\accentmark{20}{\char"1CF9}री\accentmark{22}{\char"0951}पावीत्रे\accentmark{22}{\char"0951}क्षाः।\ साहास्रादश्शा\accentmark{22}{\char"0951}तादा\accentmark{20}{\char"1CF9}भु\accentmark{27}{\char"1CD2}रीदावा\accentmark{22}{\char"0951}।\ शश्वक्तामं\accentmark{27}{\char"1CD2}\kern0.15emबर्ही\accentmark{27}{\char"1CD2}रावाज्यास्थात् \mbox{॥ ७\hspace{0pt}॥} \\
पा\accentmark{27}{\char"1CD2}वा\accentmark{22}{\char"0951}\kern0.15emस्वसो\accentmark{27}{\char"1CD2}मामाधूमंऋता\accentmark{27}{\char"1CD2}वा।\ आपो\accentmark{20}{\char"1CF9}वा\accentmark{20}{\char"1CF9}सा\accentmark{22}{\char"0951}\kern0.15emनोआ\accentmark{27}{\char"1CD2}\kern0.15emधीसा\accentmark{27}{\char"1CD2}\kern0.15emनोअ\accentmark{27}{\char"1CD2}व्ये\accentmark{22}{\char"0951}\kern0.15em।\ आ\accentmark{27}{\char"1CD2}\kern0.15emवद्रो\accentmark{27}{\char"1CD2}णानीघृता\accentmark{27}{\char"1CD2}\kern0.15emव\accentmark{20}{\char"1CF8}न्तीरोहा।\ मादिन्ता\accentmark{20}{\char"1CF8}मोमत्सारा\accentmark{20}{\char"1CF9}इन्द्रा\accentmark{22}{\char"0951}पानाः\mbox{॥ ८\hspace{0pt}॥} \\
सो\accentmark{27}{\char"1CD2}माःपवतेज\accentmark{20}{\char"1CF9}नीता\accentmark{27}{\char"1CD2}\kern0.15emमा\accentmark{20}{\char"1CF9}तीनां।\ जनीता\accentmark{27}{\char"1CD2}दीवोज\accentmark{22}{\char"0951}नीता\accentmark{20}{\char"1CF9}पृ\accentmark{22}{\char"0951}थिव्याः\accentmark{22}{\char"0951}।\ जनीता\accentmark{22}{\char"0951}ग्ने\accentmark{22}{\char"0951}र्जनीतासूर्या\accentmark{22}{\char"0951}स्या।\ जनीते\accentmark{27}{\char"1CD2}न्द्रा\accentmark{22}{\char"0951}स्याजनीतोतवी\accentmark{27}{\char"1CD2}ष्णोः\accentmark{22}{\char"0951}\kern0.15em\mbox{॥ ९\hspace{0pt}॥} \\
आ\accentmark{27}{\char"1CD2}भित्रीपृष्ठंवृ\accentmark{27}{\char"1CD2}षाणं\accentmark{22}{\char"0951}वायोधाम्।\ अ\accentmark{20}{\char"1CF8}ङ्गोषीणा\accentmark{20}{\char"1CF9}मवावाशन्तावाणीः।\ वानावसा\accentmark{22}{\char"0951}नोवा\accentmark{20}{\char"1CF9}रू\accentmark{22}{\char"0951}\kern0.15emणोनसी\accentmark{27}{\char"1CD2}न्धूः।\ वीरात्ना\accentmark{27}{\char"1CD2}\kern0.15emधा\accentmark{27}{\char"1CD2}\kern0.15emद\accentmark{27}{\char"1CD2}यातेवा\accentmark{27}{\char"1CD2}र्याणि\mbox{॥ १०\hspace{0pt}॥} \\
 \mbox{॥ इति षष्ठः खण्डः\hspace{0pt}॥} \\ 
प्रा\accentmark{20}{\char"1CF8}से\accentmark{22}{\char"0951}\kern0.15emनानी\accentmark{27}{\char"1CD2}\kern0.15emश्शू\accentmark{27}{\char"1CD2}\kern0.15emरोअ\accentmark{27}{\char"1CD2}\kern0.15emग्रे\accentmark{27}{\char"1CD2}राथा\accentmark{22}{\char"0951}नाम्।\ गव्य\accentmark{20}{\char"1CF9}न्ने\accentmark{22}{\char"0951}\kern0.15emतीह\accentmark{27}{\char"1CD2}र्षातेअस्या\accentmark{27}{\char"1CD2}\kern0.15emसे\accentmark{27}{\char"1CD2}ना\accentmark{22}{\char"0951}भद्रान्कृण्वान्निन्द्राहावान्सा\accentmark{27}{\char"1CD2}खी\accentmark{22}{\char"0951}भ्यः।\ आसोमोवा\accentmark{27}{\char"1CD2}स्त्रा\accentmark{22}{\char"0951}राभासा\accentmark{27}{\char"1CD2}नी\accentmark{22}{\char"0951}दक्ते\mbox{॥ १\hspace{0pt}॥} \\
प्रातेधा\accentmark{27}{\char"1CD2}\kern0.15emरामा\accentmark{27}{\char"1CD2}धू\accentmark{22}{\char"0951}मतिरसृगृन्।\ वा\accentmark{27}{\char"1CD2}\kern0.15emरंय\accentmark{27}{\char"1CD2}त्पूतोअत्ये\accentmark{27}{\char"1CD2}\kern0.15emष\accentmark{27}{\char"1CD2}व्यम्।\ पा\accentmark{27}{\char"1CD2}वा\accentmark{22}{\char"0951}मानापा\accentmark{20}{\char"1CF9}वा\accentmark{22}{\char"0951}\kern0.15emसेधा\accentmark{27}{\char"1CD2}मागोना\accentmark{22}{\char"0951}\kern0.15emम्।\ ज\accentmark{27}{\char"1CD2}\kern0.15emना\accentmark{27}{\char"1CD2}\kern0.15emयन्सू\accentmark{27}{\char"1CD2}र्या\accentmark{22}{\char"0951}मपिन्वोअर्कैः\accentmark{27}{\char"1CD2}\mbox{॥ २\hspace{0pt}॥} \\
प्रा\accentmark{27}{\char"1CD2}गा\accentmark{22}{\char"0951}यताभ्या\accentmark{22}{\char"0951}र्चामादेवा\accentmark{27}{\char"1CD2}\kern0.15emन्।\ सो\accentmark{27}{\char"1CD2}मं\accentmark{22}{\char"0951}हिनोतामाहातेधाना॑\accentmark{22}{\char"0951}या।\ स्वादुः\accentmark{27}{\char"1CD2}पा\accentmark{22}{\char"0951}वातामा\accentmark{27}{\char"1CD2}\kern0.15emतीवा\accentmark{27}{\char"1CD2}\kern0.15emरामा\accentmark{27}{\char"1CD2}व्यम्।\ आ\accentmark{27}{\char"1CD2}सी\accentmark{22}{\char"0951}दतुकालाश\accentmark{22}{\char"0951}\kern0.15emन्देव\accentmark{27}{\char"1CD2}\kern0.15emइ\accentmark{27}{\char"1CD2}न्दूः\accentmark{22}{\char"0951}\mbox{॥ ३\hspace{0pt}॥} \\
प्राहि\accentmark{27}{\char"1CD2}न्वानो\accentmark{20}{\char"1CF9}ज\accentmark{22}{\char"0951}नीतारो\accentmark{27}{\char"1CD2}दस्योः।\ रा\accentmark{27}{\char"1CD2}\kern0.15emथोनवा\accentmark{27}{\char"1CD2}जं\accentmark{22}{\char"0951}\kern0.15emसानीष\accentmark{27}{\char"1CD2}न्ना\accentmark{22}{\char"0951}यासीत्।\ इ\accentmark{27}{\char"1CD2}न्द्रं\accentmark{22}{\char"0951}गच्छन्ना\accentmark{27}{\char"1CD2}यु\accentmark{22}{\char"0951}धासंशी\accentmark{27}{\char"1CD2}शा\accentmark{22}{\char"0951}नाः।\ वि\accentmark{27}{\char"1CD2}श्वा\accentmark{22}{\char"0951}वा\accentmark{22}{\char"0951}\kern0.15emसूहा\accentmark{27}{\char"1CD2}स्ता\accentmark{22}{\char"0951}योरादधा\accentmark{22}{\char"0951}नः\mbox{॥ ४\hspace{0pt}॥} \\
त\accentmark{27}{\char"1CD2}क्षद्या\accentmark{27}{\char"1CD2}दिमा\accentmark{22}{\char"0951}नासावेना\accentmark{22}{\char"0951}\kern0.15emतोवा\accentmark{27}{\char"1CD2}क्।\ ज्येष्ठा\accentmark{22}{\char"0951}स्यधर्मन्द्यूक्षोरानी\accentmark{22}{\char"0951}के।\ आ\accentmark{20}{\char"1CF8}दी\accentmark{22}{\char"0951}मायन्वा\accentmark{27}{\char"1CD2}\kern0.15emरामा\accentmark{27}{\char"1CD2}वावाशानः\accentmark{27}{\char"1CD2}।\ जुष्टंपा\accentmark{20}{\char"1CF9}तिं\accentmark{22}{\char"0951}कालाशेगा\accentmark{27}{\char"1CD2}\kern0.15emवइ\accentmark{27}{\char"1CD2}न्दुम्\mbox{॥ ५\hspace{0pt}॥} \\
सा\accentmark{27}{\char"1CD2}\kern0.15emकामु\accentmark{27}{\char"1CD2}क्षोमर्जयन्तस्वासा\accentmark{22}{\char"0951}रः।\ दशधीरा\accentmark{22}{\char"0951}स्याधीता\accentmark{27}{\char"1CD2}\kern0.15emयोधा\accentmark{27}{\char"1CD2}नूत्रीः।\ हा\accentmark{27}{\char"1CD2}\kern0.15emरीःप\accentmark{27}{\char"1CD2}र्याद्रावार्जस्सू\accentmark{27}{\char"1CD2}र्या\accentmark{22}{\char"0951}स्या।\ द्रो\accentmark{27}{\char"1CD2}\kern0.15emण\accentmark{27}{\char"1CD2}न्ननक्षेअत्योना\accentmark{27}{\char"1CD2}वाजी\mbox{॥ ६\hspace{0pt}॥} \\
इ\accentmark{20}{\char"1CF8}न्दू\accentmark{22}{\char"0951}र्वाजी\accentmark{27}{\char"1CD2}पा\accentmark{22}{\char"0951}वातेगो\accentmark{27}{\char"1CD2}न्योघाः।\ इ\accentmark{27}{\char"1CD2}न्द्रेसो\accentmark{27}{\char"1CD2}\kern0.15emमस्सा\accentmark{27}{\char"1CD2}\kern0.15emहाइ\accentmark{27}{\char"1CD2}न्वन्मा\accentmark{27}{\char"1CD2}दा\accentmark{22}{\char"0951}\kern0.15emय।\ ह\accentmark{27}{\char"1CD2}न्तीरक्षो\accentmark{22}{\char"0951}बा\accentmark{20}{\char"1CF9}धा\accentmark{22}{\char"0951}तेपरियारा\accentmark{22}{\char"0951}तिं।\ वा\accentmark{27}{\char"1CD2}री\accentmark{22}{\char"0951}वःकृण्व\accentmark{27}{\char"1CD2}\kern0.15emन्वृ\accentmark{20}{\char"1CF9}ज\accentmark{20}{\char"1CF9}ना\accentmark{22}{\char"0951}\kern0.15emस्या\accentmark{27}{\char"1CD2}\kern0.15emरा\accentmark{27}{\char"1CD2}जा\accentmark{22}{\char"0951}\kern0.15em\mbox{॥ ७\hspace{0pt}॥} \\
आ\accentmark{27}{\char"1CD2}\kern0.15emधी\accentmark{27}{\char"1CD2}यादस्मिन्वा\accentmark{20}{\char"1CF9}जीनी\accentmark{22}{\char"0951}\kern0.15emवाशु\accentmark{27}{\char"1CD2}भाः।\ स्प\accentmark{20}{\char"1CF8}र्ध\accentmark{22}{\char"0951}न्तेधी\accentmark{27}{\char"1CD2}\kern0.15emयस्सू\accentmark{27}{\char"1CD2}\kern0.15emरेनवी\accentmark{27}{\char"1CD2}\kern0.15emशाः।\ आ\accentmark{27}{\char"1CD2}पो\accentmark{22}{\char"0951}\kern0.15emवृणा\accentmark{27}{\char"1CD2}नःपा\accentmark{22}{\char"0951}वातेका\accentmark{27}{\char"1CD2}वी\accentmark{22}{\char"0951}यान्।\ व्राजन्ना\accentmark{20}{\char"1CF9}प\accentmark{22}{\char"0951}\kern0.15emशुवा\accentmark{27}{\char"1CD2}र्थानायाम\accentmark{27}{\char"1CD2}न्मा\accentmark{22}{\char"0951}\mbox{॥ ८\hspace{0pt}॥} \\महत्तत्सो\accentmark{27}{\char"1CD2}मोमाहीषा\accentmark{27}{\char"1CD2}श्चा\accentmark{22}{\char"0951}कारा।\ आ\accentmark{27}{\char"1CD2}\kern0.15emपां\accentmark{27}{\char"1CD2}\kern0.15emयत्ग\accentmark{27}{\char"1CD2}\kern0.15emर्भो\accentmark{27}{\char"1CD2}वृ\accentmark{22}{\char"0951}णीतादेवा\accentmark{27}{\char"1CD2}न्।\ आद\accentmark{22}{\char"0951}धादिन्द्रेपा\accentmark{20}{\char"1CF8}वा\accentmark{22}{\char"0951}\kern0.15emमानाओ\accentmark{27}{\char"1CD2}जाः\accentmark{22}{\char"0951}।\ आज\accentmark{22}{\char"0951}नायत्सू\accentmark{27}{\char"1CD2}र्येज्यो\accentmark{27}{\char"1CD2}\kern0.15emतीरि\accentmark{27}{\char"1CD2}न्दूः\accentmark{22}{\char"0951}\mbox{॥ ९\hspace{0pt}॥} \\अपा\accentmark{27}{\char"1CD2}मीवेदूर्मायास्त\accentmark{27}{\char"1CD2}क्तूरा\accentmark{22}{\char"0951}णः।\ प्रा\accentmark{20}{\char"1CF8}मा\accentmark{22}{\char"0951}नीषा\accentmark{22}{\char"0951}ई\accentmark{22}{\char"0951}रातेसोमामा\accentmark{27}{\char"1CD2}च्छा\accentmark{22}{\char"0951}।\ नमस्य\accentmark{27}{\char"1CD2}न्तीरू\accentmark{20}{\char"1CF9}पाचायन्तीसञ्चा\accentmark{22}{\char"0951}।\ आचा\accentmark{22}{\char"0951}विशन्त्यूठ\accentmark{22}{\char"0951}\kern0.15emशाती\accentmark{27}{\char"1CD2}\kern0.15emरूश\accentmark{27}{\char"1CD2}न्त\accentmark{22}{\char"0951}म्\mbox{॥ १०\hspace{0pt}॥} \\
आया\accentmark{27}{\char"1CD2}पा\accentmark{22}{\char"0951}वापावस्वैनावा\accentmark{27}{\char"1CD2}\kern0.15emसू\accentmark{20}{\char"1CF8}नी।\ मां\accentmark{27}{\char"1CD2}श्चत्वाइन्दोसा\accentmark{20}{\char"1CF9}रा\accentmark{22}{\char"0951}साप्राध\accentmark{22}{\char"0951}न्वा।\ ब्रध्न\accentmark{22}{\char"0951}श्चिद्यस्यावा\accentmark{27}{\char"1CD2}तोनाजूतिं\accentmark{27}{\char"1CD2}\kern0.15em।\ पुरू\accentmark{27}{\char"1CD2}मेधा\accentmark{22}{\char"0951}श्चित्ताका\accentmark{22}{\char"0951}\kern0.15emवेना\accentmark{27}{\char"1CD2}रन्धात् \mbox{॥ ११\hspace{0pt}॥} \\
आ\accentmark{27}{\char"1CD2}स\accentmark{22}{\char"0951}र्जीवक्वारय्थेया\accentmark{27}{\char"1CD2}\kern0.15emथाजौ\accentmark{27}{\char"1CD2}\kern0.15em।\ धीया\accentmark{20}{\char"1CF9}मानो\accentmark{22}{\char"0951}क्ताप्राथामा\accentmark{20}{\char"1CF9}मा\accentmark{22}{\char"0951}नीषा।\ दशस्वासा\accentmark{22}{\char"0951}रोआधीसा\accentmark{22}{\char"0951}नोअव्ये।\ मृज\accentmark{27}{\char"1CD2}\kern0.15emन्ती\accentmark{27}{\char"1CD2}वहिंसा\accentmark{20}{\char"1CF9}द\accentmark{22}{\char"0951}\kern0.15emनेष्व\accentmark{27}{\char"1CD2}च्छा\accentmark{22}{\char"0951} \mbox{॥ १२\hspace{0pt}॥} \\
\mbox{॥ इति सप्तमः खण्डः\hspace{0pt}॥} \\ 
पूरो\accentmark{27}{\char"1CD2}जी\accentmark{22}{\char"0951}\kern0.15emतीवोअ\accentmark{27}{\char"1CD2}न्धासः।\ 
सूता\accentmark{27}{\char"1CD2}या\accentmark{22}{\char"0951}मादयित्ना\accentmark{27}{\char"1CD2}वे।\ 
आपश्वा\accentmark{27}{\char"1CD2}नंश्नथिष्टना।\ 
साखा\accentmark{22}{\char"0951}योदीर्घाजिठ\accentmark{22}{\char"0951}ह्व्यम्\mbox{॥ १\hspace{0pt}॥} \\
आयंपूषा\accentmark{27}{\char"1CD2}\kern0.15emरायि\accentmark{27}{\char"1CD2}र्भागाः\accentmark{27}{\char"1CD2}।\ 
सोमाः\accentmark{22}{\char"0951}पूनानोअ\accentmark{22}{\char"0951}र्ष\accentmark{22}{\char"0951}ति।\ पाति\accentmark{22}{\char"0951}र्वि\accentmark{20}{\char"1CF9}श्वा\accentmark{22}{\char"0951}स्याभू\accentmark{27}{\char"1CD2}म\accentmark{22}{\char"0951}नः।\ व्यख्यद्रोद\accentmark{22}{\char"0951}सीऊभे \mbox{॥ २\hspace{0pt}॥} \\
आह\accentmark{22}{\char"0951}रीयातायाधृ\accentmark{27}{\char"1CD2}ष्णावे।\ धा\accentmark{27}{\char"1CD2}नु\accentmark{22}{\char"0951}ष्टन्वन्तीपौंठ\accentmark{22}{\char"0951}स्यम्।\ शु\accentmark{27}{\char"1CD2}क्रावीय्यन्त्या\accentmark{27}{\char"1CD2}\kern0.15emसू\accentmark{27}{\char"1CD2}रायानिर्णीजे।\ वी\accentmark{22}{\char"0951}\kern0.15emपा\accentmark{27}{\char"1CD2}\kern0.15emमग्रे\accentmark{27}{\char"1CD2}माहीयू\accentmark{27}{\char"1CD2}वाः\accentmark{22}{\char"0951} \mbox{॥ ३\hspace{0pt}॥} \\
पा\accentmark{27}{\char"1CD2}रित्यं\accentmark{20}{\char"1CF9}हा\accentmark{22}{\char"0951}रीयातंहारीम्।\ 
बभ्रं\accentmark{27}{\char"1CD2}पु\accentmark{22}{\char"0951}नन्तीवा\accentmark{27}{\char"1CD2}रे\accentmark{22}{\char"0951}\kern0.15emण।\ 
यो\accentmark{27}{\char"1CD2}देवान्वि\accentmark{27}{\char"1CD2}श्वंइत्पारी\accentmark{22}{\char"0951}\kern0.15em।\ 
मा\accentmark{27}{\char"1CD2}दे\accentmark{22}{\char"0951}\kern0.15emनासह\accentmark{27}{\char"1CD2}गच्छा\accentmark{22}{\char"0951}ति \mbox{॥ ४\hspace{0pt}॥} \\
सूतासोमा\accentmark{27}{\char"1CD2}धूमक्तामः।\ 
सोमाइ\accentmark{27}{\char"1CD2}न्द्रा\accentmark{22}{\char"0951}\kern0.15emयाम\accentmark{27}{\char"1CD2}न्दीनाः।\ पावी\accentmark{27}{\char"1CD2}त्रवन्तोअक्षरन्।\ 
देवा\accentmark{27}{\char"1CD2}\kern0.15emन्ग\accentmark{27}{\char"1CD2}च्छन्तूवोमा\accentmark{27}{\char"1CD2}दाः\accentmark{22}{\char"0951}\mbox{॥ ५\hspace{0pt}॥} \\
सोमाःपवन्ताइन्दा\accentmark{22}{\char"0951}\kern0.15emवः।\ अ\accentmark{27}{\char"1CD2}\kern0.15emस्मा\accentmark{27}{\char"1CD2}भ्य\accentmark{22}{\char"0951}ङ्गातुवित्ता\accentmark{22}{\char"0951}मः।\ 
मित्रा\accentmark{27}{\char"1CD2}स्वाना\accentmark{20}{\char"1CF9}आ\accentmark{22}{\char"0951}\kern0.15emरेपा\accentmark{27}{\char"1CD2}साः\accentmark{22}{\char"0951}।\ स्वाध्या\accentmark{27}{\char"1CD2}ठ\accentmark{22}{\char"0951}\kern0.15emस्व\accentmark{27}{\char"1CD2}र्वीदः\accentmark{22}{\char"0951} \mbox{॥ ६\hspace{0pt}॥} \\
आ\accentmark{27}{\char"1CD2}\kern0.15emभी\accentmark{27}{\char"1CD2}नोवाजसा\accentmark{27}{\char"1CD2}तामं।\ रायीमऋ\accentmark{22}{\char"0951}षाशातस्पृह\accentmark{22}{\char"0951}म्।\ 
इ\accentmark{20}{\char"1CF8}न्दो\accentmark{22}{\char"0951}साहास्रा\accentmark{22}{\char"0951}भर्णा\accentmark{22}{\char"0951}सं।\ 
तूविद्युम्नंवीभासा\accentmark{27}{\char"1CD2}ह\accentmark{22}{\char"0951}म् \mbox{॥ ७\hspace{0pt}॥} \\
आभी\accentmark{27}{\char"1CD2}ना\accentmark{22}{\char"0951}वन्तेअद्रू\accentmark{27}{\char"1CD2}हाः।\ प्रीय\accentmark{20}{\char"1CF9}मिन्द्रा\accentmark{22}{\char"0951}\kern0.15emस्यका\accentmark{27}{\char"1CD2}म्यं\accentmark{22}{\char"0951}।\ वत्सन्नपू\accentmark{27}{\char"1CD2}\kern0.15emर्वाआ\accentmark{27}{\char"1CD2}यूनि।\ 
जातं\accentmark{27}{\char"1CD2}रिहन्तीमा\accentmark{22}{\char"0951}ताराः \mbox{॥ ८\hspace{0pt}॥} \\
प्रासुन्वानाय\accentmark{27}{\char"1CD2}\kern0.15emआ\accentmark{27}{\char"1CD2}न्धा\accentmark{22}{\char"0951}सः।\ 
मर्क्तोना\accentmark{20}{\char"1CF9}वा\accentmark{22}{\char"0951}\kern0.15emष्टात\accentmark{27}{\char"1CD2}\kern0.15emद्वा\accentmark{27}{\char"1CD2}चाः\accentmark{22}{\char"0951}।\ 
आपश्वा\accentmark{27}{\char"1CD2}ना\accentmark{22}{\char"0951}माराधा\accentmark{27}{\char"1CD2}सं\accentmark{22}{\char"0951}।\ 
हाता\accentmark{27}{\char"1CD2}\kern0.15emमाख\accentmark{27}{\char"1CD2}न्नाभ्र\accentmark{27}{\char"1CD2}ङ्गवः\accentmark{22}{\char"0951} \mbox{॥ ९\hspace{0pt}॥} \\
 \mbox{॥ इति अष्ठमः खण्डः\hspace{0pt}॥} \\ 
आ(2भी\accentmark{27}{\char"1CD2}प्रीया\accentmark{27}{\char"1CD2}णी\accentmark{22}{\char"0951}पवातेचानो\accentmark{22}{\char"0951}हितः।\ ना\accentmark{27}{\char"1CD2}मा\accentmark{22}{\char"0951}नीयह्वो\accentmark{27}{\char"1CD2}\kern0.15emआ\accentmark{27}{\char"1CD2}धीयेषूवा\accentmark{27}{\char"1CD2}र्धा\accentmark{22}{\char"0951}\kern0.15emते।\ 
आ\accentmark{27}{\char"1CD2}\kern0.15emसू\accentmark{27}{\char"1CD2}र्यास्यबृहा\accentmark{27}{\char"1CD2}\kern0.15emतो\accentmark{27}{\char"1CD2}बृहन्ना\accentmark{27}{\char"1CD2}\kern0.15emधी।\ रा\accentmark{27}{\char"1CD2}\kern0.15emथं\accentmark{27}{\char"1CD2}\kern0.15emविष्व\accentmark{27}{\char"1CD2}ञ्चमऋहद्वीचाक्षाणः\accentmark{27}{\char"1CD2}\mbox{॥ १\hspace{0pt}॥} \\
आचोद\accentmark{27}{\char"1CD2}\kern0.15emसो\accentmark{27}{\char"1CD2}नोधन्वन्त्विन्दा\accentmark{22}{\char"0951}वः।\ प्रस्वाना\accentmark{20}{\char"1CF9}सो\accentmark{22}{\char"0951}\kern0.15emबृहा\accentmark{27}{\char"1CD2}द्देवे\accentmark{27}{\char"1CD2}\kern0.15emषूहा\accentmark{27}{\char"1CD2}रा\accentmark{22}{\char"0951}\kern0.15emयः।\ वी\accentmark{27}{\char"1CD2}चीदश्नाना\accentmark{27}{\char"1CD2}\kern0.15emईषा\accentmark{27}{\char"1CD2}\kern0.15emयोआ\accentmark{27}{\char"1CD2}रा\accentmark{22}{\char"0951}तयः।\ अर्योना\accentmark{22}{\char"0951}\kern0.15emस्स\accentmark{27}{\char"1CD2}न्तुसानीषन्तुनोधी\accentmark{27}{\char"1CD2}याः\mbox{॥ २\hspace{0pt}॥} \\
ए\accentmark{27}{\char"1CD2}\kern0.15emष\accentmark{27}{\char"1CD2}\kern0.15emप्र\accentmark{27}{\char"1CD2}कोशेमाधू\accentmark{22}{\char"0951}मंअचिक्रदत्।\ इन्द्रस्याव\accentmark{20}{\char"1CF9}ज्रो\accentmark{22}{\char"0951}वापू\accentmark{20}{\char"1CF9}षो\accentmark{22}{\char"0951}\kern0.15emवापू\accentmark{27}{\char"1CD2}ष्टा\accentmark{22}{\char"0951}\kern0.15emमः।\ आ\accentmark{27}{\char"1CD2}\kern0.15emभीमृत\accentmark{20}{\char"1CF9}स्या\accentmark{22}{\char"0951}सूदूघा\accentmark{22}{\char"0951}घृतश्रूताः\accentmark{22}{\char"0951}।\ वाश्रा\accentmark{27}{\char"1CD2}\kern0.15emअ\accentmark{27}{\char"1CD2}ऋषन्तीपाया\accentmark{22}{\char"0951}साचाधे\accentmark{27}{\char"1CD2}\kern0.15emना\accentmark{27}{\char"1CD2}वाः\accentmark{22}{\char"0951} \mbox{॥ ३\hspace{0pt}॥} \\
प्रो\accentmark{27}{\char"1CD2}आ\accentmark{22}{\char"0951}यासिदिन्दूरिन्द्रा\accentmark{22}{\char"0951}स्यानिष्कृ\accentmark{22}{\char"0951}तं।\ साखासख्यु\accentmark{22}{\char"0951}र्नाप्रा\accentmark{27}{\char"1CD2}मीनातीसङ्गीर\accentmark{22}{\char"0951}\kern0.15emम्।\ 
म\accentmark{27}{\char"1CD2}र्या\accentmark{22}{\char"0951}यिवायुवातीभि\accentmark{22}{\char"0951}स्सामऋष\accentmark{27}{\char"1CD2}\kern0.15emति।\ सो\accentmark{20}{\char"1CF8}मः\accentmark{22}{\char"0951}काला\accentmark{20}{\char"1CF9}शेशाता\accentmark{27}{\char"1CD2}या\accentmark{22}{\char"0951}मानापाथा\mbox{॥ ४\hspace{0pt}॥} \\
धर्ता\accentmark{27}{\char"1CD2}\kern0.15emदीवाः\accentmark{27}{\char"1CD2}पा\accentmark{22}{\char"0951}वातेकृत्वि\accentmark{27}{\char"1CD2}योरासाः\accentmark{22}{\char"0951}।\ द\accentmark{20}{\char"1CF8}क्षोदेवा\accentmark{27}{\char"1CD2}नामानुमा\accentmark{27}{\char"1CD2}द्यो\accentmark{22}{\char"0951}नृभिः।\ हा\accentmark{27}{\char"1CD2}री\accentmark{22}{\char"0951}सृजानोअ\accentmark{27}{\char"1CD2}त्योनसत्वा\accentmark{22}{\char"0951}भीः।\ वृ\accentmark{22}{\char"0951}थापाजांसि\accentmark{22}{\char"0951}कृणूषे\accentmark{20}{\char"1CF9}नादीष्वा \mbox{॥ ५\hspace{0pt}॥} \\
वृ\accentmark{27}{\char"1CD2}षा\accentmark{22}{\char"0951}मातीनां\accentmark{27}{\char"1CD2}पा\accentmark{22}{\char"0951}वतेवीचाक्षाणः\accentmark{27}{\char"1CD2}।\ सोमोअह्रां\accentmark{22}{\char"0951}\kern0.15emप्रा\accentmark{27}{\char"1CD2}तारीतो\accentmark{20}{\char"1CF9}षा\accentmark{20}{\char"1CF9}सान्दीवाः\accentmark{27}{\char"1CD2}।\ प्राणा\accentmark{27}{\char"1CD2}सिन्धू\accentmark{22}{\char"0951}नांकाला\accentmark{27}{\char"1CD2}\kern0.15emशंअचि\accentmark{27}{\char"1CD2}क्रद\accentmark{22}{\char"0951}त्।\ इन्द्र\accentmark{22}{\char"0951}\kern0.15emस्या\accentmark{27}{\char"1CD2}\kern0.15emहा\accentmark{20}{\char"1CF9}र्द्या\accentmark{22}{\char"0951}वीशा\accentmark{20}{\char"1CF9}न्मा\accentmark{22}{\char"0951}नीषीभिः\mbox{॥ ६\hspace{0pt}॥} \\
आ\accentmark{20}{\char"1CF8}सा\accentmark{22}{\char"0951}\kern0.15emवीसो\accentmark{27}{\char"1CD2}मो\accentmark{22}{\char"0951}\kern0.15emआरूषो\accentmark{27}{\char"1CD2}वृषाहा\accentmark{27}{\char"1CD2}रिः\accentmark{22}{\char"0951}\kern0.15em।\ रा\accentmark{27}{\char"1CD2}जे\accentmark{22}{\char"0951}वादस्मो\accentmark{27}{\char"1CD2}\kern0.15emआभी\accentmark{27}{\char"1CD2}\kern0.15emगा\accentmark{27}{\char"1CD2}आ\accentmark{22}{\char"0951}चिक्रदत्।\ 
पुनानो\accentmark{27}{\char"1CD2}वारामा\accentmark{20}{\char"1CF9}त्ये\accentmark{22}{\char"0951}ष्वव्या\accentmark{22}{\char"0951}म्।\ श्येनो\accentmark{20}{\char"1CF9}नयो\accentmark{20}{\char"1CF9}निंघृता\accentmark{20}{\char"1CF9}व\accentmark{22}{\char"0951}न्तामा\accentmark{27}{\char"1CD2}सा\accentmark{22}{\char"0951}दत्\mbox{॥ ७\hspace{0pt}॥} \\
प्रा\accentmark{27}{\char"1CD2}\kern0.15emदेव\accentmark{27}{\char"1CD2}\kern0.15emमा\accentmark{27}{\char"1CD2}च्छामाधूमन्ताइन्दा\accentmark{22}{\char"0951}वः।\ आसी\accentmark{22}{\char"0951}ष्यदन्तागावाआ\accentmark{27}{\char"1CD2}\kern0.15emना\accentmark{27}{\char"1CD2}धेनावाः।\ बर्ही\accentmark{22}{\char"0951}षादो\accentmark{22}{\char"0951}वाचाना\accentmark{20}{\char"1CF9}व\accentmark{22}{\char"0951}\kern0.15emन्ताऊ\accentmark{27}{\char"1CD2}\kern0.15emधा\accentmark{20}{\char"1CF9}भिः।\ परिस्रू\accentmark{20}{\char"1CF9}ता\accentmark{22}{\char"0951}मुस्रीया\accentmark{22}{\char"0951}निर्णीज\accentmark{22}{\char"0951}न्धिरे\mbox{॥ ८\hspace{0pt}॥} \\
त्रिरा\accentmark{22}{\char"0951}स्मैसप्ता\accentmark{27}{\char"1CD2}\kern0.15emधेना\accentmark{27}{\char"1CD2}वो\accentmark{22}{\char"0951}दुदुह्रिरे।\ सत्या\accentmark{27}{\char"1CD2}माशीरंपारामे\accentmark{27}{\char"1CD2}व्यो\accentmark{22}{\char"0951}\kern0.15emमनि।\ च\accentmark{27}{\char"1CD2}त्वारिन्यो\accentmark{27}{\char"1CD2}भुवा\accentmark{22}{\char"0951}नानीनिर्णीजे।\ चा\accentmark{27}{\char"1CD2}रूणिचक्रेया\accentmark{27}{\char"1CD2}दृतैरा\accentmark{22}{\char"0951}वार्धत \mbox{॥ ९\hspace{0pt}॥} \\
इन्द्रा\accentmark{22}{\char"0951}यसोमासू\accentmark{20}{\char"1CF9}षूतःपा\accentmark{27}{\char"1CD2}री\accentmark{22}{\char"0951}स्रवा।\ अपा\accentmark{27}{\char"1CD2}मी\accentmark{22}{\char"0951}वाभवातुराक्षा\accentmark{22}{\char"0951}सासाहा।\ माते\accentmark{27}{\char"1CD2}\kern0.15emरासा\accentmark{20}{\char"1CF8}स्यमत्सतद्वा\accentmark{22}{\char"0951}यावीनाः\accentmark{22}{\char"0951}।\ द्रावी\accentmark{22}{\char"0951}णास्वन्ता\accentmark{27}{\char"1CD2}\kern0.15emईहा\accentmark{27}{\char"1CD2}सन्त्वि\accentmark{27}{\char"1CD2}न्दा\accentmark{22}{\char"0951}वः\mbox{॥ १०\hspace{0pt}॥} \\
अञ्ज\accentmark{27}{\char"1CD2}तेव्यठ\accentmark{22}{\char"0951}ञ्जतेसा\accentmark{27}{\char"1CD2}म\accentmark{22}{\char"0951}ञ्जते।\ क्रातुंरिहन्ती\accentmark{22}{\char"0951}मध्वाभ्या\accentmark{22}{\char"0951}ठ\accentmark{22}{\char"0951}ञ्जते।\ सि\accentmark{27}{\char"1CD2}न्धोरुच्छ्वासे\accentmark{27}{\char"1CD2}पाताय\accentmark{22}{\char"0951}न्तामुक्षाणं\accentmark{22}{\char"0951}।\ हिरण्यपावाःपाशू\accentmark{27}{\char"1CD2}\kern0.15emमप्सू\accentmark{27}{\char"1CD2}\kern0.15emगृ\accentmark{20}{\char"1CF8}भ्णते\mbox{॥ ११\hspace{0pt}॥} \\
पावि\accentmark{20}{\char"1CF9}त्र\accentmark{20}{\char"1CF9}न्ते\accentmark{22}{\char"0951}\kern0.15emवी\accentmark{27}{\char"1CD2}ता\accentmark{22}{\char"0951}तंब्रह्मणस्पते।\ प्राभु\accentmark{20}{\char"1CF9}र्गा\accentmark{20}{\char"1CF9}त्राणीप\accentmark{27}{\char"1CD2}र्येषिविश्वा\accentmark{27}{\char"1CD2}\kern0.15emताः।\ आ\accentmark{27}{\char"1CD2}ता\accentmark{22}{\char"0951}प्ततानुर्नाता\accentmark{27}{\char"1CD2}\kern0.15emदामो\accentmark{27}{\char"1CD2}आश्नुते।\ श्रूता\accentmark{27}{\char"1CD2}\kern0.15emसाइ\accentmark{20}{\char"1CF9}द्व\accentmark{20}{\char"1CF9}हन्तासन्तादा\accentmark{22}{\char"0951}शता \mbox{॥ १२\hspace{0pt}॥} \\
\mbox{॥ इति नवमः खण्डः\hspace{0pt}॥} \\ 
इन्द्राम\accentmark{20}{\char"1CF9}च्छा\accentmark{22}{\char"0951}\kern0.15emसूता\accentmark{27}{\char"1CD2}ईमे।\ 
वृ\accentmark{27}{\char"1CD2}षा\accentmark{22}{\char"0951}णंयन्तूहारा\accentmark{22}{\char"0951}याः।\ श्रूष्ठेजातासाइन्दावाःस्वर्वीदः \mbox{॥ १\hspace{0pt}॥} \\
प्रा\accentmark{27}{\char"1CD2}धा\accentmark{22}{\char"0951}न्वासोमा\accentmark{27}{\char"1CD2}\kern0.15emजा\accentmark{27}{\char"1CD2}ग्र\accentmark{22}{\char"0951}विः।\ इन्द्रा\accentmark{27}{\char"1CD2}\kern0.15emये\accentmark{27}{\char"1CD2}न्दोपा\accentmark{27}{\char"1CD2}री\accentmark{22}{\char"0951}स्रव।\ 
धुम\accentmark{22}{\char"0951}न्तंशुष्मामाभा\accentmark{22}{\char"0951}रास्वर्वी\accentmark{27}{\char"1CD2}दम् \mbox{॥ २\hspace{0pt}॥} \\
सा\accentmark{20}{\char"1CF8}खा\accentmark{22}{\char"0951}\kern0.15emयआ\accentmark{27}{\char"1CD2}नी\accentmark{22}{\char"0951}षीदत।\ 
पु\accentmark{27}{\char"1CD2}\kern0.15emना\accentmark{27}{\char"1CD2}नायप्रा\accentmark{27}{\char"1CD2}गा\accentmark{22}{\char"0951}यत।\ 
शी\accentmark{27}{\char"1CD2}शून्नाय\accentmark{27}{\char"1CD2}ज्ञैःपा\accentmark{27}{\char"1CD2}री\accentmark{22}{\char"0951}भूषताश्रिये\accentmark{27}{\char"1CD2} \mbox{॥ ३\hspace{0pt}॥} \\
तंवास्सखा\accentmark{22}{\char"0951}योमादा\accentmark{22}{\char"0951}\kern0.15emय।\ पु\accentmark{27}{\char"1CD2}\kern0.15emना\accentmark{27}{\char"1CD2}नामाभी\accentmark{27}{\char"1CD2}गा\accentmark{22}{\char"0951}यता।\ 
शीशून्ना\accentmark{27}{\char"1CD2}\kern0.15emहव्यै\accentmark{27}{\char"1CD2}स्वादयन्तागू\accentmark{22}{\char"0951}र्तीभिः\mbox{॥ ४\hspace{0pt}॥} \\
प्राणा\accentmark{20}{\char"1CF9}शी\accentmark{20}{\char"1CF9}शूर्माही\accentmark{27}{\char"1CD2}नां\accentmark{22}{\char"0951}।\ हिन्व\accentmark{27}{\char"1CD2}न्नृता\accentmark{27}{\char"1CD2}स्यादी\accentmark{27}{\char"1CD2}धी\accentmark{22}{\char"0951}तिं।\ 
वि\accentmark{27}{\char"1CD2}श्वापा\accentmark{20}{\char"1CF9}रीप्रीया\accentmark{20}{\char"1CF9}भु\accentmark{22}{\char"0951}वाद\accentmark{20}{\char"1CF9}धाद्वीता \mbox{॥ ५\hspace{0pt}॥} \\
पावा\accentmark{22}{\char"0951}स्वादेवा\accentmark{27}{\char"1CD2}वी\accentmark{22}{\char"0951}तये।\ 
इन्द्रोधाराभीरो\accentmark{27}{\char"1CD2}जसा।\ 
आ\accentmark{27}{\char"1CD2}\kern0.15emकाला\accentmark{27}{\char"1CD2}\kern0.15emशम्मा\accentmark{27}{\char"1CD2}धू\accentmark{22}{\char"0951}मान्सोमनस्सदः\accentmark{27}{\char"1CD2}\mbox{॥ ६\hspace{0pt}॥} \\
सो?2)माःपूनाना\accentmark{27}{\char"1CD2}ऊमींणा\accentmark{22}{\char"0951}\kern0.15em।\ अ\accentmark{27}{\char"1CD2}व्यंवा\accentmark{27}{\char"1CD2}रंवी\accentmark{22}{\char"0951}धा\accentmark{22}{\char"0951}वती।\ 
अ\accentmark{20}{\char"1CF8}ग्रे\accentmark{22}{\char"0951}\kern0.15emवाचः\accentmark{27}{\char"1CD2}पावमानःकानी\accentmark{22}{\char"0951}क्रदत् \mbox{॥ ७\hspace{0pt}॥} \\
प्रापू\accentmark{22}{\char"0951}नाना\accentmark{20}{\char"1CF9}या\accentmark{22}{\char"0951}\kern0.15emवेधा\accentmark{27}{\char"1CD2}से\accentmark{22}{\char"0951}।\ 
सो\accentmark{20}{\char"1CF8}मा\accentmark{22}{\char"0951}\kern0.15emयावा\accentmark{27}{\char"1CD2}चा\accentmark{22}{\char"0951}उच्यते।\ 
भृ\accentmark{22}{\char"0951}ति\accentmark{22}{\char"0951}र्न्ना\accentmark{27}{\char"1CD2}\kern0.15emभारा\accentmark{27}{\char"1CD2}\kern0.15emमा\accentmark{20}{\char"1CF9}तीभि\accentmark{22}{\char"0951}र्जुजो\accentmark{27}{\char"1CD2}षा\accentmark{22}{\char"0951}ते \mbox{॥ ८\hspace{0pt}॥} \\
गोम\accentmark{22}{\char"0951}न्नइन्दोअ\accentmark{27}{\char"1CD2}श्वा\accentmark{22}{\char"0951}वात्।\ सूता\accentmark{27}{\char"1CD2}स्सू\accentmark{22}{\char"0951}दक्षधनिवः।\ 
शूचि\accentmark{22}{\char"0951}\kern0.15emञ्चाव\accentmark{27}{\char"1CD2}र्ण\accentmark{22}{\char"0951}माधीगोषू\accentmark{22}{\char"0951}धाराया \mbox{॥ ९\hspace{0pt}॥} \\
अस्मा\accentmark{27}{\char"1CD2}भ्यन्त्वावासूवी\accentmark{27}{\char"1CD2}दं।\ 
आभीवाणी\accentmark{22}{\char"0951}रनूषत।\ 
गोभिष्टेव\accentmark{20}{\char"1CF9}र्णा\accentmark{22}{\char"0951}\kern0.15emमाभी\accentmark{27}{\char"1CD2}वा\accentmark{22}{\char"0951}सयामसि \mbox{॥ १०\hspace{0pt}॥} \\
पावा\accentmark{22}{\char"0951}तेहरियातोहारिः।\ 
आतिह्वा\accentmark{20}{\char"1CF9}रां\accentmark{22}{\char"0951}\kern0.15emसीरं\accentmark{27}{\char"1CD2}ह्या।\ 
अभ्यर्ष\accentmark{27}{\char"1CD2}स्तोतृ\accentmark{20}{\char"1CF9}भ्यो\accentmark{22}{\char"0951}\kern0.15emवीरा\accentmark{27}{\char"1CD2}\kern0.15emवद्या\accentmark{27}{\char"1CD2}शाः\accentmark{22}{\char"0951} \mbox{॥ ११\hspace{0pt}॥} \\
पारीको\accentmark{27}{\char"1CD2}श\accentmark{22}{\char"0951}म्माधू\accentmark{20}{\char"1CF9}श्चूतं।\ 
सो\accentmark{27}{\char"1CD2}मः\accentmark{22}{\char"0951}पूनानो\accentmark{27}{\char"1CD2}अ\accentmark{22}{\char"0951}र्षति।\ 
आ\accentmark{27}{\char"1CD2}भी\accentmark{22}{\char"0951}वाणीऋ\accentmark{22}{\char"0951}षीणां\accentmark{20}{\char"1CF9}सप्ता\accentmark{27}{\char"1CD2}नु\accentmark{22}{\char"0951}षत \mbox{॥ १२\hspace{0pt}॥} \\
\mbox{॥ इति दशमः खण्डः\hspace{0pt}॥} \\ 
पा\accentmark{20}{\char"1CF8}वा\accentmark{22}{\char"0951}स्वामा\accentmark{27}{\char"1CD2}धू\accentmark{22}{\char"0951}मक्तमः।\ 
इन्द्रा\accentmark{22}{\char"0951}यसोमक्रातूवि\accentmark{20}{\char"1CF9}क्ता\accentmark{22}{\char"0951}\kern0.15emमोमा\accentmark{27}{\char"1CD2}दाः\accentmark{22}{\char"0951}\kern0.15em।\ मा\accentmark{27}{\char"1CD2}हिद्युक्षा\accentmark{27}{\char"1CD2}तामोमा\accentmark{27}{\char"1CD2}दाः\accentmark{22}{\char"0951}\mbox{॥ १\hspace{0pt}॥} \\
आभि\accentmark{27}{\char"1CD2}द्युम्नं\accentmark{27}{\char"1CD2}बृहद्याशाः।\ 
ई\accentmark{27}{\char"1CD2}ळा\accentmark{22}{\char"0951}स्पतेदीदीही\accentmark{27}{\char"1CD2}देवादेवायूं।\ 
वीको\accentmark{27}{\char"1CD2}शं\accentmark{22}{\char"0951}माध्यामंयू\accentmark{22}{\char"0951}वा\mbox{॥ २\hspace{0pt}॥} \\
आ\accentmark{27}{\char"1CD2}सोतापा\accentmark{27}{\char"1CD2}री\accentmark{22}{\char"0951}षिञ्चता।\ अश्व\accentmark{27}{\char"1CD2}\kern0.15emन्न\accentmark{20}{\char"1CF9}स्तो\accentmark{22}{\char"0951}मामप्तूरं\accentmark{22}{\char"0951}राजस्तू\accentmark{22}{\char"0951}रं।\ वनप्राक्षा\accentmark{22}{\char"0951}मूतप्रूतम्\mbox{॥ ३\hspace{0pt}॥} \\एता\accentmark{27}{\char"1CD2}\kern0.15emमुत्य\accentmark{20}{\char"1CF9}म्मा\accentmark{22}{\char"0951}\kern0.15emदच्यू\accentmark{27}{\char"1CD2}तं\accentmark{22}{\char"0951}।\ 
साहा\accentmark{27}{\char"1CD2}स्रा\accentmark{22}{\char"0951}धारंवृषाभ\accentmark{20}{\char"1CF9}न्दी\accentmark{22}{\char"0951}वोदूहं\accentmark{22}{\char"0951}\kern0.15em।\ वि\accentmark{27}{\char"1CD2}श्वा\accentmark{22}{\char"0951}वा\accentmark{20}{\char"1CF9}सू\accentmark{22}{\char"0951}निबिभ्रा\accentmark{22}{\char"0951}तम्\mbox{॥ ४\hspace{0pt}॥} \\
सा\accentmark{20}{\char"1CF8}सुन्वोयोवा\accentmark{27}{\char"1CD2}सू\accentmark{22}{\char"0951}नां।\ 
यो\accentmark{27}{\char"1CD2}\kern0.15emराया\accentmark{20}{\char"1CF9}मानेता\accentmark{27}{\char"1CD2}याइळा\accentmark{22}{\char"0951}नां।\ सो\accentmark{27}{\char"1CD2}\kern0.15emमोय\accentmark{27}{\char"1CD2}स्सू\accentmark{22}{\char"0951}क्षीतीना\accentmark{27}{\char"1CD2}म्\mbox{॥ ५\hspace{0pt}॥} \\
त्वंह्यांट्ट्यगदै\accentmark{22}{\char"0951}व्या।\ 
पावा\accentmark{22}{\char"0951}मानाजनी\accentmark{22}{\char"0951}मानीधूमा\accentmark{27}{\char"1CD2}क्तामः।\ अ\accentmark{27}{\char"1CD2}मृता\accentmark{22}{\char"0951}त्वा\accentmark{20}{\char"1CF9}या\accentmark{22}{\char"0951}\kern0.15emघो\accentmark{27}{\char"1CD2}\kern0.15emषा\accentmark{27}{\char"1CD2}यान्\mbox{॥ ६\hspace{0pt}॥} \\
एषस्या\accentmark{27}{\char"1CD2}\kern0.15emधा\accentmark{27}{\char"1CD2}रा\accentmark{22}{\char"0951}यासूतः\accentmark{27}{\char"1CD2}।\ 
अव्यावा\accentmark{27}{\char"1CD2}रे\accentmark{22}{\char"0951}भिःपवतेमादि\accentmark{22}{\char"0951}न्तामः।\ क्रीळ\accentmark{22}{\char"0951}न्नूर्मिरापा\accentmark{27}{\char"1CD2}मी\accentmark{22}{\char"0951}वा\mbox{॥ ७\hspace{0pt}॥} \\
या\accentmark{27}{\char"1CD2}\kern0.15emउस्री\accentmark{27}{\char"1CD2}\kern0.15emयाआ\accentmark{27}{\char"1CD2}\kern0.15emपीया\accentmark{27}{\char"1CD2}अन्ताराश्मा\accentmark{22}{\char"0951}नि।\ नि\accentmark{20}{\char"1CF8}र्गा\accentmark{22}{\char"0951}आकृन्तादो\accentmark{27}{\char"1CD2}ज\accentmark{22}{\char"0951}सा।\ 
आभी\accentmark{27}{\char"1CD2}\kern0.15emव्र\accentmark{27}{\char"1CD2}जन्ता\accentmark{22}{\char"0951}त्नीषेग\accentmark{27}{\char"1CD2}व्यामाश्रयं।\ वर्मीवा\accentmark{22}{\char"0951}धृष्णावा\accentmark{22}{\char"0951}रूजा\mbox{॥ ८\hspace{0pt}॥} \\
\mbox{॥ इति एकादशः खण्डः\hspace{0pt}॥} \\   
\mbox{॥ इति पवमान पाठः समाप्तः\hspace{0pt}॥} \\ \clearpage
\mbox{॥ अथ आरण पाठः\hspace{0pt}॥} \\
इन्द्रोरा\accentmark{27}{\char"1CD2}जाजग\accentmark{22}{\char"0951}तश्वर्षाणीनां।\ आ\accentmark{20}{\char"1CF8}धि\accentmark{22}{\char"0951}क्षामा\accentmark{27}{\char"1CD2}विश्वा\accentmark{20}{\char"1CF9}रू\accentmark{22}{\char"0951}\kern0.15emपंया\accentmark{27}{\char"1CD2}द\accentmark{22}{\char"0951}स्या।\ ता\accentmark{27}{\char"1CD2}तो\accentmark{22}{\char"0951}ददातीदाशू\accentmark{27}{\char"1CD2}\kern0.15emषेवा\accentmark{27}{\char"1CD2}सू\accentmark{22}{\char"0951}नी।\ चोदद्राआधाऊ\accentmark{27}{\char"1CD2}पास्तुतञ्चीदर्वा\accentmark{27}{\char"1CD2}क्\mbox{॥ १\hspace{0pt}॥} \\
स\accentmark{27}{\char"1CD2}न्तेपा\accentmark{20}{\char"1CF9}यां\accentmark{22}{\char"0951}\kern0.15emसीसा\accentmark{27}{\char"1CD2}मूयन्तूवा\accentmark{27}{\char"1CD2}जाः\accentmark{22}{\char"0951}\kern0.15em।\ सं\accentmark{27}{\char"1CD2}वृष्ण्या\accentmark{22}{\char"0951}न्यभीमातीषा\accentmark{27}{\char"1CD2}हाः\accentmark{22}{\char"0951}।\ आप्यायमानोआमृ\accentmark{27}{\char"1CD2}ता\accentmark{22}{\char"0951}यसोमादीवि\accentmark{27}{\char"1CD2}\kern0.15emश्रा\accentmark{27}{\char"1CD2}वां\accentmark{22}{\char"0951}स्यूक्तामा\accentmark{27}{\char"1CD2}नीधत्स्वा\mbox{॥ २\hspace{0pt}॥} \\
त्वा\accentmark{27}{\char"1CD2}मीमाओषा\accentmark{22}{\char"0951}धीस्सोमा\accentmark{27}{\char"1CD2}\kern0.15emवि\accentmark{27}{\char"1CD2}श्वाः\accentmark{22}{\char"0951}\kern0.15em।\ त्वा\accentmark{27}{\char"1CD2}मापोआजनायस्त्वङगाः।\ त्वामा\accentmark{22}{\char"0951}तनोरूर्वाटट्य्न्ता\accentmark{27}{\char"1CD2}री\accentmark{22}{\char"0951}क्षाम्।\ त्व\accentmark{20}{\char"1CF8}ञ्ज्यो\accentmark{20}{\char"1CF8}तीषावीतामोववर्क्था।\ मायीवर्चीआथो\accentmark{22}{\char"0951}\kern0.15emभा\accentmark{27}{\char"1CD2}\kern0.15emगम्।\ आ\accentmark{20}{\char"1CF8}थो\accentmark{22}{\char"0951}\kern0.15emयज्ञा\accentmark{27}{\char"1CD2}स्यायत्पा\accentmark{27}{\char"1CD2}\kern0.15emयाः।\ प\accentmark{27}{\char"1CD2}रमेष्ठी\accentmark{27}{\char"1CD2}\kern0.15emप्रा\accentmark{27}{\char"1CD2}\kern0.15emजा\accentmark{27}{\char"1CD2}पा\accentmark{22}{\char"0951}तिः।\ दीवि\accentmark{27}{\char"1CD2}द्या\accentmark{22}{\char"0951}मी\accentmark{22}{\char"0951}वद्रंहतु\mbox{॥ ४\hspace{0pt}॥} \\
इ\accentmark{20}{\char"1CF9}दञ्छ\accentmark{20}{\char"1CF9}न्दो\accentmark{22}{\char"0951}जमादग्नेऋता\accentmark{27}{\char"1CD2}वृधाः\accentmark{22}{\char"0951}।\ ये\accentmark{20}{\char"1CF8}ना\accentmark{22}{\char"0951}\kern0.15emदेवा\accentmark{27}{\char"1CD2}सो\accentmark{22}{\char"0951}आमृतत्वामा\accentmark{27}{\char"1CD2}यान्।\ त\accentmark{27}{\char"1CD2}त्रद्या\accentmark{27}{\char"1CD2}वा\accentmark{22}{\char"0951}पृथिवी\accentmark{27}{\char"1CD2}\kern0.15emधक्ता\accentmark{27}{\char"1CD2}\kern0.15emमस्मा\accentmark{27}{\char"1CD2}\kern0.15emन्।\ य\accentmark{20}{\char"1CF8}त्रा\accentmark{27}{\char"1CD2}देवानांगूह्यन्नीधाम\mbox{॥ ५\hspace{0pt}॥} \\
सा\accentmark{27}{\char"1CD2}मान्याय\accentmark{27}{\char"1CD2}न्त्यूपा\accentmark{22}{\char"0951}यन्त्यन्याः\accentmark{27}{\char"1CD2}\kern0.15em।\ समा\accentmark{27}{\char"1CD2}नामुर्व\accentmark{27}{\char"1CD2}न्नद्याःठ\accentmark{22}{\char"0951}पृणन्ति।\ ता\accentmark{27}{\char"1CD2}\kern0.15emमू\accentmark{27}{\char"1CD2}\kern0.15emशू\accentmark{27}{\char"1CD2}चिंशूचायोदीदीवांसम्\accentmark{22}{\char"0951}।\ आपान्नापा\accentmark{20}{\char"1CF8}तामू\accentmark{20}{\char"1CF9}पा\accentmark{22}{\char"0951}यन्त्यापाः\accentmark{22}{\char"0951}\mbox{॥ ६\hspace{0pt}॥} \\
तेम\accentmark{22}{\char"0951}न्वतःप्रथामन्ना\accentmark{27}{\char"1CD2}मागोनां\accentmark{22}{\char"0951}।\ त्रिस्सप्ताः\accentmark{20}{\char"1CF9}पा\accentmark{20}{\char"1CF9}रा\accentmark{22}{\char"0951}\kern0.15emमन्ना\accentmark{27}{\char"1CD2}माजाना\accentmark{22}{\char"0951}म्।\ 
ताजाना\accentmark{22}{\char"0951}तिरभ्याठ\accentmark{22}{\char"0951}नूषातक्षाः\accentmark{27}{\char"1CD2}।\ आविर्भू\accentmark{22}{\char"0951}वन्नारू\accentmark{20}{\char"1CF9}णीर्याशा\accentmark{22}{\char"0951}\kern0.15emसा\accentmark{27}{\char"1CD2}\kern0.15emगा\accentmark{27}{\char"1CD2}वाः\accentmark{22}{\char"0951} \mbox{॥ ७\hspace{0pt}॥} \\
साहार्षाभासाहावात्साऊ\accentmark{27}{\char"1CD2}देता\accentmark{22}{\char"0951}।\ विश्वारूपा\accentmark{27}{\char"1CD2}\kern0.15emणीबि\accentmark{20}{\char"1CF8}भ्रा\accentmark{22}{\char"0951}तीद्यूर्ध्नीः।\ ऊरूः\accentmark{27}{\char"1CD2}\kern0.15emपृथू\accentmark{27}{\char"1CD2}\kern0.15emरायं\accentmark{27}{\char"1CD2}वोअस्तुलोकः\accentmark{27}{\char"1CD2}\kern0.15em।\ ईम\accentmark{27}{\char"1CD2}\kern0.15emआ\accentmark{27}{\char"1CD2}पा\accentmark{22}{\char"0951}सुप्रापाणा\accentmark{27}{\char"1CD2}\kern0.15emईह\accentmark{27}{\char"1CD2}स्ता\mbox{॥ ८\hspace{0pt}॥} \\
आरत्सू\accentmark{22}{\char"0951}पर्णाःपा\accentmark{22}{\char"0951}ततु।\ माश्वानामा\accentmark{27}{\char"1CD2}शी\accentmark{22}{\char"0951}तङ्करात्।\ तीर्था\accentmark{27}{\char"1CD2}नी\accentmark{22}{\char"0951}सुप्रा\accentmark{27}{\char"1CD2}\kern0.15emपा\accentmark{27}{\char"1CD2}णानी\accentmark{22}{\char"0951}\kern0.15em।\ सूग\accentmark{27}{\char"1CD2}\kern0.15emङगोषा\accentmark{27}{\char"1CD2}दनंकरात्\mbox{॥ ९\hspace{0pt}॥} \\
अग्नी\accentmark{27}{\char"1CD2}मीळेपूरोहि\accentmark{22}{\char"0951}तम्।\ यज्ञस्या\accentmark{22}{\char"0951}देवा\accentmark{22}{\char"0951}मृत्वी\accentmark{27}{\char"1CD2}जं\accentmark{22}{\char"0951}\kern0.15em।\ हो\accentmark{27}{\char"1CD2}तारंरानाधा\accentmark{27}{\char"1CD2}ता\accentmark{22}{\char"0951}मम्\mbox{॥ १०\hspace{0pt}॥} \\
\mbox{॥ इति प्रथमः खण्डः\hspace{0pt}॥} \\ 
म\accentmark{27}{\char"1CD2}न्ये\accentmark{22}{\char"0951}वान्द्यावापृथीवीसूभोज\accentmark{22}{\char"0951}सा।\ येआ\accentmark{27}{\char"1CD2}प्रा\accentmark{22}{\char"0951}थेथामामी\accentmark{27}{\char"1CD2}तामाभीयो\accentmark{27}{\char"1CD2}ज\accentmark{22}{\char"0951}नम्।\ द्या\accentmark{27}{\char"1CD2}वा\accentmark{22}{\char"0951}पृथीवीभा\accentmark{27}{\char"1CD2}वा\accentmark{22}{\char"0951}तंस्योने।\ ते\accentmark{27}{\char"1CD2}नोमुञ्चातामंहा\accentmark{22}{\char"0951}सः\mbox{॥ १\hspace{0pt}॥} \\या\accentmark{27}{\char"1CD2}शासामाद्या\accentmark{27}{\char"1CD2}वा\accentmark{22}{\char"0951}पृथिवी।\ भा\accentmark{27}{\char"1CD2}गे\accentmark{22}{\char"0951}नइन्द्रबृहास्पा\accentmark{27}{\char"1CD2}\kern0.15emती।\ या\accentmark{27}{\char"1CD2}\kern0.15emशोभा\accentmark{27}{\char"1CD2}ग\accentmark{22}{\char"0951}स्यविन्दतु।\ या\accentmark{20}{\char"1CF8}शो\accentmark{22}{\char"0951}माप्राती\accentmark{22}{\char"0951}मुच्यतां।\ यशस्व्याद्व्यस्यास्संसा\accentmark{27}{\char"1CD2}दाः\accentmark{22}{\char"0951}\kern0.15em।\ 
आहं\accentmark{27}{\char"1CD2}\kern0.15emप्रा\accentmark{27}{\char"1CD2}वादीता\accentmark{27}{\char"1CD2}स्यां\accentmark{22}{\char"0951}\kern0.15em।\ आहं\accentmark{27}{\char"1CD2}भू\accentmark{22}{\char"0951}यासमूक्तामः\accentmark{27}{\char"1CD2}\mbox{॥ २\hspace{0pt}॥} \\
प्र\accentmark{20}{\char"1CF8}क्षा\accentmark{22}{\char"0951}स्यावृ\accentmark{27}{\char"1CD2}ष्णो\accentmark{22}{\char"0951}आरूषस्यानू\accentmark{27}{\char"1CD2}\kern0.15emमा\accentmark{27}{\char"1CD2}हाः\accentmark{22}{\char"0951}\kern0.15em।\ प्रा\accentmark{27}{\char"1CD2}नुवो\accentmark{22}{\char"0951}चोवीद\accentmark{20}{\char"1CF9}था\accentmark{22}{\char"0951}\kern0.15emजाता\accentmark{27}{\char"1CD2}वेदसे।\ वै\accentmark{27}{\char"1CD2}श्वानाराया\accentmark{22}{\char"0951}मा\accentmark{20}{\char"1CF9}तिर्ना\accentmark{20}{\char"1CF9}व्या\accentmark{22}{\char"0951}\kern0.15emसेशू\accentmark{27}{\char"1CD2}चीः\accentmark{22}{\char"0951}\kern0.15em।\ सो\accentmark{27}{\char"1CD2}मा\accentmark{22}{\char"0951}इवपवातेचा\accentmark{20}{\char"1CF9}रू\accentmark{22}{\char"0951}रग्ना\accentmark{22}{\char"0951}ये\mbox{॥ ३\hspace{0pt}॥} \\
वि\accentmark{20}{\char"1CF8}श्वे\accentmark{22}{\char"0951}\kern0.15emदेवा\accentmark{27}{\char"1CD2}मामा\accentmark{22}{\char"0951}शृण्वन्तु\accentmark{27}{\char"1CD2}\kern0.15emयज्ञं\accentmark{27}{\char"1CD2}\kern0.15em।\ ऊ\accentmark{20}{\char"1CF9}भेरो\accentmark{27}{\char"1CD2}द\accentmark{22}{\char"0951}सीआपान्ना\accentmark{20}{\char"1CF9}पा\accentmark{22}{\char"0951}च्चामन्मा\accentmark{22}{\char"0951}।\ मा\accentmark{22}{\char"0951}वोवा\accentmark{22}{\char"0951}चांसिपारीचार्क्षाणि\accentmark{22}{\char"0951}वोचं।\ सुम्नेष्विद्वोअन्ता\accentmark{27}{\char"1CD2}मा\accentmark{22}{\char"0951}मदेम \mbox{॥ ४\hspace{0pt}॥} \\
एवा\accentmark{27}{\char"1CD2}\kern0.15emह्या\accentmark{27}{\char"1CD2}सी\accentmark{22}{\char"0951}वीरायुः।\ एवा\accentmark{27}{\char"1CD2}\kern0.15emशू\accentmark{27}{\char"1CD2}रा\accentmark{22}{\char"0951}\kern0.15emऊ\accentmark{27}{\char"1CD2}\kern0.15emता\accentmark{27}{\char"1CD2}स्थीरः\accentmark{27}{\char"1CD2}\kern0.15em।\ एवा\accentmark{27}{\char"1CD2}\kern0.15emतेरा\accentmark{27}{\char"1CD2}ध्याम्मा\accentmark{27}{\char"1CD2}नाः\accentmark{22}{\char"0951}\kern0.15em\mbox{॥ ५\hspace{0pt}॥} \\
आ\accentmark{27}{\char"1CD2}प्रागा\accentmark{22}{\char"0951}त्भद्रा\accentmark{20}{\char"1CF9}यू\accentmark{22}{\char"0951}वातिः\accentmark{22}{\char"0951}\kern0.15em।\ 
अ\accentmark{27}{\char"1CD2}ह्नः\accentmark{22}{\char"0951}\kern0.15emकेतु\accentmark{27}{\char"1CD2}\kern0.15emन्सा\accentmark{27}{\char"1CD2}मी\accentmark{22}{\char"0951}प्स्यती।\ आ\accentmark{20}{\char"1CF8}भु\accentmark{22}{\char"0951}त्भाद्रा\accentmark{27}{\char"1CD2}\kern0.15emनीवे\accentmark{27}{\char"1CD2}शा\accentmark{22}{\char"0951}नी।\ विश्वा\accentmark{22}{\char"0951}स्याज\accentmark{20}{\char"1CF9}ग\accentmark{22}{\char"0951}\kern0.15emतोरा\accentmark{27}{\char"1CD2}त्रिः\accentmark{22}{\char"0951}।\ उग्रं\accentmark{27}{\char"1CD2}\kern0.15emवा\accentmark{27}{\char"1CD2}\kern0.15emजी\accentmark{27}{\char"1CD2}न्याक्र\accentmark{27}{\char"1CD2}मीत्\mbox{॥ ६\hspace{0pt}॥} \\
अग्नी\accentmark{27}{\char"1CD2}राश्मीज\accentmark{27}{\char"1CD2}न्मा\accentmark{22}{\char"0951}नाजाता\accentmark{27}{\char"1CD2}वे\accentmark{22}{\char"0951}दाः।\ घृतम्मे\accentmark{27}{\char"1CD2}चाक्षु\accentmark{22}{\char"0951}रामृत\accentmark{22}{\char"0951}म्माआसन्।\ त्रिधा\accentmark{27}{\char"1CD2}स्त्वा\accentmark{20}{\char"1CF9}र्कोरा\accentmark{27}{\char"1CD2}ज\accentmark{22}{\char"0951}सोविमा\accentmark{27}{\char"1CD2}नः\accentmark{22}{\char"0951}।\ आज\accentmark{22}{\char"0951}सृञ्ज्यो\accentmark{20}{\char"1CF9}तिर्हा\accentmark{22}{\char"0951}वीरास्मिसा\accentmark{27}{\char"1CD2}र्व\accentmark{22}{\char"0951}म् \mbox{॥ ७\hspace{0pt}॥} \\
पात्य\accentmark{20}{\char"1CF9}ग्नीर्वीपो\accentmark{20}{\char"1CF9}अ\accentmark{20}{\char"1CF9}ग्रं\accentmark{22}{\char"0951}\kern0.15emपादं\accentmark{27}{\char"1CD2}वोः।\ पातीयह्न\accentmark{20}{\char"1CF9}श्चा\accentmark{20}{\char"1CF9}राणंसू\accentmark{27}{\char"1CD2}र्या\accentmark{22}{\char"0951}स्या।\ पा\accentmark{27}{\char"1CD2}\kern0.15emतीना\accentmark{20}{\char"1CF9}भा\accentmark{22}{\char"0951}\kern0.15emसप्ता\accentmark{27}{\char"1CD2}शी\accentmark{22}{\char"0951}र्षाणामग्नीः\accentmark{27}{\char"1CD2}\kern0.15em।\ पा\accentmark{27}{\char"1CD2}तीर्देवा\accentmark{27}{\char"1CD2}ना\accentmark{22}{\char"0951}मूपामा\accentmark{20}{\char"1CF8}द\accentmark{22}{\char"0951}मृष्वा\accentmark{27}{\char"1CD2}\mbox{॥ ८\hspace{0pt}॥} \\
भ्रा\accentmark{27}{\char"1CD2}ज\accentmark{22}{\char"0951}न्त्यग्नेसमीधानादी\accentmark{27}{\char"1CD2}\kern0.15emदीवः\accentmark{27}{\char"1CD2}।\ जिह्वा\accentmark{27}{\char"1CD2}चारत्यन्ता\accentmark{27}{\char"1CD2}\kern0.15emरासा\accentmark{27}{\char"1CD2}नी।\ सत्व\accentmark{27}{\char"1CD2}न्नो\accentmark{22}{\char"0951}अग्नेपा\accentmark{27}{\char"1CD2}यासावासूवी\accentmark{27}{\char"1CD2}त्।\ रायिंवर्चोंदृशे\accentmark{27}{\char"1CD2}दाः\accentmark{22}{\char"0951}\mbox{॥ ९\hspace{0pt}॥} \\
\mbox{॥ इति द्वितीयः खण्डः\hspace{0pt}॥} \\ 
इ\accentmark{20}{\char"1CF8}न्द्र\accentmark{22}{\char"0951}स्यानूवीर्या\accentmark{22}{\char"0951}णिप्रावो\accentmark{22}{\char"0951}चं।\ या\accentmark{20}{\char"1CF8}नी\accentmark{22}{\char"0951}\kern0.15emचाका\accentmark{27}{\char"1CD2}राप्राथामा\accentmark{22}{\char"0951}नीवज्रीः\accentmark{27}{\char"1CD2}\kern0.15em।\ आ\accentmark{27}{\char"1CD2}\kern0.15emहन्ना\accentmark{27}{\char"1CD2}हीमन्वापा\accentmark{27}{\char"1CD2}स्ता\accentmark{22}{\char"0951}तर्दा।\ प्रा\accentmark{27}{\char"1CD2}वक्षा\accentmark{22}{\char"0951}णाभीनत्पार्वा\accentmark{22}{\char"0951}तानाम्\mbox{॥ १\hspace{0pt}॥} \\
बिभ्राट्बृ\accentmark{22}{\char"0951}\kern0.15emहा\accentmark{27}{\char"1CD2}त्पीबतूसो\accentmark{27}{\char"1CD2}म्यम्माधू।\ आयूर्दध\accentmark{22}{\char"0951}द्यज्ञा\accentmark{20}{\char"1CF9}पा\accentmark{22}{\char"0951}तावाविहृतम्।\ वा\accentmark{27}{\char"1CD2}ताजूतोयोआ\accentmark{22}{\char"0951}भीराक्षा\accentmark{22}{\char"0951}तीत्मा\accentmark{27}{\char"1CD2}ना।\ प्राजाः\accentmark{27}{\char"1CD2}पिबर्तीबहुधा\accentmark{27}{\char"1CD2}\kern0.15emवी\accentmark{27}{\char"1CD2}रा\accentmark{22}{\char"0951}ज\accentmark{20}{\char"1CF9}ति\mbox{॥ २\hspace{0pt}॥} \\
वा\accentmark{27}{\char"1CD2}स\accentmark{22}{\char"0951}न्तइ\accentmark{20}{\char"1CF8}न्नुरन्याः\accentmark{20}{\char"1CF9}।\ ग्रीष्मइन्नु\accentmark{27}{\char"1CD2}\kern0.15emर\accentmark{27}{\char"1CD2}त्न्याः\accentmark{22}{\char"0951}\kern0.15em।\ वा\accentmark{27}{\char"1CD2}र्षाणइन्नुशारा\accentmark{27}{\char"1CD2}दाः\accentmark{22}{\char"0951}।\ हेमन्ता\accentmark{20}{\char"1CF9}शी\accentmark{20}{\char"1CF9}शीरइन्नुरन्त्याः\accentmark{22}{\char"0951}\mbox{॥ ३\hspace{0pt}॥} \\
ता\accentmark{27}{\char"1CD2}\kern0.15emवत्य\accentmark{20}{\char"1CF9}पीतोददतः।\ तावास्वादिष्ठाते\accentmark{27}{\char"1CD2}पीतो।\ प्रस्वद्वा\accentmark{27}{\char"1CD2}नोरासा\accentmark{22}{\char"0951}नां।\ तूविग्री\accentmark{27}{\char"1CD2}वा\accentmark{22}{\char"0951}इवेरते\mbox{॥ ४\hspace{0pt}॥} \\इन्द्र\accentmark{20}{\char"1CF9}स्फाढूतावृ\accentmark{22}{\char"0951}त्राहा\accentmark{27}{\char"1CD2}।\ परस्पा\accentmark{27}{\char"1CD2}\kern0.15emनोवा\accentmark{27}{\char"1CD2}रेण्यः।\ सा\accentmark{27}{\char"1CD2}नो\accentmark{22}{\char"0951}र\accentmark{20}{\char"1CF9}क्षिषच्चार\accentmark{20}{\char"1CF9}मंसा\accentmark{20}{\char"1CF9}मा\accentmark{27}{\char"1CD2}ध्यामं।\ 
सा\accentmark{27}{\char"1CD2}पश्चात्पातूसा\accentmark{27}{\char"1CD2}पू\accentmark{22}{\char"0951}राः\mbox{॥ ५\hspace{0pt}॥} \\
सह\accentmark{27}{\char"1CD2}स्राशीर्षापूरू\accentmark{22}{\char"0951}षः।\ सहस्राक्षा\accentmark{27}{\char"1CD2}स्साहस्रा\accentmark{27}{\char"1CD2}पात्।\ सा\accentmark{20}{\char"1CF8}भू\accentmark{22}{\char"0951}\kern0.15emमिंस\accentmark{27}{\char"1CD2}र्वातो\accentmark{22}{\char"0951}वृत्वा।\ अत्यातिष्ठद्दशांगूलम् \mbox{॥ ६\hspace{0pt}॥} \\
त्रीपा\accentmark{27}{\char"1CD2}दूर्ध्वा\accentmark{22}{\char"0951}\kern0.15emऊदू\accentmark{27}{\char"1CD2}यत्पूरू\accentmark{27}{\char"1CD2}षः।\ पादो\accentmark{22}{\char"0951}स्येहा\accentmark{20}{\char"1CF9}भा\accentmark{22}{\char"0951}वत्पूनाः।\ ता\accentmark{27}{\char"1CD2}थाविष्वंव्याठ\accentmark{22}{\char"0951}क्रमात्।\ अ\accentmark{20}{\char"1CF9}श\accentmark{22}{\char"0951}ना\accentmark{22}{\char"0951}\kern0.15emनशने\accentmark{27}{\char"1CD2}आभि\mbox{॥ ७\hspace{0pt}॥} \\
पूरू\accentmark{22}{\char"0951}\kern0.15emषाए\accentmark{27}{\char"1CD2}\kern0.15emवेदंस\accentmark{27}{\char"1CD2}र्वं\accentmark{22}{\char"0951}\kern0.15em।\ य\accentmark{27}{\char"1CD2}त्भूतंया\accentmark{27}{\char"1CD2}च्चाभा\accentmark{27}{\char"1CD2}व्यं\accentmark{22}{\char"0951}।\ पादो\accentmark{22}{\char"0951}स्यास\accentmark{20}{\char"1CF9}र्वा\accentmark{22}{\char"0951}भूतानी।\ त्रीपाद\accentmark{22}{\char"0951}स्यामृ\accentmark{20}{\char"1CF9}त\accentmark{22}{\char"0951}न्दीवी\accentmark{27}{\char"1CD2} \mbox{॥ ८\hspace{0pt}॥} \\
ता\accentmark{27}{\char"1CD2}वा\accentmark{22}{\char"0951}नस्यामाहीमा\accentmark{27}{\char"1CD2}\kern0.15em।\ ता\accentmark{27}{\char"1CD2}\kern0.15emतो\accentmark{20}{\char"1CF8}ज्यायांश्चापू\accentmark{27}{\char"1CD2}रु\accentmark{22}{\char"0951}षः।\ ऊता\accentmark{20}{\char"1CF9}मृ\accentmark{22}{\char"0951}तत्वस्येशा\accentmark{22}{\char"0951}नः।\ याद\accentmark{22}{\char"0951}न्नेनातीरो\accentmark{20}{\char"1CF9}हा\accentmark{22}{\char"0951}ति\mbox{॥ ९\hspace{0pt}॥} \\
ता\accentmark{20}{\char"1CF8}तो\accentmark{22}{\char"0951}\kern0.15emवीराड\accentmark{27}{\char"1CD2}\kern0.15emजा\accentmark{27}{\char"1CD2}यत।\ वीरा\accentmark{27}{\char"1CD2}\kern0.15emजोआ\accentmark{27}{\char"1CD2}\kern0.15emधीपू\accentmark{27}{\char"1CD2}रु\accentmark{22}{\char"0951}\kern0.15emषः।\ सा\accentmark{27}{\char"1CD2}\kern0.15emजातो\accentmark{27}{\char"1CD2}अत्या\accentmark{22}{\char"0951}रिच्यत।\ पश्चात्भू\accentmark{27}{\char"1CD2}\kern0.15emमि\accentmark{20}{\char"1CF8}मा\accentmark{20}{\char"1CF8}थो\accentmark{22}{\char"0951}\kern0.15emपूरः\accentmark{27}{\char"1CD2} \mbox{॥ १०\hspace{0pt}॥} \\
\mbox{॥ इति तृतीयः खण्डः\hspace{0pt}॥} \\ 
य\accentmark{27}{\char"1CD2}\kern0.15emत्पू\accentmark{27}{\char"1CD2}रूषेणाहावीषाः\accentmark{27}{\char"1CD2}\kern0.15em।\ देवा\accentmark{27}{\char"1CD2}यज्ञामा\accentmark{27}{\char"1CD2}त\accentmark{22}{\char"0951}न्वाताः।\ वस\accentmark{22}{\char"0951}न्ताये\accentmark{22}{\char"0951}षामासीदा\accentmark{27}{\char"1CD2}ज्यं\accentmark{22}{\char"0951}।\ ग्रीष्मा\accentmark{27}{\char"1CD2}इध्माः\accentmark{20}{\char"1CF9}शा\accentmark{20}{\char"1CF8}रा\accentmark{22}{\char"0951}द्धाविः\mbox{॥ १\hspace{0pt}॥} \\
सप्ता\accentmark{27}{\char"1CD2}स्या\accentmark{22}{\char"0951}सन्पारीधा\accentmark{27}{\char"1CD2}याः।\ त्रीस्सप्त\accentmark{27}{\char"1CD2}स्समीधाः\accentmark{27}{\char"1CD2}कृ\accentmark{22}{\char"0951}ताः।\ देवायद्य\accentmark{27}{\char"1CD2}\kern0.15emज्ञ\accentmark{27}{\char"1CD2}न्तन्वानः\accentmark{27}{\char"1CD2}।\ आब\accentmark{22}{\char"0951}ध्नन्पू\accentmark{27}{\char"1CD2}रू\accentmark{22}{\char"0951}षंहाविः\accentmark{27}{\char"1CD2} \mbox{॥ २\hspace{0pt}॥} \\
हा\accentmark{27}{\char"1CD2}री\accentmark{22}{\char"0951}तइन्द्रश्माश्रू\accentmark{27}{\char"1CD2}णीः।\ ऊतो\accentmark{20}{\char"1CF9}ते\accentmark{22}{\char"0951}हा\accentmark{22}{\char"0951}रीताहा\accentmark{27}{\char"1CD2}रीः।\ तन्त्वा\accentmark{22}{\char"0951}स्तुवन्तीकारा\accentmark{22}{\char"0951}वाः।\ वार्षा\accentmark{20}{\char"1CF9}सोवा\accentmark{22}{\char"0951}नार्गा\accentmark{22}{\char"0951}वः \mbox{॥ ३\hspace{0pt}॥} \\
ऋ\accentmark{27}{\char"1CD2}ष्या\accentmark{22}{\char"0951}सइन्द्रभूं\accentmark{27}{\char"1CD2}ईति।\ मा\accentmark{27}{\char"1CD2}घा\accentmark{22}{\char"0951}वन्विन्द्राभूंई\accentmark{27}{\char"1CD2}\kern0.15emति।\ इ\accentmark{27}{\char"1CD2}न्द्र\accentmark{22}{\char"0951}\kern0.15emस्ता\accentmark{27}{\char"1CD2}ता\accentmark{22}{\char"0951}रपूतः\mbox{॥ ४\hspace{0pt}॥} \\
य\accentmark{27}{\char"1CD2}द्वार्चो\accentmark{22}{\char"0951}हीरण्य\accentmark{27}{\char"1CD2}स्य\accentmark{22}{\char"0951}।\ यद्वा\accentmark{27}{\char"1CD2}\kern0.15emव\accentmark{27}{\char"1CD2}र्चो\accentmark{22}{\char"0951}ग\accentmark{20}{\char"1CF9}वा\accentmark{22}{\char"0951}मूता\accentmark{22}{\char"0951}।\ सत्या\accentmark{27}{\char"1CD2}स्या\accentmark{22}{\char"0951}ब्र\accentmark{20}{\char"1CF9}ह्माणोव\accentmark{27}{\char"1CD2}र्चाः\accentmark{22}{\char"0951}\kern0.15em।\ ते\accentmark{27}{\char"1CD2}ना\accentmark{22}{\char"0951}मसंसृजाम\accentmark{22}{\char"0951}सी \mbox{॥ ५\hspace{0pt}॥} \\
चित्रन्देवा\accentmark{27}{\char"1CD2}नामूद\accentmark{22}{\char"0951}कादनी\accentmark{22}{\char"0951}कं।\ च\accentmark{20}{\char"1CF8}क्षुर्मित्र\accentmark{27}{\char"1CD2}स्यावा\accentmark{27}{\char"1CD2}रूणास्याग्नेः\accentmark{27}{\char"1CD2}।\ 
आप्राद्या\accentmark{27}{\char"1CD2}वापृथीवी\accentmark{22}{\char"0951}अन्तारी\accentmark{22}{\char"0951}क्षं।\ सू\accentmark{20}{\char"1CF8}र्यात्माजग\accentmark{22}{\char"0951}तास्तस्थूषा\accentmark{22}{\char"0951}श्च\mbox{॥ ६\hspace{0pt}॥} \\
या\accentmark{27}{\char"1CD2}\kern0.15emदीत\accentmark{20}{\char"1CF8}स्त\accentmark{27}{\char"1CD2}न्वोमामा\accentmark{22}{\char"0951}।\ दोषा\accentmark{22}{\char"0951}रासास्यभेजिरे\accentmark{22}{\char"0951}\kern0.15em।\ ना\accentmark{27}{\char"1CD2}\kern0.15emराशं\accentmark{20}{\char"1CF9}सेनासोमे\accentmark{22}{\char"0951}\kern0.15emना।\ आ\accentmark{27}{\char"1CD2}\kern0.15emहन्त\accentmark{27}{\char"1CD2}\kern0.15emत्पू\accentmark{27}{\char"1CD2}\kern0.15emनारा\accentmark{20}{\char"1CF8}दे\accentmark{22}{\char"0951}दे\mbox{॥ ७\hspace{0pt}॥} \\
यूष्मा\accentmark{27}{\char"1CD2}दस्पारासस्पारी।\ यस्ये\accentmark{22}{\char"0951}दमारा\accentmark{22}{\char"0951}\kern0.15emजोयू\accentmark{27}{\char"1CD2}जाः\accentmark{22}{\char"0951}।\ तुजे\accentmark{22}{\char"0951}\kern0.15emज\accentmark{27}{\char"1CD2}ने\accentmark{22}{\char"0951}वानंस्वाःठ\accentmark{22}{\char"0951}।\ इन्द्र\accentmark{22}{\char"0951}स्यार\accentmark{20}{\char"1CF9}न्त्यं\accentmark{22}{\char"0951}\kern0.15emबृहा\accentmark{27}{\char"1CD2}त्\mbox{॥ ८\hspace{0pt}॥} \\
सा\accentmark{27}{\char"1CD2}हास्तन्न\accentmark{22}{\char"0951}इन्द्रादध्यो\accentmark{27}{\char"1CD2}जाः।\ ईशेह्याटट्यस्यामा\accentmark{22}{\char"0951}हातोवी\accentmark{22}{\char"0951}\kern0.15emरप्सी\accentmark{27}{\char"1CD2}न्।\ क्रा\accentmark{27}{\char"1CD2}तून्नानिर्मणं\accentmark{27}{\char"1CD2}स्था\accentmark{22}{\char"0951}\kern0.15emवीरं\accentmark{27}{\char"1CD2}\kern0.15emचावा\accentmark{27}{\char"1CD2}जं\accentmark{22}{\char"0951}।\ वृ\accentmark{20}{\char"1CF8}त्रे\accentmark{27}{\char"1CD2}\kern0.15emषुश\accentmark{20}{\char"1CF9}त्रू\accentmark{22}{\char"0951}न्सू\accentmark{20}{\char"1CF9}ह\accentmark{22}{\char"0951}ना\accentmark{22}{\char"0951}कृधी\accentmark{22}{\char"0951}नः\mbox{॥ ९\hspace{0pt}॥} \\
इन्द्राज्ये\accentmark{20}{\char"1CF9}ष्ठ\accentmark{22}{\char"0951}न्नआभा\accentmark{27}{\char"1CD2}रा।\ ओ\accentmark{22}{\char"0951}जीष्ठंपूपूरीश्रा\accentmark{27}{\char"1CD2}वाः\accentmark{22}{\char"0951}\kern0.15em।\ य\accentmark{27}{\char"1CD2}\kern0.15emद्या\accentmark{27}{\char"1CD2}धृ\accentmark{22}{\char"0951}क्षेमवज्रहा\accentmark{27}{\char"1CD2}स्तारोद\accentmark{22}{\char"0951}सि।\ 
ओभे\accentmark{27}{\char"1CD2}सू\accentmark{22}{\char"0951}शिप्रवप्राः \mbox{॥ १०\hspace{0pt}॥} \\
\mbox{॥ इति चतुर्थः खण्डः\hspace{0pt}॥} \\ 
ऊ\accentmark{27}{\char"1CD2}दु\accentmark{22}{\char"0951}क्तामंवारूणापा\accentmark{20}{\char"1CF9}शा\accentmark{22}{\char"0951}मस्मात्।\ आबा\accentmark{22}{\char"0951}धामं\accentmark{20}{\char"1CF9}वीमा\accentmark{22}{\char"0951}ध्यामं\accentmark{27}{\char"1CD2}श्ना\accentmark{22}{\char"0951}थाय।\ आथा\accentmark{22}{\char"0951}दित्यव्रातेवा\accentmark{27}{\char"1CD2}\kern0.15emयन्ता\accentmark{27}{\char"1CD2}वा\accentmark{22}{\char"0951}\kern0.15em।\ आ\accentmark{27}{\char"1CD2}नागसोआ\accentmark{27}{\char"1CD2}दीतयेस्याम\mbox{॥ १\hspace{0pt}॥} \\
त्वा\accentmark{20}{\char"1CF8}या\accentmark{22}{\char"0951}वायंपावा\accentmark{22}{\char"0951}मानेनसोम।\ भा\accentmark{20}{\char"1CF8}रे\accentmark{22}{\char"0951}\kern0.15emकृतं\accentmark{27}{\char"1CD2}\kern0.15emवी\accentmark{27}{\char"1CD2}ची\accentmark{22}{\char"0951}नुयामाशश्वात्।\ त\accentmark{20}{\char"1CF8}न्नो\accentmark{27}{\char"1CD2}मित्रोवा\accentmark{27}{\char"1CD2}रूणोमामहन्तां।\ आ\accentmark{20}{\char"1CF8}दि\accentmark{22}{\char"0951}तिस्सिन्धुःपृथीवीमूतद्यौः\mbox{॥ २\hspace{0pt}॥} \\
ईमां\accentmark{27}{\char"1CD2}वृषा\accentmark{22}{\char"0951}णंकृणूतैकम्माम्\mbox{॥ ३\hspace{0pt}॥} \\
साना\accentmark{22}{\char"0951}इ\accentmark{20}{\char"1CF9}न्द्रा\accentmark{22}{\char"0951}यायाज्या\accentmark{22}{\char"0951}\kern0.15emवे।\ वा\accentmark{27}{\char"1CD2}रू\accentmark{22}{\char"0951}\kern0.15emणा\accentmark{27}{\char"1CD2}यामारू\accentmark{27}{\char"1CD2}त्भ्यः।\ वरिवोवि\accentmark{27}{\char"1CD2}त्पारि\accentmark{22}{\char"0951}स्रवा \mbox{॥ ४\hspace{0pt}॥} \\
एना\accentmark{20}{\char"1CF9}वि\accentmark{20}{\char"1CF9}श्वा\accentmark{22}{\char"0951}नर्या\accentmark{22}{\char"0951}\kern0.15emआ\accentmark{27}{\char"1CD2}।\ 
द्युम्नानीमा\accentmark{27}{\char"1CD2}नूषाणां।\ 
सी\accentmark{27}{\char"1CD2}षा\accentmark{22}{\char"0951}सन्तोवनामहे।\ त्वामेताद\accentmark{22}{\char"0951}धारया।\ 
कृष्णासूरो\accentmark{27}{\char"1CD2}\kern0.15emही\accentmark{27}{\char"1CD2}णीषुच।\ 
पा\accentmark{27}{\char"1CD2}रू\accentmark{22}{\char"0951}ष्णीषूरू\accentmark{27}{\char"1CD2}\kern0.15emशत्पा\accentmark{27}{\char"1CD2}याः\mbox{॥ ५\hspace{0pt}॥} \\
प्राथश्चायास्या\accentmark{22}{\char"0951}सा\accentmark{22}{\char"0951}प्रा\accentmark{20}{\char"1CF9}थश्चानां।\ आनु\accentmark{22}{\char"0951}ष्टभस्याहावी\accentmark{20}{\char"1CF9}षो\accentmark{22}{\char"0951}हावीर्यात्।\ धातू\accentmark{27}{\char"1CD2}र्द्यू\accentmark{22}{\char"0951}ता\accentmark{22}{\char"0951}नास्सवी\accentmark{22}{\char"0951}तुश्चा\accentmark{20}{\char"1CF9}वि\accentmark{22}{\char"0951}ष्णोः।\ रा\accentmark{27}{\char"1CD2}थन्तारामा\accentmark{27}{\char"1CD2}जभारावासीष्ठम्\mbox{॥ ६\hspace{0pt}॥} \\
इ\accentmark{27}{\char"1CD2}न्द्राइ\accentmark{27}{\char"1CD2}द्धरीयो\accentmark{20}{\char"1CF8}स्सा\accentmark{27}{\char"1CD2}चा\accentmark{22}{\char"0951}।\ सम्मिश्लोआ\accentmark{20}{\char"1CF9}वा\accentmark{22}{\char"0951}\kern0.15emचोयू\accentmark{27}{\char"1CD2}जा\accentmark{22}{\char"0951}।\ 
इन्द्रो\accentmark{22}{\char"0951}वज्री\accentmark{20}{\char"1CF9}ही\accentmark{22}{\char"0951}रण्यायाः\mbox{॥ ७\hspace{0pt}॥} \\
इ\accentmark{27}{\char"1CD2}न्द्रावाजेषुनोअव।\ 
साहा\accentmark{27}{\char"1CD2}स्राप्रथनेषुच।\ 
उग्रा\accentmark{27}{\char"1CD2}\kern0.15emउग्रा\accentmark{20}{\char"1CF9}भीरू\accentmark{22}{\char"0951}तीभीः\accentmark{22}{\char"0951}\kern0.15em\mbox{॥ ८\hspace{0pt}॥} \\
आ\accentmark{27}{\char"1CD2}रूरूचादूषा\accentmark{27}{\char"1CD2}साः\accentmark{22}{\char"0951}।\ प्रश्नी\accentmark{22}{\char"0951}राग्रा\accentmark{27}{\char"1CD2}यूः।\ 
उक्षा\accentmark{27}{\char"1CD2}\kern0.15emमीमा\accentmark{27}{\char"1CD2}तीभूवा\accentmark{27}{\char"1CD2}नेषुवाजयुः।\ मायाविनो\accentmark{22}{\char"0951}ममिरे।\ अ\accentmark{27}{\char"1CD2}स्यामाया\accentmark{27}{\char"1CD2}या\accentmark{22}{\char"0951}नृचाक्षा\accentmark{22}{\char"0951}सः।\ 
पीतारोग\accentmark{27}{\char"1CD2}र्भामाद\accentmark{22}{\char"0951}धुः \mbox{॥ १०\hspace{0pt}॥} \\
\mbox{॥ इति पञ्चमः खण्डः\hspace{0pt}॥} \\ 
यज्जा\accentmark{27}{\char"1CD2}या\accentmark{22}{\char"0951}था\accentmark{22}{\char"0951}पूर्व्यामाघा\accentmark{22}{\char"0951}वन्वृत्राहा\accentmark{27}{\char"1CD2}था\accentmark{22}{\char"0951}या।\ त\accentmark{20}{\char"1CF8}त्पृथिवीमाप्रा\accentmark{22}{\char"0951}था\accentmark{22}{\char"0951}या।\ ताद\accentmark{22}{\char"0951}स्तभ्नाऊतादीव\accentmark{22}{\char"0951}म्\mbox{॥ १\hspace{0pt}॥} \\
अग्नआयूं\accentmark{22}{\char"0951}षिपावसे।\ आसूर्वोर्जामी\accentmark{27}{\char"1CD2}षञ्चनः।\ 
आरे\accentmark{27}{\char"1CD2}\kern0.15emबा\accentmark{27}{\char"1CD2}धास्वदु\accentmark{22}{\char"0951}च्छूना\accentmark{22}{\char"0951}म्\mbox{॥ २\hspace{0pt}॥} \\
आयं\accentmark{27}{\char"1CD2}गौःप्रश्नी\accentmark{22}{\char"0951}रक्रामीत्।\ आ\accentmark{27}{\char"1CD2}\kern0.15emसा\accentmark{27}{\char"1CD2}दन्माता\accentmark{20}{\char"1CF9}रं\accentmark{22}{\char"0951}पूनः।\ पी\accentmark{27}{\char"1CD2}तार\accentmark{22}{\char"0951}ञ्चाप्राया\accentmark{27}{\char"1CD2}न्स्वाः\mbox{॥ ३\hspace{0pt}॥} \\
अ\accentmark{27}{\char"1CD2}\kern0.15emन्ता\accentmark{27}{\char"1CD2}श्चा\accentmark{22}{\char"0951}रतिरोचन\accentmark{27}{\char"1CD2}।\ 
अस्य\accentmark{27}{\char"1CD2}प्राणा\accentmark{27}{\char"1CD2}दपानातीव्य\accentmark{27}{\char"1CD2}ख्यन्मा\accentmark{27}{\char"1CD2}हीषोदीवम् \mbox{॥ ४\hspace{0pt}॥} \\
त्रिंश\accentmark{27}{\char"1CD2}द्धामावी\accentmark{27}{\char"1CD2}राजति।\ 
वा\accentmark{20}{\char"1CF8}क्पा\accentmark{22}{\char"0951}तंगाया\accentmark{22}{\char"0951}धीयते।\ 
प्रा\accentmark{27}{\char"1CD2}\kern0.15emतीव\accentmark{27}{\char"1CD2}स्तोरा\accentmark{27}{\char"1CD2}\kern0.15emहद्यू\accentmark{27}{\char"1CD2}भीः\mbox{॥ ५\hspace{0pt}॥} \\
\mbox{॥ इति षष्ठः खण्डः\hspace{0pt}॥} \\ 
\mbox{॥ इत्यारण पाठः समाप्तः\hspace{0pt}॥} \\ \clearpage
\mbox{॥ अथ शाक्वर पर्वा\hspace{0pt}॥} \\
वीदा\accentmark{27}{\char"1CD2}मा\accentmark{22}{\char"0951}घवन्वीदा\accentmark{27}{\char"1CD2}गातूमा\accentmark{27}{\char"1CD2}नूशं\accentmark{22}{\char"0951}सीषोदीशाः\accentmark{22}{\char"0951}\kern0.15em।\ शि\accentmark{27}{\char"1CD2}क्षा\accentmark{22}{\char"0951}शचीनांपतेपूर्वीणां\accentmark{27}{\char"1CD2}पू\accentmark{22}{\char"0951}रो\accentmark{20}{\char"1CF8}वसो।\ आ\accentmark{27}{\char"1CD2}भीष्ट्‌वा\accentmark{27}{\char"1CD2}\kern0.15emमा\accentmark{20}{\char"1CF9}भीष्टी\accentmark{22}{\char"0951}भिःप्रा\accentmark{27}{\char"1CD2}चे\accentmark{22}{\char"0951}तानःप्रा\accentmark{27}{\char"1CD2}चेतायेन्द्रा\accentmark{20}{\char"1CF9}द्यु\accentmark{22}{\char"0951}\kern0.15emम्ना\accentmark{27}{\char"1CD2}या\accentmark{22}{\char"0951}नाईषे।\ एवा\accentmark{27}{\char"1CD2}हीशक्रो\accentmark{27}{\char"1CD2}रायेवा\accentmark{27}{\char"1CD2}जा\accentmark{22}{\char"0951}यवज्रीवः।\ शा\accentmark{27}{\char"1CD2}वीष्ठवज्रिन्वृञ्ज\accentmark{27}{\char"1CD2}से
मंहिष्ठ\accentmark{27}{\char"1CD2}वज्रिन्वृन्जसाआ\accentmark{20}{\char"1CF9}या\accentmark{22}{\char"0951}हीपीबामात्स्वा \mbox{॥ १\hspace{0pt}॥} \\
वीदाराये\accentmark{27}{\char"1CD2}\kern0.15emसूवी\accentmark{27}{\char"1CD2}र्यं\accentmark{22}{\char"0951}।\ भूवोवा\accentmark{20}{\char"1CF9}जा\accentmark{22}{\char"0951}नांपातिर्वाशंआ\accentmark{27}{\char"1CD2}नू।\ मंहिष्ठवज्रिन्वृञ्ज\accentmark{27}{\char"1CD2}सेयश्शावी\accentmark{22}{\char"0951}ष्ठश्शू\accentmark{20}{\char"1CF9}रा\accentmark{20}{\char"1CF9}णाम्।\ योमं\accentmark{27}{\char"1CD2}ही\accentmark{22}{\char"0951}ष्ठोमाघो\accentmark{27}{\char"1CD2}नां\accentmark{22}{\char"0951}।\ चिकित्वोआभी\accentmark{20}{\char"1CF9}नो\accentmark{22}{\char"0951}\kern0.15emनाये\accentmark{27}{\char"1CD2}न्द्रोवीदेता\accentmark{27}{\char"1CD2}मूस्तुही।\ ई\accentmark{27}{\char"1CD2}\kern0.15emशे\accentmark{20}{\char"1CF9}ही\accentmark{22}{\char"0951}शक्र\accentmark{20}{\char"1CF9}स्ता\accentmark{22}{\char"0951}मूताये\accentmark{22}{\char"0951}हावा\accentmark{22}{\char"0951}माहे\accentmark{20}{\char"1CF9}जेता\accentmark{22}{\char"0951}रामापा\accentmark{22}{\char"0951}राजितं।\ सा\accentmark{27}{\char"1CD2}नाः\accentmark{22}{\char"0951}\kern0.15emपरीषा\accentmark{27}{\char"1CD2}\kern0.15emद\accentmark{27}{\char"1CD2}तिद्वीषः\accentmark{22}{\char"0951}क्रातुच्छन्दोऋतंबृहात्\mbox{॥ २\hspace{0pt}॥} \\
इन्द्रंधानास्यासाताये।\ 
हावा\accentmark{22}{\char"0951}महेजेतारामापाराजित\accentmark{27}{\char"1CD2}\kern0.15emम्।\ सा\accentmark{27}{\char"1CD2}नाः\accentmark{22}{\char"0951}पारीषद\accentmark{22}{\char"0951}तिद्वीषस्सा\accentmark{27}{\char"1CD2}नाः\accentmark{22}{\char"0951}\kern0.15emपा\accentmark{27}{\char"1CD2}री\accentmark{22}{\char"0951}\kern0.15emषाद\accentmark{27}{\char"1CD2}तिद्वीषाः\accentmark{22}{\char"0951}।\ 
पू\accentmark{20}{\char"1CF8}र्वास्याया\accentmark{22}{\char"0951}क्ते\accentmark{22}{\char"0951}अद्रिवः।\ 
सुम्नाधे\accentmark{22}{\char"0951}हिनो\accentmark{22}{\char"0951}वसो।\ 
पूर्तीश्शा\accentmark{27}{\char"1CD2}वीष्ठशस्य\accentmark{27}{\char"1CD2}\kern0.15emते।\ 
व\accentmark{27}{\char"1CD2}शीहीशक्रोनून\accentmark{20}{\char"1CF9}न्त\accentmark{20}{\char"1CF9}न्नव्यं\accentmark{22}{\char"0951}\kern0.15emसन्या\accentmark{27}{\char"1CD2}से।\ 
प्रा\accentmark{27}{\char"1CD2}\kern0.15emभो\accentmark{27}{\char"1CD2}र्जनस्यवृत्रा\accentmark{22}{\char"0951}\kern0.15emह\accentmark{27}{\char"1CD2}न्सामर्ये\accentmark{27}{\char"1CD2}षु\accentmark{22}{\char"0951}स्तवामहे।\ 
शू\accentmark{27}{\char"1CD2}\kern0.15emरोयो\accentmark{27}{\char"1CD2}गोषूगच्छा\accentmark{22}{\char"0951}ति।\ 
सा\accentmark{20}{\char"1CF8}खा\accentmark{22}{\char"0951}\kern0.15emसूशे\accentmark{27}{\char"1CD2}वोआद्रा\accentmark{27}{\char"1CD2}\kern0.15emयुः\accentmark{27}{\char"1CD2}\mbox{॥ ३\hspace{0pt}॥} \\
एवा\accentmark{27}{\char"1CD2}ह्ये\accentmark{22}{\char"0951}वाएवाह्याग्ने\accentmark{22}{\char"0951}एवाही\accentmark{22}{\char"0951}न्द्राएवा\accentmark{27}{\char"1CD2}\kern0.15emही\accentmark{27}{\char"1CD2}पूष\accentmark{22}{\char"0951}न्एवाही\accentmark{27}{\char"1CD2}दे\accentmark{22}{\char"0951}वाः\mbox{॥ ४\hspace{0pt}॥} \\
\mbox{॥ इति शाक्वरपर्वः\hspace{0pt}॥} \\   
\mbox{॥ इति जैमिनीय पूर्वाचिक ऋक् संहिता\hspace{0pt}॥} \\ 

काण्डंखण्डंऋक्
आग्नेयम्12116
तद्वम्12118
बृहति880
असावि657
ऐन्द्रम्1097
पवमानम्11119
आरणम्654
शाक्वरम्14
Total6664595

\end{document}
